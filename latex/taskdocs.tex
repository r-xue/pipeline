%% Generated by Sphinx.
\def\sphinxdocclass{report}
\documentclass[letterpaper,10pt,openany,oneside,english]{sphinxmanual}
\ifdefined\pdfpxdimen
   \let\sphinxpxdimen\pdfpxdimen\else\newdimen\sphinxpxdimen
\fi \sphinxpxdimen=.75bp\relax
\ifdefined\pdfimageresolution
    \pdfimageresolution= \numexpr \dimexpr1in\relax/\sphinxpxdimen\relax
\fi
%% let collapsible pdf bookmarks panel have high depth per default
\PassOptionsToPackage{bookmarksdepth=5}{hyperref}

\PassOptionsToPackage{booktabs}{sphinx}
\PassOptionsToPackage{colorrows}{sphinx}

\PassOptionsToPackage{warn}{textcomp}
\usepackage[utf8]{inputenc}
\ifdefined\DeclareUnicodeCharacter
% support both utf8 and utf8x syntaxes
  \ifdefined\DeclareUnicodeCharacterAsOptional
    \def\sphinxDUC#1{\DeclareUnicodeCharacter{"#1}}
  \else
    \let\sphinxDUC\DeclareUnicodeCharacter
  \fi
  \sphinxDUC{00A0}{\nobreakspace}
  \sphinxDUC{2500}{\sphinxunichar{2500}}
  \sphinxDUC{2502}{\sphinxunichar{2502}}
  \sphinxDUC{2514}{\sphinxunichar{2514}}
  \sphinxDUC{251C}{\sphinxunichar{251C}}
  \sphinxDUC{2572}{\textbackslash}
\fi
\usepackage{cmap}
\usepackage[T1]{fontenc}
\usepackage{amsmath,amssymb,amstext}
\usepackage[english]{babel}



\usepackage{tgtermes}
\usepackage{tgheros}
\renewcommand{\ttdefault}{txtt}



\usepackage[Bjarne]{fncychap}
\usepackage{sphinx}

\fvset{fontsize=auto}
\usepackage{geometry}

\usepackage{nbsphinx}

% Include hyperref last.
\usepackage{hyperref}
% Fix anchor placement for figures with captions.
\usepackage{hypcap}% it must be loaded after hyperref.
% Set up styles of URL: it should be placed after hyperref.
\urlstyle{same}

\addto\captionsenglish{\renewcommand{\contentsname}{Pipeline Tasks}}

\usepackage{sphinxmessages}



  \usepackage{hyperref}
  \usepackage{longtable}
  \setcounter{tocdepth}{1}
  \protected\def\sphinxcode#1{\textcolor{red}{\texttt{#1}}}
  \makeatletter
  \renewcommand{\sphinxtableofcontents}{%
    \pagenumbering{roman}%
    \begingroup
      \parskip \z@skip
      \sphinxtableofcontentshook
      \tableofcontents
    \endgroup
    % before resetting page counter, let's do the right thing.
    \if@openright\cleardoublepage\else\clearpage\fi
    \pagenumbering{arabic}% 
    \makeatother
   }

\usepackage{natbib}

\makeatletter
\renewcommand\@biblabel[1]{}
\renewenvironment{sphinxthebibliography}[1]
     {\chapter*{\refname}%
      \addcontentsline{toc}{chapter}{References}
      \@mkboth{\MakeUppercase\refname}{\MakeUppercase\refname}%
      \list{}%
           {\leftmargin0pt
            \@openbib@code
            \usecounter{enumiv}}%
      \sloppy
      \clubpenalty4000
      \@clubpenalty \clubpenalty
      \widowpenalty4000%
      \sfcode`\.\@m}
     {\def\@noitemerr
       {\@latex@warning{Empty `thebibliography' environment}}%
      \endlist}
\makeatother
        

\title{Pipeline Tasks Reference Manual}
\date{Dec 13, 2024}
\release{unknown}
\author{pipeline team}
\newcommand{\sphinxlogo}{\vbox{}}
\renewcommand{\releasename}{Release}
\makeindex
\begin{document}

\ifdefined\shorthandoff
  \ifnum\catcode`\=\string=\active\shorthandoff{=}\fi
  \ifnum\catcode`\"=\active\shorthandoff{"}\fi
\fi

\pagestyle{empty}
\sphinxmaketitle
\pagestyle{plain}
\sphinxtableofcontents
\pagestyle{normal}
\phantomsection\label{\detokenize{latex::doc}}



\begin{savenotes}\sphinxattablestart
\sphinxthistablewithglobalstyle
\sphinxthistablewithnovlinesstyle
\centering
\begin{tabulary}{\linewidth}[t]{\X{1}{2}\X{1}{2}}
\sphinxtoprule
\sphinxtableatstartofbodyhook
\sphinxAtStartPar
{\hyperref[\detokenize{_autosummary/pipeline.h.cli:module-pipeline.h.cli}]{\sphinxcrossref{\sphinxcode{\sphinxupquote{pipeline.h.cli}}}}}
&
\sphinxAtStartPar
Generic Tasks
\\
\sphinxhline
\sphinxAtStartPar
{\hyperref[\detokenize{_autosummary/pipeline.hif.cli:module-pipeline.hif.cli}]{\sphinxcrossref{\sphinxcode{\sphinxupquote{pipeline.hif.cli}}}}}
&
\sphinxAtStartPar
Interferometry Generic Tasks
\\
\sphinxhline
\sphinxAtStartPar
{\hyperref[\detokenize{_autosummary/pipeline.hifa.cli:module-pipeline.hifa.cli}]{\sphinxcrossref{\sphinxcode{\sphinxupquote{pipeline.hifa.cli}}}}}
&
\sphinxAtStartPar
Interferometry ALMA Tasks
\\
\sphinxhline
\sphinxAtStartPar
{\hyperref[\detokenize{_autosummary/pipeline.hifv.cli:module-pipeline.hifv.cli}]{\sphinxcrossref{\sphinxcode{\sphinxupquote{pipeline.hifv.cli}}}}}
&
\sphinxAtStartPar
Interferometry VLA Tasks
\\
\sphinxhline
\sphinxAtStartPar
{\hyperref[\detokenize{_autosummary/pipeline.hsd.cli:module-pipeline.hsd.cli}]{\sphinxcrossref{\sphinxcode{\sphinxupquote{pipeline.hsd.cli}}}}}
&
\sphinxAtStartPar
ALMA Single Dish Tasks
\\
\sphinxhline
\sphinxAtStartPar
{\hyperref[\detokenize{_autosummary/pipeline.hsdn.cli:module-pipeline.hsdn.cli}]{\sphinxcrossref{\sphinxcode{\sphinxupquote{pipeline.hsdn.cli}}}}}
&
\sphinxAtStartPar
Nobeyama Tasks
\\
\sphinxbottomrule
\end{tabulary}
\sphinxtableafterendhook\par
\sphinxattableend\end{savenotes}

\sphinxstepscope


\chapter{pipeline.h.cli}
\label{\detokenize{_autosummary/pipeline.h.cli:module-pipeline.h.cli}}\label{\detokenize{_autosummary/pipeline.h.cli:pipeline-h-cli}}\label{\detokenize{_autosummary/pipeline.h.cli::doc}}\index{module@\spxentry{module}!pipeline.h.cli@\spxentry{pipeline.h.cli}}\index{pipeline.h.cli@\spxentry{pipeline.h.cli}!module@\spxentry{module}}
\sphinxAtStartPar
Generic Tasks
\subsubsection*{Functions}


\begin{savenotes}\sphinxattablestart
\sphinxthistablewithglobalstyle
\sphinxthistablewithnovlinesstyle
\centering
\begin{tabulary}{\linewidth}[t]{\X{1}{2}\X{1}{2}}
\sphinxtoprule
\sphinxtableatstartofbodyhook
\sphinxAtStartPar
{\hyperref[\detokenize{_autosummary/pipeline.h.cli.h_applycal:pipeline.h.cli.h_applycal}]{\sphinxcrossref{\sphinxcode{\sphinxupquote{h\_applycal}}}}}
&
\sphinxAtStartPar
Apply the calibration(s) to the data
\\
\sphinxhline
\sphinxAtStartPar
{\hyperref[\detokenize{_autosummary/pipeline.h.cli.h_export_calstate:pipeline.h.cli.h_export_calstate}]{\sphinxcrossref{\sphinxcode{\sphinxupquote{h\_export\_calstate}}}}}
&
\sphinxAtStartPar
Save the pipeline calibration state to disk
\\
\sphinxhline
\sphinxAtStartPar
{\hyperref[\detokenize{_autosummary/pipeline.h.cli.h_exportdata:pipeline.h.cli.h_exportdata}]{\sphinxcrossref{\sphinxcode{\sphinxupquote{h\_exportdata}}}}}
&
\sphinxAtStartPar
Prepare interferometry data for export
\\
\sphinxhline
\sphinxAtStartPar
{\hyperref[\detokenize{_autosummary/pipeline.h.cli.h_import_calstate:pipeline.h.cli.h_import_calstate}]{\sphinxcrossref{\sphinxcode{\sphinxupquote{h\_import\_calstate}}}}}
&
\sphinxAtStartPar
Import a calibration state from disk
\\
\sphinxhline
\sphinxAtStartPar
{\hyperref[\detokenize{_autosummary/pipeline.h.cli.h_importdata:pipeline.h.cli.h_importdata}]{\sphinxcrossref{\sphinxcode{\sphinxupquote{h\_importdata}}}}}
&
\sphinxAtStartPar
Imports data into the interferometry pipeline
\\
\sphinxhline
\sphinxAtStartPar
{\hyperref[\detokenize{_autosummary/pipeline.h.cli.h_init:pipeline.h.cli.h_init}]{\sphinxcrossref{\sphinxcode{\sphinxupquote{h\_init}}}}}
&
\sphinxAtStartPar
Initialize the pipeline
\\
\sphinxhline
\sphinxAtStartPar
{\hyperref[\detokenize{_autosummary/pipeline.h.cli.h_mssplit:pipeline.h.cli.h_mssplit}]{\sphinxcrossref{\sphinxcode{\sphinxupquote{h\_mssplit}}}}}
&
\sphinxAtStartPar
Select data from calibrated MS(s) to form new MS(s) for imaging
\\
\sphinxhline
\sphinxAtStartPar
{\hyperref[\detokenize{_autosummary/pipeline.h.cli.h_restoredata:pipeline.h.cli.h_restoredata}]{\sphinxcrossref{\sphinxcode{\sphinxupquote{h\_restoredata}}}}}
&
\sphinxAtStartPar
Restore flags and calibration state from a pipeline run
\\
\sphinxhline
\sphinxAtStartPar
{\hyperref[\detokenize{_autosummary/pipeline.h.cli.h_resume:pipeline.h.cli.h_resume}]{\sphinxcrossref{\sphinxcode{\sphinxupquote{h\_resume}}}}}
&
\sphinxAtStartPar
Restore a saved pipeline state
\\
\sphinxhline
\sphinxAtStartPar
{\hyperref[\detokenize{_autosummary/pipeline.h.cli.h_save:pipeline.h.cli.h_save}]{\sphinxcrossref{\sphinxcode{\sphinxupquote{h\_save}}}}}
&
\sphinxAtStartPar
Save the pipeline state to disk
\\
\sphinxhline
\sphinxAtStartPar
{\hyperref[\detokenize{_autosummary/pipeline.h.cli.h_show_calstate:pipeline.h.cli.h_show_calstate}]{\sphinxcrossref{\sphinxcode{\sphinxupquote{h\_show\_calstate}}}}}
&
\sphinxAtStartPar
Show the current pipeline calibration state
\\
\sphinxhline
\sphinxAtStartPar
{\hyperref[\detokenize{_autosummary/pipeline.h.cli.h_tsyscal:pipeline.h.cli.h_tsyscal}]{\sphinxcrossref{\sphinxcode{\sphinxupquote{h\_tsyscal}}}}}
&
\sphinxAtStartPar
Derive a Tsys calibration table
\\
\sphinxhline
\sphinxAtStartPar
{\hyperref[\detokenize{_autosummary/pipeline.h.cli.h_weblog:pipeline.h.cli.h_weblog}]{\sphinxcrossref{\sphinxcode{\sphinxupquote{h\_weblog}}}}}
&
\sphinxAtStartPar
Open the pipeline weblog in a browser
\\
\sphinxbottomrule
\end{tabulary}
\sphinxtableafterendhook\par
\sphinxattableend\end{savenotes}

\sphinxstepscope


\section{pipeline.h.cli.h\_applycal}
\label{\detokenize{_autosummary/pipeline.h.cli.h_applycal:pipeline-h-cli-h-applycal}}\label{\detokenize{_autosummary/pipeline.h.cli.h_applycal::doc}}\index{h\_applycal() (in module pipeline.h.cli)@\spxentry{h\_applycal()}\spxextra{in module pipeline.h.cli}}

\begin{fulllineitems}
\phantomsection\label{\detokenize{_autosummary/pipeline.h.cli.h_applycal:pipeline.h.cli.h_applycal}}
\pysigstartsignatures
\pysiglinewithargsret
{\sphinxbfcode{\sphinxupquote{h\_applycal}}}
{\sphinxparam{\DUrole{n}{vis}\DUrole{o}{=}\DUrole{default_value}{None}}\sphinxparamcomma \sphinxparam{\DUrole{n}{field}\DUrole{o}{=}\DUrole{default_value}{None}}\sphinxparamcomma \sphinxparam{\DUrole{n}{intent}\DUrole{o}{=}\DUrole{default_value}{None}}\sphinxparamcomma \sphinxparam{\DUrole{n}{spw}\DUrole{o}{=}\DUrole{default_value}{None}}\sphinxparamcomma \sphinxparam{\DUrole{n}{antenna}\DUrole{o}{=}\DUrole{default_value}{None}}\sphinxparamcomma \sphinxparam{\DUrole{n}{parang}\DUrole{o}{=}\DUrole{default_value}{None}}\sphinxparamcomma \sphinxparam{\DUrole{n}{applymode}\DUrole{o}{=}\DUrole{default_value}{None}}\sphinxparamcomma \sphinxparam{\DUrole{n}{flagbackup}\DUrole{o}{=}\DUrole{default_value}{None}}\sphinxparamcomma \sphinxparam{\DUrole{n}{flagsum}\DUrole{o}{=}\DUrole{default_value}{None}}\sphinxparamcomma \sphinxparam{\DUrole{n}{flagdetailedsum}\DUrole{o}{=}\DUrole{default_value}{None}}\sphinxparamcomma \sphinxparam{\DUrole{n}{parallel}\DUrole{o}{=}\DUrole{default_value}{None}}}
{}
\pysigstopsignatures
\sphinxAtStartPar
Apply the calibration(s) to the data

\sphinxAtStartPar
Apply precomputed calibrations to the data.

\sphinxAtStartPar
h\_applycal applies the precomputed calibration tables stored in the pipeline
context to the set of visibility files using predetermined field and
spectral window maps and default values for the interpolation schemes.

\sphinxAtStartPar
Users can interact with the pipeline calibration state using the tasks
h\_export\_calstate and h\_import\_calstate.
\begin{quote}\begin{description}
\sphinxlineitem{Parameters}\begin{itemize}
\item {} 
\sphinxAtStartPar
\sphinxstylestrong{vis} \textendash{} The list of input MeasurementSets. Defaults to the list of MeasurementSets in the pipeline context.
example: {[}‘X227.ms’{]}

\item {} 
\sphinxAtStartPar
\sphinxstylestrong{field} \textendash{} A string containing the list of field names or field ids to which the calibration will be applied. Defaults to all fields in the pipeline
context.
example: ‘3C279’, ‘3C279, M82’

\item {} 
\sphinxAtStartPar
\sphinxstylestrong{intent} \textendash{} A string containing the list of intents against which the selected fields will be matched. Defaults to all supported intents
in the pipeline context.
example: \sphinxtitleref{‘*TARGET*’}

\item {} 
\sphinxAtStartPar
\sphinxstylestrong{spw} \textendash{} The list of spectral windows and channels to which the calibration will be applied. Defaults to all science windows in the pipeline
context.
example: ‘17’, ‘11, 15’

\item {} 
\sphinxAtStartPar
\sphinxstylestrong{antenna} \textendash{} The selection of antennas to which the calibration will be applied. Defaults to all antennas. Not currently supported.

\item {} 
\sphinxAtStartPar
\sphinxstylestrong{parang} \textendash{} Apply parallactic angle correction

\item {} 
\sphinxAtStartPar
\sphinxstylestrong{applymode} \textendash{} Calibration apply mode ‘calflag’: calibrate data and apply flags from solutions
‘calflagstrict’: (default) same as above except flag spws for which calibration is
unavailable in one or more tables (instead of allowing them to pass
uncalibrated and unflagged)
‘trial’: report on flags from solutions, dataset entirely unchanged
‘flagonly’: apply flags from solutions only, data not calibrated
‘flagonlystrict’: same as above except flag spws for which calibration is
unavailable in one or more tables
‘calonly’: calibrate data only, flags from solutions NOT applied

\item {} 
\sphinxAtStartPar
\sphinxstylestrong{flagbackup} \textendash{} Backup the flags before the apply

\item {} 
\sphinxAtStartPar
\sphinxstylestrong{flagsum} \textendash{} Compute before and after flagging summary statistics

\item {} 
\sphinxAtStartPar
\sphinxstylestrong{flagdetailedsum} \textendash{} Compute detailed before and after flagging statistics summaries. Parameter available only when if flagsum is True.

\item {} 
\sphinxAtStartPar
\sphinxstylestrong{parallel} \textendash{} Execute using CASA HPC functionality, if available. options: ‘automatic’, ‘true’, ‘false’, True, False
default: None (equivalent to False)

\end{itemize}

\sphinxlineitem{Returns}
\sphinxAtStartPar
The results object for the pipeline task is returned.

\end{description}\end{quote}
\subsubsection*{Examples}
\begin{enumerate}
\sphinxsetlistlabels{\arabic}{enumi}{enumii}{}{.}%
\item {} 
\sphinxAtStartPar
Apply the calibration to the target data

\end{enumerate}

\begin{sphinxVerbatim}[commandchars=\\\{\}]
\PYG{g+gp}{\PYGZgt{}\PYGZgt{}\PYGZgt{} }\PYG{n}{hif\PYGZus{}applycal} \PYG{p}{(}\PYG{n}{intent}\PYG{o}{=}\PYG{l+s+s1}{\PYGZsq{}}\PYG{l+s+s1}{TARGET}\PYG{l+s+s1}{\PYGZsq{}}\PYG{p}{)}
\end{sphinxVerbatim}

\end{fulllineitems}


\sphinxstepscope


\section{pipeline.h.cli.h\_export\_calstate}
\label{\detokenize{_autosummary/pipeline.h.cli.h_export_calstate:pipeline-h-cli-h-export-calstate}}\label{\detokenize{_autosummary/pipeline.h.cli.h_export_calstate::doc}}\index{h\_export\_calstate() (in module pipeline.h.cli)@\spxentry{h\_export\_calstate()}\spxextra{in module pipeline.h.cli}}

\begin{fulllineitems}
\phantomsection\label{\detokenize{_autosummary/pipeline.h.cli.h_export_calstate:pipeline.h.cli.h_export_calstate}}
\pysigstartsignatures
\pysiglinewithargsret
{\sphinxbfcode{\sphinxupquote{h\_export\_calstate}}}
{\sphinxparam{\DUrole{n}{filename}\DUrole{o}{=}\DUrole{default_value}{None}}\sphinxparamcomma \sphinxparam{\DUrole{n}{state}\DUrole{o}{=}\DUrole{default_value}{None}}}
{}
\pysigstopsignatures
\sphinxAtStartPar
Save the pipeline calibration state to disk

\sphinxAtStartPar
h\_export\_calstate saves the current pipeline calibration state to disk
in the form of a set of equivalent applycal calls.

\sphinxAtStartPar
If \sphinxcode{\sphinxupquote{filename}} is not given, h\_export\_calstate saves the calibration state to
disk with a filename based on the pipeline context creation time, using the
extension ‘.calstate’

\sphinxAtStartPar
One of two calibration states can be exported: either the active calibration
state (those calibrations currently applied on\sphinxhyphen{}the\sphinxhyphen{}fly but scheduled for
permanent application to the MeasurementSet in a subsequent hif\_applycal
call) or the applied calibration state (calibrations that were previously
applied to the MeasurementSet using hif\_applycal). The default is to export
the active calibration state.
\begin{quote}\begin{description}
\sphinxlineitem{Parameters}\begin{itemize}
\item {} 
\sphinxAtStartPar
\sphinxstylestrong{filename} \textendash{} Name for saved calibration state.

\item {} 
\sphinxAtStartPar
\sphinxstylestrong{state} \textendash{} The calibration state to export. If undefined, the active
calibration state will be exported. If set to ‘applied’, the
applied calibration state will be exported instead.

\end{itemize}

\sphinxlineitem{Returns}
\sphinxAtStartPar
The results object for the pipeline task is returned.

\end{description}\end{quote}
\subsubsection*{Examples}
\begin{enumerate}
\sphinxsetlistlabels{\arabic}{enumi}{enumii}{}{.}%
\item {} 
\sphinxAtStartPar
Save the calibration state:

\end{enumerate}

\begin{sphinxVerbatim}[commandchars=\\\{\}]
\PYG{g+gp}{\PYGZgt{}\PYGZgt{}\PYGZgt{} }\PYG{n}{h\PYGZus{}export\PYGZus{}calstate}\PYG{p}{(}\PYG{p}{)}
\end{sphinxVerbatim}
\begin{enumerate}
\sphinxsetlistlabels{\arabic}{enumi}{enumii}{}{.}%
\setcounter{enumi}{1}
\item {} 
\sphinxAtStartPar
Save the active calibration state with a custom filename:

\end{enumerate}

\begin{sphinxVerbatim}[commandchars=\\\{\}]
\PYG{g+gp}{\PYGZgt{}\PYGZgt{}\PYGZgt{} }\PYG{n}{h\PYGZus{}export\PYGZus{}calstate}\PYG{p}{(}\PYG{n}{filename}\PYG{o}{=}\PYG{l+s+s1}{\PYGZsq{}}\PYG{l+s+s1}{afterbandpass.calstate}\PYG{l+s+s1}{\PYGZsq{}}\PYG{p}{)}
\end{sphinxVerbatim}
\begin{enumerate}
\sphinxsetlistlabels{\arabic}{enumi}{enumii}{}{.}%
\setcounter{enumi}{2}
\item {} 
\sphinxAtStartPar
Save the applied calibration state with a custom filename:

\end{enumerate}

\begin{sphinxVerbatim}[commandchars=\\\{\}]
\PYG{g+gp}{\PYGZgt{}\PYGZgt{}\PYGZgt{} }\PYG{n}{h\PYGZus{}export\PYGZus{}calstate}\PYG{p}{(}\PYG{n}{filename}\PYG{o}{=}\PYG{l+s+s1}{\PYGZsq{}}\PYG{l+s+s1}{applied.calstate}\PYG{l+s+s1}{\PYGZsq{}}\PYG{p}{,} \PYG{n}{state}\PYG{o}{=}\PYG{l+s+s1}{\PYGZsq{}}\PYG{l+s+s1}{applied}\PYG{l+s+s1}{\PYGZsq{}}\PYG{p}{)}
\end{sphinxVerbatim}

\end{fulllineitems}


\sphinxstepscope


\section{pipeline.h.cli.h\_exportdata}
\label{\detokenize{_autosummary/pipeline.h.cli.h_exportdata:pipeline-h-cli-h-exportdata}}\label{\detokenize{_autosummary/pipeline.h.cli.h_exportdata::doc}}\index{h\_exportdata() (in module pipeline.h.cli)@\spxentry{h\_exportdata()}\spxextra{in module pipeline.h.cli}}

\begin{fulllineitems}
\phantomsection\label{\detokenize{_autosummary/pipeline.h.cli.h_exportdata:pipeline.h.cli.h_exportdata}}
\pysigstartsignatures
\pysiglinewithargsret
{\sphinxbfcode{\sphinxupquote{h\_exportdata}}}
{\sphinxparam{\DUrole{n}{vis}\DUrole{o}{=}\DUrole{default_value}{None}}\sphinxparamcomma \sphinxparam{\DUrole{n}{session}\DUrole{o}{=}\DUrole{default_value}{None}}\sphinxparamcomma \sphinxparam{\DUrole{n}{imaging\_products\_only}\DUrole{o}{=}\DUrole{default_value}{None}}\sphinxparamcomma \sphinxparam{\DUrole{n}{exportmses}\DUrole{o}{=}\DUrole{default_value}{None}}\sphinxparamcomma \sphinxparam{\DUrole{n}{pprfile}\DUrole{o}{=}\DUrole{default_value}{None}}\sphinxparamcomma \sphinxparam{\DUrole{n}{calintents}\DUrole{o}{=}\DUrole{default_value}{None}}\sphinxparamcomma \sphinxparam{\DUrole{n}{calimages}\DUrole{o}{=}\DUrole{default_value}{None}}\sphinxparamcomma \sphinxparam{\DUrole{n}{targetimages}\DUrole{o}{=}\DUrole{default_value}{None}}\sphinxparamcomma \sphinxparam{\DUrole{n}{products\_dir}\DUrole{o}{=}\DUrole{default_value}{None}}}
{}
\pysigstopsignatures
\sphinxAtStartPar
Prepare interferometry data for export

\sphinxAtStartPar
The hif\_exportdata task exports the data defined in the pipeline context
and exports it to the data products directory, converting and or
packing it as necessary.

\sphinxAtStartPar
The current version of the task exports the following products
\begin{itemize}
\item {} 
\sphinxAtStartPar
an XML file containing the pipeline processing request

\item {} 
\sphinxAtStartPar
a tar file per ASDM / MS containing the final flags version

\item {} 
\sphinxAtStartPar
a text file per ASDM / MS containing the final calibration apply list

\item {} 
\sphinxAtStartPar
a FITS image for each selected calibrator source image

\item {} 
\sphinxAtStartPar
a FITS image for each selected science target source image

\item {} 
\sphinxAtStartPar
a tar file per session containing the caltables for that session

\item {} 
\sphinxAtStartPar
a tar file containing the file web log

\item {} 
\sphinxAtStartPar
a text file containing the final list of CASA commands

\end{itemize}

\sphinxAtStartPar
Returns

\sphinxAtStartPar
The results object for the pipeline task is returned.
\begin{quote}\begin{description}
\sphinxlineitem{Parameters}\begin{itemize}
\item {} 
\sphinxAtStartPar
\sphinxstylestrong{vis} \textendash{} List of visibility data files for which flagging and calibration information will be exported. Defaults to the list maintained in the
pipeline context.
example: vis={[}‘X227.ms’, ‘X228.ms’{]}

\item {} 
\sphinxAtStartPar
\sphinxstylestrong{session} \textendash{} List of sessions one per visibility file. Defaults to a single virtual session containing all the visibility files in vis.
example: session={[}‘session1’, ‘session2’{]}

\item {} 
\sphinxAtStartPar
\sphinxstylestrong{imaging\_products\_only} \textendash{} Export the science target image products only

\item {} 
\sphinxAtStartPar
\sphinxstylestrong{exportmses} \textendash{} Export MeasurementSets defined in vis instead of flags, caltables, and calibration instructions.
example: exportmses = True

\item {} 
\sphinxAtStartPar
\sphinxstylestrong{pprfile} \textendash{} Name of the pipeline processing request to be exported. Defaults to a file matching the template ‘PPR\_*.xml’.
example: pprfile={[}‘PPR\_GRB021004.xml’{]}

\item {} 
\sphinxAtStartPar
\sphinxstylestrong{calintents} \textendash{} List of calibrator image types to be exported. Defaults to all standard calibrator intents ‘BANDPASS’, ‘PHASE’, ‘FLUX’
example: calintents=’PHASE’

\item {} 
\sphinxAtStartPar
\sphinxstylestrong{calimages} \textendash{} List of calibrator images to be exported. Defaults to all calibrator images recorded in the pipeline context.
example: calimages={[}‘3C454.3.bandpass’, ‘3C279.phase’{]}

\item {} 
\sphinxAtStartPar
\sphinxstylestrong{targetimages} \textendash{} List of science target images to be exported. Science target images recorded in the pipeline context.
example: targetimages={[}‘NGC3256.band3’, ‘NGC3256.band6’{]}

\item {} 
\sphinxAtStartPar
\sphinxstylestrong{products\_dir} \textendash{} Name of the data products subdirectory. example: products\_dir=’../products’

\end{itemize}

\sphinxlineitem{Returns}
\sphinxAtStartPar
The results object for the pipeline task is returned.

\end{description}\end{quote}
\subsubsection*{Examples}

\sphinxAtStartPar
1. Export the pipeline results for a single session to the data products
directory

\begin{sphinxVerbatim}[commandchars=\\\{\}]
\PYGZgt{}\PYGZgt{}\PYGZgt{} !mkdir ../products
\PYGZgt{}\PYGZgt{}\PYGZgt{} hif\PYGZus{}exportdata (products\PYGZus{}dir=\PYGZsq{}../products\PYGZsq{})
\end{sphinxVerbatim}

\sphinxAtStartPar
2. Export the pipeline results to the data products directory specify that
only the gain calibrator images be saved.

\begin{sphinxVerbatim}[commandchars=\\\{\}]
\PYGZgt{}\PYGZgt{}\PYGZgt{} !mkdir ../products
\PYGZgt{}\PYGZgt{}\PYGZgt{} hif\PYGZus{}exportdata (products\PYGZus{}dir=\PYGZsq{}../products\PYGZsq{}, calintents=\PYGZsq{}*PHASE*\PYGZsq{})
\end{sphinxVerbatim}

\end{fulllineitems}


\sphinxstepscope


\section{pipeline.h.cli.h\_import\_calstate}
\label{\detokenize{_autosummary/pipeline.h.cli.h_import_calstate:pipeline-h-cli-h-import-calstate}}\label{\detokenize{_autosummary/pipeline.h.cli.h_import_calstate::doc}}\index{h\_import\_calstate() (in module pipeline.h.cli)@\spxentry{h\_import\_calstate()}\spxextra{in module pipeline.h.cli}}

\begin{fulllineitems}
\phantomsection\label{\detokenize{_autosummary/pipeline.h.cli.h_import_calstate:pipeline.h.cli.h_import_calstate}}
\pysigstartsignatures
\pysiglinewithargsret
{\sphinxbfcode{\sphinxupquote{h\_import\_calstate}}}
{\sphinxparam{\DUrole{n}{filename}}}
{}
\pysigstopsignatures
\sphinxAtStartPar
Import a calibration state from disk

\sphinxAtStartPar
h\_import\_calstate clears and then recreates the pipeline calibration state
based on the set of applycal calls given in the named file. The applycal
statements are interpreted in additive fashion; for identically specified
data selection targets, caltables specified in later statements will be added
to the state created by earlier calls.
\begin{quote}\begin{description}
\sphinxlineitem{Parameters}
\sphinxAtStartPar
\sphinxstylestrong{filename} \textendash{} Name of the saved calibration state

\sphinxlineitem{Returns}
\sphinxAtStartPar
The results object for the pipeline task is returned.

\end{description}\end{quote}
\subsubsection*{Examples}
\begin{enumerate}
\sphinxsetlistlabels{\arabic}{enumi}{enumii}{}{.}%
\item {} 
\sphinxAtStartPar
Import a calibration state from disk.

\end{enumerate}

\begin{sphinxVerbatim}[commandchars=\\\{\}]
\PYG{g+gp}{\PYGZgt{}\PYGZgt{}\PYGZgt{} }\PYG{n}{h\PYGZus{}import\PYGZus{}calstate}\PYG{p}{(}\PYG{n}{filename}\PYG{o}{=}\PYG{l+s+s1}{\PYGZsq{}}\PYG{l+s+s1}{aftergaincal.calstate}\PYG{l+s+s1}{\PYGZsq{}}\PYG{p}{)}
\end{sphinxVerbatim}

\end{fulllineitems}


\sphinxstepscope


\section{pipeline.h.cli.h\_importdata}
\label{\detokenize{_autosummary/pipeline.h.cli.h_importdata:pipeline-h-cli-h-importdata}}\label{\detokenize{_autosummary/pipeline.h.cli.h_importdata::doc}}\index{h\_importdata() (in module pipeline.h.cli)@\spxentry{h\_importdata()}\spxextra{in module pipeline.h.cli}}

\begin{fulllineitems}
\phantomsection\label{\detokenize{_autosummary/pipeline.h.cli.h_importdata:pipeline.h.cli.h_importdata}}
\pysigstartsignatures
\pysiglinewithargsret
{\sphinxbfcode{\sphinxupquote{h\_importdata}}}
{\sphinxparam{\DUrole{n}{vis}\DUrole{o}{=}\DUrole{default_value}{None}}\sphinxparamcomma \sphinxparam{\DUrole{n}{session}\DUrole{o}{=}\DUrole{default_value}{None}}\sphinxparamcomma \sphinxparam{\DUrole{n}{asis}\DUrole{o}{=}\DUrole{default_value}{None}}\sphinxparamcomma \sphinxparam{\DUrole{n}{process\_caldevice}\DUrole{o}{=}\DUrole{default_value}{None}}\sphinxparamcomma \sphinxparam{\DUrole{n}{overwrite}\DUrole{o}{=}\DUrole{default_value}{None}}\sphinxparamcomma \sphinxparam{\DUrole{n}{nocopy}\DUrole{o}{=}\DUrole{default_value}{None}}\sphinxparamcomma \sphinxparam{\DUrole{n}{bdfflags}\DUrole{o}{=}\DUrole{default_value}{None}}\sphinxparamcomma \sphinxparam{\DUrole{n}{lazy}\DUrole{o}{=}\DUrole{default_value}{None}}\sphinxparamcomma \sphinxparam{\DUrole{n}{ocorr\_mode}\DUrole{o}{=}\DUrole{default_value}{None}}\sphinxparamcomma \sphinxparam{\DUrole{n}{createmms}\DUrole{o}{=}\DUrole{default_value}{None}}}
{}
\pysigstopsignatures
\sphinxAtStartPar
Imports data into the interferometry pipeline

\sphinxAtStartPar
The h\_importdata task loads the specified visibility data into the pipeline
context unpacking and / or converting it as necessary.

\sphinxAtStartPar
If the \sphinxcode{\sphinxupquote{overwrite}} input parameter is set to False and the task is
asked to convert an input ASDM input to an MS, then when the output
MS already exists in the output directory, the importasdm
conversion step is skipped, and the existing MS will be imported
instead.
\begin{quote}\begin{description}
\sphinxlineitem{Parameters}\begin{itemize}
\item {} 
\sphinxAtStartPar
\sphinxstylestrong{vis} \textendash{} List of visibility data files. These may be ASDMs, tar files of ASDMs, MSs, or tar files of MSs, If ASDM files are specified, they will be
converted to MS format.
example: vis={[}‘X227.ms’, ‘asdms.tar.gz’{]}

\item {} 
\sphinxAtStartPar
\sphinxstylestrong{session} \textendash{} List of sessions to which the visibility files belong. Defaults to a single session containing all the visibility files, otherwise
a session must be assigned to each vis file.
example: session={[}‘session\_1’, ‘session\_2’{]}

\item {} 
\sphinxAtStartPar
\sphinxstylestrong{asis} \textendash{} Creates verbatim copies of the ASDM tables in the output MS. The value given to this option must be a list of table names
separated by space characters.
default: ‘Antenna Station Receiver CalAtmosphere’
example: ‘Receiver’, ‘’

\item {} 
\sphinxAtStartPar
\sphinxstylestrong{process\_caldevice} \textendash{} Ingest the ASDM caldevice table.

\item {} 
\sphinxAtStartPar
\sphinxstylestrong{overwrite} \textendash{} Overwrite existing files on import. When converting ASDM to MS, if overwrite=False and the MS
already exists in output directory, then this existing MS
dataset will be used instead.

\item {} 
\sphinxAtStartPar
\sphinxstylestrong{nocopy} \textendash{} When importing an MS, disable copying of the MS to the working directory.

\item {} 
\sphinxAtStartPar
\sphinxstylestrong{bdfflags} \textendash{} Apply BDF flags on import.

\item {} 
\sphinxAtStartPar
\sphinxstylestrong{lazy} \textendash{} Use the lazy import option.

\item {} 
\sphinxAtStartPar
\sphinxstylestrong{ocorr\_mode} \textendash{} Read in cross\sphinxhyphen{} and auto\sphinxhyphen{}correlation data(ca), cross\sphinxhyphen{} correlation data only (co), or autocorrelation data only (ao).

\item {} 
\sphinxAtStartPar
\sphinxstylestrong{createmms} \textendash{} Create a multi\sphinxhyphen{}MeasurementSet (‘true’) ready for parallel processing, or a standard MeasurementSet (‘false’). The default setting
(‘automatic’) creates an MMS if running in a cluster environment.

\end{itemize}

\sphinxlineitem{Returns}
\sphinxAtStartPar
The results object for the pipeline task is returned.

\end{description}\end{quote}
\subsubsection*{Examples}
\begin{enumerate}
\sphinxsetlistlabels{\arabic}{enumi}{enumii}{}{.}%
\item {} 
\sphinxAtStartPar
Load an ASDM list in the ../rawdata subdirectory into the context”

\end{enumerate}

\begin{sphinxVerbatim}[commandchars=\\\{\}]
\PYG{g+gp}{\PYGZgt{}\PYGZgt{}\PYGZgt{} }\PYG{n}{h\PYGZus{}importdata}\PYG{p}{(}\PYG{n}{vis}\PYG{o}{=}\PYG{p}{[}\PYG{l+s+s1}{\PYGZsq{}}\PYG{l+s+s1}{../rawdata/uid\PYGZus{}\PYGZus{}\PYGZus{}A002\PYGZus{}X30a93d\PYGZus{}X43e}\PYG{l+s+s1}{\PYGZsq{}}\PYG{p}{,} \PYG{l+s+s1}{\PYGZsq{}}\PYG{l+s+s1}{../rawdata/uid\PYGZus{}A002\PYGZus{}x30a93d\PYGZus{}X44e}\PYG{l+s+s1}{\PYGZsq{}}\PYG{p}{]}\PYG{p}{)}
\end{sphinxVerbatim}
\begin{enumerate}
\sphinxsetlistlabels{\arabic}{enumi}{enumii}{}{.}%
\setcounter{enumi}{1}
\item {} 
\sphinxAtStartPar
Load an MS in the current directory into the context:

\end{enumerate}

\begin{sphinxVerbatim}[commandchars=\\\{\}]
\PYG{g+gp}{\PYGZgt{}\PYGZgt{}\PYGZgt{} }\PYG{n}{h\PYGZus{}importdata}\PYG{p}{(}\PYG{n}{vis}\PYG{o}{=}\PYG{p}{[}\PYG{l+s+s1}{\PYGZsq{}}\PYG{l+s+s1}{uid\PYGZus{}\PYGZus{}\PYGZus{}A002\PYGZus{}X30a93d\PYGZus{}X43e.ms}\PYG{l+s+s1}{\PYGZsq{}}\PYG{p}{]}\PYG{p}{)}
\end{sphinxVerbatim}
\begin{enumerate}
\sphinxsetlistlabels{\arabic}{enumi}{enumii}{}{.}%
\setcounter{enumi}{2}
\item {} 
\sphinxAtStartPar
Load a tarred ASDM in ../rawdata into the context:

\end{enumerate}

\begin{sphinxVerbatim}[commandchars=\\\{\}]
\PYG{g+gp}{\PYGZgt{}\PYGZgt{}\PYGZgt{} }\PYG{n}{h\PYGZus{}importdata}\PYG{p}{(}\PYG{n}{vis}\PYG{o}{=}\PYG{p}{[}\PYG{l+s+s1}{\PYGZsq{}}\PYG{l+s+s1}{../rawdata/uid\PYGZus{}\PYGZus{}\PYGZus{}A002\PYGZus{}X30a93d\PYGZus{}X43e.tar.gz}\PYG{l+s+s1}{\PYGZsq{}}\PYG{p}{]}\PYG{p}{)}
\end{sphinxVerbatim}
\begin{enumerate}
\sphinxsetlistlabels{\arabic}{enumi}{enumii}{}{.}%
\setcounter{enumi}{3}
\item {} 
\sphinxAtStartPar
Import a list of MeasurementSets:

\end{enumerate}

\begin{sphinxVerbatim}[commandchars=\\\{\}]
\PYG{g+gp}{\PYGZgt{}\PYGZgt{}\PYGZgt{} }\PYG{n}{myvislist} \PYG{o}{=} \PYG{p}{[}\PYG{l+s+s1}{\PYGZsq{}}\PYG{l+s+s1}{uid\PYGZus{}\PYGZus{}\PYGZus{}A002\PYGZus{}X30a93d\PYGZus{}X43e.ms}\PYG{l+s+s1}{\PYGZsq{}}\PYG{p}{,} \PYG{l+s+s1}{\PYGZsq{}}\PYG{l+s+s1}{uid\PYGZus{}A002\PYGZus{}x30a93d\PYGZus{}X44e.ms}\PYG{l+s+s1}{\PYGZsq{}}\PYG{p}{]}
\PYG{g+gp}{\PYGZgt{}\PYGZgt{}\PYGZgt{} }\PYG{n}{h\PYGZus{}importdata}\PYG{p}{(}\PYG{n}{vis}\PYG{o}{=}\PYG{n}{myvislist}\PYG{p}{)}
\end{sphinxVerbatim}

\end{fulllineitems}


\sphinxstepscope


\section{pipeline.h.cli.h\_init}
\label{\detokenize{_autosummary/pipeline.h.cli.h_init:pipeline-h-cli-h-init}}\label{\detokenize{_autosummary/pipeline.h.cli.h_init::doc}}\index{h\_init() (in module pipeline.h.cli)@\spxentry{h\_init()}\spxextra{in module pipeline.h.cli}}

\begin{fulllineitems}
\phantomsection\label{\detokenize{_autosummary/pipeline.h.cli.h_init:pipeline.h.cli.h_init}}
\pysigstartsignatures
\pysiglinewithargsret
{\sphinxbfcode{\sphinxupquote{h\_init}}}
{\sphinxparam{\DUrole{n}{loglevel}\DUrole{o}{=}\DUrole{default_value}{\textquotesingle{}info\textquotesingle{}}}\sphinxparamcomma \sphinxparam{\DUrole{n}{plotlevel}\DUrole{o}{=}\DUrole{default_value}{\textquotesingle{}default\textquotesingle{}}}\sphinxparamcomma \sphinxparam{\DUrole{n}{weblog}\DUrole{o}{=}\DUrole{default_value}{True}}}
{}
\pysigstopsignatures
\sphinxAtStartPar
Initialize the pipeline

\sphinxAtStartPar
The h\_init task initializes the pipeline.

\sphinxAtStartPar
h\_init  must be called before any other pipeline task. The pipeline
can be initialized in one of two ways: by creating a new pipeline
state (h\_init) or be loading a saved pipeline state (h\_resume).

\sphinxAtStartPar
h\_init creates an empty pipeline context but does not load visibility data
into the context. hif\_importdata or hsd\_importdata can be used to load data.

\sphinxAtStartPar
The pipeline context is returned.
\begin{quote}\begin{description}
\sphinxlineitem{Parameters}\begin{itemize}
\item {} 
\sphinxAtStartPar
\sphinxstylestrong{loglevel} \textendash{} Log level for pipeline messages. Log messages below this threshold will not be displayed.

\item {} 
\sphinxAtStartPar
\sphinxstylestrong{plotlevel} \textendash{} Toggle generation of detail plots in the web log. A level of ‘all’ generates all plots; ‘summary’ omits detail plots; ‘default’ generates all plots
apart from for the hif\_applycal task.

\item {} 
\sphinxAtStartPar
\sphinxstylestrong{weblog} \textendash{} Generate the web log

\end{itemize}

\sphinxlineitem{Returns}
\sphinxAtStartPar
The results object for the pipeline task is returned.

\end{description}\end{quote}
\subsubsection*{Examples}
\begin{enumerate}
\sphinxsetlistlabels{\arabic}{enumi}{enumii}{}{.}%
\item {} 
\sphinxAtStartPar
Create the pipeline context

\end{enumerate}

\begin{sphinxVerbatim}[commandchars=\\\{\}]
\PYG{g+gp}{\PYGZgt{}\PYGZgt{}\PYGZgt{} }\PYG{n}{h\PYGZus{}init}\PYG{p}{(}\PYG{p}{)}
\end{sphinxVerbatim}

\end{fulllineitems}


\sphinxstepscope


\section{pipeline.h.cli.h\_mssplit}
\label{\detokenize{_autosummary/pipeline.h.cli.h_mssplit:pipeline-h-cli-h-mssplit}}\label{\detokenize{_autosummary/pipeline.h.cli.h_mssplit::doc}}\index{h\_mssplit() (in module pipeline.h.cli)@\spxentry{h\_mssplit()}\spxextra{in module pipeline.h.cli}}

\begin{fulllineitems}
\phantomsection\label{\detokenize{_autosummary/pipeline.h.cli.h_mssplit:pipeline.h.cli.h_mssplit}}
\pysigstartsignatures
\pysiglinewithargsret
{\sphinxbfcode{\sphinxupquote{h\_mssplit}}}
{\sphinxparam{\DUrole{n}{vis}\DUrole{o}{=}\DUrole{default_value}{None}}\sphinxparamcomma \sphinxparam{\DUrole{n}{outputvis}\DUrole{o}{=}\DUrole{default_value}{None}}\sphinxparamcomma \sphinxparam{\DUrole{n}{field}\DUrole{o}{=}\DUrole{default_value}{None}}\sphinxparamcomma \sphinxparam{\DUrole{n}{intent}\DUrole{o}{=}\DUrole{default_value}{None}}\sphinxparamcomma \sphinxparam{\DUrole{n}{spw}\DUrole{o}{=}\DUrole{default_value}{None}}\sphinxparamcomma \sphinxparam{\DUrole{n}{datacolumn}\DUrole{o}{=}\DUrole{default_value}{None}}\sphinxparamcomma \sphinxparam{\DUrole{n}{chanbin}\DUrole{o}{=}\DUrole{default_value}{None}}\sphinxparamcomma \sphinxparam{\DUrole{n}{timebin}\DUrole{o}{=}\DUrole{default_value}{None}}\sphinxparamcomma \sphinxparam{\DUrole{n}{replace}\DUrole{o}{=}\DUrole{default_value}{None}}}
{}
\pysigstopsignatures
\sphinxAtStartPar
Select data from calibrated MS(s) to form new MS(s) for imaging

\sphinxAtStartPar
Create new MeasurementSets for imaging from the corrected column of the input
MeasurementSet. By default all science target data is copied to the new MS. The new
MeasurementSet is not re\sphinxhyphen{}indexed to the selected data in the new MS will have the
same source, field, and spw names and ids as it does in the parent MS.

\sphinxAtStartPar
The results object for the pipeline task is returned.
\begin{quote}\begin{description}
\sphinxlineitem{Parameters}\begin{itemize}
\item {} 
\sphinxAtStartPar
\sphinxstylestrong{vis} \textendash{} The list of input MeasurementSets to be transformed. Defaults to the list of MeasurementSets specified in the pipeline import data task.
default ‘’: Split all MeasurementSets in the context.
example: ‘ngc5921.ms’, {[}‘ngc5921a.ms’, ngc5921b.ms’, ‘ngc5921c.ms’{]}

\item {} 
\sphinxAtStartPar
\sphinxstylestrong{outputvis} \textendash{} The list of output split MeasurementSets. The output list must be the same length as the input list and the output names must be different
from the input names.
default ‘’, The output name defaults to \textless{}msrootname\textgreater{}\_split.ms
example: ‘ngc5921.ms’, {[}‘ngc5921a.ms’, ngc5921b.ms’, ‘ngc5921c.ms’{]}

\item {} 
\sphinxAtStartPar
\sphinxstylestrong{field} \textendash{} Set of data selection field names or ids, ‘’ for all

\item {} 
\sphinxAtStartPar
\sphinxstylestrong{intent} \textendash{} Select intents to split default: ‘’, All data is selected.
example: ‘TARGET’

\item {} 
\sphinxAtStartPar
\sphinxstylestrong{spw} \textendash{} Select spectral windows to split. default: ‘’, All spws are selected
example: ‘9’, ‘9,13,15’

\item {} 
\sphinxAtStartPar
\sphinxstylestrong{datacolumn} \textendash{} Select spectral windows to split. The standard CASA options are supported. example: ‘corrected’, ‘model’

\item {} 
\sphinxAtStartPar
\sphinxstylestrong{chanbin} \textendash{} The channel binning factor. 1 for no binning, otherwise 2, 4, 8, or 16. example: 2, 4

\item {} 
\sphinxAtStartPar
\sphinxstylestrong{timebin} \textendash{} The time binning factor. ‘0s’ for no binning example: ’10s’ for 10 second binning

\item {} 
\sphinxAtStartPar
\sphinxstylestrong{replace} \textendash{} If a split was performed delete the parent MS and remove it from the context.

\end{itemize}

\sphinxlineitem{Returns}
\sphinxAtStartPar
The results object for the pipeline task is returned.

\end{description}\end{quote}
\subsubsection*{Examples}
\begin{enumerate}
\sphinxsetlistlabels{\arabic}{enumi}{enumii}{}{.}%
\item {} 
\sphinxAtStartPar
Create a 4X channel smoothed output MS from the input MS

\end{enumerate}

\begin{sphinxVerbatim}[commandchars=\\\{\}]
\PYG{g+gp}{\PYGZgt{}\PYGZgt{}\PYGZgt{} }\PYG{n}{h\PYGZus{}mssplit}\PYG{p}{(}\PYG{n}{chanbin}\PYG{o}{=}\PYG{l+m+mi}{4}\PYG{p}{)}
\end{sphinxVerbatim}

\end{fulllineitems}


\sphinxstepscope


\section{pipeline.h.cli.h\_restoredata}
\label{\detokenize{_autosummary/pipeline.h.cli.h_restoredata:pipeline-h-cli-h-restoredata}}\label{\detokenize{_autosummary/pipeline.h.cli.h_restoredata::doc}}\index{h\_restoredata() (in module pipeline.h.cli)@\spxentry{h\_restoredata()}\spxextra{in module pipeline.h.cli}}

\begin{fulllineitems}
\phantomsection\label{\detokenize{_autosummary/pipeline.h.cli.h_restoredata:pipeline.h.cli.h_restoredata}}
\pysigstartsignatures
\pysiglinewithargsret
{\sphinxbfcode{\sphinxupquote{h\_restoredata}}}
{\sphinxparam{\DUrole{n}{vis}\DUrole{o}{=}\DUrole{default_value}{None}}\sphinxparamcomma \sphinxparam{\DUrole{n}{session}\DUrole{o}{=}\DUrole{default_value}{None}}\sphinxparamcomma \sphinxparam{\DUrole{n}{products\_dir}\DUrole{o}{=}\DUrole{default_value}{None}}\sphinxparamcomma \sphinxparam{\DUrole{n}{copytoraw}\DUrole{o}{=}\DUrole{default_value}{None}}\sphinxparamcomma \sphinxparam{\DUrole{n}{rawdata\_dir}\DUrole{o}{=}\DUrole{default_value}{None}}\sphinxparamcomma \sphinxparam{\DUrole{n}{lazy}\DUrole{o}{=}\DUrole{default_value}{None}}\sphinxparamcomma \sphinxparam{\DUrole{n}{bdfflags}\DUrole{o}{=}\DUrole{default_value}{None}}\sphinxparamcomma \sphinxparam{\DUrole{n}{ocorr\_mode}\DUrole{o}{=}\DUrole{default_value}{None}}\sphinxparamcomma \sphinxparam{\DUrole{n}{asis}\DUrole{o}{=}\DUrole{default_value}{None}}}
{}
\pysigstopsignatures
\sphinxAtStartPar
Restore flags and calibration state from a pipeline run

\sphinxAtStartPar
The h\_restoredata restores flagged and calibrated data from archived ASDMs and
pipeline flagging and calibration data products. Pending archive retrieval
support h\_restoredata assumes that the required products are available in the
rawdata\_dir in the format produced by the h\_exportdata task.

\sphinxAtStartPar
h\_restoredata assumes that the following entities are available in the
rawdata\_dir directory:
\begin{itemize}
\item {} 
\sphinxAtStartPar
the ASDMs to be restored

\item {} 
\sphinxAtStartPar
for each ASDM in the input list:
\begin{itemize}
\item {} 
\sphinxAtStartPar
a compressed tar file of the final flagversions file, e.g.,
uid\_\_\_A002\_X30a93d\_X43e.ms.flagversions.tar.gz

\item {} 
\sphinxAtStartPar
a text file containing the applycal instructions, e.g.,
uid\_\_\_A002\_X30a93d\_X43e.ms.calapply.txt

\item {} 
\sphinxAtStartPar
a compressed tar file containing the caltables for the parent
session, e.g., uid\_\_\_A001\_X74\_X29.session\_3.caltables.tar.gz

\end{itemize}

\end{itemize}

\sphinxAtStartPar
h\_restore data performs the following operations:
\begin{itemize}
\item {} 
\sphinxAtStartPar
imports the ASDM(s)

\item {} 
\sphinxAtStartPar
removes the default MS.flagversions directory created by the filler

\item {} 
\sphinxAtStartPar
restores the final MS.flagversions directory stored by the pipeline

\item {} 
\sphinxAtStartPar
restores the final set of pipeline flags to the MS

\item {} 
\sphinxAtStartPar
restores the final calibration state of the MS

\item {} 
\sphinxAtStartPar
restores the final calibration tables for each MS

\item {} 
\sphinxAtStartPar
applies the calibration tables to each MS

\end{itemize}

\sphinxAtStartPar
When importing the ASDM and converting it to a Measurement Set (MS), if the
output MS already exists in the output directory, then the importasdm
conversion step is skipped, and the existing MS will be imported instead.
\begin{quote}\begin{description}
\sphinxlineitem{Parameters}\begin{itemize}
\item {} 
\sphinxAtStartPar
\sphinxstylestrong{vis} \textendash{} List of raw visibility data files to be restored. Assumed to be in the directory specified by rawdata\_dir.
Example: vis={[}‘uid\_\_\_A002\_X30a93d\_X43e’{]}

\item {} 
\sphinxAtStartPar
\sphinxstylestrong{session} \textendash{} List of sessions, one per visibility file. Example: session={[}‘session\_3’{]}

\item {} 
\sphinxAtStartPar
\sphinxstylestrong{products\_dir} \textendash{} Path to the data products directory, used to copy calibration products from. The parameter is effective only when
copytoraw=True. When copytoraw=False, calibration products in
rawdata\_dir will be used.
Example: products\_dir=’/path/to/my/products’

\item {} 
\sphinxAtStartPar
\sphinxstylestrong{copytoraw} \textendash{} Copy calibration and flagging tables from products\_dir to rawdata\_dir directory.
Example: copytoraw=False

\item {} 
\sphinxAtStartPar
\sphinxstylestrong{rawdata\_dir} \textendash{} Path to the rawdata subdirectory. example: rawdata\_dir=’/path/to/my/rawdata’

\item {} 
\sphinxAtStartPar
\sphinxstylestrong{lazy} \textendash{} Use the lazy filler option. Example: lazy=True

\item {} 
\sphinxAtStartPar
\sphinxstylestrong{bdfflags} \textendash{} Set the BDF flags. Example: bdfflags=False

\item {} 
\sphinxAtStartPar
\sphinxstylestrong{ocorr\_mode} \textendash{} Set correlation import mode. Example: ocorr\_mode=’ca’

\item {} 
\sphinxAtStartPar
\sphinxstylestrong{asis} \textendash{} Creates verbatim copies of the ASDM tables in the output MS. The value given to this option must be a list of table names separated by space characters.
example: ocorr\_mode=’Source Receiver’

\end{itemize}

\sphinxlineitem{Returns}
\sphinxAtStartPar
The results object for the pipeline task is returned.

\end{description}\end{quote}
\subsubsection*{Examples}
\begin{enumerate}
\sphinxsetlistlabels{\arabic}{enumi}{enumii}{}{.}%
\item {} 
\sphinxAtStartPar
Restore the pipeline results for a single ASDM in a single session

\end{enumerate}

\begin{sphinxVerbatim}[commandchars=\\\{\}]
\PYG{g+gp}{\PYGZgt{}\PYGZgt{}\PYGZgt{} }\PYG{n}{h\PYGZus{}restoredata} \PYG{p}{(}\PYG{n}{vis}\PYG{o}{=}\PYG{p}{[}\PYG{l+s+s1}{\PYGZsq{}}\PYG{l+s+s1}{uid\PYGZus{}\PYGZus{}\PYGZus{}A002\PYGZus{}X30a93d\PYGZus{}X43e}\PYG{l+s+s1}{\PYGZsq{}}\PYG{p}{]}\PYG{p}{,} \PYG{n}{session}\PYG{o}{=}\PYG{p}{[}\PYG{l+s+s1}{\PYGZsq{}}\PYG{l+s+s1}{session\PYGZus{}1}\PYG{l+s+s1}{\PYGZsq{}}\PYG{p}{]}\PYG{p}{,} \PYG{n}{ocorr\PYGZus{}mode}\PYG{o}{=}\PYG{l+s+s1}{\PYGZsq{}}\PYG{l+s+s1}{ca}\PYG{l+s+s1}{\PYGZsq{}}\PYG{p}{)}
\end{sphinxVerbatim}

\end{fulllineitems}


\sphinxstepscope


\section{pipeline.h.cli.h\_resume}
\label{\detokenize{_autosummary/pipeline.h.cli.h_resume:pipeline-h-cli-h-resume}}\label{\detokenize{_autosummary/pipeline.h.cli.h_resume::doc}}\index{h\_resume() (in module pipeline.h.cli)@\spxentry{h\_resume()}\spxextra{in module pipeline.h.cli}}

\begin{fulllineitems}
\phantomsection\label{\detokenize{_autosummary/pipeline.h.cli.h_resume:pipeline.h.cli.h_resume}}
\pysigstartsignatures
\pysiglinewithargsret
{\sphinxbfcode{\sphinxupquote{h\_resume}}}
{\sphinxparam{\DUrole{n}{filename}\DUrole{o}{=}\DUrole{default_value}{None}}}
{}
\pysigstopsignatures
\sphinxAtStartPar
Restore a saved pipeline state

\sphinxAtStartPar
h\_resume restores a name pipeline state from disk, allowing a
suspended pipeline reduction session to be resumed.
\begin{quote}\begin{description}
\sphinxlineitem{Parameters}
\sphinxAtStartPar
\sphinxstylestrong{filename} \textendash{} Name of the saved pipeline state. Setting filename to ‘last’ restores the most recently saved pipeline state whose name
begins with ‘context*’.

\sphinxlineitem{Returns}
\sphinxAtStartPar
The results object for the pipeline task is returned.

\end{description}\end{quote}
\subsubsection*{Examples}
\begin{enumerate}
\sphinxsetlistlabels{\arabic}{enumi}{enumii}{}{.}%
\item {} 
\sphinxAtStartPar
Resume the last saved session

\end{enumerate}

\begin{sphinxVerbatim}[commandchars=\\\{\}]
\PYG{g+gp}{\PYGZgt{}\PYGZgt{}\PYGZgt{} }\PYG{n}{h\PYGZus{}resume}\PYG{p}{(}\PYG{p}{)}
\end{sphinxVerbatim}
\begin{enumerate}
\sphinxsetlistlabels{\arabic}{enumi}{enumii}{}{.}%
\setcounter{enumi}{1}
\item {} 
\sphinxAtStartPar
Resume the named saved session

\end{enumerate}

\begin{sphinxVerbatim}[commandchars=\\\{\}]
\PYG{g+gp}{\PYGZgt{}\PYGZgt{}\PYGZgt{} }\PYG{n}{h\PYGZus{}resume}\PYG{p}{(}\PYG{n}{filename}\PYG{o}{=}\PYG{l+s+s1}{\PYGZsq{}}\PYG{l+s+s1}{context.s3.2012\PYGZhy{}02\PYGZhy{}13T10:49:11}\PYG{l+s+s1}{\PYGZsq{}}\PYG{p}{)}
\end{sphinxVerbatim}

\end{fulllineitems}


\sphinxstepscope


\section{pipeline.h.cli.h\_save}
\label{\detokenize{_autosummary/pipeline.h.cli.h_save:pipeline-h-cli-h-save}}\label{\detokenize{_autosummary/pipeline.h.cli.h_save::doc}}\index{h\_save() (in module pipeline.h.cli)@\spxentry{h\_save()}\spxextra{in module pipeline.h.cli}}

\begin{fulllineitems}
\phantomsection\label{\detokenize{_autosummary/pipeline.h.cli.h_save:pipeline.h.cli.h_save}}
\pysigstartsignatures
\pysiglinewithargsret
{\sphinxbfcode{\sphinxupquote{h\_save}}}
{\sphinxparam{\DUrole{n}{filename}\DUrole{o}{=}\DUrole{default_value}{None}}}
{}
\pysigstopsignatures
\sphinxAtStartPar
Save the pipeline state to disk

\sphinxAtStartPar
h\_save saves the current pipeline state to disk using a unique filename.
If no name is supplied one is generated automatically from a combination
of the root name, ‘context’, the current stage number, and a time stamp.
\begin{quote}\begin{description}
\sphinxlineitem{Parameters}
\sphinxAtStartPar
\sphinxstylestrong{filename} \textendash{} Name of the saved pipeline state. If filename is ‘’ then a unique name will be generated computed several components: the
root, ‘context’, the current stage number, and the time stamp.

\sphinxlineitem{Returns}
\sphinxAtStartPar
The results object for the pipeline task is returned.

\end{description}\end{quote}
\subsubsection*{Examples}
\begin{enumerate}
\sphinxsetlistlabels{\arabic}{enumi}{enumii}{}{.}%
\item {} 
\sphinxAtStartPar
Save the current state in the default file

\end{enumerate}

\begin{sphinxVerbatim}[commandchars=\\\{\}]
\PYG{g+gp}{\PYGZgt{}\PYGZgt{}\PYGZgt{} }\PYG{n}{h\PYGZus{}save}\PYG{p}{(}\PYG{p}{)}
\end{sphinxVerbatim}
\begin{enumerate}
\sphinxsetlistlabels{\arabic}{enumi}{enumii}{}{.}%
\setcounter{enumi}{1}
\item {} 
\sphinxAtStartPar
Save the current state to a file called ‘savestate\_1’

\end{enumerate}

\begin{sphinxVerbatim}[commandchars=\\\{\}]
\PYG{g+gp}{\PYGZgt{}\PYGZgt{}\PYGZgt{} }\PYG{n}{h\PYGZus{}save}\PYG{p}{(}\PYG{n}{filename}\PYG{o}{=}\PYG{l+s+s1}{\PYGZsq{}}\PYG{l+s+s1}{savestate\PYGZus{}1}\PYG{l+s+s1}{\PYGZsq{}}\PYG{p}{)}
\end{sphinxVerbatim}

\end{fulllineitems}


\sphinxstepscope


\section{pipeline.h.cli.h\_show\_calstate}
\label{\detokenize{_autosummary/pipeline.h.cli.h_show_calstate:pipeline-h-cli-h-show-calstate}}\label{\detokenize{_autosummary/pipeline.h.cli.h_show_calstate::doc}}\index{h\_show\_calstate() (in module pipeline.h.cli)@\spxentry{h\_show\_calstate()}\spxextra{in module pipeline.h.cli}}

\begin{fulllineitems}
\phantomsection\label{\detokenize{_autosummary/pipeline.h.cli.h_show_calstate:pipeline.h.cli.h_show_calstate}}
\pysigstartsignatures
\pysiglinewithargsret
{\sphinxbfcode{\sphinxupquote{h\_show\_calstate}}}
{}
{}
\pysigstopsignatures
\sphinxAtStartPar
Show the current pipeline calibration state

\sphinxAtStartPar
h\_show\_calstate displays the current on\sphinxhyphen{}the\sphinxhyphen{}fly calibration state
of the pipeline as a set of equivalent applycal calls.
\begin{quote}\begin{description}
\sphinxlineitem{Returns}
\sphinxAtStartPar
The results object for the pipeline task is returned.

\end{description}\end{quote}
\subsubsection*{Examples}
\begin{enumerate}
\sphinxsetlistlabels{\arabic}{enumi}{enumii}{}{.}%
\item {} 
\sphinxAtStartPar
Show the current on\sphinxhyphen{}the\sphinxhyphen{}fly pipeline calibration state.

\end{enumerate}

\begin{sphinxVerbatim}[commandchars=\\\{\}]
\PYG{g+gp}{\PYGZgt{}\PYGZgt{}\PYGZgt{} }\PYG{n}{h\PYGZus{}show\PYGZus{}calstate}\PYG{p}{(}\PYG{p}{)}
\end{sphinxVerbatim}

\end{fulllineitems}


\sphinxstepscope


\section{pipeline.h.cli.h\_tsyscal}
\label{\detokenize{_autosummary/pipeline.h.cli.h_tsyscal:pipeline-h-cli-h-tsyscal}}\label{\detokenize{_autosummary/pipeline.h.cli.h_tsyscal::doc}}\index{h\_tsyscal() (in module pipeline.h.cli)@\spxentry{h\_tsyscal()}\spxextra{in module pipeline.h.cli}}

\begin{fulllineitems}
\phantomsection\label{\detokenize{_autosummary/pipeline.h.cli.h_tsyscal:pipeline.h.cli.h_tsyscal}}
\pysigstartsignatures
\pysiglinewithargsret
{\sphinxbfcode{\sphinxupquote{h\_tsyscal}}}
{\sphinxparam{\DUrole{n}{vis}\DUrole{o}{=}\DUrole{default_value}{None}}\sphinxparamcomma \sphinxparam{\DUrole{n}{caltable}\DUrole{o}{=}\DUrole{default_value}{None}}\sphinxparamcomma \sphinxparam{\DUrole{n}{chantol}\DUrole{o}{=}\DUrole{default_value}{None}}}
{}
\pysigstopsignatures
\sphinxAtStartPar
Derive a Tsys calibration table

\sphinxAtStartPar
Derive the Tsys calibration for list of ALMA MeasurementSets.
\begin{quote}\begin{description}
\sphinxlineitem{Parameters}\begin{itemize}
\item {} 
\sphinxAtStartPar
\sphinxstylestrong{vis} \textendash{} List of input visibility files. example: vis={[}‘ngc5921.ms’{]}

\item {} 
\sphinxAtStartPar
\sphinxstylestrong{caltable} \textendash{} Name of output gain calibration tables. example: caltable=’ngc5921.gcal’

\item {} 
\sphinxAtStartPar
\sphinxstylestrong{chantol} \textendash{} The tolerance in channels for mapping atmospheric calibration windows (TDM) to science windows (FDM or TDM).
example: chantol=5

\end{itemize}

\sphinxlineitem{Returns}
\sphinxAtStartPar
The results object for the pipeline task is returned.

\end{description}\end{quote}
\subsubsection*{Examples}
\begin{enumerate}
\sphinxsetlistlabels{\arabic}{enumi}{enumii}{}{.}%
\item {} 
\sphinxAtStartPar
Standard call

\end{enumerate}

\begin{sphinxVerbatim}[commandchars=\\\{\}]
\PYG{g+gp}{\PYGZgt{}\PYGZgt{}\PYGZgt{} }\PYG{n}{h\PYGZus{}tsyscal}\PYG{p}{(}\PYG{p}{)}
\end{sphinxVerbatim}

\end{fulllineitems}


\sphinxstepscope


\section{pipeline.h.cli.h\_weblog}
\label{\detokenize{_autosummary/pipeline.h.cli.h_weblog:pipeline-h-cli-h-weblog}}\label{\detokenize{_autosummary/pipeline.h.cli.h_weblog::doc}}\index{h\_weblog() (in module pipeline.h.cli)@\spxentry{h\_weblog()}\spxextra{in module pipeline.h.cli}}

\begin{fulllineitems}
\phantomsection\label{\detokenize{_autosummary/pipeline.h.cli.h_weblog:pipeline.h.cli.h_weblog}}
\pysigstartsignatures
\pysiglinewithargsret
{\sphinxbfcode{\sphinxupquote{h\_weblog}}}
{\sphinxparam{\DUrole{n}{relpath}\DUrole{o}{=}\DUrole{default_value}{None}}}
{}
\pysigstopsignatures
\sphinxAtStartPar
Open the pipeline weblog in a browser

\sphinxAtStartPar
h\_weblog opens the weblog in a new browser tab or window.
\begin{quote}\begin{description}
\sphinxlineitem{Parameters}
\sphinxAtStartPar
\sphinxstylestrong{relpath} \textendash{} Relative path to the weblog index file. This file must be located
in a child directory of the CASA working directory. If relpath
is left unspecified, the most recent weblog will be located and
displayed.

\sphinxlineitem{Returns}
\sphinxAtStartPar
The results object for the pipeline task is returned.

\end{description}\end{quote}
\subsubsection*{Examples}
\begin{enumerate}
\sphinxsetlistlabels{\arabic}{enumi}{enumii}{}{.}%
\item {} 
\sphinxAtStartPar
Open pipeline weblog in a browser:

\end{enumerate}

\begin{sphinxVerbatim}[commandchars=\\\{\}]
\PYG{g+gp}{\PYGZgt{}\PYGZgt{}\PYGZgt{} }\PYG{n}{h\PYGZus{}weblog}\PYG{p}{(}\PYG{p}{)}
\end{sphinxVerbatim}

\end{fulllineitems}



\begin{savenotes}\sphinxattablestart
\sphinxthistablewithglobalstyle
\sphinxthistablewithnovlinesstyle
\centering
\begin{tabulary}{\linewidth}[t]{\X{1}{2}\X{1}{2}}
\sphinxtoprule
\sphinxtableatstartofbodyhook
\sphinxAtStartPar
{\hyperref[\detokenize{_autosummary/pipeline.h.cli.cli:module-pipeline.h.cli.cli}]{\sphinxcrossref{\sphinxcode{\sphinxupquote{cli}}}}}
&
\sphinxAtStartPar

\\
\sphinxhline
\sphinxAtStartPar
{\hyperref[\detokenize{_autosummary/pipeline.h.cli.utils:module-pipeline.h.cli.utils}]{\sphinxcrossref{\sphinxcode{\sphinxupquote{utils}}}}}
&
\sphinxAtStartPar

\\
\sphinxbottomrule
\end{tabulary}
\sphinxtableafterendhook\par
\sphinxattableend\end{savenotes}

\sphinxstepscope


\section{pipeline.h.cli.cli}
\label{\detokenize{_autosummary/pipeline.h.cli.cli:module-pipeline.h.cli.cli}}\label{\detokenize{_autosummary/pipeline.h.cli.cli:pipeline-h-cli-cli}}\label{\detokenize{_autosummary/pipeline.h.cli.cli::doc}}\index{module@\spxentry{module}!pipeline.h.cli.cli@\spxentry{pipeline.h.cli.cli}}\index{pipeline.h.cli.cli@\spxentry{pipeline.h.cli.cli}!module@\spxentry{module}}
\sphinxstepscope


\section{pipeline.h.cli.utils}
\label{\detokenize{_autosummary/pipeline.h.cli.utils:module-pipeline.h.cli.utils}}\label{\detokenize{_autosummary/pipeline.h.cli.utils:pipeline-h-cli-utils}}\label{\detokenize{_autosummary/pipeline.h.cli.utils::doc}}\index{module@\spxentry{module}!pipeline.h.cli.utils@\spxentry{pipeline.h.cli.utils}}\index{pipeline.h.cli.utils@\spxentry{pipeline.h.cli.utils}!module@\spxentry{module}}\subsubsection*{Functions}


\begin{savenotes}\sphinxattablestart
\sphinxthistablewithglobalstyle
\sphinxthistablewithnovlinesstyle
\centering
\begin{tabulary}{\linewidth}[t]{\X{1}{2}\X{1}{2}}
\sphinxtoprule
\sphinxtableatstartofbodyhook
\sphinxAtStartPar
{\hyperref[\detokenize{_autosummary/pipeline.h.cli.utils.cli_wrapper:pipeline.h.cli.utils.cli_wrapper}]{\sphinxcrossref{\sphinxcode{\sphinxupquote{cli\_wrapper}}}}}
&
\sphinxAtStartPar
Wrap pipeline task CLI functions to handle the extra \textquotesingle{}pipelinemode\textquotesingle{} argument.
\\
\sphinxhline
\sphinxAtStartPar
{\hyperref[\detokenize{_autosummary/pipeline.h.cli.utils.execute_task:pipeline.h.cli.utils.execute_task}]{\sphinxcrossref{\sphinxcode{\sphinxupquote{execute\_task}}}}}
&
\sphinxAtStartPar

\\
\sphinxhline
\sphinxAtStartPar
{\hyperref[\detokenize{_autosummary/pipeline.h.cli.utils.get_context:pipeline.h.cli.utils.get_context}]{\sphinxcrossref{\sphinxcode{\sphinxupquote{get\_context}}}}}
&
\sphinxAtStartPar

\\
\sphinxhline
\sphinxAtStartPar
{\hyperref[\detokenize{_autosummary/pipeline.h.cli.utils.get_heuristic:pipeline.h.cli.utils.get_heuristic}]{\sphinxcrossref{\sphinxcode{\sphinxupquote{get\_heuristic}}}}}
&
\sphinxAtStartPar

\\
\sphinxhline
\sphinxAtStartPar
{\hyperref[\detokenize{_autosummary/pipeline.h.cli.utils.get_ms:pipeline.h.cli.utils.get_ms}]{\sphinxcrossref{\sphinxcode{\sphinxupquote{get\_ms}}}}}
&
\sphinxAtStartPar

\\
\sphinxhline
\sphinxAtStartPar
{\hyperref[\detokenize{_autosummary/pipeline.h.cli.utils.get_output_dir:pipeline.h.cli.utils.get_output_dir}]{\sphinxcrossref{\sphinxcode{\sphinxupquote{get\_output\_dir}}}}}
&
\sphinxAtStartPar

\\
\sphinxhline
\sphinxAtStartPar
{\hyperref[\detokenize{_autosummary/pipeline.h.cli.utils.wraps:pipeline.h.cli.utils.wraps}]{\sphinxcrossref{\sphinxcode{\sphinxupquote{wraps}}}}}
&
\sphinxAtStartPar
Decorator factory to apply update\_wrapper() to a wrapper function
\\
\sphinxbottomrule
\end{tabulary}
\sphinxtableafterendhook\par
\sphinxattableend\end{savenotes}

\sphinxstepscope


\subsection{pipeline.h.cli.utils.cli\_wrapper}
\label{\detokenize{_autosummary/pipeline.h.cli.utils.cli_wrapper:pipeline-h-cli-utils-cli-wrapper}}\label{\detokenize{_autosummary/pipeline.h.cli.utils.cli_wrapper::doc}}\index{cli\_wrapper() (in module pipeline.h.cli.utils)@\spxentry{cli\_wrapper()}\spxextra{in module pipeline.h.cli.utils}}

\begin{fulllineitems}
\phantomsection\label{\detokenize{_autosummary/pipeline.h.cli.utils.cli_wrapper:pipeline.h.cli.utils.cli_wrapper}}
\pysigstartsignatures
\pysiglinewithargsret
{\sphinxbfcode{\sphinxupquote{cli\_wrapper}}}
{\sphinxparam{\DUrole{n}{func}\DUrole{p}{:}\DUrole{w}{ }\DUrole{n}{Callable}}}
{}
\pysigstopsignatures
\sphinxAtStartPar
Wrap pipeline task CLI functions to handle the extra ‘pipelinemode’ argument.

\sphinxAtStartPar
PIPE\sphinxhyphen{}1884: this decorator function removes the “pipelinemode” argument from pipeline CLI task
calls, which commonly exists in archival casa\_pipescript.py/casa\_pipestorescript.py scripts generated
by old pipeline versions before PIPE\sphinxhyphen{}1686.

\end{fulllineitems}


\sphinxstepscope


\subsection{pipeline.h.cli.utils.execute\_task}
\label{\detokenize{_autosummary/pipeline.h.cli.utils.execute_task:pipeline-h-cli-utils-execute-task}}\label{\detokenize{_autosummary/pipeline.h.cli.utils.execute_task::doc}}\index{execute\_task() (in module pipeline.h.cli.utils)@\spxentry{execute\_task()}\spxextra{in module pipeline.h.cli.utils}}

\begin{fulllineitems}
\phantomsection\label{\detokenize{_autosummary/pipeline.h.cli.utils.execute_task:pipeline.h.cli.utils.execute_task}}
\pysigstartsignatures
\pysiglinewithargsret
{\sphinxbfcode{\sphinxupquote{execute\_task}}}
{\sphinxparam{\DUrole{n}{context}}\sphinxparamcomma \sphinxparam{\DUrole{n}{casa\_task}}\sphinxparamcomma \sphinxparam{\DUrole{n}{casa\_args}}}
{}
\pysigstopsignatures
\end{fulllineitems}


\sphinxstepscope


\subsection{pipeline.h.cli.utils.get\_context}
\label{\detokenize{_autosummary/pipeline.h.cli.utils.get_context:pipeline-h-cli-utils-get-context}}\label{\detokenize{_autosummary/pipeline.h.cli.utils.get_context::doc}}\index{get\_context() (in module pipeline.h.cli.utils)@\spxentry{get\_context()}\spxextra{in module pipeline.h.cli.utils}}

\begin{fulllineitems}
\phantomsection\label{\detokenize{_autosummary/pipeline.h.cli.utils.get_context:pipeline.h.cli.utils.get_context}}
\pysigstartsignatures
\pysiglinewithargsret
{\sphinxbfcode{\sphinxupquote{get\_context}}}
{}
{}
\pysigstopsignatures
\end{fulllineitems}


\sphinxstepscope


\subsection{pipeline.h.cli.utils.get\_heuristic}
\label{\detokenize{_autosummary/pipeline.h.cli.utils.get_heuristic:pipeline-h-cli-utils-get-heuristic}}\label{\detokenize{_autosummary/pipeline.h.cli.utils.get_heuristic::doc}}\index{get\_heuristic() (in module pipeline.h.cli.utils)@\spxentry{get\_heuristic()}\spxextra{in module pipeline.h.cli.utils}}

\begin{fulllineitems}
\phantomsection\label{\detokenize{_autosummary/pipeline.h.cli.utils.get_heuristic:pipeline.h.cli.utils.get_heuristic}}
\pysigstartsignatures
\pysiglinewithargsret
{\sphinxbfcode{\sphinxupquote{get\_heuristic}}}
{\sphinxparam{\DUrole{n}{arg}}}
{}
\pysigstopsignatures
\end{fulllineitems}


\sphinxstepscope


\subsection{pipeline.h.cli.utils.get\_ms}
\label{\detokenize{_autosummary/pipeline.h.cli.utils.get_ms:pipeline-h-cli-utils-get-ms}}\label{\detokenize{_autosummary/pipeline.h.cli.utils.get_ms::doc}}\index{get\_ms() (in module pipeline.h.cli.utils)@\spxentry{get\_ms()}\spxextra{in module pipeline.h.cli.utils}}

\begin{fulllineitems}
\phantomsection\label{\detokenize{_autosummary/pipeline.h.cli.utils.get_ms:pipeline.h.cli.utils.get_ms}}
\pysigstartsignatures
\pysiglinewithargsret
{\sphinxbfcode{\sphinxupquote{get\_ms}}}
{\sphinxparam{\DUrole{n}{vis}}}
{}
\pysigstopsignatures
\end{fulllineitems}


\sphinxstepscope


\subsection{pipeline.h.cli.utils.get\_output\_dir}
\label{\detokenize{_autosummary/pipeline.h.cli.utils.get_output_dir:pipeline-h-cli-utils-get-output-dir}}\label{\detokenize{_autosummary/pipeline.h.cli.utils.get_output_dir::doc}}\index{get\_output\_dir() (in module pipeline.h.cli.utils)@\spxentry{get\_output\_dir()}\spxextra{in module pipeline.h.cli.utils}}

\begin{fulllineitems}
\phantomsection\label{\detokenize{_autosummary/pipeline.h.cli.utils.get_output_dir:pipeline.h.cli.utils.get_output_dir}}
\pysigstartsignatures
\pysiglinewithargsret
{\sphinxbfcode{\sphinxupquote{get\_output\_dir}}}
{}
{}
\pysigstopsignatures
\end{fulllineitems}


\sphinxstepscope


\subsection{pipeline.h.cli.utils.wraps}
\label{\detokenize{_autosummary/pipeline.h.cli.utils.wraps:pipeline-h-cli-utils-wraps}}\label{\detokenize{_autosummary/pipeline.h.cli.utils.wraps::doc}}\index{wraps() (in module pipeline.h.cli.utils)@\spxentry{wraps()}\spxextra{in module pipeline.h.cli.utils}}

\begin{fulllineitems}
\phantomsection\label{\detokenize{_autosummary/pipeline.h.cli.utils.wraps:pipeline.h.cli.utils.wraps}}
\pysigstartsignatures
\pysiglinewithargsret
{\sphinxbfcode{\sphinxupquote{wraps}}}
{\sphinxparam{\DUrole{n}{wrapped}}\sphinxparamcomma \sphinxparam{\DUrole{n}{assigned}\DUrole{o}{=}\DUrole{default_value}{(\textquotesingle{}\_\_module\_\_\textquotesingle{}, \textquotesingle{}\_\_name\_\_\textquotesingle{}, \textquotesingle{}\_\_qualname\_\_\textquotesingle{}, \textquotesingle{}\_\_doc\_\_\textquotesingle{}, \textquotesingle{}\_\_annotations\_\_\textquotesingle{})}}\sphinxparamcomma \sphinxparam{\DUrole{n}{updated}\DUrole{o}{=}\DUrole{default_value}{(\textquotesingle{}\_\_dict\_\_\textquotesingle{},)}}}
{}
\pysigstopsignatures
\sphinxAtStartPar
Decorator factory to apply update\_wrapper() to a wrapper function

\sphinxAtStartPar
Returns a decorator that invokes update\_wrapper() with the decorated
function as the wrapper argument and the arguments to wraps() as the
remaining arguments. Default arguments are as for update\_wrapper().
This is a convenience function to simplify applying partial() to
update\_wrapper().

\end{fulllineitems}


\sphinxstepscope


\chapter{pipeline.hif.cli}
\label{\detokenize{_autosummary/pipeline.hif.cli:module-pipeline.hif.cli}}\label{\detokenize{_autosummary/pipeline.hif.cli:pipeline-hif-cli}}\label{\detokenize{_autosummary/pipeline.hif.cli::doc}}\index{module@\spxentry{module}!pipeline.hif.cli@\spxentry{pipeline.hif.cli}}\index{pipeline.hif.cli@\spxentry{pipeline.hif.cli}!module@\spxentry{module}}
\sphinxAtStartPar
Interferometry Generic Tasks
\subsubsection*{Functions}


\begin{savenotes}\sphinxattablestart
\sphinxthistablewithglobalstyle
\sphinxthistablewithnovlinesstyle
\centering
\begin{tabulary}{\linewidth}[t]{\X{1}{2}\X{1}{2}}
\sphinxtoprule
\sphinxtableatstartofbodyhook
\sphinxAtStartPar
{\hyperref[\detokenize{_autosummary/pipeline.hif.cli.hif_analyzealpha:pipeline.hif.cli.hif_analyzealpha}]{\sphinxcrossref{\sphinxcode{\sphinxupquote{hif\_analyzealpha}}}}}
&
\sphinxAtStartPar
Extract spectral index from intensity peak in VLA/VLASS images
\\
\sphinxhline
\sphinxAtStartPar
{\hyperref[\detokenize{_autosummary/pipeline.hif.cli.hif_antpos:pipeline.hif.cli.hif_antpos}]{\sphinxcrossref{\sphinxcode{\sphinxupquote{hif\_antpos}}}}}
&
\sphinxAtStartPar
Derive an antenna position calibration table
\\
\sphinxhline
\sphinxAtStartPar
{\hyperref[\detokenize{_autosummary/pipeline.hif.cli.hif_applycal:pipeline.hif.cli.hif_applycal}]{\sphinxcrossref{\sphinxcode{\sphinxupquote{hif\_applycal}}}}}
&
\sphinxAtStartPar
Apply the calibration(s) to the data
\\
\sphinxhline
\sphinxAtStartPar
{\hyperref[\detokenize{_autosummary/pipeline.hif.cli.hif_bandpass:pipeline.hif.cli.hif_bandpass}]{\sphinxcrossref{\sphinxcode{\sphinxupquote{hif\_bandpass}}}}}
&
\sphinxAtStartPar
Compute bandpass calibration solutions
\\
\sphinxhline
\sphinxAtStartPar
{\hyperref[\detokenize{_autosummary/pipeline.hif.cli.hif_checkproductsize:pipeline.hif.cli.hif_checkproductsize}]{\sphinxcrossref{\sphinxcode{\sphinxupquote{hif\_checkproductsize}}}}}
&
\sphinxAtStartPar
Check imaging product size
\\
\sphinxhline
\sphinxAtStartPar
{\hyperref[\detokenize{_autosummary/pipeline.hif.cli.hif_correctedampflag:pipeline.hif.cli.hif_correctedampflag}]{\sphinxcrossref{\sphinxcode{\sphinxupquote{hif\_correctedampflag}}}}}
&
\sphinxAtStartPar
Flag corrected \sphinxhyphen{} model amplitudes based on calibrators.
\\
\sphinxhline
\sphinxAtStartPar
{\hyperref[\detokenize{_autosummary/pipeline.hif.cli.hif_editimlist:pipeline.hif.cli.hif_editimlist}]{\sphinxcrossref{\sphinxcode{\sphinxupquote{hif\_editimlist}}}}}
&
\sphinxAtStartPar
Add to a list of images to be produced with hif\_makeimages()
\\
\sphinxhline
\sphinxAtStartPar
{\hyperref[\detokenize{_autosummary/pipeline.hif.cli.hif_findcont:pipeline.hif.cli.hif_findcont}]{\sphinxcrossref{\sphinxcode{\sphinxupquote{hif\_findcont}}}}}
&
\sphinxAtStartPar
Find continuum frequency ranges
\\
\sphinxhline
\sphinxAtStartPar
{\hyperref[\detokenize{_autosummary/pipeline.hif.cli.hif_gaincal:pipeline.hif.cli.hif_gaincal}]{\sphinxcrossref{\sphinxcode{\sphinxupquote{hif\_gaincal}}}}}
&
\sphinxAtStartPar
Determine temporal gains from calibrator observations
\\
\sphinxhline
\sphinxAtStartPar
{\hyperref[\detokenize{_autosummary/pipeline.hif.cli.hif_lowgainflag:pipeline.hif.cli.hif_lowgainflag}]{\sphinxcrossref{\sphinxcode{\sphinxupquote{hif\_lowgainflag}}}}}
&
\sphinxAtStartPar
Flag antennas with low or high gain
\\
\sphinxhline
\sphinxAtStartPar
{\hyperref[\detokenize{_autosummary/pipeline.hif.cli.hif_makecutoutimages:pipeline.hif.cli.hif_makecutoutimages}]{\sphinxcrossref{\sphinxcode{\sphinxupquote{hif\_makecutoutimages}}}}}
&
\sphinxAtStartPar
Cutout central 1 sq.
\\
\sphinxhline
\sphinxAtStartPar
{\hyperref[\detokenize{_autosummary/pipeline.hif.cli.hif_makeimages:pipeline.hif.cli.hif_makeimages}]{\sphinxcrossref{\sphinxcode{\sphinxupquote{hif\_makeimages}}}}}
&
\sphinxAtStartPar
Compute clean map
\\
\sphinxhline
\sphinxAtStartPar
{\hyperref[\detokenize{_autosummary/pipeline.hif.cli.hif_makeimlist:pipeline.hif.cli.hif_makeimlist}]{\sphinxcrossref{\sphinxcode{\sphinxupquote{hif\_makeimlist}}}}}
&
\sphinxAtStartPar
Compute list of clean images to be produced
\\
\sphinxhline
\sphinxAtStartPar
{\hyperref[\detokenize{_autosummary/pipeline.hif.cli.hif_makermsimages:pipeline.hif.cli.hif_makermsimages}]{\sphinxcrossref{\sphinxcode{\sphinxupquote{hif\_makermsimages}}}}}
&
\sphinxAtStartPar
Create RMS images for VLASS data.
\\
\sphinxhline
\sphinxAtStartPar
{\hyperref[\detokenize{_autosummary/pipeline.hif.cli.hif_mstransform:pipeline.hif.cli.hif_mstransform}]{\sphinxcrossref{\sphinxcode{\sphinxupquote{hif\_mstransform}}}}}
&
\sphinxAtStartPar
Create new MeasurementSets for science target imaging
\\
\sphinxhline
\sphinxAtStartPar
{\hyperref[\detokenize{_autosummary/pipeline.hif.cli.hif_rawflagchans:pipeline.hif.cli.hif_rawflagchans}]{\sphinxcrossref{\sphinxcode{\sphinxupquote{hif\_rawflagchans}}}}}
&
\sphinxAtStartPar
Flag deviant baseline/channels in raw data
\\
\sphinxhline
\sphinxAtStartPar
{\hyperref[\detokenize{_autosummary/pipeline.hif.cli.hif_refant:pipeline.hif.cli.hif_refant}]{\sphinxcrossref{\sphinxcode{\sphinxupquote{hif\_refant}}}}}
&
\sphinxAtStartPar
Select the best reference antennas
\\
\sphinxhline
\sphinxAtStartPar
{\hyperref[\detokenize{_autosummary/pipeline.hif.cli.hif_selfcal:pipeline.hif.cli.hif_selfcal}]{\sphinxcrossref{\sphinxcode{\sphinxupquote{hif\_selfcal}}}}}
&
\sphinxAtStartPar
Determine and apply self\sphinxhyphen{}calibration with the science target data
\\
\sphinxhline
\sphinxAtStartPar
{\hyperref[\detokenize{_autosummary/pipeline.hif.cli.hif_setjy:pipeline.hif.cli.hif_setjy}]{\sphinxcrossref{\sphinxcode{\sphinxupquote{hif\_setjy}}}}}
&
\sphinxAtStartPar
Fill the model column with calibrated visibilities
\\
\sphinxhline
\sphinxAtStartPar
{\hyperref[\detokenize{_autosummary/pipeline.hif.cli.hif_setmodels:pipeline.hif.cli.hif_setmodels}]{\sphinxcrossref{\sphinxcode{\sphinxupquote{hif\_setmodels}}}}}
&
\sphinxAtStartPar
Set calibrator source models
\\
\sphinxhline
\sphinxAtStartPar
{\hyperref[\detokenize{_autosummary/pipeline.hif.cli.hif_transformimagedata:pipeline.hif.cli.hif_transformimagedata}]{\sphinxcrossref{\sphinxcode{\sphinxupquote{hif\_transformimagedata}}}}}
&
\sphinxAtStartPar
Extract fields for the desired VLASS image to a new MS and reset weights if desired
\\
\sphinxhline
\sphinxAtStartPar
{\hyperref[\detokenize{_autosummary/pipeline.hif.cli.hif_uvcontsub:pipeline.hif.cli.hif_uvcontsub}]{\sphinxcrossref{\sphinxcode{\sphinxupquote{hif\_uvcontsub}}}}}
&
\sphinxAtStartPar
Fit and subtract continuum from the data
\\
\sphinxbottomrule
\end{tabulary}
\sphinxtableafterendhook\par
\sphinxattableend\end{savenotes}

\sphinxstepscope


\section{pipeline.hif.cli.hif\_analyzealpha}
\label{\detokenize{_autosummary/pipeline.hif.cli.hif_analyzealpha:pipeline-hif-cli-hif-analyzealpha}}\label{\detokenize{_autosummary/pipeline.hif.cli.hif_analyzealpha::doc}}\index{hif\_analyzealpha() (in module pipeline.hif.cli)@\spxentry{hif\_analyzealpha()}\spxextra{in module pipeline.hif.cli}}

\begin{fulllineitems}
\phantomsection\label{\detokenize{_autosummary/pipeline.hif.cli.hif_analyzealpha:pipeline.hif.cli.hif_analyzealpha}}
\pysigstartsignatures
\pysiglinewithargsret
{\sphinxbfcode{\sphinxupquote{hif\_analyzealpha}}}
{\sphinxparam{\DUrole{n}{vis}\DUrole{o}{=}\DUrole{default_value}{None}}\sphinxparamcomma \sphinxparam{\DUrole{n}{image}\DUrole{o}{=}\DUrole{default_value}{None}}\sphinxparamcomma \sphinxparam{\DUrole{n}{alphafile}\DUrole{o}{=}\DUrole{default_value}{None}}\sphinxparamcomma \sphinxparam{\DUrole{n}{alphaerrorfile}\DUrole{o}{=}\DUrole{default_value}{None}}}
{}
\pysigstopsignatures
\sphinxAtStartPar
Extract spectral index from intensity peak in VLA/VLASS images

\sphinxAtStartPar
The results object for the pipeline task is returned.
\begin{quote}\begin{description}
\sphinxlineitem{Parameters}\begin{itemize}
\item {} 
\sphinxAtStartPar
\sphinxstylestrong{vis} \textendash{} List of visisbility  data files. These may be ASDMs, tar files of ASDMs, MSs, or tar files of MSs, If ASDM files are specified, they will be
converted  to MS format.
example: vis={[}‘X227.ms’, ‘asdms.tar.gz’{]}

\item {} 
\sphinxAtStartPar
\sphinxstylestrong{image} \textendash{} Restored subimage

\item {} 
\sphinxAtStartPar
\sphinxstylestrong{alphafile} \textendash{} Input spectral index map

\item {} 
\sphinxAtStartPar
\sphinxstylestrong{alphaerrorfile} \textendash{} Input spectral index error map

\end{itemize}

\sphinxlineitem{Returns}
\sphinxAtStartPar
The results object for the pipeline task is returned.

\end{description}\end{quote}
\subsubsection*{Examples}
\begin{enumerate}
\sphinxsetlistlabels{\arabic}{enumi}{enumii}{}{.}%
\item {} 
\sphinxAtStartPar
Basic analyzealpha task

\end{enumerate}

\begin{sphinxVerbatim}[commandchars=\\\{\}]
\PYG{g+gp}{\PYGZgt{}\PYGZgt{}\PYGZgt{} }\PYG{n}{hif\PYGZus{}analyzealpha}\PYG{p}{(}\PYG{p}{)}
\end{sphinxVerbatim}

\end{fulllineitems}


\sphinxstepscope


\section{pipeline.hif.cli.hif\_antpos}
\label{\detokenize{_autosummary/pipeline.hif.cli.hif_antpos:pipeline-hif-cli-hif-antpos}}\label{\detokenize{_autosummary/pipeline.hif.cli.hif_antpos::doc}}\index{hif\_antpos() (in module pipeline.hif.cli)@\spxentry{hif\_antpos()}\spxextra{in module pipeline.hif.cli}}

\begin{fulllineitems}
\phantomsection\label{\detokenize{_autosummary/pipeline.hif.cli.hif_antpos:pipeline.hif.cli.hif_antpos}}
\pysigstartsignatures
\pysiglinewithargsret
{\sphinxbfcode{\sphinxupquote{hif\_antpos}}}
{\sphinxparam{\DUrole{n}{vis}\DUrole{o}{=}\DUrole{default_value}{None}}\sphinxparamcomma \sphinxparam{\DUrole{n}{caltable}\DUrole{o}{=}\DUrole{default_value}{None}}\sphinxparamcomma \sphinxparam{\DUrole{n}{hm\_antpos}\DUrole{o}{=}\DUrole{default_value}{None}}\sphinxparamcomma \sphinxparam{\DUrole{n}{antenna}\DUrole{o}{=}\DUrole{default_value}{None}}\sphinxparamcomma \sphinxparam{\DUrole{n}{offsets}\DUrole{o}{=}\DUrole{default_value}{None}}\sphinxparamcomma \sphinxparam{\DUrole{n}{antposfile}\DUrole{o}{=}\DUrole{default_value}{None}}}
{}
\pysigstopsignatures
\sphinxAtStartPar
Derive an antenna position calibration table

\sphinxAtStartPar
Derive the antenna position calibration for list of MeasurementSets.

\sphinxAtStartPar
The hif\_antpos task corrects the antenna positions recorded in the ASDMs using
updated antenna position calibration information determined after the
observation was taken.

\sphinxAtStartPar
Corrections can be input by hand, read from a file on disk, or in the future
by querying an ALMA database service.

\sphinxAtStartPar
The antenna positions file is in ‘csv’ format containing 6 comma\sphinxhyphen{}delimited
columns as shown below. The default name of this file is ‘antennapos.csv’.

\sphinxAtStartPar
Contents of example ‘antennapos.csv’ file:

\sphinxAtStartPar
ms,antenna,xoffset,yoffset,zoffset,comment
uid\_\_\_A002\_X30a93d\_X43e.ms,DV11,0.000,0.010,0.000,”No comment”
uid\_\_\_A002\_X30a93d\_X43e.dup.ms,DV11,0.000,\sphinxhyphen{}0.010,0.000,”No comment”

\sphinxAtStartPar
The corrections are used to generate a calibration table which is recorded
in the pipeline context and applied to the raw visibility data, on the fly to
generate other calibration tables, or permanently to generate calibrated
visibilities for imaging.
\begin{quote}\begin{description}
\sphinxlineitem{Parameters}\begin{itemize}
\item {} 
\sphinxAtStartPar
\sphinxstylestrong{vis} \textendash{} List of input visibility files. example: {[}‘ngc5921.ms’{]}

\item {} 
\sphinxAtStartPar
\sphinxstylestrong{caltable} \textendash{} Name of output gain calibration tables. example: {[}‘ngc5921.gcal’{]}

\item {} 
\sphinxAtStartPar
\sphinxstylestrong{hm\_antpos} \textendash{} Heuristics method for retrieving the antenna position corrections. The options are ‘online’ (not yet implemented), ‘manual’,
and ‘file’.

\item {} 
\sphinxAtStartPar
\sphinxstylestrong{antenna} \textendash{} The list of antennas for which the positions are to be corrected. Available when hm\_antpos=’manual’.
example: antenna=’DV05,DV07’

\item {} 
\sphinxAtStartPar
\sphinxstylestrong{offsets} \textendash{} The list of antenna offsets for each antenna in ‘antennas’. Each offset is a set of 3 floating point numbers separated by commas, specified
in the ITRF frame. Available when hm\_antpos=’manual’.
example: offsets={[}0.01, 0.02, 0.03, 0.03, 0.02, 0.01{]}

\item {} 
\sphinxAtStartPar
\sphinxstylestrong{antposfile} \textendash{} The file(s) containing the antenna offsets. Used if hm\_antpos is ‘file’.
example: ‘antennapos.csv’

\end{itemize}

\sphinxlineitem{Returns}
\sphinxAtStartPar
The results object for the pipeline task is returned.

\end{description}\end{quote}
\subsubsection*{Examples}

\sphinxAtStartPar
1. Correct the position of antenna 5 for all the visibility files in a single
pipeline run:

\begin{sphinxVerbatim}[commandchars=\\\{\}]
\PYG{g+gp}{\PYGZgt{}\PYGZgt{}\PYGZgt{} }\PYG{n}{hif\PYGZus{}antpos}\PYG{p}{(}\PYG{n}{antenna}\PYG{o}{=}\PYG{l+s+s1}{\PYGZsq{}}\PYG{l+s+s1}{DV05}\PYG{l+s+s1}{\PYGZsq{}}\PYG{p}{,} \PYG{n}{offsets}\PYG{o}{=}\PYG{p}{[}\PYG{l+m+mf}{0.01}\PYG{p}{,} \PYG{l+m+mf}{0.02}\PYG{p}{,} \PYG{l+m+mf}{0.03}\PYG{p}{]}\PYG{p}{)}
\end{sphinxVerbatim}

\sphinxAtStartPar
2. Correct the position of antennas for all the visibility files in a single
pipeline run using antenna positions files on disk. These files are assumed
to conform to a default naming scheme if \sphinxcode{\sphinxupquote{antposfile}} is unspecified by the
user:

\begin{sphinxVerbatim}[commandchars=\\\{\}]
\PYG{g+gp}{\PYGZgt{}\PYGZgt{}\PYGZgt{} }\PYG{n}{hif\PYGZus{}antpos}\PYG{p}{(}\PYG{n}{hm\PYGZus{}antpos}\PYG{o}{=}\PYG{l+s+s1}{\PYGZsq{}}\PYG{l+s+s1}{file}\PYG{l+s+s1}{\PYGZsq{}}\PYG{p}{,} \PYG{n}{antposfile}\PYG{o}{=}\PYG{l+s+s1}{\PYGZsq{}}\PYG{l+s+s1}{myantposfile.csv}\PYG{l+s+s1}{\PYGZsq{}}\PYG{p}{)}
\end{sphinxVerbatim}

\end{fulllineitems}


\sphinxstepscope


\section{pipeline.hif.cli.hif\_applycal}
\label{\detokenize{_autosummary/pipeline.hif.cli.hif_applycal:pipeline-hif-cli-hif-applycal}}\label{\detokenize{_autosummary/pipeline.hif.cli.hif_applycal::doc}}\index{hif\_applycal() (in module pipeline.hif.cli)@\spxentry{hif\_applycal()}\spxextra{in module pipeline.hif.cli}}

\begin{fulllineitems}
\phantomsection\label{\detokenize{_autosummary/pipeline.hif.cli.hif_applycal:pipeline.hif.cli.hif_applycal}}
\pysigstartsignatures
\pysiglinewithargsret
{\sphinxbfcode{\sphinxupquote{hif\_applycal}}}
{\sphinxparam{\DUrole{n}{vis}\DUrole{o}{=}\DUrole{default_value}{None}}\sphinxparamcomma \sphinxparam{\DUrole{n}{field}\DUrole{o}{=}\DUrole{default_value}{None}}\sphinxparamcomma \sphinxparam{\DUrole{n}{intent}\DUrole{o}{=}\DUrole{default_value}{None}}\sphinxparamcomma \sphinxparam{\DUrole{n}{spw}\DUrole{o}{=}\DUrole{default_value}{None}}\sphinxparamcomma \sphinxparam{\DUrole{n}{antenna}\DUrole{o}{=}\DUrole{default_value}{None}}\sphinxparamcomma \sphinxparam{\DUrole{n}{parang}\DUrole{o}{=}\DUrole{default_value}{None}}\sphinxparamcomma \sphinxparam{\DUrole{n}{applymode}\DUrole{o}{=}\DUrole{default_value}{None}}\sphinxparamcomma \sphinxparam{\DUrole{n}{calwt}\DUrole{o}{=}\DUrole{default_value}{None}}\sphinxparamcomma \sphinxparam{\DUrole{n}{flagbackup}\DUrole{o}{=}\DUrole{default_value}{None}}\sphinxparamcomma \sphinxparam{\DUrole{n}{flagsum}\DUrole{o}{=}\DUrole{default_value}{None}}\sphinxparamcomma \sphinxparam{\DUrole{n}{flagdetailedsum}\DUrole{o}{=}\DUrole{default_value}{None}}\sphinxparamcomma \sphinxparam{\DUrole{n}{parallel}\DUrole{o}{=}\DUrole{default_value}{None}}}
{}
\pysigstopsignatures
\sphinxAtStartPar
Apply the calibration(s) to the data

\sphinxAtStartPar
Apply precomputed calibrations to the data.

\sphinxAtStartPar
hif\_applycal applies the precomputed calibration tables stored in the pipeline
context to the set of visibility files using predetermined field and
spectral window maps and default values for the interpolation schemes.

\sphinxAtStartPar
Users can interact with the pipeline calibration state using the tasks
h\_export\_calstate and h\_import\_calstate.
\begin{quote}\begin{description}
\sphinxlineitem{Parameters}\begin{itemize}
\item {} 
\sphinxAtStartPar
\sphinxstylestrong{vis} \textendash{} The list of input MeasurementSets. Defaults to the list of MeasurementSets in the pipeline context.
example: {[}‘X227.ms’{]}

\item {} 
\sphinxAtStartPar
\sphinxstylestrong{field} \textendash{} A string containing the list of field names or field ids to which the calibration will be applied. Defaults to all fields in the pipeline
context.
example: ‘3C279’, ‘3C279, M82’

\item {} 
\sphinxAtStartPar
\sphinxstylestrong{intent} \textendash{} A string containing the list of intents against which the selected fields will be matched. Defaults to all supported intents
in the pipeline context.
example: \sphinxtitleref{‘*TARGET*’}

\item {} 
\sphinxAtStartPar
\sphinxstylestrong{spw} \textendash{} The list of spectral windows and channels to which the calibration will be applied. Defaults to all science windows in the pipeline
context.
example: ‘17’, ‘11, 15’

\item {} 
\sphinxAtStartPar
\sphinxstylestrong{antenna} \textendash{} The selection of antennas to which the calibration will be applied. Defaults to all antennas. Not currently supported.

\item {} 
\sphinxAtStartPar
\sphinxstylestrong{parang} \textendash{} Apply parallactic angle correction

\item {} 
\sphinxAtStartPar
\sphinxstylestrong{applymode} \textendash{} Calibration apply mode ‘calflag’: calibrate data and apply flags from solutions
‘calflagstrict’: (default) same as above except flag spws for which calibration is
unavailable in one or more tables (instead of allowing them to pass
uncalibrated and unflagged)
‘trial’: report on flags from solutions, dataset entirely unchanged
‘flagonly’: apply flags from solutions only, data not calibrated
‘flagonlystrict’: same as above except flag spws for which calibration is
unavailable in one or more tables
‘calonly’: calibrate data only, flags from solutions NOT applied

\item {} 
\sphinxAtStartPar
\sphinxstylestrong{calwt} \textendash{} Calibrate the weights as well as the data

\item {} 
\sphinxAtStartPar
\sphinxstylestrong{flagbackup} \textendash{} Backup the flags before the apply

\item {} 
\sphinxAtStartPar
\sphinxstylestrong{flagsum} \textendash{} Compute before and after flagging summary statistics

\item {} 
\sphinxAtStartPar
\sphinxstylestrong{flagdetailedsum} \textendash{} Compute detailed before and after flagging statistics summaries. Parameter available only when if flagsum is True.

\item {} 
\sphinxAtStartPar
\sphinxstylestrong{parallel} \textendash{} Execute using CASA HPC functionality, if available. options: ‘automatic’, ‘true’, ‘false’, True, False
default: None (equivalent to False)

\end{itemize}

\sphinxlineitem{Returns}
\sphinxAtStartPar
The results object for the pipeline task is returned.

\end{description}\end{quote}
\subsubsection*{Examples}
\begin{enumerate}
\sphinxsetlistlabels{\arabic}{enumi}{enumii}{}{.}%
\item {} 
\sphinxAtStartPar
Apply the calibration to the target data

\end{enumerate}

\begin{sphinxVerbatim}[commandchars=\\\{\}]
\PYG{g+gp}{\PYGZgt{}\PYGZgt{}\PYGZgt{} }\PYG{n}{hif\PYGZus{}applycal} \PYG{p}{(}\PYG{n}{intent}\PYG{o}{=}\PYG{l+s+s1}{\PYGZsq{}}\PYG{l+s+s1}{TARGET}\PYG{l+s+s1}{\PYGZsq{}}\PYG{p}{)}
\end{sphinxVerbatim}

\end{fulllineitems}


\sphinxstepscope


\section{pipeline.hif.cli.hif\_bandpass}
\label{\detokenize{_autosummary/pipeline.hif.cli.hif_bandpass:pipeline-hif-cli-hif-bandpass}}\label{\detokenize{_autosummary/pipeline.hif.cli.hif_bandpass::doc}}\index{hif\_bandpass() (in module pipeline.hif.cli)@\spxentry{hif\_bandpass()}\spxextra{in module pipeline.hif.cli}}

\begin{fulllineitems}
\phantomsection\label{\detokenize{_autosummary/pipeline.hif.cli.hif_bandpass:pipeline.hif.cli.hif_bandpass}}
\pysigstartsignatures
\pysiglinewithargsret
{\sphinxbfcode{\sphinxupquote{hif\_bandpass}}}
{\sphinxparam{\DUrole{n}{vis}\DUrole{o}{=}\DUrole{default_value}{None}}\sphinxparamcomma \sphinxparam{\DUrole{n}{caltable}\DUrole{o}{=}\DUrole{default_value}{None}}\sphinxparamcomma \sphinxparam{\DUrole{n}{field}\DUrole{o}{=}\DUrole{default_value}{None}}\sphinxparamcomma \sphinxparam{\DUrole{n}{intent}\DUrole{o}{=}\DUrole{default_value}{None}}\sphinxparamcomma \sphinxparam{\DUrole{n}{spw}\DUrole{o}{=}\DUrole{default_value}{None}}\sphinxparamcomma \sphinxparam{\DUrole{n}{antenna}\DUrole{o}{=}\DUrole{default_value}{None}}\sphinxparamcomma \sphinxparam{\DUrole{n}{phaseup}\DUrole{o}{=}\DUrole{default_value}{None}}\sphinxparamcomma \sphinxparam{\DUrole{n}{phaseupsolint}\DUrole{o}{=}\DUrole{default_value}{None}}\sphinxparamcomma \sphinxparam{\DUrole{n}{phaseupbw}\DUrole{o}{=}\DUrole{default_value}{None}}\sphinxparamcomma \sphinxparam{\DUrole{n}{solint}\DUrole{o}{=}\DUrole{default_value}{None}}\sphinxparamcomma \sphinxparam{\DUrole{n}{combine}\DUrole{o}{=}\DUrole{default_value}{None}}\sphinxparamcomma \sphinxparam{\DUrole{n}{refant}\DUrole{o}{=}\DUrole{default_value}{None}}\sphinxparamcomma \sphinxparam{\DUrole{n}{solnorm}\DUrole{o}{=}\DUrole{default_value}{None}}\sphinxparamcomma \sphinxparam{\DUrole{n}{minblperant}\DUrole{o}{=}\DUrole{default_value}{None}}\sphinxparamcomma \sphinxparam{\DUrole{n}{minsnr}\DUrole{o}{=}\DUrole{default_value}{None}}}
{}
\pysigstopsignatures
\sphinxAtStartPar
Compute bandpass calibration solutions

\sphinxAtStartPar
Compute amplitude and phase as a function of frequency for each spectral
window in each MeasurementSet.

\sphinxAtStartPar
Previous calibration can be applied on the fly.

\sphinxAtStartPar
hif\_bandpass computes a bandpass solution for every specified science
spectral window. By default a ‘phaseup’ pre\sphinxhyphen{}calibration is performed
and applied on the fly to the data, before the bandpass is computed.

\sphinxAtStartPar
The hif\_refant task may be used to precompute a prioritized list of
reference antennas.
\begin{quote}\begin{description}
\sphinxlineitem{Parameters}\begin{itemize}
\item {} 
\sphinxAtStartPar
\sphinxstylestrong{vis} \textendash{} The list of input MeasurementSets. Defaults to the list of MeasurementSets specified in the h\_init or hif\_importdata task.
‘’: use all MeasurementSets in the context
Examples: ‘ngc5921.ms’, {[}‘ngc5921a.ms’, ngc5921b.ms’, ‘ngc5921c.ms’{]}

\item {} 
\sphinxAtStartPar
\sphinxstylestrong{caltable} \textendash{} The list of output calibration tables. Defaults to the standard pipeline naming convention.
Example: caltable={[}‘M82.gcal’, ‘M82B.gcal’{]}

\item {} 
\sphinxAtStartPar
\sphinxstylestrong{field} \textendash{} The list of field names or field ids for which bandpasses are computed. Defaults to all fields.
Examples: field=’3C279’, field=’3C279, M82’

\item {} 
\sphinxAtStartPar
\sphinxstylestrong{intent} \textendash{} A string containing a comma delimited list of intents against which the selected fields are matched.  Defaults to all data
with bandpass intent.
Example: intent=’\sphinxtitleref{*PHASE*}’

\item {} 
\sphinxAtStartPar
\sphinxstylestrong{spw} \textendash{} The list of spectral windows and channels for which bandpasses are computed. Defaults to all science spectral windows.
Example: spw=’11,13,15,17’

\item {} 
\sphinxAtStartPar
\sphinxstylestrong{antenna} \textendash{} Set of data selection antenna IDs

\item {} 
\sphinxAtStartPar
\sphinxstylestrong{phaseup} \textendash{} Do a phaseup on the data before computing the bandpass solution.

\item {} 
\sphinxAtStartPar
\sphinxstylestrong{phaseupsolint} \textendash{} The phase correction solution interval in CASA syntax. Used when phaseup is True.
Example: phaseupsolint=300

\item {} 
\sphinxAtStartPar
\sphinxstylestrong{phaseupbw} \textendash{} Bandwidth to be used for phaseup. Defaults to 500MHz. Used when phaseup is True.
Examples: phaseupbw=’’ to use entire bandpass
phaseupbw=’500MHz’ to use central 500MHz

\item {} 
\sphinxAtStartPar
\sphinxstylestrong{solint} \textendash{} Time and channel solution intervals in CASA syntax. Examples: solint=’inf,10ch’, solint=’inf’

\item {} 
\sphinxAtStartPar
\sphinxstylestrong{combine} \textendash{} Data axes to combine for solving. Axes are ‘’, ‘scan’, ‘spw’, ‘field’ or any comma\sphinxhyphen{}separated combination.
Example: combine=’scan,field’

\item {} 
\sphinxAtStartPar
\sphinxstylestrong{refant} \textendash{} Reference antenna names. Defaults to the value(s) stored in the pipeline context. If undefined in the pipeline context
defaults to the CASA reference antenna naming scheme.
Examples: refant=’DV01’, refant=’DV06,DV07’

\item {} 
\sphinxAtStartPar
\sphinxstylestrong{solnorm} \textendash{} Normalise the bandpass solution

\item {} 
\sphinxAtStartPar
\sphinxstylestrong{minblperant} \textendash{} Minimum number of baselines required per antenna for each solve. Antennas with fewer baselines are excluded from
solutions.

\item {} 
\sphinxAtStartPar
\sphinxstylestrong{minsnr} \textendash{} Reject solutions below this SNR

\end{itemize}

\sphinxlineitem{Returns}
\sphinxAtStartPar
The results object for the pipeline task is returned.

\end{description}\end{quote}
\subsubsection*{Examples}

\sphinxAtStartPar
1. Compute a channel bandpass for all visibility files in the pipeline
context using the CASA reference antenna determination scheme:

\begin{sphinxVerbatim}[commandchars=\\\{\}]
\PYG{g+gp}{\PYGZgt{}\PYGZgt{}\PYGZgt{} }\PYG{n}{hif\PYGZus{}bandpass}\PYG{p}{(}\PYG{p}{)}
\end{sphinxVerbatim}
\begin{enumerate}
\sphinxsetlistlabels{\arabic}{enumi}{enumii}{}{.}%
\setcounter{enumi}{1}
\item {} 
\sphinxAtStartPar
Same as the above but precompute a prioritized reference antenna list:

\end{enumerate}

\begin{sphinxVerbatim}[commandchars=\\\{\}]
\PYG{g+gp}{\PYGZgt{}\PYGZgt{}\PYGZgt{} }\PYG{n}{hif\PYGZus{}refant}\PYG{p}{(}\PYG{p}{)}
\PYG{g+gp}{\PYGZgt{}\PYGZgt{}\PYGZgt{} }\PYG{n}{hif\PYGZus{}bandpass}\PYG{p}{(}\PYG{p}{)}
\end{sphinxVerbatim}

\end{fulllineitems}


\sphinxstepscope


\section{pipeline.hif.cli.hif\_checkproductsize}
\label{\detokenize{_autosummary/pipeline.hif.cli.hif_checkproductsize:pipeline-hif-cli-hif-checkproductsize}}\label{\detokenize{_autosummary/pipeline.hif.cli.hif_checkproductsize::doc}}\index{hif\_checkproductsize() (in module pipeline.hif.cli)@\spxentry{hif\_checkproductsize()}\spxextra{in module pipeline.hif.cli}}

\begin{fulllineitems}
\phantomsection\label{\detokenize{_autosummary/pipeline.hif.cli.hif_checkproductsize:pipeline.hif.cli.hif_checkproductsize}}
\pysigstartsignatures
\pysiglinewithargsret
{\sphinxbfcode{\sphinxupquote{hif\_checkproductsize}}}
{\sphinxparam{\DUrole{n}{vis}\DUrole{o}{=}\DUrole{default_value}{None}}\sphinxparamcomma \sphinxparam{\DUrole{n}{maxcubesize}\DUrole{o}{=}\DUrole{default_value}{None}}\sphinxparamcomma \sphinxparam{\DUrole{n}{maxcubelimit}\DUrole{o}{=}\DUrole{default_value}{None}}\sphinxparamcomma \sphinxparam{\DUrole{n}{maxproductsize}\DUrole{o}{=}\DUrole{default_value}{None}}\sphinxparamcomma \sphinxparam{\DUrole{n}{maximsize}\DUrole{o}{=}\DUrole{default_value}{None}}\sphinxparamcomma \sphinxparam{\DUrole{n}{calcsb}\DUrole{o}{=}\DUrole{default_value}{None}}\sphinxparamcomma \sphinxparam{\DUrole{n}{parallel}\DUrole{o}{=}\DUrole{default_value}{None}}}
{}
\pysigstopsignatures
\sphinxAtStartPar
Check imaging product size

\sphinxAtStartPar
Check interferometry imaging product size and try to mitigate to maximum
allowed values. The task implements a mitigation cascade computing the largest
cube size and tries to reduce it below a given limit by adjusting the \sphinxcode{\sphinxupquote{nbins}},
\sphinxcode{\sphinxupquote{hm\_imsize}} and \sphinxcode{\sphinxupquote{hm\_cell}} parameters. If this step succeeds, it also checks the
overall imaging product size and if necessary reduces the number of fields to
be imaged.

\sphinxAtStartPar
Alternatively, if \sphinxcode{\sphinxupquote{maximsize}} is set, the image product pixel count is
mitigated by trying to adjust \sphinxcode{\sphinxupquote{hm\_cell}} parameter. If the pixel count is still
greater than \sphinxcode{\sphinxupquote{maximsize}} at \sphinxcode{\sphinxupquote{hm\_cell}} of 4ppb, then this value is kept and
the image field is truncated around the phase center by forcing \sphinxcode{\sphinxupquote{hm\_imsize}}
= \sphinxcode{\sphinxupquote{maximsize}}.

\sphinxAtStartPar
Note that mitigation for image pixel count and for the product size currently
are mutually exclusive, with maximsize taking precedence if set.
\begin{quote}\begin{description}
\sphinxlineitem{Parameters}\begin{itemize}
\item {} 
\sphinxAtStartPar
\sphinxstylestrong{vis} \textendash{} The list of input MeasurementSets. Defaults to the list of MeasurementSets specified in the h\_init or hif\_importdata task.
‘’: use all MeasurementSets in the context
Examples: ‘ngc5921.ms’, {[}‘ngc5921a.ms’, ngc5921b.ms’, ‘ngc5921c.ms’{]}

\item {} 
\sphinxAtStartPar
\sphinxstylestrong{maxcubesize} \textendash{} Maximum allowed cube size in gigabytes (mitigation goal) \sphinxhyphen{}1: automatic from performance parameters

\item {} 
\sphinxAtStartPar
\sphinxstylestrong{maxcubelimit} \textendash{} Maximum allowed cube limit in gigabytes (mitigation failure limit)
\sphinxhyphen{}1: automatic from performance parameters

\item {} 
\sphinxAtStartPar
\sphinxstylestrong{maxproductsize} \textendash{} Maximum allowed product size in gigabytes (mitigation goal and failure limit)
\sphinxhyphen{}1: automatic from performance parameters

\item {} 
\sphinxAtStartPar
\sphinxstylestrong{maximsize} \textendash{} Maximum allowed image count size (mitigation goal and hard maximum).
Parameter \sphinxcode{\sphinxupquote{maximsize}} must be even and divisible by 2,3,5,7 only.
Note that \sphinxcode{\sphinxupquote{maximsize}} is disabled by default and cannot be set at
the same time as \sphinxcode{\sphinxupquote{maxcubesize}}, \sphinxcode{\sphinxupquote{maxcubelimit}} and \sphinxcode{\sphinxupquote{maxproductsize}}!
\sphinxhyphen{}1: disables mitigation for this parameter

\item {} 
\sphinxAtStartPar
\sphinxstylestrong{calcsb} \textendash{} Force (re\sphinxhyphen{})calculation of sensitivities and beams

\item {} 
\sphinxAtStartPar
\sphinxstylestrong{parallel} \textendash{} Use MPI cluster where possible

\end{itemize}

\sphinxlineitem{Returns}
\sphinxAtStartPar
The results object for the pipeline task is returned.

\end{description}\end{quote}
\subsubsection*{Examples}
\begin{enumerate}
\sphinxsetlistlabels{\arabic}{enumi}{enumii}{}{.}%
\item {} 
\sphinxAtStartPar
Basic call to check the product sizes using internal defaults

\end{enumerate}

\begin{sphinxVerbatim}[commandchars=\\\{\}]
\PYG{g+gp}{\PYGZgt{}\PYGZgt{}\PYGZgt{} }\PYG{n}{hif\PYGZus{}checkproductsize}\PYG{p}{(}\PYG{p}{)}
\end{sphinxVerbatim}
\begin{enumerate}
\sphinxsetlistlabels{\arabic}{enumi}{enumii}{}{.}%
\setcounter{enumi}{1}
\item {} 
\sphinxAtStartPar
Typical ALMA call

\end{enumerate}

\begin{sphinxVerbatim}[commandchars=\\\{\}]
\PYG{g+gp}{\PYGZgt{}\PYGZgt{}\PYGZgt{} }\PYG{n}{hif\PYGZus{}checkproductsize}\PYG{p}{(}\PYG{n}{maxcubesize}\PYG{o}{=}\PYG{l+m+mf}{40.0}\PYG{p}{,} \PYG{n}{maxcubelimit}\PYG{o}{=}\PYG{l+m+mf}{60.0}\PYG{p}{,} \PYG{n}{maxproductsize}\PYG{o}{=}\PYG{l+m+mf}{350.0}\PYG{p}{)}
\end{sphinxVerbatim}

\end{fulllineitems}


\sphinxstepscope


\section{pipeline.hif.cli.hif\_correctedampflag}
\label{\detokenize{_autosummary/pipeline.hif.cli.hif_correctedampflag:pipeline-hif-cli-hif-correctedampflag}}\label{\detokenize{_autosummary/pipeline.hif.cli.hif_correctedampflag::doc}}\index{hif\_correctedampflag() (in module pipeline.hif.cli)@\spxentry{hif\_correctedampflag()}\spxextra{in module pipeline.hif.cli}}

\begin{fulllineitems}
\phantomsection\label{\detokenize{_autosummary/pipeline.hif.cli.hif_correctedampflag:pipeline.hif.cli.hif_correctedampflag}}
\pysigstartsignatures
\pysiglinewithargsret
{\sphinxbfcode{\sphinxupquote{hif\_correctedampflag}}}
{\sphinxparam{\DUrole{n}{vis}\DUrole{o}{=}\DUrole{default_value}{None}}\sphinxparamcomma \sphinxparam{\DUrole{n}{intent}\DUrole{o}{=}\DUrole{default_value}{None}}\sphinxparamcomma \sphinxparam{\DUrole{n}{field}\DUrole{o}{=}\DUrole{default_value}{None}}\sphinxparamcomma \sphinxparam{\DUrole{n}{spw}\DUrole{o}{=}\DUrole{default_value}{None}}\sphinxparamcomma \sphinxparam{\DUrole{n}{antnegsig}\DUrole{o}{=}\DUrole{default_value}{None}}\sphinxparamcomma \sphinxparam{\DUrole{n}{antpossig}\DUrole{o}{=}\DUrole{default_value}{None}}\sphinxparamcomma \sphinxparam{\DUrole{n}{tmantint}\DUrole{o}{=}\DUrole{default_value}{None}}\sphinxparamcomma \sphinxparam{\DUrole{n}{tmint}\DUrole{o}{=}\DUrole{default_value}{None}}\sphinxparamcomma \sphinxparam{\DUrole{n}{tmbl}\DUrole{o}{=}\DUrole{default_value}{None}}\sphinxparamcomma \sphinxparam{\DUrole{n}{antblnegsig}\DUrole{o}{=}\DUrole{default_value}{None}}\sphinxparamcomma \sphinxparam{\DUrole{n}{antblpossig}\DUrole{o}{=}\DUrole{default_value}{None}}\sphinxparamcomma \sphinxparam{\DUrole{n}{relaxed\_factor}\DUrole{o}{=}\DUrole{default_value}{None}}\sphinxparamcomma \sphinxparam{\DUrole{n}{niter}\DUrole{o}{=}\DUrole{default_value}{None}}}
{}
\pysigstopsignatures
\sphinxAtStartPar
Flag corrected \sphinxhyphen{} model amplitudes based on calibrators.

\sphinxAtStartPar
This task computes the flagging heuristics on a calibrator by calling hif\_correctedampflag
which looks for outlier visibility points by statistically examining the scalar
difference of corrected amplitudes minus model amplitudes, and flags those outliers.
The philosophy is that only outlier data points that have remained outliers after
calibration will be flagged. The heuristic works equally well on resolved calibrators
and point sources because it is not performing a vector difference, and thus is not
sensitive to nulls in the flux density vs. uvdistance domain. Note that the phase of
the data is not assessed.
\begin{quote}\begin{description}
\sphinxlineitem{Parameters}\begin{itemize}
\item {} 
\sphinxAtStartPar
\sphinxstylestrong{vis} \textendash{} The list of input MeasurementSets. Defaults to the list of MeasurementSets specified in the h\_init or hif\_importdata task.
‘’: use all MeasurementSets in the context
Examples: ‘ngc5921.ms’, {[}‘ngc5921a.ms’, ngc5921b.ms’, ‘ngc5921c.ms’{]}

\item {} 
\sphinxAtStartPar
\sphinxstylestrong{intent} \textendash{} A string containing a comma delimited list of intents against which the selected fields are matched. If undefined (default),
it will select all data with the BANDPASS intent.
Example: intent=’\sphinxtitleref{*PHASE*}’

\item {} 
\sphinxAtStartPar
\sphinxstylestrong{field} \textendash{} The list of field names or field ids for which bandpasses are computed. If undefined (default), it will select all fields.
Examples: field=’3C279’, ‘3C279, M82’

\item {} 
\sphinxAtStartPar
\sphinxstylestrong{spw} \textendash{} The list of spectral windows and channels for which bandpasses are computed. If undefined (default), it will select all
science spectral windows.
Example: spw=’11,13,15,17’

\item {} 
\sphinxAtStartPar
\sphinxstylestrong{antnegsig} \textendash{} Lower sigma threshold for identifying outliers as a result of bad antennas within individual timestamps

\item {} 
\sphinxAtStartPar
\sphinxstylestrong{antpossig} \textendash{} Upper sigma threshold for identifying outliers as a result of bad antennas within individual timestamps

\item {} 
\sphinxAtStartPar
\sphinxstylestrong{tmantint} \textendash{} Threshold for maximum fraction of timestamps that are allowed to contain outliers

\item {} 
\sphinxAtStartPar
\sphinxstylestrong{tmint} \textendash{} Initial threshold for maximum fraction of “outlier timestamps” over “total timestamps” that a baseline may be a part of

\item {} 
\sphinxAtStartPar
\sphinxstylestrong{tmbl} \textendash{} Initial threshold for maximum fraction of “bad baselines” over “all timestamps” that an antenna may be a part of

\item {} 
\sphinxAtStartPar
\sphinxstylestrong{antblnegsig} \textendash{} Lower sigma threshold for identifying outliers as a result of “bad baselines” and/or “bad antennas” within baselines (across all timestamps)

\item {} 
\sphinxAtStartPar
\sphinxstylestrong{antblpossig} \textendash{} Upper sigma threshold for identifying outliers as a result of “bad baselines” and/or “bad antennas” within baselines (across all timestamps)

\item {} 
\sphinxAtStartPar
\sphinxstylestrong{relaxed\_factor} \textendash{} Relaxed value to set the threshold scaling factor to under certain conditions (see task description)

\item {} 
\sphinxAtStartPar
\sphinxstylestrong{niter} \textendash{} Maximum number of times to iterate on evaluation of flagging heuristics. If an iteration results in no new flags, then
subsequent iterations are skipped.

\end{itemize}

\sphinxlineitem{Returns}
\sphinxAtStartPar
The results object for the pipeline task is returned.

\end{description}\end{quote}
\subsubsection*{Examples}

\sphinxAtStartPar
Run default flagging on bandpass calibrator with recommended settings:

\begin{sphinxVerbatim}[commandchars=\\\{\}]
\PYG{g+gp}{\PYGZgt{}\PYGZgt{}\PYGZgt{} }\PYG{n}{hif\PYGZus{}correctedampflag}\PYG{p}{(}\PYG{p}{)}
\end{sphinxVerbatim}

\end{fulllineitems}


\sphinxstepscope


\section{pipeline.hif.cli.hif\_editimlist}
\label{\detokenize{_autosummary/pipeline.hif.cli.hif_editimlist:pipeline-hif-cli-hif-editimlist}}\label{\detokenize{_autosummary/pipeline.hif.cli.hif_editimlist::doc}}\index{hif\_editimlist() (in module pipeline.hif.cli)@\spxentry{hif\_editimlist()}\spxextra{in module pipeline.hif.cli}}

\begin{fulllineitems}
\phantomsection\label{\detokenize{_autosummary/pipeline.hif.cli.hif_editimlist:pipeline.hif.cli.hif_editimlist}}
\pysigstartsignatures
\pysiglinewithargsret
{\sphinxbfcode{\sphinxupquote{hif\_editimlist}}}
{\sphinxparam{\DUrole{n}{imagename}\DUrole{o}{=}\DUrole{default_value}{None}}\sphinxparamcomma \sphinxparam{\DUrole{n}{search\_radius\_arcsec}\DUrole{o}{=}\DUrole{default_value}{None}}\sphinxparamcomma \sphinxparam{\DUrole{n}{cell}\DUrole{o}{=}\DUrole{default_value}{None}}\sphinxparamcomma \sphinxparam{\DUrole{n}{cfcache}\DUrole{o}{=}\DUrole{default_value}{None}}\sphinxparamcomma \sphinxparam{\DUrole{n}{conjbeams}\DUrole{o}{=}\DUrole{default_value}{None}}\sphinxparamcomma \sphinxparam{\DUrole{n}{cyclefactor}\DUrole{o}{=}\DUrole{default_value}{None}}\sphinxparamcomma \sphinxparam{\DUrole{n}{cycleniter}\DUrole{o}{=}\DUrole{default_value}{None}}\sphinxparamcomma \sphinxparam{\DUrole{n}{nmajor}\DUrole{o}{=}\DUrole{default_value}{None}}\sphinxparamcomma \sphinxparam{\DUrole{n}{datatype}\DUrole{o}{=}\DUrole{default_value}{None}}\sphinxparamcomma \sphinxparam{\DUrole{n}{datacolumn}\DUrole{o}{=}\DUrole{default_value}{None}}\sphinxparamcomma \sphinxparam{\DUrole{n}{deconvolver}\DUrole{o}{=}\DUrole{default_value}{None}}\sphinxparamcomma \sphinxparam{\DUrole{n}{editmode}\DUrole{o}{=}\DUrole{default_value}{None}}\sphinxparamcomma \sphinxparam{\DUrole{n}{field}\DUrole{o}{=}\DUrole{default_value}{None}}\sphinxparamcomma \sphinxparam{\DUrole{n}{imaging\_mode}\DUrole{o}{=}\DUrole{default_value}{None}}\sphinxparamcomma \sphinxparam{\DUrole{n}{imsize}\DUrole{o}{=}\DUrole{default_value}{None}}\sphinxparamcomma \sphinxparam{\DUrole{n}{intent}\DUrole{o}{=}\DUrole{default_value}{None}}\sphinxparamcomma \sphinxparam{\DUrole{n}{gridder}\DUrole{o}{=}\DUrole{default_value}{None}}\sphinxparamcomma \sphinxparam{\DUrole{n}{mask}\DUrole{o}{=}\DUrole{default_value}{None}}\sphinxparamcomma \sphinxparam{\DUrole{n}{pbmask}\DUrole{o}{=}\DUrole{default_value}{None}}\sphinxparamcomma \sphinxparam{\DUrole{n}{nbin}\DUrole{o}{=}\DUrole{default_value}{None}}\sphinxparamcomma \sphinxparam{\DUrole{n}{nchan}\DUrole{o}{=}\DUrole{default_value}{None}}\sphinxparamcomma \sphinxparam{\DUrole{n}{niter}\DUrole{o}{=}\DUrole{default_value}{None}}\sphinxparamcomma \sphinxparam{\DUrole{n}{nterms}\DUrole{o}{=}\DUrole{default_value}{None}}\sphinxparamcomma \sphinxparam{\DUrole{n}{parameter\_file}\DUrole{o}{=}\DUrole{default_value}{None}}\sphinxparamcomma \sphinxparam{\DUrole{n}{pblimit}\DUrole{o}{=}\DUrole{default_value}{None}}\sphinxparamcomma \sphinxparam{\DUrole{n}{phasecenter}\DUrole{o}{=}\DUrole{default_value}{None}}\sphinxparamcomma \sphinxparam{\DUrole{n}{reffreq}\DUrole{o}{=}\DUrole{default_value}{None}}\sphinxparamcomma \sphinxparam{\DUrole{n}{restfreq}\DUrole{o}{=}\DUrole{default_value}{None}}\sphinxparamcomma \sphinxparam{\DUrole{n}{robust}\DUrole{o}{=}\DUrole{default_value}{None}}\sphinxparamcomma \sphinxparam{\DUrole{n}{scales}\DUrole{o}{=}\DUrole{default_value}{None}}\sphinxparamcomma \sphinxparam{\DUrole{n}{specmode}\DUrole{o}{=}\DUrole{default_value}{None}}\sphinxparamcomma \sphinxparam{\DUrole{n}{spw}\DUrole{o}{=}\DUrole{default_value}{None}}\sphinxparamcomma \sphinxparam{\DUrole{n}{start}\DUrole{o}{=}\DUrole{default_value}{None}}\sphinxparamcomma \sphinxparam{\DUrole{n}{stokes}\DUrole{o}{=}\DUrole{default_value}{None}}\sphinxparamcomma \sphinxparam{\DUrole{n}{sensitivity}\DUrole{o}{=}\DUrole{default_value}{None}}\sphinxparamcomma \sphinxparam{\DUrole{n}{threshold}\DUrole{o}{=}\DUrole{default_value}{None}}\sphinxparamcomma \sphinxparam{\DUrole{n}{nsigma}\DUrole{o}{=}\DUrole{default_value}{None}}\sphinxparamcomma \sphinxparam{\DUrole{n}{uvtaper}\DUrole{o}{=}\DUrole{default_value}{None}}\sphinxparamcomma \sphinxparam{\DUrole{n}{uvrange}\DUrole{o}{=}\DUrole{default_value}{None}}\sphinxparamcomma \sphinxparam{\DUrole{n}{width}\DUrole{o}{=}\DUrole{default_value}{None}}\sphinxparamcomma \sphinxparam{\DUrole{n}{vlass\_plane\_reject\_ms}\DUrole{o}{=}\DUrole{default_value}{None}}}
{}
\pysigstopsignatures
\sphinxAtStartPar
Add to a list of images to be produced with hif\_makeimages()

\sphinxAtStartPar
Add to a list of images to be produced with hif\_makeimages(), which uses hif\_tclean() to invoke CASA tclean.
Many of the hif\_editimlist() inputs map directly to tclean parameters.

\sphinxAtStartPar
The results object for the pipeline task is returned.
\begin{quote}\begin{description}
\sphinxlineitem{Parameters}\begin{itemize}
\item {} 
\sphinxAtStartPar
\sphinxstylestrong{imagename} \textendash{} Prefix for output image names.

\item {} 
\sphinxAtStartPar
\sphinxstylestrong{search\_radius\_arcsec} \textendash{} Size of the field finding beam search radius in arcsec.

\item {} 
\sphinxAtStartPar
\sphinxstylestrong{cell} \textendash{} Image X and Y cell size(s) with units or pixels per beam. Single value same for both. ‘\textless{}number\textgreater{}ppb’ for pixels per beam.
Compute cell size based on the UV coverage of all the fields
to be imaged and use a 5 pix per beam sampling.
The pix per beam specification uses the above default cell size
(‘5ppb’) and scales it accordingly.
example: {[}‘0.5arcsec’, ‘0.5arcsec’{]} ‘3ppb’

\item {} 
\sphinxAtStartPar
\sphinxstylestrong{cfcache} \textendash{} Convolution function cache directory name

\item {} 
\sphinxAtStartPar
\sphinxstylestrong{conjbeams} \textendash{} Use conjugate frequency in tclean for wideband A\sphinxhyphen{}terms.

\item {} 
\sphinxAtStartPar
\sphinxstylestrong{cyclefactor} \textendash{} Controls the depth of clean in minor cycles based on PSF.

\item {} 
\sphinxAtStartPar
\sphinxstylestrong{cycleniter} \textendash{} Controls max number of minor cycle iterations in a single major cycle.

\item {} 
\sphinxAtStartPar
\sphinxstylestrong{nmajor} \textendash{} Controls the maximum number of major cycles to evaluate.

\item {} 
\sphinxAtStartPar
\sphinxstylestrong{datatype} \textendash{} Data type(s) to image. The default ‘’ selects the best available data type (e.g. selfcal over regcal) with
an automatic fallback to the next available data type.
With the \sphinxcode{\sphinxupquote{datatype}} parameter of ‘regcal’ or ‘selfcal’, one
can force the use of only given data type(s).
Note that this parameter is only for non\sphinxhyphen{}VLASS data when the datacolumn
is not explictly set by user or imaging heuristics.

\item {} 
\sphinxAtStartPar
\sphinxstylestrong{datacolumn} \textendash{} Data column to image; this will take precedence over the datatype parameter.

\item {} 
\sphinxAtStartPar
\sphinxstylestrong{deconvolver} \textendash{} Minor cycle algorithm (multiscale or mtmfs)

\item {} 
\sphinxAtStartPar
\sphinxstylestrong{editmode} \textendash{} The edit mode of the task (‘add’ or ‘replace’). Defaults to ‘add’.

\item {} 
\sphinxAtStartPar
\sphinxstylestrong{field} \textendash{} Set of data selection field names or ids.

\item {} 
\sphinxAtStartPar
\sphinxstylestrong{imaging\_mode} \textendash{} Identity of product type (e.g. VLASS quick look) desired.  This will determine the heuristics used.

\item {} 
\sphinxAtStartPar
\sphinxstylestrong{imsize} \textendash{} Image X and Y size(s) in pixels or PB level (single fields), ‘’ for default. Single value same for both. ‘\textless{}number\textgreater{}pb’ for PB level.

\item {} 
\sphinxAtStartPar
\sphinxstylestrong{intent} \textendash{} Set of data selection intents

\item {} 
\sphinxAtStartPar
\sphinxstylestrong{gridder} \textendash{} Name of the gridder to use with tclean

\item {} 
\sphinxAtStartPar
\sphinxstylestrong{mask} \textendash{} Used to declare whether to use a predefined mask for tclean.

\item {} 
\sphinxAtStartPar
\sphinxstylestrong{pbmask} \textendash{} Used to declare primary beam gain level for cleaning with primary beam mask (usemask=’pb’), used only for VLASS\sphinxhyphen{}SE\sphinxhyphen{}CONT imaging mode.

\item {} 
\sphinxAtStartPar
\sphinxstylestrong{nbin} \textendash{} Channel binning factor.

\item {} 
\sphinxAtStartPar
\sphinxstylestrong{nchan} \textendash{} Number of channels, \sphinxhyphen{}1 = all

\item {} 
\sphinxAtStartPar
\sphinxstylestrong{niter} \textendash{} The max total number of minor cycle iterations allowed for tclean

\item {} 
\sphinxAtStartPar
\sphinxstylestrong{nterms} \textendash{} Number of Taylor coefficients in the spectral model

\item {} 
\sphinxAtStartPar
\sphinxstylestrong{parameter\_file} \textendash{} keyword=value text file as alternative method of input parameters

\item {} 
\sphinxAtStartPar
\sphinxstylestrong{pblimit} \textendash{} PB gain level at which to cut off normalizations

\item {} 
\sphinxAtStartPar
\sphinxstylestrong{phasecenter} \textendash{} The default phase center is set to the mean of the field directions of all fields that are to be image together.
example: 0, ‘J2000 19h30m00 \sphinxhyphen{}40d00m00’

\item {} 
\sphinxAtStartPar
\sphinxstylestrong{reffreq} \textendash{} Reference frequency of the output image coordinate system

\item {} 
\sphinxAtStartPar
\sphinxstylestrong{restfreq} \textendash{} List of rest frequencies or a rest frequency in a string for output image.

\item {} 
\sphinxAtStartPar
\sphinxstylestrong{robust} \textendash{} Briggs robustness parameter for tclean

\item {} 
\sphinxAtStartPar
\sphinxstylestrong{scales} \textendash{} The scales for multi\sphinxhyphen{}scale imaging.

\item {} 
\sphinxAtStartPar
\sphinxstylestrong{specmode} \textendash{} Spectral gridding type (mfs, cont, cube, ‘’ for default)

\item {} 
\sphinxAtStartPar
\sphinxstylestrong{spw} \textendash{} Set of data selection spectral window/channels, ‘’ for all

\item {} 
\sphinxAtStartPar
\sphinxstylestrong{start} \textendash{} First channel for frequency mode images. Starts at first input channel of the spw.
example: ‘22.3GHz’

\item {} 
\sphinxAtStartPar
\sphinxstylestrong{stokes} \textendash{} Stokes Planes to make

\item {} 
\sphinxAtStartPar
\sphinxstylestrong{sensitivity} \textendash{} Theoretical sensitivity (override internal calculation)

\item {} 
\sphinxAtStartPar
\sphinxstylestrong{threshold} \textendash{} Stopping threshold (number in units of Jy, or string)

\item {} 
\sphinxAtStartPar
\sphinxstylestrong{nsigma} \textendash{} Multiplicative factor for rms\sphinxhyphen{}based threshold stopping

\item {} 
\sphinxAtStartPar
\sphinxstylestrong{uvtaper} \textendash{} Used to set a uv\sphinxhyphen{}taper during clean.

\item {} 
\sphinxAtStartPar
\sphinxstylestrong{uvrange} \textendash{} Set of data selection uv ranges, ‘’ for all.

\item {} 
\sphinxAtStartPar
\sphinxstylestrong{width} \textendash{} Channel width

\item {} 
\sphinxAtStartPar
\sphinxstylestrong{vlass\_plane\_reject\_ms} \textendash{} Only used for the ‘VLASS\sphinxhyphen{}SE\sphinxhyphen{}CUBE’ imaging mode. default: True If True, reject VLASS Coarse Cube planes with high flagging percentages (see the heuristics details below)
If False, do not perform flagging\sphinxhyphen{}based VLASS Coarse Cube plane rejection.
If the input value is a dictionary, the plane rejection heuristics will be performed with custom thresholds.
The optional keys are:
\begin{itemize}
\item {} 
\sphinxAtStartPar
exclude\_spw, default: ‘’
Spectral windows to be excluded from the VLASS Coarse Cube plane rejection consideration, i.e. always preserve.

\item {} 
\sphinxAtStartPar
flagpct\_thresh, default: 0.9
Flagging percentage threshold per field for the plane rejection.

\item {} 
\sphinxAtStartPar
nfield\_thresh: default: 12
A minimal number of fields above the flagging percentage threshold is required for the plane rejection.

\end{itemize}

\end{itemize}

\sphinxlineitem{Returns}
\sphinxAtStartPar
The results object for the pipeline task is returned.

\end{description}\end{quote}

\sphinxAtStartPar
Examples:

\end{fulllineitems}


\sphinxstepscope


\section{pipeline.hif.cli.hif\_findcont}
\label{\detokenize{_autosummary/pipeline.hif.cli.hif_findcont:pipeline-hif-cli-hif-findcont}}\label{\detokenize{_autosummary/pipeline.hif.cli.hif_findcont::doc}}\index{hif\_findcont() (in module pipeline.hif.cli)@\spxentry{hif\_findcont()}\spxextra{in module pipeline.hif.cli}}

\begin{fulllineitems}
\phantomsection\label{\detokenize{_autosummary/pipeline.hif.cli.hif_findcont:pipeline.hif.cli.hif_findcont}}
\pysigstartsignatures
\pysiglinewithargsret
{\sphinxbfcode{\sphinxupquote{hif\_findcont}}}
{\sphinxparam{\DUrole{n}{vis}\DUrole{o}{=}\DUrole{default_value}{None}}\sphinxparamcomma \sphinxparam{\DUrole{n}{target\_list}\DUrole{o}{=}\DUrole{default_value}{None}}\sphinxparamcomma \sphinxparam{\DUrole{n}{hm\_mosweight}\DUrole{o}{=}\DUrole{default_value}{None}}\sphinxparamcomma \sphinxparam{\DUrole{n}{hm\_perchanweightdensity}\DUrole{o}{=}\DUrole{default_value}{None}}\sphinxparamcomma \sphinxparam{\DUrole{n}{hm\_weighting}\DUrole{o}{=}\DUrole{default_value}{None}}\sphinxparamcomma \sphinxparam{\DUrole{n}{datacolumn}\DUrole{o}{=}\DUrole{default_value}{None}}\sphinxparamcomma \sphinxparam{\DUrole{n}{parallel}\DUrole{o}{=}\DUrole{default_value}{None}}}
{}
\pysigstopsignatures
\sphinxAtStartPar
Find continuum frequency ranges

\sphinxAtStartPar
Find continuum frequency ranges for a list of specified targets.

\sphinxAtStartPar
If the cont.dat file is not already present in the working directory, then dirty image
cubes are created for each spectral window of each science target at the native channel
resolution unless the \sphinxcode{\sphinxupquote{nbins}} parameter was used in the preceding hif\_makeimlist stage.
Robust=1 Briggs weighting is used for optimal line sensitivity, even if a different
robust had been chosen in hifa\_imageprecheck to match the PI requested angular resolution.
Using moment0 and moment8 images of each cube, SNR\sphinxhyphen{}based masks are created, and the mean
spectrum of the joint mask is computed and evaluated with extensive heuristics to find the
channel ranges that are likely to be free of line emission.  Warnings are generated if
the channel ranges contain a small fraction of the bandwidth, or sample only a limited
extent of the spectrum.

\sphinxAtStartPar
If the cont.dat file already exists in the working directory before this task is executed,
then it will first examine the contents. For any spw that already has frequency ranges
defined in this file, it will not perform the analysis described above in favor of the
a priori ranges. For spws not listed in a pre\sphinxhyphen{}existing file, it will analyze them as
normal and update the file. In either case, the cont.dat file is used by the subsequent
hif\_uvcontsub and hif\_makeimages stages.
\begin{quote}\begin{description}
\sphinxlineitem{Parameters}\begin{itemize}
\item {} 
\sphinxAtStartPar
\sphinxstylestrong{vis} \textendash{} The list of input MeasurementSets. Defaults to the list of MeasurementSets specified in the h\_init or hif\_importdata task.
‘’: use all MeasurementSets in the context
Examples: ‘ngc5921.ms’, {[}‘ngc5921a.ms’, ngc5921b.ms’, ‘ngc5921c.ms’{]}

\item {} 
\sphinxAtStartPar
\sphinxstylestrong{target\_list} \textendash{} Dictionary specifying targets to be imaged; blank will read list from context.

\item {} 
\sphinxAtStartPar
\sphinxstylestrong{hm\_mosweight} \textendash{} Mosaic weighting. Defaults to ‘’ to enable the automatic heuristics calculation.
Can be set to True or False manually.

\item {} 
\sphinxAtStartPar
\sphinxstylestrong{hm\_perchanweightdensity} \textendash{} Calculate the weight density for each channel independently.
Defaults to ‘’ to enable the automatic heuristics calculation. Can be set to True or False manually.

\item {} 
\sphinxAtStartPar
\sphinxstylestrong{hm\_weighting} \textendash{} Weighting scheme (natural,uniform,briggs,briggsabs{[}experimental{]},briggsbwtaper{[}experimental{]})

\item {} 
\sphinxAtStartPar
\sphinxstylestrong{datacolumn} \textendash{} Data column to image. Only to be used for manual overriding when the automatic choice by data type is not appropriate.

\item {} 
\sphinxAtStartPar
\sphinxstylestrong{parallel} \textendash{} Use MPI cluster where possible.

\end{itemize}

\sphinxlineitem{Returns}
\sphinxAtStartPar
The results object for the pipeline task is returned.

\end{description}\end{quote}
\subsubsection*{Examples}
\begin{enumerate}
\sphinxsetlistlabels{\arabic}{enumi}{enumii}{}{.}%
\item {} 
\sphinxAtStartPar
Perform continuum frequency range detection for all science targets and spws:

\end{enumerate}

\begin{sphinxVerbatim}[commandchars=\\\{\}]
\PYG{g+gp}{\PYGZgt{}\PYGZgt{}\PYGZgt{} }\PYG{n}{hif\PYGZus{}findcont}\PYG{p}{(}\PYG{p}{)}
\end{sphinxVerbatim}

\end{fulllineitems}


\sphinxstepscope


\section{pipeline.hif.cli.hif\_gaincal}
\label{\detokenize{_autosummary/pipeline.hif.cli.hif_gaincal:pipeline-hif-cli-hif-gaincal}}\label{\detokenize{_autosummary/pipeline.hif.cli.hif_gaincal::doc}}\index{hif\_gaincal() (in module pipeline.hif.cli)@\spxentry{hif\_gaincal()}\spxextra{in module pipeline.hif.cli}}

\begin{fulllineitems}
\phantomsection\label{\detokenize{_autosummary/pipeline.hif.cli.hif_gaincal:pipeline.hif.cli.hif_gaincal}}
\pysigstartsignatures
\pysiglinewithargsret
{\sphinxbfcode{\sphinxupquote{hif\_gaincal}}}
{\sphinxparam{\DUrole{n}{vis}\DUrole{o}{=}\DUrole{default_value}{None}}\sphinxparamcomma \sphinxparam{\DUrole{n}{caltable}\DUrole{o}{=}\DUrole{default_value}{None}}\sphinxparamcomma \sphinxparam{\DUrole{n}{field}\DUrole{o}{=}\DUrole{default_value}{None}}\sphinxparamcomma \sphinxparam{\DUrole{n}{intent}\DUrole{o}{=}\DUrole{default_value}{None}}\sphinxparamcomma \sphinxparam{\DUrole{n}{spw}\DUrole{o}{=}\DUrole{default_value}{None}}\sphinxparamcomma \sphinxparam{\DUrole{n}{antenna}\DUrole{o}{=}\DUrole{default_value}{None}}\sphinxparamcomma \sphinxparam{\DUrole{n}{hm\_gaintype}\DUrole{o}{=}\DUrole{default_value}{None}}\sphinxparamcomma \sphinxparam{\DUrole{n}{calmode}\DUrole{o}{=}\DUrole{default_value}{None}}\sphinxparamcomma \sphinxparam{\DUrole{n}{solint}\DUrole{o}{=}\DUrole{default_value}{None}}\sphinxparamcomma \sphinxparam{\DUrole{n}{combine}\DUrole{o}{=}\DUrole{default_value}{None}}\sphinxparamcomma \sphinxparam{\DUrole{n}{refant}\DUrole{o}{=}\DUrole{default_value}{None}}\sphinxparamcomma \sphinxparam{\DUrole{n}{refantmode}\DUrole{o}{=}\DUrole{default_value}{None}}\sphinxparamcomma \sphinxparam{\DUrole{n}{solnorm}\DUrole{o}{=}\DUrole{default_value}{None}}\sphinxparamcomma \sphinxparam{\DUrole{n}{minblperant}\DUrole{o}{=}\DUrole{default_value}{None}}\sphinxparamcomma \sphinxparam{\DUrole{n}{minsnr}\DUrole{o}{=}\DUrole{default_value}{None}}\sphinxparamcomma \sphinxparam{\DUrole{n}{smodel}\DUrole{o}{=}\DUrole{default_value}{None}}\sphinxparamcomma \sphinxparam{\DUrole{n}{splinetime}\DUrole{o}{=}\DUrole{default_value}{None}}\sphinxparamcomma \sphinxparam{\DUrole{n}{npointaver}\DUrole{o}{=}\DUrole{default_value}{None}}\sphinxparamcomma \sphinxparam{\DUrole{n}{phasewrap}\DUrole{o}{=}\DUrole{default_value}{None}}}
{}
\pysigstopsignatures
\sphinxAtStartPar
Determine temporal gains from calibrator observations

\sphinxAtStartPar
The complex gains are derived from the data column (raw data) divided by the
model column (usually set with hif\_setjy). The gains are obtained for a
specified solution interval, spw combination and field combination.

\sphinxAtStartPar
Good candidate reference antennas can be determined using the hif\_refant
task.

\sphinxAtStartPar
Previous calibrations that have been stored in the pipeline context are
applied on the fly. Users can interact with these calibrations via the
h\_export\_calstate and h\_import\_calstate tasks.
\begin{quote}\begin{description}
\sphinxlineitem{Parameters}\begin{itemize}
\item {} 
\sphinxAtStartPar
\sphinxstylestrong{vis} \textendash{} The list of input MeasurementSets. Defaults to the list of MeasurementSets specified in the h\_init or hif\_importdata task.
‘’: use all MeasurementSets in the context
Examples: ‘ngc5921.ms’, {[}‘ngc5921a.ms’, ngc5921b.ms’, ‘ngc5921c.ms’{]}

\item {} 
\sphinxAtStartPar
\sphinxstylestrong{caltable} \textendash{} The list of output calibration tables. Defaults to the standard pipeline naming convention.
Example: caltable={[}‘M82.gcal’, ‘M82B.gcal’{]}

\item {} 
\sphinxAtStartPar
\sphinxstylestrong{field} \textendash{} The list of field names or field ids for which gain solutions are to be computed. Defaults to all fields with the standard
intent.
Example: field=’3C279’, field=’3C279, M82’

\item {} 
\sphinxAtStartPar
\sphinxstylestrong{intent} \textendash{} A string containing a comma delimited list of intents against which the selected fields are matched. Defaults to \sphinxtitleref{*PHASE*}.
Examples: intent=’’, intent=’\sphinxtitleref{*AMP*,*PHASE*}’

\item {} 
\sphinxAtStartPar
\sphinxstylestrong{spw} \textendash{} The list of spectral windows and channels for which gain solutions are computed. Defaults to all science spectral
windows.
Examples: spw=’21’, spw=’21, 23’

\item {} 
\sphinxAtStartPar
\sphinxstylestrong{antenna} \textendash{} Set of data selection antenna ids

\item {} 
\sphinxAtStartPar
\sphinxstylestrong{hm\_gaintype} \textendash{} The type of gain calibration. The options are ‘gtype’ and ‘gspline’ for CASA gain types = ‘G’ and ‘GSPLINE’ respectively.

\item {} 
\sphinxAtStartPar
\sphinxstylestrong{calmode} \textendash{} Type of solution. The options are ‘ap’ (amp and phase), ‘p’ (phase only) and ‘a’ (amp only).
Examples: calmode=’p’, calmode=’a’, calmode=’ap’

\item {} 
\sphinxAtStartPar
\sphinxstylestrong{solint} \textendash{} Time solution intervals in CASA syntax. Works for hm\_gaintype=’gtype’ only.
Examples: solint=’inf’, solint=’int’, solint=’100sec’

\item {} 
\sphinxAtStartPar
\sphinxstylestrong{combine} \textendash{} Data axes to combine for solving. Options are  ‘’, ‘scan’, ‘spw’, ‘field’ or any comma\sphinxhyphen{}separated combination. Works for
hm\_gaintype=’gtype’ only.

\item {} 
\sphinxAtStartPar
\sphinxstylestrong{refant} \textendash{} Reference antenna name(s) in priority order. Defaults to most recent values set in the pipeline context. If no reference
antenna is defined in the pipeline context use the CASA
defaults.
Examples: refant=’DV01’, refant=’DV05,DV07’

\item {} 
\sphinxAtStartPar
\sphinxstylestrong{refantmode} \textendash{} Controls how the refant is applied. Currently available choices are ‘flex’, ‘strict’, and the default value of ‘’.
Setting to ‘’ allows the pipeline to select the appropriate
mode based on the state of the reference antenna list.
Examples: refantmode=’strict’, refantmode=’’

\item {} 
\sphinxAtStartPar
\sphinxstylestrong{solnorm} \textendash{} Normalize average solution amplitudes to 1.0

\item {} 
\sphinxAtStartPar
\sphinxstylestrong{minblperant} \textendash{} Minimum number of baselines required per antenna for each solve. Antennas with fewer baselines are excluded from
solutions. Works for hm\_gaintype=’gtype’ only.

\item {} 
\sphinxAtStartPar
\sphinxstylestrong{minsnr} \textendash{} Solutions below this SNR are rejected. Works for hm\_gaintype=’channel’ only.

\item {} 
\sphinxAtStartPar
\sphinxstylestrong{smodel} \textendash{} Point source Stokes parameters for source model (experimental). Defaults to using standard MODEL\_DATA column data.
Example: smodel={[}1,0,0,0{]}  \sphinxhyphen{} (I=1, unpolarized)

\item {} 
\sphinxAtStartPar
\sphinxstylestrong{splinetime} \textendash{} Spline timescale (sec). Used for hm\_gaintype=’gspline’. Typical splinetime should cover about 3 to 5 calibrator scans.

\item {} 
\sphinxAtStartPar
\sphinxstylestrong{npointaver} \textendash{} Tune phase\sphinxhyphen{}unwrapping algorithm. Used for hm\_gaintype=’gspline’. Keep at default value.

\item {} 
\sphinxAtStartPar
\sphinxstylestrong{phasewrap} \textendash{} Wrap the phase for changes larger than this amount (degrees). Used for hm\_gaintype=’gspline’. Keep at default value.

\end{itemize}

\sphinxlineitem{Returns}
\sphinxAtStartPar
The results object for the pipeline task is returned.

\end{description}\end{quote}
\subsubsection*{Examples}

\sphinxAtStartPar
Compute standard per scan gain solutions that will be used to calibrate
the target:

\begin{sphinxVerbatim}[commandchars=\\\{\}]
\PYG{g+gp}{\PYGZgt{}\PYGZgt{}\PYGZgt{} }\PYG{n}{hif\PYGZus{}gaincal}\PYG{p}{(}\PYG{p}{)}
\end{sphinxVerbatim}

\end{fulllineitems}


\sphinxstepscope


\section{pipeline.hif.cli.hif\_lowgainflag}
\label{\detokenize{_autosummary/pipeline.hif.cli.hif_lowgainflag:pipeline-hif-cli-hif-lowgainflag}}\label{\detokenize{_autosummary/pipeline.hif.cli.hif_lowgainflag::doc}}\index{hif\_lowgainflag() (in module pipeline.hif.cli)@\spxentry{hif\_lowgainflag()}\spxextra{in module pipeline.hif.cli}}

\begin{fulllineitems}
\phantomsection\label{\detokenize{_autosummary/pipeline.hif.cli.hif_lowgainflag:pipeline.hif.cli.hif_lowgainflag}}
\pysigstartsignatures
\pysiglinewithargsret
{\sphinxbfcode{\sphinxupquote{hif\_lowgainflag}}}
{\sphinxparam{\DUrole{n}{vis}\DUrole{o}{=}\DUrole{default_value}{None}}\sphinxparamcomma \sphinxparam{\DUrole{n}{intent}\DUrole{o}{=}\DUrole{default_value}{None}}\sphinxparamcomma \sphinxparam{\DUrole{n}{spw}\DUrole{o}{=}\DUrole{default_value}{None}}\sphinxparamcomma \sphinxparam{\DUrole{n}{refant}\DUrole{o}{=}\DUrole{default_value}{None}}\sphinxparamcomma \sphinxparam{\DUrole{n}{flag\_nmedian}\DUrole{o}{=}\DUrole{default_value}{None}}\sphinxparamcomma \sphinxparam{\DUrole{n}{fnm\_lo\_limit}\DUrole{o}{=}\DUrole{default_value}{None}}\sphinxparamcomma \sphinxparam{\DUrole{n}{fnm\_hi\_limit}\DUrole{o}{=}\DUrole{default_value}{None}}\sphinxparamcomma \sphinxparam{\DUrole{n}{tmef1\_limit}\DUrole{o}{=}\DUrole{default_value}{None}}}
{}
\pysigstopsignatures
\sphinxAtStartPar
Flag antennas with low or high gain

\sphinxAtStartPar
Deviant antennas are detected by outlier analysis of a view showing their
amplitude gains, pre\sphinxhyphen{}applying a temporary bandpass and phase solution.
This view is a list of 2D images with axes ‘Scan’ and ‘Antenna’; there
is one image for each spectral window and intent. A flagcmd to flag all data
for an antenna will be generated by any gain that is outside the range
{[}fnm\_lo\_limit * median, fnm\_hi\_limit * median{]}.
\begin{quote}\begin{description}
\sphinxlineitem{Parameters}\begin{itemize}
\item {} 
\sphinxAtStartPar
\sphinxstylestrong{vis} \textendash{} The list of input MeasurementSets. Defaults to the list of
MeasurementSets specified in the h\_init or hif\_importdata task.
‘’: use all MeasurementSets in the context
Examples: ‘ngc5921.ms’, {[}‘ngc5921a.ms’, ngc5921b.ms’, ‘ngc5921c.ms’{]}

\item {} 
\sphinxAtStartPar
\sphinxstylestrong{intent} \textendash{} A string containing the list of intents to be checked for
antennas with deviant gains. The default is blank, which
causes the task to select the ‘BANDPASS’ intent.

\item {} 
\sphinxAtStartPar
\sphinxstylestrong{spw} \textendash{} The list of spectral windows and channels to which the
calibration will be applied. Defaults to all science windows
in the pipeline context.
Examples: spw=’17’, spw=’11, 15’

\item {} 
\sphinxAtStartPar
\sphinxstylestrong{refant} \textendash{} A string containing a prioritized list of reference antenna
name(s) to be used to produce the gain table. Defaults to the
value(s) stored in the pipeline context. If undefined in the
pipeline context defaults to the CASA reference antenna naming
scheme.
Examples: refant=’DV01’, refant=’DV06,DV07’

\item {} 
\sphinxAtStartPar
\sphinxstylestrong{flag\_nmedian} \textendash{} Whether to flag figures of merit greater than
\sphinxcode{\sphinxupquote{fnm\_hi\_limit}} * median or lower than \sphinxcode{\sphinxupquote{fnm\_lo\_limit}} * median.
(default: True)

\item {} 
\sphinxAtStartPar
\sphinxstylestrong{fnm\_lo\_limit} \textendash{} Flag values lower than \sphinxcode{\sphinxupquote{fnm\_lo\_limit}} * median (default: 0.5)

\item {} 
\sphinxAtStartPar
\sphinxstylestrong{fnm\_hi\_limit} \textendash{} Flag values higher than \sphinxcode{\sphinxupquote{fnm\_hi\_limit}} * median (default: 1.5)

\item {} 
\sphinxAtStartPar
\sphinxstylestrong{niter} \textendash{} The maximum number of iterations to run of the sequence:
solve for amplitude gains, assess statistics, flag spw/antenna
combinations that are outliers (default: 2)

\item {} 
\sphinxAtStartPar
\sphinxstylestrong{tmef1\_limit} \textendash{} Threshold for “too many entirely flagged” \sphinxhyphen{}
the critical fraction of antennas whose solutions are entirely
flagged in the flagging view of a spw for this stage:
if the fraction is equal or greater than this value, then flag
the visibility data from all antennas in this spw
(default: 0.666)

\end{itemize}

\sphinxlineitem{Returns}
\sphinxAtStartPar
The results object for the pipeline task is returned.

\end{description}\end{quote}
\subsubsection*{Examples}
\begin{enumerate}
\sphinxsetlistlabels{\arabic}{enumi}{enumii}{}{.}%
\item {} 
\sphinxAtStartPar
Flag antennas with low or high gain using recommended thresholds:

\end{enumerate}

\begin{sphinxVerbatim}[commandchars=\\\{\}]
\PYG{g+gp}{\PYGZgt{}\PYGZgt{}\PYGZgt{} }\PYG{n}{hif\PYGZus{}lowgainflag}\PYG{p}{(}\PYG{p}{)}
\end{sphinxVerbatim}

\end{fulllineitems}


\sphinxstepscope


\section{pipeline.hif.cli.hif\_makecutoutimages}
\label{\detokenize{_autosummary/pipeline.hif.cli.hif_makecutoutimages:pipeline-hif-cli-hif-makecutoutimages}}\label{\detokenize{_autosummary/pipeline.hif.cli.hif_makecutoutimages::doc}}\index{hif\_makecutoutimages() (in module pipeline.hif.cli)@\spxentry{hif\_makecutoutimages()}\spxextra{in module pipeline.hif.cli}}

\begin{fulllineitems}
\phantomsection\label{\detokenize{_autosummary/pipeline.hif.cli.hif_makecutoutimages:pipeline.hif.cli.hif_makecutoutimages}}
\pysigstartsignatures
\pysiglinewithargsret
{\sphinxbfcode{\sphinxupquote{hif\_makecutoutimages}}}
{\sphinxparam{\DUrole{n}{vis}\DUrole{o}{=}\DUrole{default_value}{None}}\sphinxparamcomma \sphinxparam{\DUrole{n}{offsetblc}\DUrole{o}{=}\DUrole{default_value}{None}}\sphinxparamcomma \sphinxparam{\DUrole{n}{offsettrc}\DUrole{o}{=}\DUrole{default_value}{None}}}
{}
\pysigstopsignatures
\sphinxAtStartPar
Cutout central 1 sq. degree from VLASS QL, SE, and Coarse Cube images

\sphinxAtStartPar
Cutout central 1 sq. degree from VLASS QL, SE, and Coarse Cube images
\begin{quote}\begin{description}
\sphinxlineitem{Parameters}\begin{itemize}
\item {} 
\sphinxAtStartPar
\sphinxstylestrong{vis} \textendash{} List of visibility data files. These may be ASDMs, tar files of ASDMs, MSs,
or tar files of MSs.
If ASDM files are specified, they will be converted to
MS format.
example: vis={[}‘X227.ms’, ‘asdms.tar.gz’{]}

\item {} 
\sphinxAtStartPar
\sphinxstylestrong{offsetblc} \textendash{} \sphinxhyphen{}x and \sphinxhyphen{}y offsets to the bottom lower corner (blc) in arcseconds

\item {} 
\sphinxAtStartPar
\sphinxstylestrong{offsettrc} \textendash{} +x and +y offsets to the top right corner (trc) in arcseconds

\end{itemize}

\sphinxlineitem{Returns}
\sphinxAtStartPar
The results object for the pipeline task is returned.

\end{description}\end{quote}
\subsubsection*{Examples}
\begin{enumerate}
\sphinxsetlistlabels{\arabic}{enumi}{enumii}{}{.}%
\item {} 
\sphinxAtStartPar
Basic makecutoutimages task

\end{enumerate}

\begin{sphinxVerbatim}[commandchars=\\\{\}]
\PYG{g+gp}{\PYGZgt{}\PYGZgt{}\PYGZgt{} }\PYG{n}{hif\PYGZus{}makecutoutimages}\PYG{p}{(}\PYG{p}{)}
\end{sphinxVerbatim}

\end{fulllineitems}


\sphinxstepscope


\section{pipeline.hif.cli.hif\_makeimages}
\label{\detokenize{_autosummary/pipeline.hif.cli.hif_makeimages:pipeline-hif-cli-hif-makeimages}}\label{\detokenize{_autosummary/pipeline.hif.cli.hif_makeimages::doc}}\index{hif\_makeimages() (in module pipeline.hif.cli)@\spxentry{hif\_makeimages()}\spxextra{in module pipeline.hif.cli}}

\begin{fulllineitems}
\phantomsection\label{\detokenize{_autosummary/pipeline.hif.cli.hif_makeimages:pipeline.hif.cli.hif_makeimages}}
\pysigstartsignatures
\pysiglinewithargsret
{\sphinxbfcode{\sphinxupquote{hif\_makeimages}}}
{\sphinxparam{\DUrole{n}{vis}\DUrole{o}{=}\DUrole{default_value}{None}}\sphinxparamcomma \sphinxparam{\DUrole{n}{target\_list}\DUrole{o}{=}\DUrole{default_value}{None}}\sphinxparamcomma \sphinxparam{\DUrole{n}{hm\_masking}\DUrole{o}{=}\DUrole{default_value}{None}}\sphinxparamcomma \sphinxparam{\DUrole{n}{hm\_sidelobethreshold}\DUrole{o}{=}\DUrole{default_value}{None}}\sphinxparamcomma \sphinxparam{\DUrole{n}{hm\_noisethreshold}\DUrole{o}{=}\DUrole{default_value}{None}}\sphinxparamcomma \sphinxparam{\DUrole{n}{hm\_lownoisethreshold}\DUrole{o}{=}\DUrole{default_value}{None}}\sphinxparamcomma \sphinxparam{\DUrole{n}{hm\_negativethreshold}\DUrole{o}{=}\DUrole{default_value}{None}}\sphinxparamcomma \sphinxparam{\DUrole{n}{hm\_minbeamfrac}\DUrole{o}{=}\DUrole{default_value}{None}}\sphinxparamcomma \sphinxparam{\DUrole{n}{hm\_growiterations}\DUrole{o}{=}\DUrole{default_value}{None}}\sphinxparamcomma \sphinxparam{\DUrole{n}{hm\_dogrowprune}\DUrole{o}{=}\DUrole{default_value}{None}}\sphinxparamcomma \sphinxparam{\DUrole{n}{hm\_minpercentchange}\DUrole{o}{=}\DUrole{default_value}{None}}\sphinxparamcomma \sphinxparam{\DUrole{n}{hm\_fastnoise}\DUrole{o}{=}\DUrole{default_value}{None}}\sphinxparamcomma \sphinxparam{\DUrole{n}{hm\_nsigma}\DUrole{o}{=}\DUrole{default_value}{None}}\sphinxparamcomma \sphinxparam{\DUrole{n}{hm\_perchanweightdensity}\DUrole{o}{=}\DUrole{default_value}{None}}\sphinxparamcomma \sphinxparam{\DUrole{n}{hm\_npixels}\DUrole{o}{=}\DUrole{default_value}{None}}\sphinxparamcomma \sphinxparam{\DUrole{n}{hm\_cyclefactor}\DUrole{o}{=}\DUrole{default_value}{None}}\sphinxparamcomma \sphinxparam{\DUrole{n}{hm\_minpsffraction}\DUrole{o}{=}\DUrole{default_value}{None}}\sphinxparamcomma \sphinxparam{\DUrole{n}{hm\_maxpsffraction}\DUrole{o}{=}\DUrole{default_value}{None}}\sphinxparamcomma \sphinxparam{\DUrole{n}{hm\_weighting}\DUrole{o}{=}\DUrole{default_value}{None}}\sphinxparamcomma \sphinxparam{\DUrole{n}{hm\_cleaning}\DUrole{o}{=}\DUrole{default_value}{None}}\sphinxparamcomma \sphinxparam{\DUrole{n}{tlimit}\DUrole{o}{=}\DUrole{default_value}{None}}\sphinxparamcomma \sphinxparam{\DUrole{n}{drcorrect}\DUrole{o}{=}\DUrole{default_value}{None}}\sphinxparamcomma \sphinxparam{\DUrole{n}{masklimit}\DUrole{o}{=}\DUrole{default_value}{None}}\sphinxparamcomma \sphinxparam{\DUrole{n}{cleancontranges}\DUrole{o}{=}\DUrole{default_value}{None}}\sphinxparamcomma \sphinxparam{\DUrole{n}{calcsb}\DUrole{o}{=}\DUrole{default_value}{None}}\sphinxparamcomma \sphinxparam{\DUrole{n}{hm\_mosweight}\DUrole{o}{=}\DUrole{default_value}{None}}\sphinxparamcomma \sphinxparam{\DUrole{n}{overwrite\_on\_export}\DUrole{o}{=}\DUrole{default_value}{None}}\sphinxparamcomma \sphinxparam{\DUrole{n}{vlass\_plane\_reject\_im}\DUrole{o}{=}\DUrole{default_value}{None}}\sphinxparamcomma \sphinxparam{\DUrole{n}{parallel}\DUrole{o}{=}\DUrole{default_value}{None}}}
{}
\pysigstopsignatures
\sphinxAtStartPar
Compute clean map

\sphinxAtStartPar
Compute clean results from a list of specified targets.
\begin{quote}\begin{description}
\sphinxlineitem{Parameters}\begin{itemize}
\item {} 
\sphinxAtStartPar
\sphinxstylestrong{vis} \textendash{} The list of input MeasurementSets. Defaults to the list of MeasurementSets specified in the h\_init or hif\_importdata task.
‘’: use all MeasurementSets in the context
Examples: ‘ngc5921.ms’, {[}‘ngc5921a.ms’, ngc5921b.ms’, ‘ngc5921c.ms’{]}

\item {} 
\sphinxAtStartPar
\sphinxstylestrong{target\_list} \textendash{} Dictionary specifying targets to be imaged; blank will read list from context

\item {} 
\sphinxAtStartPar
\sphinxstylestrong{hm\_masking} \textendash{} Clean masking mode. Options are ‘centralregion’, ‘auto’, ‘manual’ and ‘none’

\item {} 
\sphinxAtStartPar
\sphinxstylestrong{hm\_sidelobethreshold} \textendash{} sidelobethreshold * the max sidelobe level

\item {} 
\sphinxAtStartPar
\sphinxstylestrong{hm\_noisethreshold} \textendash{} noisethreshold * rms in residual image

\item {} 
\sphinxAtStartPar
\sphinxstylestrong{hm\_lownoisethreshold} \textendash{} lownoisethreshold * rms in residual image

\item {} 
\sphinxAtStartPar
\sphinxstylestrong{hm\_negativethreshold} \textendash{} negativethreshold * rms in residual image

\item {} 
\sphinxAtStartPar
\sphinxstylestrong{hm\_minbeamfrac} \textendash{} Minimum beam fraction for pruning

\item {} 
\sphinxAtStartPar
\sphinxstylestrong{hm\_growiterations} \textendash{} Number of binary dilation iterations for growing the mask

\item {} 
\sphinxAtStartPar
\sphinxstylestrong{hm\_dogrowprune} \textendash{} Do pruning on the grow mask. Defaults to ‘’ to enable the automatic heuristics calculation.
Can be set to True or False manually.

\item {} 
\sphinxAtStartPar
\sphinxstylestrong{hm\_minpercentchange} \textendash{} Mask size change threshold

\item {} 
\sphinxAtStartPar
\sphinxstylestrong{hm\_fastnoise} \textendash{} Faster noise calculation for automask or nsigma stopping. Defaults to ‘’ to enable the automatic heuristics calculation.
Can be set to True or False manually.

\item {} 
\sphinxAtStartPar
\sphinxstylestrong{hm\_nsigma} \textendash{} Multiplicative factor for rms\sphinxhyphen{}based threshold stopping

\item {} 
\sphinxAtStartPar
\sphinxstylestrong{hm\_perchanweightdensity} \textendash{} Calculate the weight density for each channel independently. Defaults to ‘’ to enable the automatic heuristics calculation.
Can be set to True or False manually.

\item {} 
\sphinxAtStartPar
\sphinxstylestrong{hm\_npixels} \textendash{} Number of pixels to determine uv\sphinxhyphen{}cell size for super\sphinxhyphen{}uniform weighting

\item {} 
\sphinxAtStartPar
\sphinxstylestrong{hm\_cyclefactor} \textendash{} Scaling on PSF sidelobe level to compute the minor\sphinxhyphen{}cycle stopping threshold

\item {} 
\sphinxAtStartPar
\sphinxstylestrong{hm\_minpsffraction} \textendash{} PSF fraction that marks the max depth of cleaning in the minor cycle

\item {} 
\sphinxAtStartPar
\sphinxstylestrong{hm\_maxpsffraction} \textendash{} PSF fraction that marks the minimum depth of cleaning in the minor cycle

\item {} 
\sphinxAtStartPar
\sphinxstylestrong{hm\_weighting} \textendash{} Weighting scheme (natural,uniform,briggs,briggsabs{[}experimental{]},briggsbwtaper{[}experimental{]})

\item {} 
\sphinxAtStartPar
\sphinxstylestrong{hm\_cleaning} \textendash{} Pipeline cleaning mode

\item {} 
\sphinxAtStartPar
\sphinxstylestrong{tlimit} \textendash{} Times the sensitivity limit for cleaning

\item {} 
\sphinxAtStartPar
\sphinxstylestrong{drcorrect} \textendash{} Override the default heuristics\sphinxhyphen{}based DR correction (for ALMA data only)

\item {} 
\sphinxAtStartPar
\sphinxstylestrong{masklimit} \textendash{} Times good mask pixels for cleaning

\item {} 
\sphinxAtStartPar
\sphinxstylestrong{cleancontranges} \textendash{} Clean continuum frequency ranges in cubes

\item {} 
\sphinxAtStartPar
\sphinxstylestrong{calcsb} \textendash{} Force (re\sphinxhyphen{})calculation of sensitivities and beams

\item {} 
\sphinxAtStartPar
\sphinxstylestrong{hm\_mosweight} \textendash{} Mosaic weighting Defaults to ‘’ to enable the automatic heuristics calculation.
Can be set to True or False manually.

\item {} 
\sphinxAtStartPar
\sphinxstylestrong{overwrite\_on\_export} \textendash{} Replace existing image products when h/hifa/hifv\_exportdata is called.
If False, images that would have the same FITS name on export,
are amended to include a version number.  For example, if
oussid.J1248\sphinxhyphen{}4559\_ph.spw21.mfs.I.pbcor.fits would already be
exported by a previous call to hif\_makeimags, then
‘oussid.J1248\sphinxhyphen{}4559\_ph.spw21.mfs.I.pbcor.v2.fits’ would also be
exported to the products/ directory. The first exported
product retains the same name.  Additional products start
counting with ‘v2’, ‘v3’, etc.

\item {} 
\sphinxAtStartPar
\sphinxstylestrong{vlass\_plane\_reject\_im} \textendash{} Only used for the ‘VLASS\sphinxhyphen{}SE\sphinxhyphen{}CUBE’ imaging mode.

\sphinxAtStartPar
Default: True

\sphinxAtStartPar
If True, reject VLASS Coarse Cube planes with high flagging percentages or outlier beam sizes (see the heuristics details below)

\sphinxAtStartPar
If False, do not perform the post\sphinxhyphen{}imaging VLASS Coarse Cube plane rejection.
If the input value is a dictionary, the plane rejection heuristics will be performed with custom thresholds.
The optional keys could be:
\begin{itemize}
\item {} 
\sphinxAtStartPar
exclude\_spw, default: ‘’
Spectral windows to be excluded from the VLASS Coarse Cube post\sphinxhyphen{}imaging plane rejection consideration, i.e. always preserve.

\item {} 
\sphinxAtStartPar
flagpct\_thresh, default: 0.8
The flagging percentage across the entire mosaic to be considered to be high flagging level for the plane rejection.

\item {} 
\sphinxAtStartPar
beamdev\_thresh: default: 0.2
Threshold for the fractional beam deviation from the expected value required for the plane rejection.

\end{itemize}

\item {} 
\sphinxAtStartPar
\sphinxstylestrong{parallel} \textendash{} Clean images using MPI cluster

\end{itemize}

\sphinxlineitem{Returns}
\sphinxAtStartPar
The results object for the pipeline task is returned.

\end{description}\end{quote}
\subsubsection*{Examples}
\begin{enumerate}
\sphinxsetlistlabels{\arabic}{enumi}{enumii}{}{.}%
\item {} 
\sphinxAtStartPar
Compute clean results for all imaging targets defined in a previous hif\_makeimlist or hif\_editimlist call:

\end{enumerate}

\begin{sphinxVerbatim}[commandchars=\\\{\}]
\PYG{g+gp}{\PYGZgt{}\PYGZgt{}\PYGZgt{} }\PYG{n}{hif\PYGZus{}makeimages}\PYG{p}{(}\PYG{p}{)}
\end{sphinxVerbatim}
\begin{enumerate}
\sphinxsetlistlabels{\arabic}{enumi}{enumii}{}{.}%
\setcounter{enumi}{1}
\item {} 
\sphinxAtStartPar
Compute clean results overriding automatic masking choice:

\end{enumerate}

\begin{sphinxVerbatim}[commandchars=\\\{\}]
\PYG{g+gp}{\PYGZgt{}\PYGZgt{}\PYGZgt{} }\PYG{n}{hif\PYGZus{}makeimages}\PYG{p}{(}\PYG{n}{hm\PYGZus{}masking}\PYG{o}{=}\PYG{l+s+s1}{\PYGZsq{}}\PYG{l+s+s1}{centralregion}\PYG{l+s+s1}{\PYGZsq{}}\PYG{p}{)}
\end{sphinxVerbatim}

\end{fulllineitems}


\sphinxstepscope


\section{pipeline.hif.cli.hif\_makeimlist}
\label{\detokenize{_autosummary/pipeline.hif.cli.hif_makeimlist:pipeline-hif-cli-hif-makeimlist}}\label{\detokenize{_autosummary/pipeline.hif.cli.hif_makeimlist::doc}}\index{hif\_makeimlist() (in module pipeline.hif.cli)@\spxentry{hif\_makeimlist()}\spxextra{in module pipeline.hif.cli}}

\begin{fulllineitems}
\phantomsection\label{\detokenize{_autosummary/pipeline.hif.cli.hif_makeimlist:pipeline.hif.cli.hif_makeimlist}}
\pysigstartsignatures
\pysiglinewithargsret
{\sphinxbfcode{\sphinxupquote{hif\_makeimlist}}}
{\sphinxparam{\DUrole{n}{vis}\DUrole{o}{=}\DUrole{default_value}{None}}\sphinxparamcomma \sphinxparam{\DUrole{n}{imagename}\DUrole{o}{=}\DUrole{default_value}{None}}\sphinxparamcomma \sphinxparam{\DUrole{n}{intent}\DUrole{o}{=}\DUrole{default_value}{None}}\sphinxparamcomma \sphinxparam{\DUrole{n}{field}\DUrole{o}{=}\DUrole{default_value}{None}}\sphinxparamcomma \sphinxparam{\DUrole{n}{spw}\DUrole{o}{=}\DUrole{default_value}{None}}\sphinxparamcomma \sphinxparam{\DUrole{n}{contfile}\DUrole{o}{=}\DUrole{default_value}{None}}\sphinxparamcomma \sphinxparam{\DUrole{n}{linesfile}\DUrole{o}{=}\DUrole{default_value}{None}}\sphinxparamcomma \sphinxparam{\DUrole{n}{uvrange}\DUrole{o}{=}\DUrole{default_value}{None}}\sphinxparamcomma \sphinxparam{\DUrole{n}{specmode}\DUrole{o}{=}\DUrole{default_value}{None}}\sphinxparamcomma \sphinxparam{\DUrole{n}{outframe}\DUrole{o}{=}\DUrole{default_value}{None}}\sphinxparamcomma \sphinxparam{\DUrole{n}{hm\_imsize}\DUrole{o}{=}\DUrole{default_value}{None}}\sphinxparamcomma \sphinxparam{\DUrole{n}{hm\_cell}\DUrole{o}{=}\DUrole{default_value}{None}}\sphinxparamcomma \sphinxparam{\DUrole{n}{calmaxpix}\DUrole{o}{=}\DUrole{default_value}{None}}\sphinxparamcomma \sphinxparam{\DUrole{n}{phasecenter}\DUrole{o}{=}\DUrole{default_value}{None}}\sphinxparamcomma \sphinxparam{\DUrole{n}{nchan}\DUrole{o}{=}\DUrole{default_value}{None}}\sphinxparamcomma \sphinxparam{\DUrole{n}{start}\DUrole{o}{=}\DUrole{default_value}{None}}\sphinxparamcomma \sphinxparam{\DUrole{n}{width}\DUrole{o}{=}\DUrole{default_value}{None}}\sphinxparamcomma \sphinxparam{\DUrole{n}{nbins}\DUrole{o}{=}\DUrole{default_value}{None}}\sphinxparamcomma \sphinxparam{\DUrole{n}{robust}\DUrole{o}{=}\DUrole{default_value}{None}}\sphinxparamcomma \sphinxparam{\DUrole{n}{uvtaper}\DUrole{o}{=}\DUrole{default_value}{None}}\sphinxparamcomma \sphinxparam{\DUrole{n}{clearlist}\DUrole{o}{=}\DUrole{default_value}{None}}\sphinxparamcomma \sphinxparam{\DUrole{n}{per\_eb}\DUrole{o}{=}\DUrole{default_value}{None}}\sphinxparamcomma \sphinxparam{\DUrole{n}{per\_session}\DUrole{o}{=}\DUrole{default_value}{None}}\sphinxparamcomma \sphinxparam{\DUrole{n}{calcsb}\DUrole{o}{=}\DUrole{default_value}{None}}\sphinxparamcomma \sphinxparam{\DUrole{n}{datatype}\DUrole{o}{=}\DUrole{default_value}{None}}\sphinxparamcomma \sphinxparam{\DUrole{n}{datacolumn}\DUrole{o}{=}\DUrole{default_value}{None}}\sphinxparamcomma \sphinxparam{\DUrole{n}{parallel}\DUrole{o}{=}\DUrole{default_value}{None}}}
{}
\pysigstopsignatures
\sphinxAtStartPar
Compute list of clean images to be produced

\sphinxAtStartPar
Generate a list of images to be cleaned. By default, the list will include
one image per science target per spw. Calibrator targets can be selected
by setting appropriate values for \sphinxcode{\sphinxupquote{intent}}.

\sphinxAtStartPar
By default, the output image cell size is set to the minimum cell size
consistent with the UV coverage.

\sphinxAtStartPar
By default, the image size in pixels is set to values determined by the
cell size and the primary beam size. If a calibrator is being
imaged (intents ‘PHASE’, ‘BANDPASS’, ‘FLUX’ or ‘AMPLITUDE’) then the
image dimensions are limited to ‘calmaxpix’ pixels.

\sphinxAtStartPar
By default, science target images are cubes and calibrator target images
are mfs. Science target images may be mosaics or single fields.
\begin{quote}\begin{description}
\sphinxlineitem{Parameters}\begin{itemize}
\item {} 
\sphinxAtStartPar
\sphinxstylestrong{vis} \textendash{} The list of input MeasurementSets. Defaults to the list of MeasurementSets specified in the h\_init or hif\_importdata task.
“”: use all MeasurementSets in the context
Examples: ‘ngc5921.ms’, {[}‘ngc5921a.ms’, ngc5921b.ms’, ‘ngc5921c.ms’{]}

\item {} 
\sphinxAtStartPar
\sphinxstylestrong{imagename} \textendash{} Prefix for output image names, “” for automatic.

\item {} 
\sphinxAtStartPar
\sphinxstylestrong{intent} \textendash{} Select intents for which associated fields will be imaged. Possible choices are PHASE, BANDPASS, AMPLITUDE, CHECK and
TARGET or combinations thereof.
Examples: ‘PHASE,BANDPASS’, ‘TARGET’

\item {} 
\sphinxAtStartPar
\sphinxstylestrong{field} \textendash{} Select fields to image. Use field name(s) NOT id(s). Mosaics are assumed to have common source / field names.  If intent is
specified only fields with data matching the intent will be
selected. The fields will be selected from MeasurementSets in
“vis”.
“” Fields matching intent, one image per target source.

\item {} 
\sphinxAtStartPar
\sphinxstylestrong{spw} \textendash{} Select spectral windows to image. “”: Images will be computed for all science spectral windows.

\item {} 
\sphinxAtStartPar
\sphinxstylestrong{contfile} \textendash{} Name of file with frequency ranges to use for continuum images.

\item {} 
\sphinxAtStartPar
\sphinxstylestrong{linesfile} \textendash{} Name of file with line frequency ranges to exclude for continuum images.

\item {} 
\sphinxAtStartPar
\sphinxstylestrong{uvrange} \textendash{} Select a set of uv ranges to image. “”: All uv data is included
Examples: ‘0\textasciitilde{}1000klambda’, {[}‘0\textasciitilde{}100klambda’, 100\textasciitilde{}1000klambda{]}

\item {} 
\sphinxAtStartPar
\sphinxstylestrong{specmode} \textendash{} Frequency imaging mode, ‘mfs’, ‘cont’, ‘cube’, ‘repBW’. ‘’ defaults to ‘cube’ if \sphinxcode{\sphinxupquote{intent}} parameter includes ‘TARGET’
otherwise ‘mfs’.
specmode=’mfs’ produce one image per source and spw
specmode=’cont’ produce one image per source and aggregate
over all specified spws
specmode=’cube’ produce an LSRK frequency cube, channels are
specified in frequency
specmode=’repBW’ produce an LSRK frequency cube at
representative channel width

\item {} 
\sphinxAtStartPar
\sphinxstylestrong{outframe} \textendash{} velocity frame of output image (LSRK, ‘’ for automatic) (not implemented)

\item {} 
\sphinxAtStartPar
\sphinxstylestrong{hm\_imsize} \textendash{} Image X and Y size in pixels or PB level for single fields. The explicit sizes must be even and divisible by 2,3,5,7 only.
The default values are derived as follows:
1. Determine phase center and spread of field centers
around it.
2. Set the size of the image to cover the spread of field
centers plus a border of width 0.75 * beam radius, to
first null.
3. Divide X and Y extents by cell size to arrive at the
number of pixels required.
The PB level setting for single fields leads to an imsize
extending to the specified level plus 5\% padding in all
directions.
Examples: ‘0.3pb’, {[}120, 120{]}

\item {} 
\sphinxAtStartPar
\sphinxstylestrong{hm\_cell} \textendash{} Image X and Y cell sizes. “” computes the cell size based on the UV coverage of all the fields to be imaged and uses a
5 pix per beam sampling.
The pix per beam specification (‘\textless{}number\textgreater{}ppb’) uses the above
default cell size (‘5ppb’) and scales it accordingly.
The cells can also be specified as explicit measures.
Examples: ‘3ppb’, {[}‘0.5arcsec’, ‘0.5arcsec’{]}

\item {} 
\sphinxAtStartPar
\sphinxstylestrong{calmaxpix} \textendash{} Maximum image X or Y size in pixels if a calibrator is being imaged (‘PHASE’, ‘BANDPASS’, ‘AMPLITUDE’ or ‘FLUX’ intent).

\item {} 
\sphinxAtStartPar
\sphinxstylestrong{phasecenter} \textendash{} Direction measure or field id of the image center. The default phase center is set to the mean of the field
directions of all fields that are to be image together.
Examples: ‘J2000 19h30m00 \sphinxhyphen{}40d00m00’, 0

\item {} 
\sphinxAtStartPar
\sphinxstylestrong{nchan} \textendash{} Total number of channels in the output image(s) \sphinxhyphen{}1 selects enough channels to cover the data selected by
spw consistent with start and width.

\item {} 
\sphinxAtStartPar
\sphinxstylestrong{start} \textendash{} Start of image frequency axis as frequency or velocity. “” selects start frequency automatically.

\item {} 
\sphinxAtStartPar
\sphinxstylestrong{width} \textendash{} Output channel width. Difference in frequency between 2 selected channels for
frequency mode images.
‘pilotimage’ for 15 MHz / 8 channel heuristic

\item {} 
\sphinxAtStartPar
\sphinxstylestrong{nbins} \textendash{} Channel binning factors for each spw. Format: ‘spw1:nb1,spw2:nb2,…’ with optional wildcards: ‘\sphinxstyleemphasis{:nb’
Examples: ‘9:2,11:4,13:2,15:8’, ‘}:2’

\item {} 
\sphinxAtStartPar
\sphinxstylestrong{robust} \textendash{} Briggs robustness parameter Values range from \sphinxhyphen{}2.0 (uniform) to 2.0 (natural)

\item {} 
\sphinxAtStartPar
\sphinxstylestrong{uvtaper} \textendash{} uv\sphinxhyphen{}taper on outer baselines

\item {} 
\sphinxAtStartPar
\sphinxstylestrong{clearlist} \textendash{} Clear any existing target list

\item {} 
\sphinxAtStartPar
\sphinxstylestrong{per\_eb} \textendash{} Make an image target per EB

\item {} 
\sphinxAtStartPar
\sphinxstylestrong{per\_session} \textendash{} Make an image target per session

\item {} 
\sphinxAtStartPar
\sphinxstylestrong{calcsb} \textendash{} Force (re\sphinxhyphen{})calculation of sensitivities and beams

\item {} 
\sphinxAtStartPar
\sphinxstylestrong{datatype} \textendash{} Data type(s) to image. The default ‘’ selects the best available data type (e.g. selfcal over regcal) with
an automatic fallback to the next available data type.
With the \sphinxcode{\sphinxupquote{datatype}} parameter one can force the use of only
given data type(s) without a fallback. The data type(s) are
specified as comma separated string of keywords. Accepted
values are the standard data types such as
‘REGCAL\_CONTLINE\_ALL’, ‘REGCAL\_CONTLINE\_SCIENCE’,
‘SELFCAL\_CONTLINE\_SCIENCE’, ‘REGCAL\_LINE\_SCIENCE’,
‘SELFCAL\_LINE\_SCIENCE’. The shortcuts ‘regcal’ and ‘selfcal’
are also accepted. They are expanded into the full data types
using the \sphinxcode{\sphinxupquote{specmode}} parameter and the available data types for
the given MSes. In addition the strings ‘best’ and ‘all’ are
accepted, where ‘best’ means the above mentioned automatic
mode and ‘all’ means all available data types for a given
specmode. The data type strings are case insensitive.
Examples: ‘selfcal’, ‘regcal, ‘selfcal,regcal’,
‘REGCAL\_LINE\_SCIENCE,selfcal\_line\_science’

\item {} 
\sphinxAtStartPar
\sphinxstylestrong{datacolumn} \textendash{} Data column to image. Only to be used for manual overriding when the automatic choice by data type is not appropriate.

\item {} 
\sphinxAtStartPar
\sphinxstylestrong{parallel} \textendash{} Use MPI cluster where possible

\end{itemize}

\sphinxlineitem{Returns}
\sphinxAtStartPar
The results object for the pipeline task is returned.

\end{description}\end{quote}
\subsubsection*{Examples}

\sphinxAtStartPar
1. Make a list of science target images to be cleaned, one image per science
spw.

\begin{sphinxVerbatim}[commandchars=\\\{\}]
\PYG{g+gp}{\PYGZgt{}\PYGZgt{}\PYGZgt{} }\PYG{n}{hif\PYGZus{}makeimlist}\PYG{p}{(}\PYG{p}{)}
\end{sphinxVerbatim}

\sphinxAtStartPar
2. Make a list of PHASE and BANDPASS calibrator targets to be imaged,
one image per science spw.

\begin{sphinxVerbatim}[commandchars=\\\{\}]
\PYG{g+gp}{\PYGZgt{}\PYGZgt{}\PYGZgt{} }\PYG{n}{hif\PYGZus{}makeimlist}\PYG{p}{(}\PYG{n}{intent}\PYG{o}{=}\PYG{l+s+s1}{\PYGZsq{}}\PYG{l+s+s1}{PHASE,BANDPASS}\PYG{l+s+s1}{\PYGZsq{}}\PYG{p}{)}
\end{sphinxVerbatim}

\sphinxAtStartPar
3. Make a list of PHASE calibrator images observed in spw 1, images limited to
50 pixels on a side.

\begin{sphinxVerbatim}[commandchars=\\\{\}]
\PYG{g+gp}{\PYGZgt{}\PYGZgt{}\PYGZgt{} }\PYG{n}{hif\PYGZus{}makeimlist}\PYG{p}{(}\PYG{n}{intent}\PYG{o}{=}\PYG{l+s+s1}{\PYGZsq{}}\PYG{l+s+s1}{PHASE}\PYG{l+s+s1}{\PYGZsq{}}\PYG{p}{,}\PYG{n}{spw}\PYG{o}{=}\PYG{l+s+s1}{\PYGZsq{}}\PYG{l+s+s1}{1}\PYG{l+s+s1}{\PYGZsq{}}\PYG{p}{,}\PYG{n}{calmaxpix}\PYG{o}{=}\PYG{l+m+mi}{50}\PYG{p}{)}
\end{sphinxVerbatim}

\end{fulllineitems}


\sphinxstepscope


\section{pipeline.hif.cli.hif\_makermsimages}
\label{\detokenize{_autosummary/pipeline.hif.cli.hif_makermsimages:pipeline-hif-cli-hif-makermsimages}}\label{\detokenize{_autosummary/pipeline.hif.cli.hif_makermsimages::doc}}\index{hif\_makermsimages() (in module pipeline.hif.cli)@\spxentry{hif\_makermsimages()}\spxextra{in module pipeline.hif.cli}}

\begin{fulllineitems}
\phantomsection\label{\detokenize{_autosummary/pipeline.hif.cli.hif_makermsimages:pipeline.hif.cli.hif_makermsimages}}
\pysigstartsignatures
\pysiglinewithargsret
{\sphinxbfcode{\sphinxupquote{hif\_makermsimages}}}
{\sphinxparam{\DUrole{n}{vis}\DUrole{o}{=}\DUrole{default_value}{None}}}
{}
\pysigstopsignatures
\sphinxAtStartPar
Create RMS images for VLASS data.

\sphinxAtStartPar
Create RMS images for VLASS data.
\begin{quote}\begin{description}
\sphinxlineitem{Parameters}
\sphinxAtStartPar
\sphinxstylestrong{vis} \textendash{} List of visibility data files. These may be ASDMs, tar files of ASDMs, MSs, or tar files of MSs, If ASDM files are specified, they will be
converted  to MS format.
example: vis={[}‘X227.ms’, ‘asdms.tar.gz’{]}

\sphinxlineitem{Returns}
\sphinxAtStartPar
The results object for the pipeline task is returned.

\end{description}\end{quote}
\subsubsection*{Examples}
\begin{enumerate}
\sphinxsetlistlabels{\arabic}{enumi}{enumii}{}{.}%
\item {} 
\sphinxAtStartPar
Basic makermsimages task

\end{enumerate}

\begin{sphinxVerbatim}[commandchars=\\\{\}]
\PYG{g+gp}{\PYGZgt{}\PYGZgt{}\PYGZgt{} }\PYG{n}{hif\PYGZus{}makermsimages}\PYG{p}{(}\PYG{p}{)}
\end{sphinxVerbatim}

\end{fulllineitems}


\sphinxstepscope


\section{pipeline.hif.cli.hif\_mstransform}
\label{\detokenize{_autosummary/pipeline.hif.cli.hif_mstransform:pipeline-hif-cli-hif-mstransform}}\label{\detokenize{_autosummary/pipeline.hif.cli.hif_mstransform::doc}}\index{hif\_mstransform() (in module pipeline.hif.cli)@\spxentry{hif\_mstransform()}\spxextra{in module pipeline.hif.cli}}

\begin{fulllineitems}
\phantomsection\label{\detokenize{_autosummary/pipeline.hif.cli.hif_mstransform:pipeline.hif.cli.hif_mstransform}}
\pysigstartsignatures
\pysiglinewithargsret
{\sphinxbfcode{\sphinxupquote{hif\_mstransform}}}
{\sphinxparam{\DUrole{n}{vis}\DUrole{o}{=}\DUrole{default_value}{None}}\sphinxparamcomma \sphinxparam{\DUrole{n}{outputvis}\DUrole{o}{=}\DUrole{default_value}{None}}\sphinxparamcomma \sphinxparam{\DUrole{n}{field}\DUrole{o}{=}\DUrole{default_value}{None}}\sphinxparamcomma \sphinxparam{\DUrole{n}{intent}\DUrole{o}{=}\DUrole{default_value}{None}}\sphinxparamcomma \sphinxparam{\DUrole{n}{spw}\DUrole{o}{=}\DUrole{default_value}{None}}\sphinxparamcomma \sphinxparam{\DUrole{n}{chanbin}\DUrole{o}{=}\DUrole{default_value}{None}}\sphinxparamcomma \sphinxparam{\DUrole{n}{timebin}\DUrole{o}{=}\DUrole{default_value}{None}}}
{}
\pysigstopsignatures
\sphinxAtStartPar
Create new MeasurementSets for science target imaging

\sphinxAtStartPar
Create new MeasurementSets for imaging from the corrected column of the input
MeasurementSet via a single call to mstransform with all data selection parameters.
By default, all science target data is copied to the new MS. The
new MeasurementSet is not re\sphinxhyphen{}indexed to the selected data and the new MS will
have the same source, field, and spw names and ids as it does in the parent MS.
\begin{quote}\begin{description}
\sphinxlineitem{Parameters}\begin{itemize}
\item {} 
\sphinxAtStartPar
\sphinxstylestrong{vis} \textendash{} The list of input MeasurementSets. Defaults to the list of MeasurementSets specified in the h\_init or hif\_importdata task.
‘’: use all MeasurementSets in the context
Examples: ‘ngc5921.ms’, {[}‘ngc5921a.ms’, ngc5921b.ms’, ‘ngc5921c.ms’{]}

\item {} 
\sphinxAtStartPar
\sphinxstylestrong{outputvis} \textendash{} The list of output transformed MeasurementSets to be used for imaging. The output list must be the same length as the input
list. The default output name defaults to
\textless{}msrootname\textgreater{}\_targets.ms
Examples: ‘ngc5921.ms’,
{[}‘ngc5921a.ms’, ngc5921b.ms’, ‘ngc5921c.ms’{]}

\item {} 
\sphinxAtStartPar
\sphinxstylestrong{field} \textendash{} Select fields name(s) or id(s) to transform. Only fields with data matching the intent will be selected.
Examples: ‘3C279’, ‘Centaurus*’, ‘3C279,J1427\sphinxhyphen{}421’

\item {} 
\sphinxAtStartPar
\sphinxstylestrong{intent} \textendash{} Select intents for which associated fields will be imaged. By default only TARGET data is selected.
Examples: ‘PHASE,BANDPASS’

\item {} 
\sphinxAtStartPar
\sphinxstylestrong{spw} \textendash{} Select spectral window/channels to image. By default all science spws for which the specified intent is valid are
selected.

\item {} 
\sphinxAtStartPar
\sphinxstylestrong{chanbin} \textendash{} Width (bin) of input channels to average to form an output channel. If chanbin \textgreater{} 1 then chanaverage is automatically
switched to True.

\item {} 
\sphinxAtStartPar
\sphinxstylestrong{timebin} \textendash{} Bin width for time averaging. If timebin \textgreater{} 0s then timeaverage is automatically switched to True.

\end{itemize}

\sphinxlineitem{Returns}
\sphinxAtStartPar
The results object for the pipeline task is returned.

\end{description}\end{quote}
\subsubsection*{Examples}
\begin{enumerate}
\sphinxsetlistlabels{\arabic}{enumi}{enumii}{}{.}%
\item {} 
\sphinxAtStartPar
Create a science target MS from the corrected column in the input MS.

\end{enumerate}

\begin{sphinxVerbatim}[commandchars=\\\{\}]
\PYG{g+gp}{\PYGZgt{}\PYGZgt{}\PYGZgt{} }\PYG{n}{hif\PYGZus{}mstransform}\PYG{p}{(}\PYG{p}{)}
\end{sphinxVerbatim}

\sphinxAtStartPar
2. Make a phase and bandpass calibrator targets MS from the corrected
column in the input MS.

\begin{sphinxVerbatim}[commandchars=\\\{\}]
\PYG{g+gp}{\PYGZgt{}\PYGZgt{}\PYGZgt{} }\PYG{n}{hif\PYGZus{}mstransform}\PYG{p}{(}\PYG{n}{intent}\PYG{o}{=}\PYG{l+s+s1}{\PYGZsq{}}\PYG{l+s+s1}{PHASE,BANDPASS}\PYG{l+s+s1}{\PYGZsq{}}\PYG{p}{)}
\end{sphinxVerbatim}

\end{fulllineitems}


\sphinxstepscope


\section{pipeline.hif.cli.hif\_rawflagchans}
\label{\detokenize{_autosummary/pipeline.hif.cli.hif_rawflagchans:pipeline-hif-cli-hif-rawflagchans}}\label{\detokenize{_autosummary/pipeline.hif.cli.hif_rawflagchans::doc}}\index{hif\_rawflagchans() (in module pipeline.hif.cli)@\spxentry{hif\_rawflagchans()}\spxextra{in module pipeline.hif.cli}}

\begin{fulllineitems}
\phantomsection\label{\detokenize{_autosummary/pipeline.hif.cli.hif_rawflagchans:pipeline.hif.cli.hif_rawflagchans}}
\pysigstartsignatures
\pysiglinewithargsret
{\sphinxbfcode{\sphinxupquote{hif\_rawflagchans}}}
{\sphinxparam{\DUrole{n}{vis}\DUrole{o}{=}\DUrole{default_value}{None}}\sphinxparamcomma \sphinxparam{\DUrole{n}{spw}\DUrole{o}{=}\DUrole{default_value}{None}}\sphinxparamcomma \sphinxparam{\DUrole{n}{intent}\DUrole{o}{=}\DUrole{default_value}{None}}\sphinxparamcomma \sphinxparam{\DUrole{n}{flag\_hilo}\DUrole{o}{=}\DUrole{default_value}{None}}\sphinxparamcomma \sphinxparam{\DUrole{n}{fhl\_limit}\DUrole{o}{=}\DUrole{default_value}{None}}\sphinxparamcomma \sphinxparam{\DUrole{n}{fhl\_minsample}\DUrole{o}{=}\DUrole{default_value}{None}}\sphinxparamcomma \sphinxparam{\DUrole{n}{flag\_bad\_quadrant}\DUrole{o}{=}\DUrole{default_value}{None}}\sphinxparamcomma \sphinxparam{\DUrole{n}{fbq\_hilo\_limit}\DUrole{o}{=}\DUrole{default_value}{None}}\sphinxparamcomma \sphinxparam{\DUrole{n}{fbq\_antenna\_frac\_limit}\DUrole{o}{=}\DUrole{default_value}{None}}\sphinxparamcomma \sphinxparam{\DUrole{n}{fbq\_baseline\_frac\_limit}\DUrole{o}{=}\DUrole{default_value}{None}}}
{}
\pysigstopsignatures
\sphinxAtStartPar
Flag deviant baseline/channels in raw data

\sphinxAtStartPar
hif\_rawflagchans flags deviant baseline/channels in the raw data.

\sphinxAtStartPar
The flagging views used are derived from the raw data for the specified
intent \sphinxhyphen{} default is BANDPASS.

\sphinxAtStartPar
Bad baseline/channels are flagged for all intents, not just the
one that is the basis of the flagging views.

\sphinxAtStartPar
For each spectral window the flagging view is a 2d image with axes
‘channel’ and ‘baseline’. The pixel for each channel, baseline is the
time average of the underlying unflagged raw data.

\sphinxAtStartPar
The baseline axis is labeled by numbers of form id1.id2 where id1 and id2
are the IDs of the baseline antennas. Both id1 and id2 run over all
antenna IDs in the observation. This means that each baseline is shown
twice but has the benefit that ‘bad’ antennas are easily identified by
eye.

\sphinxAtStartPar
Three flagging methods are available:

\sphinxAtStartPar
If parameter \sphinxcode{\sphinxupquote{flag\_hilo}} is set True then outliers from the median
of each flagging view will be flagged.

\sphinxAtStartPar
If parameter \sphinxcode{\sphinxupquote{flag\_bad\_quadrant}} is set True then a simple 2 part
test is used to check for bad antenna quadrants and/or bad baseline
quadrants. Here a ‘quadrant’ is defined simply as one quarter of the
channel axis. The first part of the test is to note as ‘suspect’ those
points further from the view median than \sphinxcode{\sphinxupquote{fbq\_hilo\_limit}} * MAD.
The second part is to flag entire antenna/quadrants if their
fraction of suspect points exceeds \sphinxcode{\sphinxupquote{fbq\_antenna\_frac\_limit}}.
Failing that, entire baseline/quadrants may be flagged if their
fraction of suspect points exceeds \sphinxcode{\sphinxupquote{fbq\_baseline\_frac\_limit}}.
Suspect points are not flagged unless as part of a bad antenna or
baseline quadrant.
\begin{quote}\begin{description}
\sphinxlineitem{Parameters}\begin{itemize}
\item {} 
\sphinxAtStartPar
\sphinxstylestrong{vis} \textendash{} List of input MeasurementSets. default: {[}{]} \sphinxhyphen{} Use the MeasurementSets currently known to the pipeline
context.

\item {} 
\sphinxAtStartPar
\sphinxstylestrong{spw} \textendash{} The list of spectral windows and channels to which the calibration will be applied. Defaults to all science windows in the pipeline
context.
example: spw=’17’, spw=’11, 15’

\item {} 
\sphinxAtStartPar
\sphinxstylestrong{intent} \textendash{} A string containing the list of intents to be checked for antennas with deviant gains. The default is blank, which causes the task to select
the ‘BANDPASS’ intent.
example: intent=’\sphinxtitleref{*BANDPASS*}’

\item {} 
\sphinxAtStartPar
\sphinxstylestrong{flag\_hilo} \textendash{} True to flag channel/baseline data further from the view median than fhl\_limit * MAD.

\item {} 
\sphinxAtStartPar
\sphinxstylestrong{fhl\_limit} \textendash{} If flag\_hilo is True then flag channel/baseline data further from the view median than fhl\_limit * MAD.

\item {} 
\sphinxAtStartPar
\sphinxstylestrong{fhl\_minsample} \textendash{} Do no flagging if the view median and MAD are derived from fewer than fhl\_minsample view pixels.

\item {} 
\sphinxAtStartPar
\sphinxstylestrong{flag\_bad\_quadrant} \textendash{} True to search for and flag bad antenna quadrants and baseline quadrants. Here a /’quadrant/’ is one
quarter of the channel axis.

\item {} 
\sphinxAtStartPar
\sphinxstylestrong{fbq\_hilo\_limit} \textendash{} If flag\_bad\_quadrant is True then channel/baselines further from the view median than fbq\_hilo\_limit * MAD will be noted as
‘suspect’. If there are enough of them to indicate that an antenna or
baseline quadrant is bad then all channel/baselines in that quadrant will
be flagged.

\item {} 
\sphinxAtStartPar
\sphinxstylestrong{fbq\_antenna\_frac\_limit} \textendash{} If flag\_bad\_quadrant is True and the fraction of suspect channel/baselines in a particular antenna/quadrant exceeds
fbq\_antenna\_frac\_limit then all data for that antenna/quadrant will
be flagged.

\item {} 
\sphinxAtStartPar
\sphinxstylestrong{fbq\_baseline\_frac\_limit} \textendash{} If flag\_bad\_quadrant is True and the fraction of suspect channel/baselines in a particular baseline/quadrant exceeds
fbq\_baseline\_frac\_limit then all data for that baseline/quadrant will
be flagged.

\end{itemize}

\sphinxlineitem{Returns}
\sphinxAtStartPar
The results object for the pipeline task is returned.

\end{description}\end{quote}
\subsubsection*{Examples}
\begin{enumerate}
\sphinxsetlistlabels{\arabic}{enumi}{enumii}{}{.}%
\item {} 
\sphinxAtStartPar
Flag bad quadrants and wild outliers, default method:

\end{enumerate}

\begin{sphinxVerbatim}[commandchars=\\\{\}]
\PYG{g+gp}{\PYGZgt{}\PYGZgt{}\PYGZgt{} }\PYG{n}{hif\PYGZus{}rawflagchans}\PYG{p}{(}\PYG{p}{)}
\end{sphinxVerbatim}

\sphinxAtStartPar
equivalent to:

\begin{sphinxVerbatim}[commandchars=\\\{\}]
\PYG{g+gp}{\PYGZgt{}\PYGZgt{}\PYGZgt{} }\PYG{n}{hif\PYGZus{}rawflagchans}\PYG{p}{(}\PYG{n}{flag\PYGZus{}hilo}\PYG{o}{=}\PYG{k+kc}{True}\PYG{p}{,} \PYG{n}{fhl\PYGZus{}limit}\PYG{o}{=}\PYG{l+m+mi}{20}\PYG{p}{,} \PYG{n}{flag\PYGZus{}bad\PYGZus{}quadrant}\PYG{o}{=}\PYG{k+kc}{True}\PYG{p}{,} \PYG{n}{fbq\PYGZus{}hilo\PYGZus{}limit}\PYG{o}{=}\PYG{l+m+mi}{8}\PYG{p}{,}
\PYG{g+gp}{... }                 \PYG{n}{fbq\PYGZus{}antenna\PYGZus{}frac\PYGZus{}limit}\PYG{o}{=}\PYG{l+m+mf}{0.2}\PYG{p}{,} \PYG{n}{fbq\PYGZus{}baseline\PYGZus{}frac\PYGZus{}limit}\PYG{o}{=}\PYG{l+m+mf}{1.0}\PYG{p}{)}
\end{sphinxVerbatim}

\end{fulllineitems}


\sphinxstepscope


\section{pipeline.hif.cli.hif\_refant}
\label{\detokenize{_autosummary/pipeline.hif.cli.hif_refant:pipeline-hif-cli-hif-refant}}\label{\detokenize{_autosummary/pipeline.hif.cli.hif_refant::doc}}\index{hif\_refant() (in module pipeline.hif.cli)@\spxentry{hif\_refant()}\spxextra{in module pipeline.hif.cli}}

\begin{fulllineitems}
\phantomsection\label{\detokenize{_autosummary/pipeline.hif.cli.hif_refant:pipeline.hif.cli.hif_refant}}
\pysigstartsignatures
\pysiglinewithargsret
{\sphinxbfcode{\sphinxupquote{hif\_refant}}}
{\sphinxparam{\DUrole{n}{vis}\DUrole{o}{=}\DUrole{default_value}{None}}\sphinxparamcomma \sphinxparam{\DUrole{n}{field}\DUrole{o}{=}\DUrole{default_value}{None}}\sphinxparamcomma \sphinxparam{\DUrole{n}{spw}\DUrole{o}{=}\DUrole{default_value}{None}}\sphinxparamcomma \sphinxparam{\DUrole{n}{intent}\DUrole{o}{=}\DUrole{default_value}{None}}\sphinxparamcomma \sphinxparam{\DUrole{n}{hm\_refant}\DUrole{o}{=}\DUrole{default_value}{None}}\sphinxparamcomma \sphinxparam{\DUrole{n}{refant}\DUrole{o}{=}\DUrole{default_value}{None}}\sphinxparamcomma \sphinxparam{\DUrole{n}{geometry}\DUrole{o}{=}\DUrole{default_value}{None}}\sphinxparamcomma \sphinxparam{\DUrole{n}{flagging}\DUrole{o}{=}\DUrole{default_value}{None}}\sphinxparamcomma \sphinxparam{\DUrole{n}{parallel}\DUrole{o}{=}\DUrole{default_value}{None}}\sphinxparamcomma \sphinxparam{\DUrole{n}{refantignore}\DUrole{o}{=}\DUrole{default_value}{None}}}
{}
\pysigstopsignatures
\sphinxAtStartPar
Select the best reference antennas

\sphinxAtStartPar
The hif\_refant task selects a list of reference antennas and stores them
in the pipeline context in priority order.

\sphinxAtStartPar
The priority order is determined by a weighted combination of scores derived
by the antenna selection heuristics. In manual mode the reference antennas
can be set by hand.
\begin{quote}\begin{description}
\sphinxlineitem{Parameters}\begin{itemize}
\item {} 
\sphinxAtStartPar
\sphinxstylestrong{vis} \textendash{} The list of input MeasurementSets. Defaults to the list of MeasurementSets in the pipeline context.
example: {[}‘M31.ms’{]}

\item {} 
\sphinxAtStartPar
\sphinxstylestrong{field} \textendash{} The comma delimited list of field names or field ids for which flagging scores are computed if hm\_refant=’automatic’ and  flagging = True
example: ‘’ (Default to fields with the specified intents), ‘3C279’, ‘3C279,M82’

\item {} 
\sphinxAtStartPar
\sphinxstylestrong{spw} \textendash{} A string containing the comma delimited list of spectral window ids for which flagging scores are computed if hm\_refant=’automatic’ and  flagging = True.
example: ‘’ (all spws observed with the specified intents), ‘11,13,15,17’

\item {} 
\sphinxAtStartPar
\sphinxstylestrong{intent} \textendash{} A string containing a comma delimited list of intents against which the selected fields are matched. Defaults to all supported
intents.
example: ‘BANDPASS’, ‘AMPLITUDE,BANDPASS,PHASE,POLARIZATION’

\item {} 
\sphinxAtStartPar
\sphinxstylestrong{hm\_refant} \textendash{} The heuristics method or mode for selection the reference antenna. The options are ‘manual’ and ‘automatic. In manual
mode a user supplied reference antenna refant is supplied.
In ‘automatic’ mode the antennas are selected automatically.

\item {} 
\sphinxAtStartPar
\sphinxstylestrong{refant} \textendash{} The user supplied reference antenna for hm\_refant=’manual’. If no antenna list is supplied an empty list is returned.
example: ‘DV05’

\item {} 
\sphinxAtStartPar
\sphinxstylestrong{geometry} \textendash{} Score antenna by proximity to the center of the array. This option is quick as only the ANTENNA table must be read.
Parameter is available when {\color{red}\bfseries{}\textasciigrave{}\textasciigrave{}}hm\_refant\textasciigrave{}\textasciigrave{}=’automatic’.

\item {} 
\sphinxAtStartPar
\sphinxstylestrong{flagging} \textendash{} Score antennas by percentage of unflagged data.  This option requires computing flagging statistics.
Parameter is available when {\color{red}\bfseries{}\textasciigrave{}\textasciigrave{}}hm\_refant\textasciigrave{}\textasciigrave{}=’automatic’.

\item {} 
\sphinxAtStartPar
\sphinxstylestrong{parallel} \textendash{} Execute using CASA HPC functionality, if available. options: ‘automatic’, ‘true’, ‘false’, True, False
default: None (equivalent to False)

\item {} 
\sphinxAtStartPar
\sphinxstylestrong{refantignore} \textendash{} string list to be ignored as reference antennas. example:  refantignore=’ea02,ea03’

\end{itemize}

\sphinxlineitem{Returns}
\sphinxAtStartPar
The results object for the pipeline task is returned.

\end{description}\end{quote}
\subsubsection*{Examples}
\begin{enumerate}
\sphinxsetlistlabels{\arabic}{enumi}{enumii}{}{.}%
\item {} 
\sphinxAtStartPar
Compute the references antennas to be used for bandpass and gain calibration.

\end{enumerate}

\begin{sphinxVerbatim}[commandchars=\\\{\}]
\PYG{g+gp}{\PYGZgt{}\PYGZgt{}\PYGZgt{} }\PYG{n}{hif\PYGZus{}refant}\PYG{p}{(}\PYG{p}{)}
\end{sphinxVerbatim}

\end{fulllineitems}


\sphinxstepscope


\section{pipeline.hif.cli.hif\_selfcal}
\label{\detokenize{_autosummary/pipeline.hif.cli.hif_selfcal:pipeline-hif-cli-hif-selfcal}}\label{\detokenize{_autosummary/pipeline.hif.cli.hif_selfcal::doc}}\index{hif\_selfcal() (in module pipeline.hif.cli)@\spxentry{hif\_selfcal()}\spxextra{in module pipeline.hif.cli}}

\begin{fulllineitems}
\phantomsection\label{\detokenize{_autosummary/pipeline.hif.cli.hif_selfcal:pipeline.hif.cli.hif_selfcal}}
\pysigstartsignatures
\pysiglinewithargsret
{\sphinxbfcode{\sphinxupquote{hif\_selfcal}}}
{\sphinxparam{\DUrole{n}{vis}\DUrole{o}{=}\DUrole{default_value}{None}}\sphinxparamcomma \sphinxparam{\DUrole{n}{field}\DUrole{o}{=}\DUrole{default_value}{None}}\sphinxparamcomma \sphinxparam{\DUrole{n}{spw}\DUrole{o}{=}\DUrole{default_value}{None}}\sphinxparamcomma \sphinxparam{\DUrole{n}{contfile}\DUrole{o}{=}\DUrole{default_value}{None}}\sphinxparamcomma \sphinxparam{\DUrole{n}{apply}\DUrole{o}{=}\DUrole{default_value}{None}}\sphinxparamcomma \sphinxparam{\DUrole{n}{recal}\DUrole{o}{=}\DUrole{default_value}{None}}\sphinxparamcomma \sphinxparam{\DUrole{n}{refantignore}\DUrole{o}{=}\DUrole{default_value}{None}}\sphinxparamcomma \sphinxparam{\DUrole{n}{restore\_resources}\DUrole{o}{=}\DUrole{default_value}{None}}\sphinxparamcomma \sphinxparam{\DUrole{n}{n\_solints}\DUrole{o}{=}\DUrole{default_value}{None}}\sphinxparamcomma \sphinxparam{\DUrole{n}{amplitude\_selfcal}\DUrole{o}{=}\DUrole{default_value}{None}}\sphinxparamcomma \sphinxparam{\DUrole{n}{gaincal\_minsnr}\DUrole{o}{=}\DUrole{default_value}{None}}\sphinxparamcomma \sphinxparam{\DUrole{n}{minsnr\_to\_proceed}\DUrole{o}{=}\DUrole{default_value}{None}}\sphinxparamcomma \sphinxparam{\DUrole{n}{delta\_beam\_thresh}\DUrole{o}{=}\DUrole{default_value}{None}}\sphinxparamcomma \sphinxparam{\DUrole{n}{apply\_cal\_mode\_default}\DUrole{o}{=}\DUrole{default_value}{None}}\sphinxparamcomma \sphinxparam{\DUrole{n}{rel\_thresh\_scaling}\DUrole{o}{=}\DUrole{default_value}{None}}\sphinxparamcomma \sphinxparam{\DUrole{n}{dividing\_factor}\DUrole{o}{=}\DUrole{default_value}{None}}\sphinxparamcomma \sphinxparam{\DUrole{n}{check\_all\_spws}\DUrole{o}{=}\DUrole{default_value}{None}}\sphinxparamcomma \sphinxparam{\DUrole{n}{inf\_EB\_gaincal\_combine}\DUrole{o}{=}\DUrole{default_value}{None}}\sphinxparamcomma \sphinxparam{\DUrole{n}{parallel}\DUrole{o}{=}\DUrole{default_value}{None}}}
{}
\pysigstopsignatures
\sphinxAtStartPar
Determine and apply self\sphinxhyphen{}calibration with the science target data

\sphinxAtStartPar
Determine and apply self\sphinxhyphen{}calibration with the science target data
\begin{quote}\begin{description}
\sphinxlineitem{Parameters}\begin{itemize}
\item {} 
\sphinxAtStartPar
\sphinxstylestrong{vis} \textendash{} The list of input MeasurementSets. Defaults to the list of MeasurementSets specified in the h\_init or hif\_importdata task.
default = “”: use all MeasurementSets in the context

\item {} 
\sphinxAtStartPar
\sphinxstylestrong{field} \textendash{} Select fields to image. Use field name(s) NOT id(s). Mosaics are assumed to have common source / field names.  If intent is
specified only fields with data matching the intent will be
selected. The fields will be selected from MeasurementSets in
“vis”.
default= “” Fields matching intent, one image per target source.

\item {} 
\sphinxAtStartPar
\sphinxstylestrong{spw} \textendash{} Select spectral windows to image. “”: Images will be computed for all science spectral windows.

\item {} 
\sphinxAtStartPar
\sphinxstylestrong{contfile} \textendash{} Name of file to specify line\sphinxhyphen{}free frequency ranges for selfcal continuum imaging. default=”cont.dat”

\item {} 
\sphinxAtStartPar
\sphinxstylestrong{apply} \textendash{} Apply final selfcal solutions back to the input MeasurementSets. default = True

\item {} 
\sphinxAtStartPar
\sphinxstylestrong{recal} \textendash{} Always re\sphinxhyphen{}do self\sphinxhyphen{}calibration even solutions/caltables are found in the Pipeline context or json restore file.
default = False

\item {} 
\sphinxAtStartPar
\sphinxstylestrong{refantignore} \textendash{} string list to be ignored as reference antennas. example:  refantignore=’ea02,ea03’

\item {} 
\sphinxAtStartPar
\sphinxstylestrong{restore\_resources} \textendash{} Path to the restore resources from a standard run of hif\_selfcal. hif\_selfcal will automatically do an exhaustive search to lookup/extract/verify
the selfcal restore resources, i.e., selfcal.json and all selfcal\sphinxhyphen{}caltable referred
in selfcal.json, starting from working/, to products/ and rawdata/.
If restore\_resources is specified, this file path will be evaluated first
before the pre\sphinxhyphen{}defined exhaustive search list.
The value can be the file path of {\color{red}\bfseries{}*}auxproducts.tgz file or {\color{red}\bfseries{}*}selfcal.json file.

\item {} 
\sphinxAtStartPar
\sphinxstylestrong{n\_solints} \textendash{} number of solution intervals to attempt for self\sphinxhyphen{}calibration. default: 4

\item {} 
\sphinxAtStartPar
\sphinxstylestrong{amplitude\_selfcal} \textendash{} Attempt amplitude self\sphinxhyphen{}calibration following phase\sphinxhyphen{}only self\sphinxhyphen{}calibration; if median time between scans of a given target is \textless{} 150s,
solution intervals of 300s and inf will be attempted, otherwise just
inf will be attempted.
default = False

\item {} 
\sphinxAtStartPar
\sphinxstylestrong{gaincal\_minsnr} \textendash{} Minimum S/N for a solution to not be flagged by gaincal. default = 2.0

\item {} 
\sphinxAtStartPar
\sphinxstylestrong{minsnr\_to\_proceed} \textendash{} Minimum estimated S/N on a per antenna basis to attempt self\sphinxhyphen{}calibration of a source.
default = 3.0

\item {} 
\sphinxAtStartPar
\sphinxstylestrong{delta\_beam\_thresh} \textendash{} Allowed fractional change in beam size for selfcalibration to accept results of a solution interval.
default = 0.05

\item {} 
\sphinxAtStartPar
\sphinxstylestrong{apply\_cal\_mode\_default} \textendash{} Apply mode to use for applycal task during self\sphinxhyphen{}calibration; same options as applycal.
default = ‘calflag’

\item {} 
\sphinxAtStartPar
\sphinxstylestrong{rel\_thresh\_scaling} \textendash{} Scaling type to determine how clean thresholds per solution interval should be determined going from the starting
clean threshold to 3.0 * RMS for the final solution interval.
default=’log10’, options: ‘linear’, ‘log10’, or ‘loge’ (natural log)

\item {} 
\sphinxAtStartPar
\sphinxstylestrong{dividing\_factor} \textendash{} Scaling factor to determine clean threshold for first self\sphinxhyphen{}calibration solution interval.
Equivalent to (Peak S/N / dividing\_factor) {\color{red}\bfseries{}*}RMS = First clean threshold;
however, if (Peak S/N / dividing\_factor) {\color{red}\bfseries{}*}RMS is \textless{} 5.0; a value of 5.0
is used for the first clean threshold.
default = 40 for \textless{} 8 GHz; 15 for \textgreater{} 8 GHz

\item {} 
\sphinxAtStartPar
\sphinxstylestrong{check\_all\_spws} \textendash{} If True, the S/N of mfs images created on a per\sphinxhyphen{}spectral\sphinxhyphen{}window basis will be compared at the initial stages final self\sphinxhyphen{}calibration.
default=False

\item {} 
\sphinxAtStartPar
\sphinxstylestrong{inf\_EB\_gaincal\_combine} \textendash{} change gain solution combination parameters for the inf\_EB solution interval. if True, the gaincal combine parameter will be set to ‘scan,spw’; if False,
the gaincal combine parameter will be set to ‘scan’.
default=False

\item {} 
\sphinxAtStartPar
\sphinxstylestrong{parallel} \textendash{} Use MPI cluster where possible, default=’automatic’. options: ‘automatic’, ‘true’, ‘false’, True, False

\end{itemize}

\sphinxlineitem{Returns}
\sphinxAtStartPar
The results object for the pipeline task is returned.

\end{description}\end{quote}
\subsubsection*{Examples}
\begin{enumerate}
\sphinxsetlistlabels{\arabic}{enumi}{enumii}{}{.}%
\item {} 
\sphinxAtStartPar
Run self\sphinxhyphen{}calibration and apply solutions to all science targets and spws

\end{enumerate}

\begin{sphinxVerbatim}[commandchars=\\\{\}]
\PYG{g+gp}{\PYGZgt{}\PYGZgt{}\PYGZgt{} }\PYG{n}{hif\PYGZus{}selfcal}\PYG{p}{(}\PYG{p}{)}
\end{sphinxVerbatim}
\begin{enumerate}
\sphinxsetlistlabels{\arabic}{enumi}{enumii}{}{.}%
\setcounter{enumi}{1}
\item {} 
\sphinxAtStartPar
Run self\sphinxhyphen{}calibration and apply solutions to a single science target

\end{enumerate}

\begin{sphinxVerbatim}[commandchars=\\\{\}]
\PYG{g+gp}{\PYGZgt{}\PYGZgt{}\PYGZgt{} }\PYG{n}{hif\PYGZus{}selfcal}\PYG{p}{(}\PYG{n}{field}\PYG{o}{=}\PYG{l+s+s2}{\PYGZdq{}}\PYG{l+s+s2}{3C279}\PYG{l+s+s2}{\PYGZdq{}}\PYG{p}{)}
\end{sphinxVerbatim}
\begin{enumerate}
\sphinxsetlistlabels{\arabic}{enumi}{enumii}{}{.}%
\setcounter{enumi}{2}
\item {} 
\sphinxAtStartPar
Run self\sphinxhyphen{}calibration with a more relaxed allowed fractional change in the beam size for a solution interval to be successful

\end{enumerate}

\begin{sphinxVerbatim}[commandchars=\\\{\}]
\PYG{g+gp}{\PYGZgt{}\PYGZgt{}\PYGZgt{} }\PYG{n}{hif\PYGZus{}selfcal}\PYG{p}{(}\PYG{n}{delta\PYGZus{}beam\PYGZus{}thresh}\PYG{o}{=}\PYG{l+m+mf}{0.15}\PYG{p}{)}
\end{sphinxVerbatim}

\end{fulllineitems}


\sphinxstepscope


\section{pipeline.hif.cli.hif\_setjy}
\label{\detokenize{_autosummary/pipeline.hif.cli.hif_setjy:pipeline-hif-cli-hif-setjy}}\label{\detokenize{_autosummary/pipeline.hif.cli.hif_setjy::doc}}\index{hif\_setjy() (in module pipeline.hif.cli)@\spxentry{hif\_setjy()}\spxextra{in module pipeline.hif.cli}}

\begin{fulllineitems}
\phantomsection\label{\detokenize{_autosummary/pipeline.hif.cli.hif_setjy:pipeline.hif.cli.hif_setjy}}
\pysigstartsignatures
\pysiglinewithargsret
{\sphinxbfcode{\sphinxupquote{hif\_setjy}}}
{\sphinxparam{\DUrole{n}{vis}\DUrole{o}{=}\DUrole{default_value}{None}}\sphinxparamcomma \sphinxparam{\DUrole{n}{field}\DUrole{o}{=}\DUrole{default_value}{None}}\sphinxparamcomma \sphinxparam{\DUrole{n}{intent}\DUrole{o}{=}\DUrole{default_value}{None}}\sphinxparamcomma \sphinxparam{\DUrole{n}{spw}\DUrole{o}{=}\DUrole{default_value}{None}}\sphinxparamcomma \sphinxparam{\DUrole{n}{model}\DUrole{o}{=}\DUrole{default_value}{None}}\sphinxparamcomma \sphinxparam{\DUrole{n}{reffile}\DUrole{o}{=}\DUrole{default_value}{None}}\sphinxparamcomma \sphinxparam{\DUrole{n}{normfluxes}\DUrole{o}{=}\DUrole{default_value}{None}}\sphinxparamcomma \sphinxparam{\DUrole{n}{reffreq}\DUrole{o}{=}\DUrole{default_value}{None}}\sphinxparamcomma \sphinxparam{\DUrole{n}{fluxdensity}\DUrole{o}{=}\DUrole{default_value}{None}}\sphinxparamcomma \sphinxparam{\DUrole{n}{spix}\DUrole{o}{=}\DUrole{default_value}{None}}\sphinxparamcomma \sphinxparam{\DUrole{n}{scalebychan}\DUrole{o}{=}\DUrole{default_value}{None}}\sphinxparamcomma \sphinxparam{\DUrole{n}{standard}\DUrole{o}{=}\DUrole{default_value}{None}}}
{}
\pysigstopsignatures
\sphinxAtStartPar
Fill the model column with calibrated visibilities

\sphinxAtStartPar
Fills the model column with the model visibilities.
\begin{quote}\begin{description}
\sphinxlineitem{Parameters}\begin{itemize}
\item {} 
\sphinxAtStartPar
\sphinxstylestrong{vis} \textendash{} The list of input MeasurementSets. Defaults to the list of MeasurementSets defined in the pipeline context.

\item {} 
\sphinxAtStartPar
\sphinxstylestrong{field} \textendash{} The list of field names or field ids for which the models are to be set. Defaults to all fields with intent ‘\sphinxtitleref{*AMPLITUDE*}’.
example: field=’3C279’, field=’3C279, M82’

\item {} 
\sphinxAtStartPar
\sphinxstylestrong{intent} \textendash{} A string containing a comma delimited list of intents against which the selected fields are matched. Defaults to all data
with amplitude intent.
example: intent=’\sphinxtitleref{*AMPLITUDE*}’

\item {} 
\sphinxAtStartPar
\sphinxstylestrong{spw} \textendash{} The list of spectral windows and channels for which bandpasses are computed. Defaults to all science spectral windows.
example: spw=’11,13,15,17’

\item {} 
\sphinxAtStartPar
\sphinxstylestrong{model} \textendash{} Model image for setting model visibilities. Not fully supported.
example: see details in help for CASA setjy task

\item {} 
\sphinxAtStartPar
\sphinxstylestrong{reffile} \textendash{} Path to a file containing flux densities for calibrators unknown to CASA. Values given in this file take precedence over the CASA\sphinxhyphen{}derived
values for all calibrators except solar system calibrators. By default the
path is set to the CSV file created by h\_importdata, consisting of
catalogue fluxes extracted from the ASDM.
example: reffile=’’, reffile=’working/flux.csv’

\item {} 
\sphinxAtStartPar
\sphinxstylestrong{normfluxes} \textendash{} Normalize lookup fluxes.

\item {} 
\sphinxAtStartPar
\sphinxstylestrong{reffreq} \textendash{} The reference frequency for spix, given with units. Provided to avoid division by zero. If the flux density is being scaled by spectral
index, then reffreq must be set to whatever reference frequency is correct
for the given fluxdensity and spix. It cannot be determined from vis. On
the other hand, if spix is 0, then any positive frequency can be used and
will be ignored.
example: reffreq=’86.0GHz’, reffreq=’4.65e9Hz’

\item {} 
\sphinxAtStartPar
\sphinxstylestrong{fluxdensity} \textendash{} Specified flux density {[}I,Q,U,V{]} in Jy. Uses {[}1,0,0,0{]} flux density for unrecognized sources, and standard flux densities for
ones recognized by ‘standard’, including 3C286, 3C48, 3C147, and several
planets, moons, and asteroids.
example: {[}3.06,0.0,0.0,0.0{]}

\item {} 
\sphinxAtStartPar
\sphinxstylestrong{spix} \textendash{} Spectral index for fluxdensity S = fluxdensity * (freq/reffreq)**spix Only used if fluxdensity is being used. If fluxdensity is positive, and
spix is nonzero, then reffreq must be set too. It is applied in the same
way to all polarizations, and does not account for Faraday rotation or
depolarization.

\item {} 
\sphinxAtStartPar
\sphinxstylestrong{scalebychan} \textendash{} This determines whether the fluxdensity set in the model is calculated on a per channel basis. If False then only one fluxdensity
value is calculated per spw.

\item {} 
\sphinxAtStartPar
\sphinxstylestrong{standard} \textendash{} Flux density standard, used if fluxdensity{[}0{]} less than 0.0. The options are: ‘Baars’,’Perley 90’,’Perley\sphinxhyphen{}Taylor 95’, ‘Perley\sphinxhyphen{}Taylor 99’,
‘Perley\sphinxhyphen{}Butler 2010’ and ‘Butler\sphinxhyphen{}JPL\sphinxhyphen{}Horizons 2010’.
default: ‘Butler\sphinxhyphen{}JPL\sphinxhyphen{}Horizons 2012’ for solar system object
‘Perley\sphinxhyphen{}Butler 2010’ otherwise

\end{itemize}

\sphinxlineitem{Returns}
\sphinxAtStartPar
The results object for the pipeline task is returned.

\end{description}\end{quote}
\subsubsection*{Examples}
\begin{enumerate}
\sphinxsetlistlabels{\arabic}{enumi}{enumii}{}{.}%
\item {} 
\sphinxAtStartPar
Set the model flux densities for all the amplitude calibrators:

\end{enumerate}

\begin{sphinxVerbatim}[commandchars=\\\{\}]
\PYG{g+gp}{\PYGZgt{}\PYGZgt{}\PYGZgt{} }\PYG{n}{hif\PYGZus{}setjy}\PYG{p}{(}\PYG{p}{)}
\end{sphinxVerbatim}

\end{fulllineitems}


\sphinxstepscope


\section{pipeline.hif.cli.hif\_setmodels}
\label{\detokenize{_autosummary/pipeline.hif.cli.hif_setmodels:pipeline-hif-cli-hif-setmodels}}\label{\detokenize{_autosummary/pipeline.hif.cli.hif_setmodels::doc}}\index{hif\_setmodels() (in module pipeline.hif.cli)@\spxentry{hif\_setmodels()}\spxextra{in module pipeline.hif.cli}}

\begin{fulllineitems}
\phantomsection\label{\detokenize{_autosummary/pipeline.hif.cli.hif_setmodels:pipeline.hif.cli.hif_setmodels}}
\pysigstartsignatures
\pysiglinewithargsret
{\sphinxbfcode{\sphinxupquote{hif\_setmodels}}}
{\sphinxparam{\DUrole{n}{vis}\DUrole{o}{=}\DUrole{default_value}{None}}\sphinxparamcomma \sphinxparam{\DUrole{n}{reference}\DUrole{o}{=}\DUrole{default_value}{None}}\sphinxparamcomma \sphinxparam{\DUrole{n}{refintent}\DUrole{o}{=}\DUrole{default_value}{None}}\sphinxparamcomma \sphinxparam{\DUrole{n}{transfer}\DUrole{o}{=}\DUrole{default_value}{None}}\sphinxparamcomma \sphinxparam{\DUrole{n}{transintent}\DUrole{o}{=}\DUrole{default_value}{None}}\sphinxparamcomma \sphinxparam{\DUrole{n}{reffile}\DUrole{o}{=}\DUrole{default_value}{None}}\sphinxparamcomma \sphinxparam{\DUrole{n}{normfluxes}\DUrole{o}{=}\DUrole{default_value}{None}}\sphinxparamcomma \sphinxparam{\DUrole{n}{scalebychan}\DUrole{o}{=}\DUrole{default_value}{None}}}
{}
\pysigstopsignatures
\sphinxAtStartPar
Set calibrator source models

\sphinxAtStartPar
Set model fluxes values for calibrator reference and transfer sources using lookup
values. By default the reference sources are the flux calibrators and the transfer
sources are the bandpass, phase, and check source calibrators. Reference sources
which are also in the transfer source list are removed from the transfer source list.

\sphinxAtStartPar
Built\sphinxhyphen{}in lookup tables are used to compute models for solar system object calibrators.
Point source models are used for other calibrators with flux densities provided in the reference file.
Normalized fluxes are computed for transfer sources if the \sphinxcode{\sphinxupquote{normfluxes}} parameter is
set to True.

\sphinxAtStartPar
The default reference file is ‘flux.csv’ in the current working directory.
This file is usually created in the importdata stage. The file is in
‘csv’ format and contains the following comma delimited columns.

\sphinxAtStartPar
vis,fieldid,spwid,I,Q,U,V,pix,comment
\begin{quote}\begin{description}
\sphinxlineitem{Parameters}\begin{itemize}
\item {} 
\sphinxAtStartPar
\sphinxstylestrong{vis} \textendash{} The list of input MeasurementSets. Defaults to the list of MeasurementSets specified in the pipeline context.
example: {[}‘M32A.ms’, ‘M32B.ms’{]}

\item {} 
\sphinxAtStartPar
\sphinxstylestrong{reference} \textendash{} A string containing a comma delimited list of  field names defining the reference calibrators. Defaults to field names with
intent ‘AMPLITUDE’.
example: ‘M82,3C273’

\item {} 
\sphinxAtStartPar
\sphinxstylestrong{refintent} \textendash{} A string containing a comma delimited list of intents used to select the reference calibrators. Defaults to ‘AMPLITUDE’.
example: ‘BANDPASS’

\item {} 
\sphinxAtStartPar
\sphinxstylestrong{transfer} \textendash{} A string containing a comma delimited list of  field names defining the transfer calibrators. Defaults to field names with
intent ‘’.
example: ‘J1328+041,J1206+30’

\item {} 
\sphinxAtStartPar
\sphinxstylestrong{transintent} \textendash{} A string containing a comma delimited list of intents defining the transfer calibrators. Defaults to ‘BANDPASS,PHASE,CHECK’.
‘’ stands for no transfer sources.
example: ‘PHASE’

\item {} 
\sphinxAtStartPar
\sphinxstylestrong{reffile} \textendash{} The reference file containing a lookup table of point source models This file currently defaults to ‘flux.csv’ in the working directory. This
file must conform to the standard pipeline ‘flux.csv’ format
example: ‘myfluxes.csv’

\item {} 
\sphinxAtStartPar
\sphinxstylestrong{normfluxes} \textendash{} Normalize the transfer source flux densities.

\item {} 
\sphinxAtStartPar
\sphinxstylestrong{scalebychan} \textendash{} Scale the flux density on a per channel basis or else on a per spw basis

\end{itemize}

\sphinxlineitem{Returns}
\sphinxAtStartPar
The results object for the pipeline task is returned.

\end{description}\end{quote}
\subsubsection*{Examples}
\begin{enumerate}
\sphinxsetlistlabels{\arabic}{enumi}{enumii}{}{.}%
\item {} 
\sphinxAtStartPar
Set model fluxes for the flux, bandpass, phase, and check sources.

\end{enumerate}

\begin{sphinxVerbatim}[commandchars=\\\{\}]
\PYG{g+gp}{\PYGZgt{}\PYGZgt{}\PYGZgt{} }\PYG{n}{hif\PYGZus{}setmodels}\PYG{p}{(}\PYG{p}{)}
\end{sphinxVerbatim}

\end{fulllineitems}


\sphinxstepscope


\section{pipeline.hif.cli.hif\_transformimagedata}
\label{\detokenize{_autosummary/pipeline.hif.cli.hif_transformimagedata:pipeline-hif-cli-hif-transformimagedata}}\label{\detokenize{_autosummary/pipeline.hif.cli.hif_transformimagedata::doc}}\index{hif\_transformimagedata() (in module pipeline.hif.cli)@\spxentry{hif\_transformimagedata()}\spxextra{in module pipeline.hif.cli}}

\begin{fulllineitems}
\phantomsection\label{\detokenize{_autosummary/pipeline.hif.cli.hif_transformimagedata:pipeline.hif.cli.hif_transformimagedata}}
\pysigstartsignatures
\pysiglinewithargsret
{\sphinxbfcode{\sphinxupquote{hif\_transformimagedata}}}
{\sphinxparam{\DUrole{n}{vis}\DUrole{o}{=}\DUrole{default_value}{None}}\sphinxparamcomma \sphinxparam{\DUrole{n}{outputvis}\DUrole{o}{=}\DUrole{default_value}{None}}\sphinxparamcomma \sphinxparam{\DUrole{n}{field}\DUrole{o}{=}\DUrole{default_value}{None}}\sphinxparamcomma \sphinxparam{\DUrole{n}{intent}\DUrole{o}{=}\DUrole{default_value}{None}}\sphinxparamcomma \sphinxparam{\DUrole{n}{spw}\DUrole{o}{=}\DUrole{default_value}{None}}\sphinxparamcomma \sphinxparam{\DUrole{n}{datacolumn}\DUrole{o}{=}\DUrole{default_value}{None}}\sphinxparamcomma \sphinxparam{\DUrole{n}{chanbin}\DUrole{o}{=}\DUrole{default_value}{None}}\sphinxparamcomma \sphinxparam{\DUrole{n}{timebin}\DUrole{o}{=}\DUrole{default_value}{None}}\sphinxparamcomma \sphinxparam{\DUrole{n}{replace}\DUrole{o}{=}\DUrole{default_value}{None}}\sphinxparamcomma \sphinxparam{\DUrole{n}{clear\_pointing}\DUrole{o}{=}\DUrole{default_value}{None}}\sphinxparamcomma \sphinxparam{\DUrole{n}{modify\_weights}\DUrole{o}{=}\DUrole{default_value}{None}}\sphinxparamcomma \sphinxparam{\DUrole{n}{wtmode}\DUrole{o}{=}\DUrole{default_value}{None}}}
{}
\pysigstopsignatures
\sphinxAtStartPar
Extract fields for the desired VLASS image to a new MS and reset weights if desired

\sphinxAtStartPar
Extract fields for the desired VLASS image to a new MS and reset weights if desired
\begin{quote}\begin{description}
\sphinxlineitem{Parameters}\begin{itemize}
\item {} 
\sphinxAtStartPar
\sphinxstylestrong{vis} \textendash{} List of visibility data files. These may be ASDMs, tar files of ASDMs, MSs, or tar files of MSs, If ASDM files are specified, they will be
converted  to MS format.
example: vis={[}‘X227.ms’, ‘asdms.tar.gz’{]}

\item {} 
\sphinxAtStartPar
\sphinxstylestrong{outputvis} \textendash{} The output MeasurementSet.

\item {} 
\sphinxAtStartPar
\sphinxstylestrong{field} \textendash{} Set of data selection field names or ids, ‘’ for all.

\item {} 
\sphinxAtStartPar
\sphinxstylestrong{intent} \textendash{} Set of data selection intents, ‘’ for all.

\item {} 
\sphinxAtStartPar
\sphinxstylestrong{spw} \textendash{} Set of data selection spectral window ids ‘’ for all.

\item {} 
\sphinxAtStartPar
\sphinxstylestrong{datacolumn} \textendash{} Select spectral windows to split. The standard CASA options are supported
example: ‘data’, ‘model’

\item {} 
\sphinxAtStartPar
\sphinxstylestrong{chanbin} \textendash{} Bin width for channel averaging.

\item {} 
\sphinxAtStartPar
\sphinxstylestrong{timebin} \textendash{} Bin width for time averaging.

\item {} 
\sphinxAtStartPar
\sphinxstylestrong{replace} \textendash{} If a split was performed delete the parent MS and remove it from the context. example: True or False

\item {} 
\sphinxAtStartPar
\sphinxstylestrong{clear\_pointing} \textendash{} Clear the pointing table.

\item {} 
\sphinxAtStartPar
\sphinxstylestrong{modify\_weights} \textendash{} Re\sphinxhyphen{}initialize the weights.

\item {} 
\sphinxAtStartPar
\sphinxstylestrong{wtmode} \textendash{} optional weight initialization mode when modify\_weights=True

\end{itemize}

\sphinxlineitem{Returns}
\sphinxAtStartPar
The results object for the pipeline task is returned.

\end{description}\end{quote}
\subsubsection*{Examples}
\begin{enumerate}
\sphinxsetlistlabels{\arabic}{enumi}{enumii}{}{.}%
\item {} 
\sphinxAtStartPar
Basic transformimagedata task

\end{enumerate}

\begin{sphinxVerbatim}[commandchars=\\\{\}]
\PYG{g+gp}{\PYGZgt{}\PYGZgt{}\PYGZgt{} }\PYG{n}{hif\PYGZus{}transformimagedata}\PYG{p}{(}\PYG{p}{)}
\end{sphinxVerbatim}

\end{fulllineitems}


\sphinxstepscope


\section{pipeline.hif.cli.hif\_uvcontsub}
\label{\detokenize{_autosummary/pipeline.hif.cli.hif_uvcontsub:pipeline-hif-cli-hif-uvcontsub}}\label{\detokenize{_autosummary/pipeline.hif.cli.hif_uvcontsub::doc}}\index{hif\_uvcontsub() (in module pipeline.hif.cli)@\spxentry{hif\_uvcontsub()}\spxextra{in module pipeline.hif.cli}}

\begin{fulllineitems}
\phantomsection\label{\detokenize{_autosummary/pipeline.hif.cli.hif_uvcontsub:pipeline.hif.cli.hif_uvcontsub}}
\pysigstartsignatures
\pysiglinewithargsret
{\sphinxbfcode{\sphinxupquote{hif\_uvcontsub}}}
{\sphinxparam{\DUrole{n}{vis}\DUrole{o}{=}\DUrole{default_value}{None}}\sphinxparamcomma \sphinxparam{\DUrole{n}{field}\DUrole{o}{=}\DUrole{default_value}{None}}\sphinxparamcomma \sphinxparam{\DUrole{n}{intent}\DUrole{o}{=}\DUrole{default_value}{None}}\sphinxparamcomma \sphinxparam{\DUrole{n}{spw}\DUrole{o}{=}\DUrole{default_value}{None}}\sphinxparamcomma \sphinxparam{\DUrole{n}{fitorder}\DUrole{o}{=}\DUrole{default_value}{None}}\sphinxparamcomma \sphinxparam{\DUrole{n}{parallel}\DUrole{o}{=}\DUrole{default_value}{None}}}
{}
\pysigstopsignatures
\sphinxAtStartPar
Fit and subtract continuum from the data

\sphinxAtStartPar
hif\_uvcontsub fits the continuum for the frequency ranges given in the cont.dat
file, subtracts that fit from the uv data and generates a new set of MSes
containing the continuum subtracted (i.e. line) data. The fit is attempted
for all science targets and spws. If a fit is impossible, the corresponding
data selection is not written to the output line MS.
\begin{quote}\begin{description}
\sphinxlineitem{Parameters}\begin{itemize}
\item {} 
\sphinxAtStartPar
\sphinxstylestrong{vis} \textendash{} The list of input MeasurementSets. Defaults to the list of MeasurementSets specified in the h\_init or hif\_importdata task.
‘’: use all MeasurementSets in the context
Examples: ‘ngc5921.ms’, {[}‘ngc5921a.ms’, ngc5921b.ms’, ‘ngc5921c.ms’{]}

\item {} 
\sphinxAtStartPar
\sphinxstylestrong{field} \textendash{} The list of field names or field ids for which UV continuum fits are computed. Defaults to all fields.
Examples: ‘3C279’, ‘3C279,M82’

\item {} 
\sphinxAtStartPar
\sphinxstylestrong{intent} \textendash{} A string containing a comma delimited list of intents against which the selected fields are matched.
‘’: Defaults to all data with TARGET intent.

\item {} 
\sphinxAtStartPar
\sphinxstylestrong{spw} \textendash{} The list of spectral windows and channels for which uv continuum fits are computed.
‘’, Defaults to all science spectral windows.
Example: ‘11,13,15,17’

\item {} 
\sphinxAtStartPar
\sphinxstylestrong{fitorder} \textendash{} Polynomial order for the continuum fits per source and spw. Defaults to \{\} which means fit order 1 for all sources and
spws. If an explicit dictionary is given then all unspecified
selections still default to 1.
Example: \{‘3C279’: \{‘15’: 1, ‘17’: 2\}, ‘M82’: \{‘13’: 2\}\}

\item {} 
\sphinxAtStartPar
\sphinxstylestrong{parallel} \textendash{} Execute using CASA HPC functionality, if available.

\end{itemize}

\sphinxlineitem{Returns}
\sphinxAtStartPar
The results object for the pipeline task is returned.

\end{description}\end{quote}
\subsubsection*{Examples}
\begin{enumerate}
\sphinxsetlistlabels{\arabic}{enumi}{enumii}{}{.}%
\item {} 
\sphinxAtStartPar
Fit and subtract continuum for all science targets and spws

\end{enumerate}

\begin{sphinxVerbatim}[commandchars=\\\{\}]
\PYG{g+gp}{\PYGZgt{}\PYGZgt{}\PYGZgt{} }\PYG{n}{hif\PYGZus{}uvcontsub}\PYG{p}{(}\PYG{p}{)}
\end{sphinxVerbatim}
\begin{enumerate}
\sphinxsetlistlabels{\arabic}{enumi}{enumii}{}{.}%
\setcounter{enumi}{1}
\item {} 
\sphinxAtStartPar
Fit and subtract continuum only for a subsect of fields

\end{enumerate}

\begin{sphinxVerbatim}[commandchars=\\\{\}]
\PYG{g+gp}{\PYGZgt{}\PYGZgt{}\PYGZgt{} }\PYG{n}{hif\PYGZus{}uvcontsub}\PYG{p}{(}\PYG{n}{field}\PYG{o}{=}\PYG{l+s+s1}{\PYGZsq{}}\PYG{l+s+s1}{3C279,M82}\PYG{l+s+s1}{\PYGZsq{}}
\end{sphinxVerbatim}
\begin{enumerate}
\sphinxsetlistlabels{\arabic}{enumi}{enumii}{}{.}%
\setcounter{enumi}{2}
\item {} 
\sphinxAtStartPar
Fit and subtract continuum only for a subsect of spws

\end{enumerate}

\begin{sphinxVerbatim}[commandchars=\\\{\}]
\PYG{g+gp}{\PYGZgt{}\PYGZgt{}\PYGZgt{} }\PYG{n}{hif\PYGZus{}uvcontsub}\PYG{p}{(}\PYG{n}{spw}\PYG{o}{=}\PYG{l+s+s1}{\PYGZsq{}}\PYG{l+s+s1}{11,13}\PYG{l+s+s1}{\PYGZsq{}}\PYG{p}{)}
\end{sphinxVerbatim}
\begin{enumerate}
\sphinxsetlistlabels{\arabic}{enumi}{enumii}{}{.}%
\setcounter{enumi}{3}
\item {} 
\sphinxAtStartPar
Override automatic fit order choice

\end{enumerate}

\begin{sphinxVerbatim}[commandchars=\\\{\}]
\PYG{g+gp}{\PYGZgt{}\PYGZgt{}\PYGZgt{} }\PYG{n}{hif\PYGZus{}uvcontsub}\PYG{p}{(}\PYG{n}{fitorder}\PYG{o}{=}\PYG{p}{\PYGZob{}}\PYG{l+s+s1}{\PYGZsq{}}\PYG{l+s+s1}{3C279}\PYG{l+s+s1}{\PYGZsq{}}\PYG{p}{:} \PYG{p}{\PYGZob{}}\PYG{l+s+s1}{\PYGZsq{}}\PYG{l+s+s1}{15}\PYG{l+s+s1}{\PYGZsq{}}\PYG{p}{:} \PYG{l+m+mi}{1}\PYG{p}{,} \PYG{l+s+s1}{\PYGZsq{}}\PYG{l+s+s1}{17}\PYG{l+s+s1}{\PYGZsq{}}\PYG{p}{:} \PYG{l+m+mi}{2}\PYG{p}{\PYGZcb{}}\PYG{p}{,} \PYG{l+s+s1}{\PYGZsq{}}\PYG{l+s+s1}{M82}\PYG{l+s+s1}{\PYGZsq{}}\PYG{p}{:} \PYG{p}{\PYGZob{}}\PYG{l+s+s1}{\PYGZsq{}}\PYG{l+s+s1}{13}\PYG{l+s+s1}{\PYGZsq{}}\PYG{p}{:} \PYG{l+m+mi}{2}\PYG{p}{\PYGZcb{}}\PYG{p}{\PYGZcb{}}\PYG{p}{)}
\end{sphinxVerbatim}

\end{fulllineitems}


\sphinxstepscope


\chapter{pipeline.hifa.cli}
\label{\detokenize{_autosummary/pipeline.hifa.cli:module-pipeline.hifa.cli}}\label{\detokenize{_autosummary/pipeline.hifa.cli:pipeline-hifa-cli}}\label{\detokenize{_autosummary/pipeline.hifa.cli::doc}}\index{module@\spxentry{module}!pipeline.hifa.cli@\spxentry{pipeline.hifa.cli}}\index{pipeline.hifa.cli@\spxentry{pipeline.hifa.cli}!module@\spxentry{module}}
\sphinxAtStartPar
Interferometry ALMA Tasks
\subsubsection*{Functions}


\begin{savenotes}\sphinxattablestart
\sphinxthistablewithglobalstyle
\sphinxthistablewithnovlinesstyle
\centering
\begin{tabulary}{\linewidth}[t]{\X{1}{2}\X{1}{2}}
\sphinxtoprule
\sphinxtableatstartofbodyhook
\sphinxAtStartPar
{\hyperref[\detokenize{_autosummary/pipeline.hifa.cli.hifa_antpos:pipeline.hifa.cli.hifa_antpos}]{\sphinxcrossref{\sphinxcode{\sphinxupquote{hifa\_antpos}}}}}
&
\sphinxAtStartPar
Derive an antenna position calibration table
\\
\sphinxhline
\sphinxAtStartPar
{\hyperref[\detokenize{_autosummary/pipeline.hifa.cli.hifa_bandpass:pipeline.hifa.cli.hifa_bandpass}]{\sphinxcrossref{\sphinxcode{\sphinxupquote{hifa\_bandpass}}}}}
&
\sphinxAtStartPar
Compute bandpass calibration solutions
\\
\sphinxhline
\sphinxAtStartPar
{\hyperref[\detokenize{_autosummary/pipeline.hifa.cli.hifa_bandpassflag:pipeline.hifa.cli.hifa_bandpassflag}]{\sphinxcrossref{\sphinxcode{\sphinxupquote{hifa\_bandpassflag}}}}}
&
\sphinxAtStartPar
Bandpass calibration flagging
\\
\sphinxhline
\sphinxAtStartPar
{\hyperref[\detokenize{_autosummary/pipeline.hifa.cli.hifa_bpsolint:pipeline.hifa.cli.hifa_bpsolint}]{\sphinxcrossref{\sphinxcode{\sphinxupquote{hifa\_bpsolint}}}}}
&
\sphinxAtStartPar
Compute optimal bandpass calibration solution intervals
\\
\sphinxhline
\sphinxAtStartPar
{\hyperref[\detokenize{_autosummary/pipeline.hifa.cli.hifa_diffgaincal:pipeline.hifa.cli.hifa_diffgaincal}]{\sphinxcrossref{\sphinxcode{\sphinxupquote{hifa\_diffgaincal}}}}}
&
\sphinxAtStartPar
Derive SpW phase offsets from differential gain calibrator.
\\
\sphinxhline
\sphinxAtStartPar
{\hyperref[\detokenize{_autosummary/pipeline.hifa.cli.hifa_exportdata:pipeline.hifa.cli.hifa_exportdata}]{\sphinxcrossref{\sphinxcode{\sphinxupquote{hifa\_exportdata}}}}}
&
\sphinxAtStartPar
Prepare interferometry data for export
\\
\sphinxhline
\sphinxAtStartPar
{\hyperref[\detokenize{_autosummary/pipeline.hifa.cli.hifa_flagdata:pipeline.hifa.cli.hifa_flagdata}]{\sphinxcrossref{\sphinxcode{\sphinxupquote{hifa\_flagdata}}}}}
&
\sphinxAtStartPar
Do metadata based flagging of a list of MeasurementSets.
\\
\sphinxhline
\sphinxAtStartPar
{\hyperref[\detokenize{_autosummary/pipeline.hifa.cli.hifa_flagtargets:pipeline.hifa.cli.hifa_flagtargets}]{\sphinxcrossref{\sphinxcode{\sphinxupquote{hifa\_flagtargets}}}}}
&
\sphinxAtStartPar
Do science target flagging
\\
\sphinxhline
\sphinxAtStartPar
{\hyperref[\detokenize{_autosummary/pipeline.hifa.cli.hifa_fluxcalflag:pipeline.hifa.cli.hifa_fluxcalflag}]{\sphinxcrossref{\sphinxcode{\sphinxupquote{hifa\_fluxcalflag}}}}}
&
\sphinxAtStartPar
Locate and flag line regions in solar system flux calibrators
\\
\sphinxhline
\sphinxAtStartPar
{\hyperref[\detokenize{_autosummary/pipeline.hifa.cli.hifa_gaincalsnr:pipeline.hifa.cli.hifa_gaincalsnr}]{\sphinxcrossref{\sphinxcode{\sphinxupquote{hifa\_gaincalsnr}}}}}
&
\sphinxAtStartPar
Compute gaincal signal\sphinxhyphen{}to\sphinxhyphen{}noise ratios per spw
\\
\sphinxhline
\sphinxAtStartPar
{\hyperref[\detokenize{_autosummary/pipeline.hifa.cli.hifa_gfluxscale:pipeline.hifa.cli.hifa_gfluxscale}]{\sphinxcrossref{\sphinxcode{\sphinxupquote{hifa\_gfluxscale}}}}}
&
\sphinxAtStartPar
Derive flux density scales from standard calibrators
\\
\sphinxhline
\sphinxAtStartPar
{\hyperref[\detokenize{_autosummary/pipeline.hifa.cli.hifa_gfluxscaleflag:pipeline.hifa.cli.hifa_gfluxscaleflag}]{\sphinxcrossref{\sphinxcode{\sphinxupquote{hifa\_gfluxscaleflag}}}}}
&
\sphinxAtStartPar
Flag the flux, diffgain, phase calibrators and check source
\\
\sphinxhline
\sphinxAtStartPar
{\hyperref[\detokenize{_autosummary/pipeline.hifa.cli.hifa_imageprecheck:pipeline.hifa.cli.hifa_imageprecheck}]{\sphinxcrossref{\sphinxcode{\sphinxupquote{hifa\_imageprecheck}}}}}
&
\sphinxAtStartPar
Calculates the best Briggs robust parameter to achieve sensitivity and angular resolution goals.
\\
\sphinxhline
\sphinxAtStartPar
{\hyperref[\detokenize{_autosummary/pipeline.hifa.cli.hifa_importdata:pipeline.hifa.cli.hifa_importdata}]{\sphinxcrossref{\sphinxcode{\sphinxupquote{hifa\_importdata}}}}}
&
\sphinxAtStartPar
Imports data into the interferometry pipeline
\\
\sphinxhline
\sphinxAtStartPar
{\hyperref[\detokenize{_autosummary/pipeline.hifa.cli.hifa_lock_refant:pipeline.hifa.cli.hifa_lock_refant}]{\sphinxcrossref{\sphinxcode{\sphinxupquote{hifa\_lock\_refant}}}}}
&
\sphinxAtStartPar
Lock reference antenna list
\\
\sphinxhline
\sphinxAtStartPar
{\hyperref[\detokenize{_autosummary/pipeline.hifa.cli.hifa_polcal:pipeline.hifa.cli.hifa_polcal}]{\sphinxcrossref{\sphinxcode{\sphinxupquote{hifa\_polcal}}}}}
&
\sphinxAtStartPar
Derive instrumental polarization calibration for ALMA.
\\
\sphinxhline
\sphinxAtStartPar
{\hyperref[\detokenize{_autosummary/pipeline.hifa.cli.hifa_polcalflag:pipeline.hifa.cli.hifa_polcalflag}]{\sphinxcrossref{\sphinxcode{\sphinxupquote{hifa\_polcalflag}}}}}
&
\sphinxAtStartPar
Flag polarization calibrators
\\
\sphinxhline
\sphinxAtStartPar
{\hyperref[\detokenize{_autosummary/pipeline.hifa.cli.hifa_renorm:pipeline.hifa.cli.hifa_renorm}]{\sphinxcrossref{\sphinxcode{\sphinxupquote{hifa\_renorm}}}}}
&
\sphinxAtStartPar
ALMA renormalization task
\\
\sphinxhline
\sphinxAtStartPar
{\hyperref[\detokenize{_autosummary/pipeline.hifa.cli.hifa_restoredata:pipeline.hifa.cli.hifa_restoredata}]{\sphinxcrossref{\sphinxcode{\sphinxupquote{hifa\_restoredata}}}}}
&
\sphinxAtStartPar
Restore flagged and calibration interferometry data from a pipeline run
\\
\sphinxhline
\sphinxAtStartPar
{\hyperref[\detokenize{_autosummary/pipeline.hifa.cli.hifa_session_refant:pipeline.hifa.cli.hifa_session_refant}]{\sphinxcrossref{\sphinxcode{\sphinxupquote{hifa\_session\_refant}}}}}
&
\sphinxAtStartPar
Select best reference antenna for session(s)
\\
\sphinxhline
\sphinxAtStartPar
{\hyperref[\detokenize{_autosummary/pipeline.hifa.cli.hifa_spwphaseup:pipeline.hifa.cli.hifa_spwphaseup}]{\sphinxcrossref{\sphinxcode{\sphinxupquote{hifa\_spwphaseup}}}}}
&
\sphinxAtStartPar
Compute phase calibration spw map and per spw phase offsets
\\
\sphinxhline
\sphinxAtStartPar
{\hyperref[\detokenize{_autosummary/pipeline.hifa.cli.hifa_targetflag:pipeline.hifa.cli.hifa_targetflag}]{\sphinxcrossref{\sphinxcode{\sphinxupquote{hifa\_targetflag}}}}}
&
\sphinxAtStartPar
Flag target source outliers
\\
\sphinxhline
\sphinxAtStartPar
{\hyperref[\detokenize{_autosummary/pipeline.hifa.cli.hifa_timegaincal:pipeline.hifa.cli.hifa_timegaincal}]{\sphinxcrossref{\sphinxcode{\sphinxupquote{hifa\_timegaincal}}}}}
&
\sphinxAtStartPar
Determine temporal gains from calibrator observations
\\
\sphinxhline
\sphinxAtStartPar
{\hyperref[\detokenize{_autosummary/pipeline.hifa.cli.hifa_tsysflag:pipeline.hifa.cli.hifa_tsysflag}]{\sphinxcrossref{\sphinxcode{\sphinxupquote{hifa\_tsysflag}}}}}
&
\sphinxAtStartPar
Flag deviant system temperatures for ALMA interferometry measurements.
\\
\sphinxhline
\sphinxAtStartPar
{\hyperref[\detokenize{_autosummary/pipeline.hifa.cli.hifa_tsysflagcontamination:pipeline.hifa.cli.hifa_tsysflagcontamination}]{\sphinxcrossref{\sphinxcode{\sphinxupquote{hifa\_tsysflagcontamination}}}}}
&
\sphinxAtStartPar
Flag line contamination in ALMA interferometric Tsys caltables
\\
\sphinxhline
\sphinxAtStartPar
{\hyperref[\detokenize{_autosummary/pipeline.hifa.cli.hifa_unlock_refant:pipeline.hifa.cli.hifa_unlock_refant}]{\sphinxcrossref{\sphinxcode{\sphinxupquote{hifa\_unlock\_refant}}}}}
&
\sphinxAtStartPar
Unlock reference antenna list
\\
\sphinxhline
\sphinxAtStartPar
{\hyperref[\detokenize{_autosummary/pipeline.hifa.cli.hifa_wvrgcal:pipeline.hifa.cli.hifa_wvrgcal}]{\sphinxcrossref{\sphinxcode{\sphinxupquote{hifa\_wvrgcal}}}}}
&
\sphinxAtStartPar
Generate a gain table based on Water Vapor Radiometer data, and calculate a QA score based on its effect on the interferometric data.
\\
\sphinxhline
\sphinxAtStartPar
{\hyperref[\detokenize{_autosummary/pipeline.hifa.cli.hifa_wvrgcalflag:pipeline.hifa.cli.hifa_wvrgcalflag}]{\sphinxcrossref{\sphinxcode{\sphinxupquote{hifa\_wvrgcalflag}}}}}
&
\sphinxAtStartPar
Generate a gain table based on Water Vapor Radiometer data, interpolating over antennas with bad radiometers.
\\
\sphinxbottomrule
\end{tabulary}
\sphinxtableafterendhook\par
\sphinxattableend\end{savenotes}

\sphinxstepscope


\section{pipeline.hifa.cli.hifa\_antpos}
\label{\detokenize{_autosummary/pipeline.hifa.cli.hifa_antpos:pipeline-hifa-cli-hifa-antpos}}\label{\detokenize{_autosummary/pipeline.hifa.cli.hifa_antpos::doc}}\index{hifa\_antpos() (in module pipeline.hifa.cli)@\spxentry{hifa\_antpos()}\spxextra{in module pipeline.hifa.cli}}

\begin{fulllineitems}
\phantomsection\label{\detokenize{_autosummary/pipeline.hifa.cli.hifa_antpos:pipeline.hifa.cli.hifa_antpos}}
\pysigstartsignatures
\pysiglinewithargsret
{\sphinxbfcode{\sphinxupquote{hifa\_antpos}}}
{\sphinxparam{\DUrole{n}{vis}\DUrole{o}{=}\DUrole{default_value}{None}}\sphinxparamcomma \sphinxparam{\DUrole{n}{caltable}\DUrole{o}{=}\DUrole{default_value}{None}}\sphinxparamcomma \sphinxparam{\DUrole{n}{hm\_antpos}\DUrole{o}{=}\DUrole{default_value}{None}}\sphinxparamcomma \sphinxparam{\DUrole{n}{antenna}\DUrole{o}{=}\DUrole{default_value}{None}}\sphinxparamcomma \sphinxparam{\DUrole{n}{offsets}\DUrole{o}{=}\DUrole{default_value}{None}}\sphinxparamcomma \sphinxparam{\DUrole{n}{antposfile}\DUrole{o}{=}\DUrole{default_value}{None}}\sphinxparamcomma \sphinxparam{\DUrole{n}{threshold}\DUrole{o}{=}\DUrole{default_value}{None}}}
{}
\pysigstopsignatures
\sphinxAtStartPar
Derive an antenna position calibration table

\sphinxAtStartPar
The hifa\_antpos task corrects the antenna positions recorded in the ASDMs using
updated antenna position calibration information determined after the
observation was taken.

\sphinxAtStartPar
Corrections can be input by hand, read from a file on disk, or in the future
by querying an ALMA database service.

\sphinxAtStartPar
The antenna positions file is in ‘csv’ format containing 6 comma\sphinxhyphen{}delimited
columns as shown below. This file should not include blank lines, including
after the end of the last entry. The default name of this file is ‘antennapos.csv’.

\sphinxAtStartPar
Example of contents for an ‘antennapos.csv’ file:
\begin{quote}

\sphinxAtStartPar
ms,antenna,xoffset,yoffset,zoffset,comment

\sphinxAtStartPar
uid\_\_\_A002\_X30a93d\_X43e.ms,DV11,0.000,0.010,0.000,”No comment”

\sphinxAtStartPar
uid\_\_\_A002\_X30a93d\_X43e.dup.ms,DV11,0.000,\sphinxhyphen{}0.010,0.000,”No comment”
\end{quote}

\sphinxAtStartPar
The offset values in this file are in meters.

\sphinxAtStartPar
The corrections are used to generate a calibration table which is recorded
in the pipeline context and applied to the raw visibility data, on the fly to
generate other calibration tables, or permanently to generate calibrated
visibilities for imaging.

\sphinxAtStartPar
Note: the \sphinxcode{\sphinxupquote{hm\_antpos}} ‘online’ option will be implemented when the
observing system provides an antenna position determination service.
\begin{quote}\begin{description}
\sphinxlineitem{Parameters}\begin{itemize}
\item {} 
\sphinxAtStartPar
\sphinxstylestrong{vis} \textendash{} List of input MeasurementSets. Defaults to the list of
MeasurementSets specified in the pipeline context.
Example: vis={[}‘ngc5921.ms’{]}

\item {} 
\sphinxAtStartPar
\sphinxstylestrong{caltable} \textendash{} List of names for the output calibration tables. Defaults
to the standard pipeline naming convention.
Example: caltable={[}‘ngc5921.gcal’{]}

\item {} 
\sphinxAtStartPar
\sphinxstylestrong{hm\_antpos} \textendash{} Heuristics method for retrieving the antenna position
corrections. The options are ‘online’ (not yet implemented),
‘manual’, and ‘file’.
Example: hm\_antpos=’manual’

\item {} 
\sphinxAtStartPar
\sphinxstylestrong{antenna} \textendash{} The list of antennas for which the positions are to be corrected
if \sphinxcode{\sphinxupquote{hm\_antpos}} is ‘manual’.
Example: antenna=’DV05,DV07’

\item {} 
\sphinxAtStartPar
\sphinxstylestrong{offsets} \textendash{} The list of antenna offsets for each antenna in ‘antennas’.
Each offset is a set of 3 floating point numbers separated by
commas, specified in the ITRF frame.
Example: offsets={[}0.01, 0.02, 0.03, 0.03, 0.02, 0.01{]}

\item {} 
\sphinxAtStartPar
\sphinxstylestrong{antposfile} \textendash{} The file(s) containing the antenna offsets. Used if
\sphinxcode{\sphinxupquote{hm\_antpos}} is ‘file’.

\item {} 
\sphinxAtStartPar
\sphinxstylestrong{threshold} \textendash{} Highlight antenna position offsets greater than this value in
the weblog. Units are wavelengths and the default is 1.0.
Example: threshold=1.0

\end{itemize}

\sphinxlineitem{Returns}
\sphinxAtStartPar
The results object for the pipeline task is returned.

\end{description}\end{quote}
\subsubsection*{Examples}

\sphinxAtStartPar
1. Correct the position of antenna ‘DV05’ for all the visibility files in a
single pipeline run:

\begin{sphinxVerbatim}[commandchars=\\\{\}]
\PYG{g+gp}{\PYGZgt{}\PYGZgt{}\PYGZgt{} }\PYG{n}{hifa\PYGZus{}antpos}\PYG{p}{(}\PYG{n}{antenna}\PYG{o}{=}\PYG{l+s+s1}{\PYGZsq{}}\PYG{l+s+s1}{DV05}\PYG{l+s+s1}{\PYGZsq{}}\PYG{p}{,} \PYG{n}{offsets}\PYG{o}{=}\PYG{p}{[}\PYG{l+m+mf}{0.01}\PYG{p}{,} \PYG{l+m+mf}{0.02}\PYG{p}{,} \PYG{l+m+mf}{0.03}\PYG{p}{]}\PYG{p}{)}
\end{sphinxVerbatim}

\sphinxAtStartPar
2. Correct the position of antennas for all the visibility files in a single
pipeline run using antenna positions files on disk. These files are assumed
to conform to a default naming scheme if \sphinxcode{\sphinxupquote{antposfile}} is unspecified by the
user:

\begin{sphinxVerbatim}[commandchars=\\\{\}]
\PYG{g+gp}{\PYGZgt{}\PYGZgt{}\PYGZgt{} }\PYG{n}{hifa\PYGZus{}antpos}\PYG{p}{(}\PYG{n}{hm\PYGZus{}antpos}\PYG{o}{=}\PYG{l+s+s1}{\PYGZsq{}}\PYG{l+s+s1}{file}\PYG{l+s+s1}{\PYGZsq{}}\PYG{p}{,} \PYG{n}{antposfile}\PYG{o}{=}\PYG{l+s+s1}{\PYGZsq{}}\PYG{l+s+s1}{myantposfile.csv}\PYG{l+s+s1}{\PYGZsq{}}\PYG{p}{)}
\end{sphinxVerbatim}

\end{fulllineitems}


\sphinxstepscope


\section{pipeline.hifa.cli.hifa\_bandpass}
\label{\detokenize{_autosummary/pipeline.hifa.cli.hifa_bandpass:pipeline-hifa-cli-hifa-bandpass}}\label{\detokenize{_autosummary/pipeline.hifa.cli.hifa_bandpass::doc}}\index{hifa\_bandpass() (in module pipeline.hifa.cli)@\spxentry{hifa\_bandpass()}\spxextra{in module pipeline.hifa.cli}}

\begin{fulllineitems}
\phantomsection\label{\detokenize{_autosummary/pipeline.hifa.cli.hifa_bandpass:pipeline.hifa.cli.hifa_bandpass}}
\pysigstartsignatures
\pysiglinewithargsret
{\sphinxbfcode{\sphinxupquote{hifa\_bandpass}}}
{\sphinxparam{\DUrole{n}{vis}\DUrole{o}{=}\DUrole{default_value}{None}}\sphinxparamcomma \sphinxparam{\DUrole{n}{caltable}\DUrole{o}{=}\DUrole{default_value}{None}}\sphinxparamcomma \sphinxparam{\DUrole{n}{field}\DUrole{o}{=}\DUrole{default_value}{None}}\sphinxparamcomma \sphinxparam{\DUrole{n}{intent}\DUrole{o}{=}\DUrole{default_value}{None}}\sphinxparamcomma \sphinxparam{\DUrole{n}{spw}\DUrole{o}{=}\DUrole{default_value}{None}}\sphinxparamcomma \sphinxparam{\DUrole{n}{antenna}\DUrole{o}{=}\DUrole{default_value}{None}}\sphinxparamcomma \sphinxparam{\DUrole{n}{hm\_phaseup}\DUrole{o}{=}\DUrole{default_value}{None}}\sphinxparamcomma \sphinxparam{\DUrole{n}{phaseupsolint}\DUrole{o}{=}\DUrole{default_value}{None}}\sphinxparamcomma \sphinxparam{\DUrole{n}{phaseupbw}\DUrole{o}{=}\DUrole{default_value}{None}}\sphinxparamcomma \sphinxparam{\DUrole{n}{phaseupsnr}\DUrole{o}{=}\DUrole{default_value}{None}}\sphinxparamcomma \sphinxparam{\DUrole{n}{phaseupnsols}\DUrole{o}{=}\DUrole{default_value}{None}}\sphinxparamcomma \sphinxparam{\DUrole{n}{hm\_bandpass}\DUrole{o}{=}\DUrole{default_value}{None}}\sphinxparamcomma \sphinxparam{\DUrole{n}{solint}\DUrole{o}{=}\DUrole{default_value}{None}}\sphinxparamcomma \sphinxparam{\DUrole{n}{maxchannels}\DUrole{o}{=}\DUrole{default_value}{None}}\sphinxparamcomma \sphinxparam{\DUrole{n}{evenbpints}\DUrole{o}{=}\DUrole{default_value}{None}}\sphinxparamcomma \sphinxparam{\DUrole{n}{bpsnr}\DUrole{o}{=}\DUrole{default_value}{None}}\sphinxparamcomma \sphinxparam{\DUrole{n}{minbpsnr}\DUrole{o}{=}\DUrole{default_value}{None}}\sphinxparamcomma \sphinxparam{\DUrole{n}{bpnsols}\DUrole{o}{=}\DUrole{default_value}{None}}\sphinxparamcomma \sphinxparam{\DUrole{n}{combine}\DUrole{o}{=}\DUrole{default_value}{None}}\sphinxparamcomma \sphinxparam{\DUrole{n}{refant}\DUrole{o}{=}\DUrole{default_value}{None}}\sphinxparamcomma \sphinxparam{\DUrole{n}{solnorm}\DUrole{o}{=}\DUrole{default_value}{None}}\sphinxparamcomma \sphinxparam{\DUrole{n}{minblperant}\DUrole{o}{=}\DUrole{default_value}{None}}\sphinxparamcomma \sphinxparam{\DUrole{n}{minsnr}\DUrole{o}{=}\DUrole{default_value}{None}}\sphinxparamcomma \sphinxparam{\DUrole{n}{unregister\_existing}\DUrole{o}{=}\DUrole{default_value}{None}}\sphinxparamcomma \sphinxparam{\DUrole{n}{hm\_auto\_fillgaps}\DUrole{o}{=}\DUrole{default_value}{None}}}
{}
\pysigstopsignatures
\sphinxAtStartPar
Compute bandpass calibration solutions

\sphinxAtStartPar
The hifa\_bandpass task computes a bandpass solution for every specified
science spectral window. By default, a ‘phaseup’ pre\sphinxhyphen{}calibration is
performed and applied on the fly to the data, before the bandpass is
computed.

\sphinxAtStartPar
The hif\_refant task may be used to pre\sphinxhyphen{}compute a prioritized list of
reference antennas.
\begin{quote}\begin{description}
\sphinxlineitem{Parameters}\begin{itemize}
\item {} 
\sphinxAtStartPar
\sphinxstylestrong{vis} \textendash{} List of input MeasurementSets. Defaults to the list of
MeasurementSets specified in the pipeline context.
Example: vis={[}‘ngc5921.ms’{]}

\item {} 
\sphinxAtStartPar
\sphinxstylestrong{caltable} \textendash{} List of names for the output calibration tables. Defaults
to the standard pipeline naming convention.
Example: caltable={[}‘ngc5921.gcal’{]}

\item {} 
\sphinxAtStartPar
\sphinxstylestrong{field} \textendash{} The list of field names or field ids for which
bandpasses are computed. Set to field=’’ by default,
which means the task will select all fields.
Example: field=’3C279’, field=’3C279,M82’

\item {} 
\sphinxAtStartPar
\sphinxstylestrong{intent} \textendash{} A string containing a comma delimited list of intents
against which the selected fields are matched. Set to
intent=’’ by default, which means the task will select
all data with the BANDPASS intent.
Example: intent=’\sphinxtitleref{*PHASE*}’

\item {} 
\sphinxAtStartPar
\sphinxstylestrong{spw} \textendash{} The list of spectral windows and channels for which
bandpasses are computed. Set to spw=’’ by default, which
means the task will select all science spectral windows.
Example: spw=’11,13,15,17’

\item {} 
\sphinxAtStartPar
\sphinxstylestrong{antenna} \textendash{} Set of data selection antenna IDs

\item {} 
\sphinxAtStartPar
\sphinxstylestrong{hm\_phaseup} \textendash{} The pre\sphinxhyphen{}bandpass solution phaseup gain heuristics. The
options are:
\begin{itemize}
\item {} 
\sphinxAtStartPar
‘snr’: compute solution required to achieve the specified SNR

\item {} 
\sphinxAtStartPar
‘manual’: use manual solution parameters

\item {} 
\sphinxAtStartPar
‘’: skip phaseup

\end{itemize}

\sphinxAtStartPar
Example: hm\_phaseup=’manual’

\item {} 
\sphinxAtStartPar
\sphinxstylestrong{phaseupsolint} \textendash{} The phase correction solution interval in CASA syntax.
Used when \sphinxcode{\sphinxupquote{hm\_phaseup}} = ‘manual’ or as a default if
the \sphinxcode{\sphinxupquote{hm\_phaseup}} = ‘snr’ heuristic computation fails.
Example: phaseupsolint=’300s’

\item {} 
\sphinxAtStartPar
\sphinxstylestrong{phaseupbw} \textendash{} Bandwidth to be used for phaseup. Used when
\sphinxcode{\sphinxupquote{hm\_phaseup}} = ‘manual’.

\sphinxAtStartPar
Example:
\begin{itemize}
\item {} 
\sphinxAtStartPar
phaseupbw=’’ to use entire bandpass

\item {} 
\sphinxAtStartPar
phaseupbw=’500MHz’ to use central 500MHz

\end{itemize}

\item {} 
\sphinxAtStartPar
\sphinxstylestrong{phaseupsnr} \textendash{} The required SNR for the phaseup solution. Used to calculate
the phaseup time solint, and only if
\sphinxcode{\sphinxupquote{hm\_phaseup}} = ‘snr’.
Example: phaseupsnr=10.0

\item {} 
\sphinxAtStartPar
\sphinxstylestrong{phaseupnsols} \textendash{} The minimum number of phaseup gain solutions. Used only
if \sphinxcode{\sphinxupquote{hm\_phaseup}} = ‘snr’.
Example: phaseupnsols=4

\item {} 
\sphinxAtStartPar
\sphinxstylestrong{hm\_bandpass} \textendash{} The bandpass solution heuristics. The options are:
‘snr’: compute the solution required to achieve the specified SNR
‘smoothed’: simple ‘smoothing’ i.e. spectral solint\textgreater{}1chan
‘fixed’: use the user defined parameters for all spws
Example: hm\_bandpass=’snr’

\item {} 
\sphinxAtStartPar
\sphinxstylestrong{solint} \textendash{} Time and channel solution intervals in CASA syntax.
Default is solint=’inf’, which is used when
\sphinxcode{\sphinxupquote{hm\_bandpass}} = ‘fixed’.
If \sphinxcode{\sphinxupquote{hm\_bandpass}} = ‘snr’, then the task will attempt
to compute and use an optimal SNR\sphinxhyphen{}based solint (and warn
if this solint is not good enough).
If \sphinxcode{\sphinxupquote{hm\_bandpass}} = ‘smoothed’, the task will override
the spectral solint with bandwidth/maxchannels.
Example: solint=’int’

\item {} 
\sphinxAtStartPar
\sphinxstylestrong{maxchannels} \textendash{} The bandpass solution ‘smoothing’ factor in channels,
i.e. spectral solint will be set to bandwidth / maxchannels
Set to 0 for no smoothing. Used if
\sphinxcode{\sphinxupquote{hm\_bandpass}} = ‘smoothed’.
Example: maxchannels=240

\item {} 
\sphinxAtStartPar
\sphinxstylestrong{evenbpints} \textendash{} Force the per spw frequency solint to be evenly divisible
into the spw bandpass if \sphinxcode{\sphinxupquote{hm\_bandpass}} = ‘snr’.
Example: evenbpints=False

\item {} 
\sphinxAtStartPar
\sphinxstylestrong{bpsnr} \textendash{} The required SNR for the bandpass solution. Used only if
\sphinxcode{\sphinxupquote{hm\_bandpass}} = ‘snr’.
Example: bpsnr=30.0

\item {} 
\sphinxAtStartPar
\sphinxstylestrong{minbpsnr} \textendash{} The minimum required SNR for the bandpass solution
when strong atmospheric lines exist in Tsys spectra.
Used only if {\color{red}\bfseries{}\textasciigrave{}\textasciigrave{}}hm\_bandpass\textasciigrave{}\textasciigrave{}=’snr’.
Example: minbpsnr=10.0

\item {} 
\sphinxAtStartPar
\sphinxstylestrong{bpnsols} \textendash{} The minimum number of bandpass solutions. Used only if
\sphinxcode{\sphinxupquote{hm\_bandpass}} = ‘snr’.
Example: bpnsols=8

\item {} 
\sphinxAtStartPar
\sphinxstylestrong{combine} \textendash{} Data axes to combine for solving. Axes are ‘’, ‘scan’,
‘spw’, ‘field’ or any comma\sphinxhyphen{}separated combination.
Example: combine=’scan,field’

\item {} 
\sphinxAtStartPar
\sphinxstylestrong{refant} \textendash{} List of reference antenna names. Defaults to the
value(s) stored in the pipeline context. If undefined in
the pipeline context defaults to the CASA reference
antenna naming scheme.
Example: refant=’DV06,DV07’

\item {} 
\sphinxAtStartPar
\sphinxstylestrong{solnorm} \textendash{} Normalise the bandpass solution; defaults to True.
Example: solnorm=False

\item {} 
\sphinxAtStartPar
\sphinxstylestrong{minblperant} \textendash{} Minimum number of baselines required per antenna for
each solve. Antennas with fewer baselines are excluded
from solutions.
Example: minblperant=4

\item {} 
\sphinxAtStartPar
\sphinxstylestrong{minsnr} \textendash{} Solutions below this SNR are rejected in the phaseup and
bandpass solves.
Example: minsnr=3.0

\item {} 
\sphinxAtStartPar
\sphinxstylestrong{unregister\_existing} \textendash{} Unregister all bandpass calibrations from the pipeline
context before registering the new bandpass calibrations
from this task. Defaults to False.
Example: unregister\_existing=True

\item {} 
\sphinxAtStartPar
\sphinxstylestrong{hm\_auto\_fillgaps} \textendash{} If True, then the \sphinxcode{\sphinxupquote{hm\_bandpass}} = ‘snr’ or ‘smoothed’
modes, that solve bandpass per SpW, are performed with
CASA bandpass task parameter ‘fillgaps’ set to a quarter
of the respective SpW bandwidth (in channels).
If False, then these bandpass solves will use
fillgaps=0.
The \sphinxcode{\sphinxupquote{hm\_bandpass}} = ‘fixed’ mode is unaffected by
\sphinxcode{\sphinxupquote{hm\_auto\_fillgaps}} and always uses fillgaps=0.

\end{itemize}

\sphinxlineitem{Returns}
\sphinxAtStartPar
The results object for the pipeline task is returned.

\end{description}\end{quote}
\subsubsection*{Examples}

\sphinxAtStartPar
1. Compute a channel bandpass for all visibility files in the pipeline
context using the CASA reference antenna determination scheme:

\begin{sphinxVerbatim}[commandchars=\\\{\}]
\PYG{g+gp}{\PYGZgt{}\PYGZgt{}\PYGZgt{} }\PYG{n}{hifa\PYGZus{}bandpass}\PYG{p}{(}\PYG{p}{)}
\end{sphinxVerbatim}
\begin{enumerate}
\sphinxsetlistlabels{\arabic}{enumi}{enumii}{}{.}%
\setcounter{enumi}{1}
\item {} 
\sphinxAtStartPar
Same as the above but precompute a prioritized reference antenna list:

\end{enumerate}

\begin{sphinxVerbatim}[commandchars=\\\{\}]
\PYG{g+gp}{\PYGZgt{}\PYGZgt{}\PYGZgt{} }\PYG{n}{hif\PYGZus{}refant}\PYG{p}{(}\PYG{p}{)}
\PYG{g+gp}{\PYGZgt{}\PYGZgt{}\PYGZgt{} }\PYG{n}{hifa\PYGZus{}bandpass}\PYG{p}{(}\PYG{p}{)}
\end{sphinxVerbatim}

\end{fulllineitems}


\sphinxstepscope


\section{pipeline.hifa.cli.hifa\_bandpassflag}
\label{\detokenize{_autosummary/pipeline.hifa.cli.hifa_bandpassflag:pipeline-hifa-cli-hifa-bandpassflag}}\label{\detokenize{_autosummary/pipeline.hifa.cli.hifa_bandpassflag::doc}}\index{hifa\_bandpassflag() (in module pipeline.hifa.cli)@\spxentry{hifa\_bandpassflag()}\spxextra{in module pipeline.hifa.cli}}

\begin{fulllineitems}
\phantomsection\label{\detokenize{_autosummary/pipeline.hifa.cli.hifa_bandpassflag:pipeline.hifa.cli.hifa_bandpassflag}}
\pysigstartsignatures
\pysiglinewithargsret
{\sphinxbfcode{\sphinxupquote{hifa\_bandpassflag}}}
{\sphinxparam{\DUrole{n}{vis}\DUrole{o}{=}\DUrole{default_value}{None}}\sphinxparamcomma \sphinxparam{\DUrole{n}{caltable}\DUrole{o}{=}\DUrole{default_value}{None}}\sphinxparamcomma \sphinxparam{\DUrole{n}{intent}\DUrole{o}{=}\DUrole{default_value}{None}}\sphinxparamcomma \sphinxparam{\DUrole{n}{field}\DUrole{o}{=}\DUrole{default_value}{None}}\sphinxparamcomma \sphinxparam{\DUrole{n}{spw}\DUrole{o}{=}\DUrole{default_value}{None}}\sphinxparamcomma \sphinxparam{\DUrole{n}{antenna}\DUrole{o}{=}\DUrole{default_value}{None}}\sphinxparamcomma \sphinxparam{\DUrole{n}{hm\_phaseup}\DUrole{o}{=}\DUrole{default_value}{None}}\sphinxparamcomma \sphinxparam{\DUrole{n}{phaseupsolint}\DUrole{o}{=}\DUrole{default_value}{None}}\sphinxparamcomma \sphinxparam{\DUrole{n}{phaseupbw}\DUrole{o}{=}\DUrole{default_value}{None}}\sphinxparamcomma \sphinxparam{\DUrole{n}{phaseupsnr}\DUrole{o}{=}\DUrole{default_value}{None}}\sphinxparamcomma \sphinxparam{\DUrole{n}{phaseupnsols}\DUrole{o}{=}\DUrole{default_value}{None}}\sphinxparamcomma \sphinxparam{\DUrole{n}{hm\_bandpass}\DUrole{o}{=}\DUrole{default_value}{None}}\sphinxparamcomma \sphinxparam{\DUrole{n}{solint}\DUrole{o}{=}\DUrole{default_value}{None}}\sphinxparamcomma \sphinxparam{\DUrole{n}{maxchannels}\DUrole{o}{=}\DUrole{default_value}{None}}\sphinxparamcomma \sphinxparam{\DUrole{n}{evenbpints}\DUrole{o}{=}\DUrole{default_value}{None}}\sphinxparamcomma \sphinxparam{\DUrole{n}{bpsnr}\DUrole{o}{=}\DUrole{default_value}{None}}\sphinxparamcomma \sphinxparam{\DUrole{n}{minbpsnr}\DUrole{o}{=}\DUrole{default_value}{None}}\sphinxparamcomma \sphinxparam{\DUrole{n}{bpnsols}\DUrole{o}{=}\DUrole{default_value}{None}}\sphinxparamcomma \sphinxparam{\DUrole{n}{combine}\DUrole{o}{=}\DUrole{default_value}{None}}\sphinxparamcomma \sphinxparam{\DUrole{n}{refant}\DUrole{o}{=}\DUrole{default_value}{None}}\sphinxparamcomma \sphinxparam{\DUrole{n}{minblperant}\DUrole{o}{=}\DUrole{default_value}{None}}\sphinxparamcomma \sphinxparam{\DUrole{n}{minsnr}\DUrole{o}{=}\DUrole{default_value}{None}}\sphinxparamcomma \sphinxparam{\DUrole{n}{solnorm}\DUrole{o}{=}\DUrole{default_value}{None}}\sphinxparamcomma \sphinxparam{\DUrole{n}{antnegsig}\DUrole{o}{=}\DUrole{default_value}{None}}\sphinxparamcomma \sphinxparam{\DUrole{n}{antpossig}\DUrole{o}{=}\DUrole{default_value}{None}}\sphinxparamcomma \sphinxparam{\DUrole{n}{tmantint}\DUrole{o}{=}\DUrole{default_value}{None}}\sphinxparamcomma \sphinxparam{\DUrole{n}{tmint}\DUrole{o}{=}\DUrole{default_value}{None}}\sphinxparamcomma \sphinxparam{\DUrole{n}{tmbl}\DUrole{o}{=}\DUrole{default_value}{None}}\sphinxparamcomma \sphinxparam{\DUrole{n}{antblnegsig}\DUrole{o}{=}\DUrole{default_value}{None}}\sphinxparamcomma \sphinxparam{\DUrole{n}{antblpossig}\DUrole{o}{=}\DUrole{default_value}{None}}\sphinxparamcomma \sphinxparam{\DUrole{n}{relaxed\_factor}\DUrole{o}{=}\DUrole{default_value}{None}}\sphinxparamcomma \sphinxparam{\DUrole{n}{niter}\DUrole{o}{=}\DUrole{default_value}{None}}\sphinxparamcomma \sphinxparam{\DUrole{n}{hm\_auto\_fillgaps}\DUrole{o}{=}\DUrole{default_value}{None}}}
{}
\pysigstopsignatures
\sphinxAtStartPar
Bandpass calibration flagging

\sphinxAtStartPar
This task performs a preliminary phased\sphinxhyphen{}up bandpass solution and temporarily
applies it, then computes the flagging heuristics by calling
hif\_correctedampflag which looks for outlier visibility points by statistically
examining the scalar difference of the corrected amplitudes minus model
amplitudes, and then flags those outliers. The philosophy is that only outlier
data points that have remained outliers after calibration will be flagged. Note
that the phase of the data is not assessed.

\sphinxAtStartPar
Plots are generated at two points in this workflow: after bandpass calibration
but before flagging heuristics are run, and after flagging heuristics have been
run and applied. If no points were flagged, the ‘after’ plots are not generated
or displayed.
\begin{quote}\begin{description}
\sphinxlineitem{Parameters}\begin{itemize}
\item {} 
\sphinxAtStartPar
\sphinxstylestrong{vis} \textendash{} List of input MeasurementSets. Defaults to the list of
MeasurementSets specified in the pipeline context.
Example: vis={[}‘ngc5921.ms’{]}

\item {} 
\sphinxAtStartPar
\sphinxstylestrong{caltable} \textendash{} List of names for the output calibration tables. Defaults
to the standard pipeline naming convention.
Example: caltable={[}‘ngc5921.gcal’{]}

\item {} 
\sphinxAtStartPar
\sphinxstylestrong{intent} \textendash{} A string containing a comma delimited list of intents against
which the selected fields are matched. Set to intent=’’ by default, which
means the task will select all data with the BANDPASS intent.
Example: intent=’\sphinxtitleref{*PHASE*}’

\item {} 
\sphinxAtStartPar
\sphinxstylestrong{field} \textendash{} The list of field names or field ids for which bandpasses are
computed. Set to field=’’ by default, which means the task
will select all fields.
Example: field=’3C279’, field=’3C279,M82’

\item {} 
\sphinxAtStartPar
\sphinxstylestrong{spw} \textendash{} The list of spectral windows and channels for which bandpasses are
computed. Set to spw=’’ by default, which means the task will select all
science spectral windows.
Example: spw=’11,13,15,17’

\item {} 
\sphinxAtStartPar
\sphinxstylestrong{antenna} \textendash{} Set of data selection antenna IDs

\item {} 
\sphinxAtStartPar
\sphinxstylestrong{hm\_phaseup} \textendash{} The pre\sphinxhyphen{}bandpass solution phaseup gain heuristics. The options are:
‘snr’: compute solution required to achieve the specified SNR
‘manual’: use manual solution parameters
‘’: skip phaseup
Example: hm\_phaseup=’manual’

\item {} 
\sphinxAtStartPar
\sphinxstylestrong{phaseupsolint} \textendash{} The phase correction solution interval in CASA syntax.
Used when \sphinxcode{\sphinxupquote{hm\_phaseup}} = ‘manual’ or as a default if the
\sphinxcode{\sphinxupquote{hm\_phaseup}} = ‘snr’ heuristic computation fails.
Example: phaseupsolint=’300s’

\item {} 
\sphinxAtStartPar
\sphinxstylestrong{phaseupbw} \textendash{} Bandwidth to be used for phaseup. Used when
\sphinxcode{\sphinxupquote{hm\_phaseup}} = ‘manual’.
Example:
phaseupbw=’’ to use entire bandpass
phaseupbw=’500MHz’ to use central 500MHz

\item {} 
\sphinxAtStartPar
\sphinxstylestrong{phaseupsnr} \textendash{} The required SNR for the phaseup solution. Used only if
hm\_phaseup=’snr’.
Example: phaseupsnr=10.0

\item {} 
\sphinxAtStartPar
\sphinxstylestrong{phaseupnsols} \textendash{} The minimum number of phaseup gain solutions. Used only if
hm\_phaseup=’snr’.
Example: phaseupnsols=4

\item {} 
\sphinxAtStartPar
\sphinxstylestrong{hm\_bandpass} \textendash{} The bandpass solution heuristics. The options are:
‘snr’: compute the solution required to achieve the specified SNR
‘smoothed’: simple ‘smoothing’ i.e. spectral solint\textgreater{}1chan
‘fixed’: use the user defined parameters for all spws

\item {} 
\sphinxAtStartPar
\sphinxstylestrong{solint} \textendash{} Time and channel solution intervals in CASA syntax.
Default is solint=’inf’, which is used when
\sphinxcode{\sphinxupquote{hm\_bandpass}} = ‘fixed’.
If \sphinxcode{\sphinxupquote{hm\_bandpass}} = ‘snr’, then the task will attempt to
compute and use an optimal SNR\sphinxhyphen{}based solint (and warn if this
solint is not good enough).
If \sphinxcode{\sphinxupquote{hm\_bandpass}} = ‘smoothed’, the task will override the
spectral solint with bandwidth/maxchannels.

\item {} 
\sphinxAtStartPar
\sphinxstylestrong{maxchannels} \textendash{} The bandpass solution ‘smoothing’ factor in channels, i.e.
spectral solint will be set to bandwidth/maxchannels
Set to 0 for no smoothing.
Used if \sphinxcode{\sphinxupquote{hm\_bandpass}} = ‘smoothed’.
Example: maxchannels=240

\item {} 
\sphinxAtStartPar
\sphinxstylestrong{evenbpints} \textendash{} Force the per spw frequency solint to be evenly divisible
into the spw bandpass if \sphinxcode{\sphinxupquote{hm\_bandpass}} = ‘snr’.
Example: evenbpints=False

\item {} 
\sphinxAtStartPar
\sphinxstylestrong{bpsnr} \textendash{} The required SNR for the bandpass solution. Used only if
\sphinxcode{\sphinxupquote{hm\_bandpass}} = ‘snr’.
Example: bpsnr=30.0

\item {} 
\sphinxAtStartPar
\sphinxstylestrong{minbpsnr} \textendash{} The minimum required SNR for the bandpass solution
when strong atmospheric lines exist in Tsys spectra.
Used only if \sphinxcode{\sphinxupquote{hm\_bandpass}} = ‘snr’.
Example: minbpsnr=10.0

\item {} 
\sphinxAtStartPar
\sphinxstylestrong{bpnsols} \textendash{} The minimum number of bandpass solutions. Used only if
\sphinxcode{\sphinxupquote{hm\_bandpass}} = ‘snr’.
Example: bpnsols=8

\item {} 
\sphinxAtStartPar
\sphinxstylestrong{combine} \textendash{} Data axes to combine for solving. Axes are ‘’, ‘scan’, ‘spw’,
‘field’ or any comma\sphinxhyphen{}separated combination.
Example: combine=’scan,field’

\item {} 
\sphinxAtStartPar
\sphinxstylestrong{refant} \textendash{} List of reference antenna names. Defaults to the value(s) stored in the
pipeline context. If undefined in the pipeline context defaults to
the CASA reference antenna naming scheme.
Example: refant=’DV06,DV07’

\item {} 
\sphinxAtStartPar
\sphinxstylestrong{minblperant} \textendash{} Minimum number of baselines required per antenna for each solve.
Antennas with fewer baselines are excluded from solutions.
Example: minblperant=4

\item {} 
\sphinxAtStartPar
\sphinxstylestrong{minsnr} \textendash{} Solutions below this SNR are rejected.
Example: minsnr=3.0

\item {} 
\sphinxAtStartPar
\sphinxstylestrong{solnorm} \textendash{} Normalise the bandpass solution; defaults to True.

\item {} 
\sphinxAtStartPar
\sphinxstylestrong{antnegsig} \textendash{} Lower sigma threshold for identifying outliers as a result of
bad antennas within individual timestamps.
Example: antnegsig=4.0

\item {} 
\sphinxAtStartPar
\sphinxstylestrong{antpossig} \textendash{} Upper sigma threshold for identifying outliers as a result of
bad antennas within individual timestamps.
Example: antpossig=4.6

\item {} 
\sphinxAtStartPar
\sphinxstylestrong{tmantint} \textendash{} Threshold for maximum fraction of timestamps that are allowed
to contain outliers.
Example: tmantint=0.063

\item {} 
\sphinxAtStartPar
\sphinxstylestrong{tmint} \textendash{} Initial threshold for maximum fraction of ‘outlier timestamps’
over ‘total timestamps’ that a baseline may be a part of.
Example: tmint=0.085

\item {} 
\sphinxAtStartPar
\sphinxstylestrong{tmbl} \textendash{} Initial threshold for maximum fraction of ‘bad baselines’ over
‘all baselines’ that an antenna may be a part of.
Example: tmbl=0.175

\item {} 
\sphinxAtStartPar
\sphinxstylestrong{antblnegsig} \textendash{} Lower sigma threshold for identifying outliers as a result of
‘bad baselines’ and/or ‘bad antennas’ within baselines (across all
timestamps).
Example: antblnegsig=3.4

\item {} 
\sphinxAtStartPar
\sphinxstylestrong{antblpossig} \textendash{} Upper sigma threshold for identifying outliers as a result of
‘bad baselines’ and/or ‘bad antennas’ within baselines (across all
timestamps).
Example: antblpossig=3.2

\item {} 
\sphinxAtStartPar
\sphinxstylestrong{relaxed\_factor} \textendash{} Relaxed value to set the threshold scaling factor to under
certain conditions (see documentation of the underlying correctedampflag task).
Example: relaxed\_factor=2.0

\item {} 
\sphinxAtStartPar
\sphinxstylestrong{niter} \textendash{} Maximum number of times to iterate on evaluation of flagging
heuristics. If an iteration results in no new flags, then subsequent
iterations are skipped.
Example: niter=2

\item {} 
\sphinxAtStartPar
\sphinxstylestrong{hm\_auto\_fillgaps} \textendash{} If True, then the \sphinxcode{\sphinxupquote{hm\_bandpass}} = ‘snr’ or ‘smoothed’
modes, that solve bandpass per SpW, are performed with
CASA bandpass task parameter ‘fillgaps’ set to a quarter
of the respective SpW bandwidth (in channels).
If False, then these bandpass solves will use
fillgaps=0.
The \sphinxcode{\sphinxupquote{hm\_bandpass}} = ‘fixed’ mode is unaffected by
\sphinxcode{\sphinxupquote{hm\_auto\_fillgaps}} and always uses fillgaps=0.

\end{itemize}

\sphinxlineitem{Returns}
\sphinxAtStartPar
The results object for the pipeline task is returned.

\end{description}\end{quote}
\subsubsection*{Examples}

\sphinxAtStartPar
1. run with recommended settings to create bandpass solution with flagging
using recommended thresholds:

\begin{sphinxVerbatim}[commandchars=\\\{\}]
\PYG{g+gp}{\PYGZgt{}\PYGZgt{}\PYGZgt{} }\PYG{n}{hifa\PYGZus{}bandpassflag}\PYG{p}{(}\PYG{p}{)}
\end{sphinxVerbatim}

\end{fulllineitems}


\sphinxstepscope


\section{pipeline.hifa.cli.hifa\_bpsolint}
\label{\detokenize{_autosummary/pipeline.hifa.cli.hifa_bpsolint:pipeline-hifa-cli-hifa-bpsolint}}\label{\detokenize{_autosummary/pipeline.hifa.cli.hifa_bpsolint::doc}}\index{hifa\_bpsolint() (in module pipeline.hifa.cli)@\spxentry{hifa\_bpsolint()}\spxextra{in module pipeline.hifa.cli}}

\begin{fulllineitems}
\phantomsection\label{\detokenize{_autosummary/pipeline.hifa.cli.hifa_bpsolint:pipeline.hifa.cli.hifa_bpsolint}}
\pysigstartsignatures
\pysiglinewithargsret
{\sphinxbfcode{\sphinxupquote{hifa\_bpsolint}}}
{\sphinxparam{\DUrole{n}{vis}\DUrole{o}{=}\DUrole{default_value}{None}}\sphinxparamcomma \sphinxparam{\DUrole{n}{field}\DUrole{o}{=}\DUrole{default_value}{None}}\sphinxparamcomma \sphinxparam{\DUrole{n}{intent}\DUrole{o}{=}\DUrole{default_value}{None}}\sphinxparamcomma \sphinxparam{\DUrole{n}{spw}\DUrole{o}{=}\DUrole{default_value}{None}}\sphinxparamcomma \sphinxparam{\DUrole{n}{phaseupsnr}\DUrole{o}{=}\DUrole{default_value}{None}}\sphinxparamcomma \sphinxparam{\DUrole{n}{minphaseupints}\DUrole{o}{=}\DUrole{default_value}{None}}\sphinxparamcomma \sphinxparam{\DUrole{n}{evenbpints}\DUrole{o}{=}\DUrole{default_value}{None}}\sphinxparamcomma \sphinxparam{\DUrole{n}{bpsnr}\DUrole{o}{=}\DUrole{default_value}{None}}\sphinxparamcomma \sphinxparam{\DUrole{n}{minbpsnr}\DUrole{o}{=}\DUrole{default_value}{None}}\sphinxparamcomma \sphinxparam{\DUrole{n}{minbpnchan}\DUrole{o}{=}\DUrole{default_value}{None}}\sphinxparamcomma \sphinxparam{\DUrole{n}{hm\_nantennas}\DUrole{o}{=}\DUrole{default_value}{None}}\sphinxparamcomma \sphinxparam{\DUrole{n}{maxfracflagged}\DUrole{o}{=}\DUrole{default_value}{None}}}
{}
\pysigstopsignatures
\sphinxAtStartPar
Compute optimal bandpass calibration solution intervals

\sphinxAtStartPar
The optimal bandpass phaseup time and frequency solution intervals required
to achieve the required signal\sphinxhyphen{}to\sphinxhyphen{}noise ratio is estimated based on nominal
ALMA array characteristics the metadata associated with the observation.

\sphinxAtStartPar
The phaseup gain time and bandpass frequency intervals are determined as
follows:
\begin{itemize}
\item {} 
\sphinxAtStartPar
For each data set the list of source(s) to use for bandpass solution
signal\sphinxhyphen{}to\sphinxhyphen{}noise estimation is compiled based on the values of the
\sphinxcode{\sphinxupquote{field}}, \sphinxcode{\sphinxupquote{intent}}, and \sphinxcode{\sphinxupquote{spw}} parameters.

\item {} 
\sphinxAtStartPar
Source fluxes are determined for each spw and source combination.

\item {} 
\sphinxAtStartPar
Fluxes in Jy are derived from the pipeline context.

\item {} 
\sphinxAtStartPar
Pipeline context fluxes are derived from the online flux calibrator
catalog, the ASDM, or the user via the flux.csv file.

\item {} 
\sphinxAtStartPar
If no fluxes are available the task terminates.

\item {} 
\sphinxAtStartPar
Atmospheric calibration and observations scans are determined for each
spw and source combination.

\item {} 
\sphinxAtStartPar
If \sphinxcode{\sphinxupquote{intent}} is set to ‘PHASE’ are there are no atmospheric scans
associated with the ‘PHASE’ calibrator, ‘TARGET’ atmospheric scans will be
used instead.

\item {} 
\sphinxAtStartPar
If atmospheric scans cannot be associated with any of the spw and source
combinations the task terminates.

\item {} 
\sphinxAtStartPar
Science spws are mapped to atmospheric spws for each science spw and
source combination.

\item {} 
\sphinxAtStartPar
If mappings cannot be determined for any of the spws the task terminates.

\item {} 
\sphinxAtStartPar
The median Tsys value for each atmospheric spw and source combination is
determined from the SYSCAL table. Medians are computed first by channel,
then by antenna, in order to reduce sensitivity to deviant values.

\item {} 
\sphinxAtStartPar
The science spw parameters, exposure time(s), and integration time(s) are
determined.

\item {} 
\sphinxAtStartPar
The phase\sphinxhyphen{}up time interval, in time units and number of integrations
required to meet the \sphinxcode{\sphinxupquote{phaseupsnr}} are computed, along with the
sensitivity in mJy and the signal\sphinxhyphen{}to\sphinxhyphen{}noise per integration. Nominal Tsys
and sensitivity values per receiver band provided by the ALMA project are
used for this estimate.

\item {} 
\sphinxAtStartPar
Warnings are issued if estimated phase\sphinxhyphen{}up gain time solution would contain
fewer than \sphinxcode{\sphinxupquote{minphaseupints}} solutions.

\item {} 
\sphinxAtStartPar
The frequency interval, in MHz and number of channels required to meet the
\sphinxcode{\sphinxupquote{bpsnr}} are computed, along with the per channel sensitivity in mJy and
the per channel signal\sphinxhyphen{}to\sphinxhyphen{}noise. Nominal Tsys and sensitivity values per
receiver band provided by the ALMA project are used for this estimate.

\item {} 
\sphinxAtStartPar
Warnings are issued if estimated bandpass solution would contain fewer
than \sphinxcode{\sphinxupquote{minbpnchan}} solutions.

\item {} 
\sphinxAtStartPar
If strong atmospheric features are detected in the Tsys spectrum for a
given spw, the frequency interval of bandpass solution is recalculated to
meet the lower threshold, \sphinxcode{\sphinxupquote{minbpsnr}} \sphinxhyphen{} i.e. a lower snr is tolerated in
order to preserve enough frequency intervals to capture the atmospheric
line.

\end{itemize}
\begin{quote}\begin{description}
\sphinxlineitem{Parameters}\begin{itemize}
\item {} 
\sphinxAtStartPar
\sphinxstylestrong{vis} \textendash{} The list of input MeasurementSets. Defaults to the list of
MeasurementSets specified in the pipeline context.
example: vis={[}‘M82A.ms’, ‘M82B.ms’{]}

\item {} 
\sphinxAtStartPar
\sphinxstylestrong{field} \textendash{} The list of field names of sources to be used for
signal\sphinxhyphen{}to\sphinxhyphen{}noise estimation. Defaults to all fields with the
standard intent.
example: field=’3C279’

\item {} 
\sphinxAtStartPar
\sphinxstylestrong{intent} \textendash{} A string containing a comma delimited list of intents against
which the selected fields are matched. Defaults to
‘BANDPASS’.
example: intent=’PHASE’

\item {} 
\sphinxAtStartPar
\sphinxstylestrong{spw} \textendash{} The list of spectral windows and channels for which gain
solutions are computed. Defaults to all the science spectral
windows for which there are both ‘intent’ and TARGET intents.
example: spw=’13,15’

\item {} 
\sphinxAtStartPar
\sphinxstylestrong{phaseupsnr} \textendash{} The required phase\sphinxhyphen{}up gain time interval solution
signal\sphinxhyphen{}to\sphinxhyphen{}noise.
example: phaseupsnr=10.0

\item {} 
\sphinxAtStartPar
\sphinxstylestrong{minphaseupints} \textendash{} The minimum number of time intervals in the phase\sphinxhyphen{}up gain
solution.
example: minphaseupints=4

\item {} 
\sphinxAtStartPar
\sphinxstylestrong{evenbpints} \textendash{} Use a bandpass frequency solint that is an integer divisor of
the spw bandwidth, to prevent the occurrence of one narrower
fractional frequency interval.

\item {} 
\sphinxAtStartPar
\sphinxstylestrong{bpsnr} \textendash{} The required bandpass frequency interval solution
signal\sphinxhyphen{}to\sphinxhyphen{}noise.
example: bpsnr=30.0

\item {} 
\sphinxAtStartPar
\sphinxstylestrong{minbpsnr} \textendash{} The minimum required bandpass frequency interval solution
signal\sphinxhyphen{}to\sphinxhyphen{}noise when strong atmospheric lines exist in Tsys
spectra.
example: minbpsnr=10.0

\item {} 
\sphinxAtStartPar
\sphinxstylestrong{minbpnchan} \textendash{} The minimum number of frequency intervals in the bandpass
solution.
example: minbpnchan=16

\item {} 
\sphinxAtStartPar
\sphinxstylestrong{hm\_nantennas} \textendash{} The heuristics for determines the number of antennas to use
in the signal\sphinxhyphen{}to\sphinxhyphen{}noise estimate. The options are ‘all’ and
‘unflagged’. The ‘unflagged’ options is not currently
supported.
example: hm\_nantennas=’unflagged’

\item {} 
\sphinxAtStartPar
\sphinxstylestrong{maxfracflagged} \textendash{} The maximum fraction of an antenna that can be flagged before
it is excluded from the signal\sphinxhyphen{}to\sphinxhyphen{}noise estimate.
example: maxfracflagged=0.80

\end{itemize}

\sphinxlineitem{Returns}
\sphinxAtStartPar
The results object for the pipeline task is returned.

\end{description}\end{quote}
\subsubsection*{Examples}

\sphinxAtStartPar
1. Estimate the phaseup gain time interval and the bandpass frequency
interval required to match the desired signal\sphinxhyphen{}to\sphinxhyphen{}noise for bandpass
solutions:

\begin{sphinxVerbatim}[commandchars=\\\{\}]
\PYG{g+gp}{\PYGZgt{}\PYGZgt{}\PYGZgt{} }\PYG{n}{hifa\PYGZus{}bpsolint}\PYG{p}{(}\PYG{p}{)}
\end{sphinxVerbatim}

\end{fulllineitems}


\sphinxstepscope


\section{pipeline.hifa.cli.hifa\_diffgaincal}
\label{\detokenize{_autosummary/pipeline.hifa.cli.hifa_diffgaincal:pipeline-hifa-cli-hifa-diffgaincal}}\label{\detokenize{_autosummary/pipeline.hifa.cli.hifa_diffgaincal::doc}}\index{hifa\_diffgaincal() (in module pipeline.hifa.cli)@\spxentry{hifa\_diffgaincal()}\spxextra{in module pipeline.hifa.cli}}

\begin{fulllineitems}
\phantomsection\label{\detokenize{_autosummary/pipeline.hifa.cli.hifa_diffgaincal:pipeline.hifa.cli.hifa_diffgaincal}}
\pysigstartsignatures
\pysiglinewithargsret
{\sphinxbfcode{\sphinxupquote{hifa\_diffgaincal}}}
{\sphinxparam{\DUrole{n}{vis}\DUrole{o}{=}\DUrole{default_value}{None}}}
{}
\pysigstopsignatures
\sphinxAtStartPar
Derive SpW phase offsets from differential gain calibrator.

\sphinxAtStartPar
This task creates the spectral window phase gain offset table used to allow
calibrating the “on\sphinxhyphen{}source” spectral setup with phase gains from a
“reference” spectral setup. A bright point source Quasar, called the
Differential Gain Calibrator (DIFFGAIN) source, is used for this purpose.
This DIFFGAIN source typically observed in groups of interleaved “reference”
and “on\sphinxhyphen{}source” scans, once at the start and once at the end of the
observations. In very long observations, there may be a group of scans
occurring during the middle. Scan groups are combined while solving for SpW
offsets between “reference” and “on\sphinxhyphen{}source” spectral setups.
\begin{quote}\begin{description}
\sphinxlineitem{Parameters}
\sphinxAtStartPar
\sphinxstylestrong{vis} \textendash{} The list of input MeasurementSets. Defaults to the list of
MeasurementSets specified in the pipeline context.
Example: {[}‘M32A.ms’, ‘M32B.ms’{]}

\sphinxlineitem{Returns}
\sphinxAtStartPar
The results object for the pipeline task is returned.

\end{description}\end{quote}
\subsubsection*{Examples}
\begin{enumerate}
\sphinxsetlistlabels{\arabic}{enumi}{enumii}{}{.}%
\item {} 
\sphinxAtStartPar
Derive SpW phase offsets from differential gain calibrator.

\end{enumerate}

\begin{sphinxVerbatim}[commandchars=\\\{\}]
\PYG{g+gp}{\PYGZgt{}\PYGZgt{}\PYGZgt{} }\PYG{n}{hifa\PYGZus{}diffgaincal}\PYG{p}{(}\PYG{p}{)}
\end{sphinxVerbatim}

\end{fulllineitems}


\sphinxstepscope


\section{pipeline.hifa.cli.hifa\_exportdata}
\label{\detokenize{_autosummary/pipeline.hifa.cli.hifa_exportdata:pipeline-hifa-cli-hifa-exportdata}}\label{\detokenize{_autosummary/pipeline.hifa.cli.hifa_exportdata::doc}}\index{hifa\_exportdata() (in module pipeline.hifa.cli)@\spxentry{hifa\_exportdata()}\spxextra{in module pipeline.hifa.cli}}

\begin{fulllineitems}
\phantomsection\label{\detokenize{_autosummary/pipeline.hifa.cli.hifa_exportdata:pipeline.hifa.cli.hifa_exportdata}}
\pysigstartsignatures
\pysiglinewithargsret
{\sphinxbfcode{\sphinxupquote{hifa\_exportdata}}}
{\sphinxparam{\DUrole{n}{vis}\DUrole{o}{=}\DUrole{default_value}{None}}\sphinxparamcomma \sphinxparam{\DUrole{n}{session}\DUrole{o}{=}\DUrole{default_value}{None}}\sphinxparamcomma \sphinxparam{\DUrole{n}{imaging\_products\_only}\DUrole{o}{=}\DUrole{default_value}{None}}\sphinxparamcomma \sphinxparam{\DUrole{n}{exportmses}\DUrole{o}{=}\DUrole{default_value}{None}}\sphinxparamcomma \sphinxparam{\DUrole{n}{pprfile}\DUrole{o}{=}\DUrole{default_value}{None}}\sphinxparamcomma \sphinxparam{\DUrole{n}{calintents}\DUrole{o}{=}\DUrole{default_value}{None}}\sphinxparamcomma \sphinxparam{\DUrole{n}{calimages}\DUrole{o}{=}\DUrole{default_value}{None}}\sphinxparamcomma \sphinxparam{\DUrole{n}{targetimages}\DUrole{o}{=}\DUrole{default_value}{None}}\sphinxparamcomma \sphinxparam{\DUrole{n}{products\_dir}\DUrole{o}{=}\DUrole{default_value}{None}}}
{}
\pysigstopsignatures
\sphinxAtStartPar
Prepare interferometry data for export

\sphinxAtStartPar
The hifa\_exportdata task for ALMA CASA pipeline exports the data defined
in the pipeline context and exports it to the data products directory,
converting and or packing it as necessary.

\sphinxAtStartPar
The current version of the task exports the following products
\begin{itemize}
\item {} 
\sphinxAtStartPar
an XML file containing the pipeline processing request

\item {} 
\sphinxAtStartPar
a tar file per ASDM / MS containing the final flags version

\item {} 
\sphinxAtStartPar
a text file per ASDM / MS containing the final calibration apply list

\item {} 
\sphinxAtStartPar
a FITS image for each selected calibrator source image

\item {} 
\sphinxAtStartPar
a FITS image for each selected science target source image

\item {} 
\sphinxAtStartPar
a tar file per session containing the caltables for that session

\item {} 
\sphinxAtStartPar
a tar file containing the file web log

\item {} 
\sphinxAtStartPar
a text file containing the final list of CASA commands

\item {} 
\sphinxAtStartPar
an XML “manifest” file listing the products

\item {} 
\sphinxAtStartPar
an XML “aquareport” file listing the QA scores and sub\sphinxhyphen{}scores,
image sensitivities, and other numerical information

\end{itemize}
\begin{quote}\begin{description}
\sphinxlineitem{Parameters}\begin{itemize}
\item {} 
\sphinxAtStartPar
\sphinxstylestrong{vis} \textendash{} List of visibility data files for which flagging and calibration
information will be exported. Defaults to the list maintained in the
pipeline context.
Example: vis={[}‘X227.ms’, ‘X228.ms’{]}

\item {} 
\sphinxAtStartPar
\sphinxstylestrong{session} \textendash{} List of sessions one per visibility file. Currently defaults
to a single virtual session containing all the visibility files in vis.
In the future, this will default to the set of observing sessions defined
in the context.
Example: session={[}‘session1’, ‘session2’{]}

\item {} 
\sphinxAtStartPar
\sphinxstylestrong{imaging\_products\_only} \textendash{} Export science target imaging products only

\item {} 
\sphinxAtStartPar
\sphinxstylestrong{exportmses} \textendash{} Export the final MeasurementSets instead of the final flags,
calibration tables, and calibration instructions.

\item {} 
\sphinxAtStartPar
\sphinxstylestrong{pprfile} \textendash{} Name of the pipeline processing request to be exported. Defaults
to a file matching the template ‘PPR\_*.xml’.
Example: pprfile={[}‘PPR\_GRB021004.xml’{]}

\item {} 
\sphinxAtStartPar
\sphinxstylestrong{calintents} \textendash{} List of calibrator image types to be exported. Defaults to
all standard calibrator intents, ‘BANDPASS’, ‘PHASE’, ‘FLUX’.
Example: ‘PHASE’

\item {} 
\sphinxAtStartPar
\sphinxstylestrong{calimages} \textendash{} List of calibrator images to be exported. Defaults to all
calibrator images recorded in the pipeline context.
Example: calimages={[}‘3C454.3.bandpass’, ‘3C279.phase’{]}

\item {} 
\sphinxAtStartPar
\sphinxstylestrong{targetimages} \textendash{} List of science target images to be exported. Defaults to all
science target images recorded in the pipeline context.
Example: targetimages={[}‘NGC3256.band3’, ‘NGC3256.band6’{]}

\item {} 
\sphinxAtStartPar
\sphinxstylestrong{products\_dir} \textendash{} Name of the data products subdirectory. Defaults to ‘./’.
Example: products\_dir=’../products’

\end{itemize}

\sphinxlineitem{Returns}
\sphinxAtStartPar
The results object for the pipeline task is returned.

\end{description}\end{quote}
\subsubsection*{Examples}

\sphinxAtStartPar
1. Export the pipeline results for a single session to the data products
directory:

\begin{sphinxVerbatim}[commandchars=\\\{\}]
\PYGZgt{}\PYGZgt{}\PYGZgt{} !mkdir ../products
\PYGZgt{}\PYGZgt{}\PYGZgt{} hif\PYGZus{}exportdata(products\PYGZus{}dir=\PYGZsq{}../products\PYGZsq{})
\end{sphinxVerbatim}

\sphinxAtStartPar
2. Export the pipeline results to the data products directory specify that
only the gain calibrator images be saved:

\begin{sphinxVerbatim}[commandchars=\\\{\}]
\PYGZgt{}\PYGZgt{}\PYGZgt{} !mkdir ../products
\PYGZgt{}\PYGZgt{}\PYGZgt{} hif\PYGZus{}exportdata(products\PYGZus{}dir=\PYGZsq{}../products\PYGZsq{}, calintents=\PYGZsq{}*PHASE*\PYGZsq{})
\end{sphinxVerbatim}

\end{fulllineitems}


\sphinxstepscope


\section{pipeline.hifa.cli.hifa\_flagdata}
\label{\detokenize{_autosummary/pipeline.hifa.cli.hifa_flagdata:pipeline-hifa-cli-hifa-flagdata}}\label{\detokenize{_autosummary/pipeline.hifa.cli.hifa_flagdata::doc}}\index{hifa\_flagdata() (in module pipeline.hifa.cli)@\spxentry{hifa\_flagdata()}\spxextra{in module pipeline.hifa.cli}}

\begin{fulllineitems}
\phantomsection\label{\detokenize{_autosummary/pipeline.hifa.cli.hifa_flagdata:pipeline.hifa.cli.hifa_flagdata}}
\pysigstartsignatures
\pysiglinewithargsret
{\sphinxbfcode{\sphinxupquote{hifa\_flagdata}}}
{\sphinxparam{\DUrole{n}{vis}\DUrole{o}{=}\DUrole{default_value}{None}}\sphinxparamcomma \sphinxparam{\DUrole{n}{autocorr}\DUrole{o}{=}\DUrole{default_value}{None}}\sphinxparamcomma \sphinxparam{\DUrole{n}{shadow}\DUrole{o}{=}\DUrole{default_value}{None}}\sphinxparamcomma \sphinxparam{\DUrole{n}{tolerance}\DUrole{o}{=}\DUrole{default_value}{None}}\sphinxparamcomma \sphinxparam{\DUrole{n}{scan}\DUrole{o}{=}\DUrole{default_value}{None}}\sphinxparamcomma \sphinxparam{\DUrole{n}{scannumber}\DUrole{o}{=}\DUrole{default_value}{None}}\sphinxparamcomma \sphinxparam{\DUrole{n}{intents}\DUrole{o}{=}\DUrole{default_value}{None}}\sphinxparamcomma \sphinxparam{\DUrole{n}{edgespw}\DUrole{o}{=}\DUrole{default_value}{None}}\sphinxparamcomma \sphinxparam{\DUrole{n}{fracspw}\DUrole{o}{=}\DUrole{default_value}{None}}\sphinxparamcomma \sphinxparam{\DUrole{n}{fracspwfps}\DUrole{o}{=}\DUrole{default_value}{None}}\sphinxparamcomma \sphinxparam{\DUrole{n}{online}\DUrole{o}{=}\DUrole{default_value}{None}}\sphinxparamcomma \sphinxparam{\DUrole{n}{partialpol}\DUrole{o}{=}\DUrole{default_value}{None}}\sphinxparamcomma \sphinxparam{\DUrole{n}{lowtrans}\DUrole{o}{=}\DUrole{default_value}{None}}\sphinxparamcomma \sphinxparam{\DUrole{n}{mintransrepspw}\DUrole{o}{=}\DUrole{default_value}{None}}\sphinxparamcomma \sphinxparam{\DUrole{n}{mintransnonrepspws}\DUrole{o}{=}\DUrole{default_value}{None}}\sphinxparamcomma \sphinxparam{\DUrole{n}{fileonline}\DUrole{o}{=}\DUrole{default_value}{None}}\sphinxparamcomma \sphinxparam{\DUrole{n}{template}\DUrole{o}{=}\DUrole{default_value}{None}}\sphinxparamcomma \sphinxparam{\DUrole{n}{filetemplate}\DUrole{o}{=}\DUrole{default_value}{None}}\sphinxparamcomma \sphinxparam{\DUrole{n}{hm\_tbuff}\DUrole{o}{=}\DUrole{default_value}{None}}\sphinxparamcomma \sphinxparam{\DUrole{n}{tbuff}\DUrole{o}{=}\DUrole{default_value}{None}}\sphinxparamcomma \sphinxparam{\DUrole{n}{qa0}\DUrole{o}{=}\DUrole{default_value}{None}}\sphinxparamcomma \sphinxparam{\DUrole{n}{qa2}\DUrole{o}{=}\DUrole{default_value}{None}}\sphinxparamcomma \sphinxparam{\DUrole{n}{flagbackup}\DUrole{o}{=}\DUrole{default_value}{None}}}
{}
\pysigstopsignatures
\sphinxAtStartPar
Do metadata based flagging of a list of MeasurementSets.

\sphinxAtStartPar
The hifa\_flagdata data performs basic flagging operations on a list of
measurements including:
\begin{itemize}
\item {} 
\sphinxAtStartPar
applying online flags

\item {} 
\sphinxAtStartPar
applying a flagging template

\item {} 
\sphinxAtStartPar
partial polarization flagging

\item {} 
\sphinxAtStartPar
autocorrelation data flagging

\item {} 
\sphinxAtStartPar
shadowed antenna data flagging

\item {} 
\sphinxAtStartPar
scan\sphinxhyphen{}based flagging by intent or scan number

\item {} 
\sphinxAtStartPar
edge channel flagging, as needed

\item {} 
\sphinxAtStartPar
low atmospheric transmission flagging

\end{itemize}

\sphinxAtStartPar
About the spectral window edge channel flagging:
\begin{itemize}
\item {} 
\sphinxAtStartPar
For TDM spectral windows, a number of edge channels are always flagged,
according to the \sphinxcode{\sphinxupquote{fracspw}} and \sphinxcode{\sphinxupquote{fracspwfps}} parameters (the latter
operates only on spectral windows with 62, 124, or 248 channels). With the
default setting of \sphinxcode{\sphinxupquote{fracspw}}, the number of channels flagged on each edge
is 2, 4, or 8 for 64, 128, or 256\sphinxhyphen{}channel spectral windows, respectively.

\item {} 
\sphinxAtStartPar
For most FDM spectral windows, no edge flagging is done. The only exceptions
are ACA spectral windows that encroach too close to the baseband edge.
Channels that lie closer to the baseband edge than the following values are
flagged: 62.5, 40, 20, 10, and 5 MHz for spectral windows with bandwidths of
1000, 500, 250, 125, and 62.5 MHz, respectively. A warning is generated in
the weblog if flagging occurs due to proximity to the baseband edge.
By definition, 2000 MHz spectral windows always encroach the baseband edge on
both sides of the spectral window, and thus are always flagged on both sides
in order to achieve 1875 MHz bandwidth (in effect, they are flagged by
62.5 MHz on each side), and thus no warning is generated.

\end{itemize}
\begin{quote}\begin{description}
\sphinxlineitem{Parameters}\begin{itemize}
\item {} 
\sphinxAtStartPar
\sphinxstylestrong{vis} \textendash{} The list of input MeasurementSets. Defaults to the list of
MeasurementSets defined in the pipeline context.

\item {} 
\sphinxAtStartPar
\sphinxstylestrong{autocorr} \textendash{} Flag autocorrelation data.

\item {} 
\sphinxAtStartPar
\sphinxstylestrong{shadow} \textendash{} Flag shadowed antennas.

\item {} 
\sphinxAtStartPar
\sphinxstylestrong{tolerance} \textendash{} Amount of antenna shadowing tolerated, in meters. A positive number
allows antennas to overlap in projection. A negative number forces antennas
apart in projection. Zero implies a distance of radius\_1+radius\_2 between
antenna centers.

\item {} 
\sphinxAtStartPar
\sphinxstylestrong{scan} \textendash{} Flag a list of specified scans.

\item {} 
\sphinxAtStartPar
\sphinxstylestrong{scannumber} \textendash{} A string containing a comma delimited list of scans to be
flagged.
Example: scannumber=’3,5,6’

\item {} 
\sphinxAtStartPar
\sphinxstylestrong{intents} \textendash{} A string containing a comma delimited list of intents against
which the scans to be flagged are matched.
Example: intents=’\sphinxtitleref{*BANDPASS*}’

\item {} 
\sphinxAtStartPar
\sphinxstylestrong{edgespw} \textendash{} Flag the edge spectral window channels.

\item {} 
\sphinxAtStartPar
\sphinxstylestrong{fracspw} \textendash{} Fraction of channels to flag at both edges of TDM spectral windows.

\item {} 
\sphinxAtStartPar
\sphinxstylestrong{fracspwfps} \textendash{} Fraction of channels to flag at both edges of ACA TDM
spectral windows that were created with the earlier (original)
implementation of the frequency profile synthesis (FPS) algorithm.

\item {} 
\sphinxAtStartPar
\sphinxstylestrong{online} \textendash{} Apply the online flags.

\item {} 
\sphinxAtStartPar
\sphinxstylestrong{partialpol} \textendash{} Identify integrations in multi\sphinxhyphen{}polarisation data where part
of the polarization products are already flagged, and flag the other
polarization products in those integrations.

\item {} 
\sphinxAtStartPar
\sphinxstylestrong{lowtrans} \textendash{} Flag spectral windows for which a significant fraction of
the channels have atmospheric transmission below the
threshold (\sphinxcode{\sphinxupquote{mintransrepspw}}, \sphinxcode{\sphinxupquote{mintransnonrepspws}}).

\item {} 
\sphinxAtStartPar
\sphinxstylestrong{mintransnonrepspws} \textendash{} This atmospheric transmissivity threshold is used to flag
a non\sphinxhyphen{}representative science spectral window when more than 60\% of
its channels have a transmissivity below this level.

\item {} 
\sphinxAtStartPar
\sphinxstylestrong{mintransrepspw} \textendash{} This atmospheric transmissivity threshold is used to flag the
representative science spectral window when more than 60\% of its channels
have a transmissivity below this level.

\item {} 
\sphinxAtStartPar
\sphinxstylestrong{fileonline} \textendash{} File containing the online flags. These are computed by the
h\_init or hif\_importdata data tasks. If the online flags files
are undefined a name of the form ‘msname.flagonline.txt’ is assumed.

\item {} 
\sphinxAtStartPar
\sphinxstylestrong{template} \textendash{} Apply flagging templates

\item {} 
\sphinxAtStartPar
\sphinxstylestrong{filetemplate} \textendash{} The name of a text file that contains the flagging template
for RFI, birdies, telluric lines, etc. If the template flags files
is undefined a name of the form ‘msname.flagtemplate.txt’ is assumed.

\item {} 
\sphinxAtStartPar
\sphinxstylestrong{hm\_tbuff} \textendash{} The heuristic for computing the default time interval padding
parameter. The options are ‘halfint’ and ‘manual’. In ‘halfint’ mode tbuff
is set to half the maximum of the median integration time of the science
and calibrator target observations. The value of 0.048 seconds is
subtracted from the lower time limit to accommodate the behavior of the
ALMA Control system.

\item {} 
\sphinxAtStartPar
\sphinxstylestrong{tbuff} \textendash{} The time in seconds used to pad flagging command time
intervals if \sphinxcode{\sphinxupquote{hm\_tbuff}} = ‘manual’. The default in
manual mode is no flagging.

\item {} 
\sphinxAtStartPar
\sphinxstylestrong{qa0} \textendash{} QA0 flags.

\item {} 
\sphinxAtStartPar
\sphinxstylestrong{qa2} \textendash{} QA2 flags.

\item {} 
\sphinxAtStartPar
\sphinxstylestrong{flagbackup} \textendash{} Back up any pre\sphinxhyphen{}existing flags.

\end{itemize}

\sphinxlineitem{Returns}
\sphinxAtStartPar
The results object for the pipeline task is returned.

\end{description}\end{quote}
\subsubsection*{Examples}
\begin{enumerate}
\sphinxsetlistlabels{\arabic}{enumi}{enumii}{}{.}%
\item {} 
\sphinxAtStartPar
Do basic flagging on a MeasurementSet:

\end{enumerate}

\begin{sphinxVerbatim}[commandchars=\\\{\}]
\PYG{g+gp}{\PYGZgt{}\PYGZgt{}\PYGZgt{} }\PYG{n}{hifa\PYGZus{}flagdata}\PYG{p}{(}\PYG{p}{)}
\end{sphinxVerbatim}

\sphinxAtStartPar
2. Do basic flagging on a MeasurementSet flagging additional scans selected
by number as well:

\begin{sphinxVerbatim}[commandchars=\\\{\}]
\PYG{g+gp}{\PYGZgt{}\PYGZgt{}\PYGZgt{} }\PYG{n}{hifa\PYGZus{}flagdata}\PYG{p}{(}\PYG{n}{scannumber}\PYG{o}{=}\PYG{l+s+s1}{\PYGZsq{}}\PYG{l+s+s1}{13,18}\PYG{l+s+s1}{\PYGZsq{}}\PYG{p}{)}
\end{sphinxVerbatim}

\end{fulllineitems}


\sphinxstepscope


\section{pipeline.hifa.cli.hifa\_flagtargets}
\label{\detokenize{_autosummary/pipeline.hifa.cli.hifa_flagtargets:pipeline-hifa-cli-hifa-flagtargets}}\label{\detokenize{_autosummary/pipeline.hifa.cli.hifa_flagtargets::doc}}\index{hifa\_flagtargets() (in module pipeline.hifa.cli)@\spxentry{hifa\_flagtargets()}\spxextra{in module pipeline.hifa.cli}}

\begin{fulllineitems}
\phantomsection\label{\detokenize{_autosummary/pipeline.hifa.cli.hifa_flagtargets:pipeline.hifa.cli.hifa_flagtargets}}
\pysigstartsignatures
\pysiglinewithargsret
{\sphinxbfcode{\sphinxupquote{hifa\_flagtargets}}}
{\sphinxparam{\DUrole{n}{vis}\DUrole{o}{=}\DUrole{default_value}{None}}\sphinxparamcomma \sphinxparam{\DUrole{n}{template}\DUrole{o}{=}\DUrole{default_value}{None}}\sphinxparamcomma \sphinxparam{\DUrole{n}{filetemplate}\DUrole{o}{=}\DUrole{default_value}{None}}\sphinxparamcomma \sphinxparam{\DUrole{n}{flagbackup}\DUrole{o}{=}\DUrole{default_value}{None}}}
{}
\pysigstopsignatures
\sphinxAtStartPar
Do science target flagging

\sphinxAtStartPar
The hifa\_flagtargets task performs basic flagging operations on a list of
science target MeasurementSets, including:
\begin{itemize}
\item {} 
\sphinxAtStartPar
applying a flagging template

\end{itemize}
\begin{quote}\begin{description}
\sphinxlineitem{Parameters}\begin{itemize}
\item {} 
\sphinxAtStartPar
\sphinxstylestrong{vis} \textendash{} The list of input MeasurementSets. Defaults to the list
of MeasurementSets defined in the pipeline context.

\item {} 
\sphinxAtStartPar
\sphinxstylestrong{template} \textendash{} Apply flagging templates; defaults to True.

\item {} 
\sphinxAtStartPar
\sphinxstylestrong{filetemplate} \textendash{} The name of a text file that contains the flagging
template for issues with the science target data etc.
If the template flags files is undefined a name of the
form ‘msname\_flagtargetstemplate.txt’ is assumed.

\item {} 
\sphinxAtStartPar
\sphinxstylestrong{flagbackup} \textendash{} Back up any pre\sphinxhyphen{}existing flags; defaults to False.

\end{itemize}

\sphinxlineitem{Returns}
\sphinxAtStartPar
The results object for the pipeline task is returned.

\end{description}\end{quote}
\subsubsection*{Examples}
\begin{enumerate}
\sphinxsetlistlabels{\arabic}{enumi}{enumii}{}{.}%
\item {} 
\sphinxAtStartPar
Do basic flagging on a science target MeasurementSet:

\end{enumerate}

\begin{sphinxVerbatim}[commandchars=\\\{\}]
\PYG{g+gp}{\PYGZgt{}\PYGZgt{}\PYGZgt{} }\PYG{n}{hifa\PYGZus{}flagtargets}\PYG{p}{(}\PYG{p}{)}
\end{sphinxVerbatim}

\end{fulllineitems}


\sphinxstepscope


\section{pipeline.hifa.cli.hifa\_fluxcalflag}
\label{\detokenize{_autosummary/pipeline.hifa.cli.hifa_fluxcalflag:pipeline-hifa-cli-hifa-fluxcalflag}}\label{\detokenize{_autosummary/pipeline.hifa.cli.hifa_fluxcalflag::doc}}\index{hifa\_fluxcalflag() (in module pipeline.hifa.cli)@\spxentry{hifa\_fluxcalflag()}\spxextra{in module pipeline.hifa.cli}}

\begin{fulllineitems}
\phantomsection\label{\detokenize{_autosummary/pipeline.hifa.cli.hifa_fluxcalflag:pipeline.hifa.cli.hifa_fluxcalflag}}
\pysigstartsignatures
\pysiglinewithargsret
{\sphinxbfcode{\sphinxupquote{hifa\_fluxcalflag}}}
{\sphinxparam{\DUrole{n}{vis}\DUrole{o}{=}\DUrole{default_value}{None}}\sphinxparamcomma \sphinxparam{\DUrole{n}{field}\DUrole{o}{=}\DUrole{default_value}{None}}\sphinxparamcomma \sphinxparam{\DUrole{n}{intent}\DUrole{o}{=}\DUrole{default_value}{None}}\sphinxparamcomma \sphinxparam{\DUrole{n}{spw}\DUrole{o}{=}\DUrole{default_value}{None}}\sphinxparamcomma \sphinxparam{\DUrole{n}{threshold}\DUrole{o}{=}\DUrole{default_value}{None}}\sphinxparamcomma \sphinxparam{\DUrole{n}{appendlines}\DUrole{o}{=}\DUrole{default_value}{None}}\sphinxparamcomma \sphinxparam{\DUrole{n}{linesfiles}\DUrole{o}{=}\DUrole{default_value}{None}}\sphinxparamcomma \sphinxparam{\DUrole{n}{applyflags}\DUrole{o}{=}\DUrole{default_value}{None}}}
{}
\pysigstopsignatures
\sphinxAtStartPar
Locate and flag line regions in solar system flux calibrators

\sphinxAtStartPar
Search the built\sphinxhyphen{}in solar system flux calibrator line catalog for overlaps with
the science spectral windows. Generate a list of line overlap regions and
flagging commands.
\begin{quote}\begin{description}
\sphinxlineitem{Parameters}\begin{itemize}
\item {} 
\sphinxAtStartPar
\sphinxstylestrong{vis} \textendash{} The list of input MeasurementSets. Defaults to the list of
MeasurementSets defined in the pipeline context.

\item {} 
\sphinxAtStartPar
\sphinxstylestrong{field} \textendash{} The list of field names or field ids for which the models are
to be set. Defaults to all fields with intent ‘AMPLITUDE’.
Example: field=’3C279’, field=’3C279, M82’

\item {} 
\sphinxAtStartPar
\sphinxstylestrong{intent} \textendash{} A string containing a comma delimited list of intents against
which the selected fields are matched. Defaults to all data
with amplitude intent.
Example: intent=’AMPLITUDE’

\item {} 
\sphinxAtStartPar
\sphinxstylestrong{spw} \textendash{} Spectral windows and channels for which bandpasses are
computed. Defaults to all science spectral windows.
Example: spw=’11,13,15,17’

\item {} 
\sphinxAtStartPar
\sphinxstylestrong{threshold} \textendash{} If the fraction of a spectral window occupied by line regions
is greater than this threshold value, then flag the entire
spectral window.

\item {} 
\sphinxAtStartPar
\sphinxstylestrong{appendlines} \textendash{} Append user defined line regions to the line dictionary.

\item {} 
\sphinxAtStartPar
\sphinxstylestrong{linesfile} \textendash{} Read in a file containing lines regions and append it to the
builtin dictionary. Blank lines and comments beginning with \#
are skipped. The data is contained in 4 whitespace delimited
fields containing the solar system object field name, e.g.
‘Callisto’, the molecular species name, e.g. ‘13CO’, and the
starting and ending frequency in GHz.

\item {} 
\sphinxAtStartPar
\sphinxstylestrong{applyflags} \textendash{} Boolean for whether to apply the generated flag commands. (default True)

\end{itemize}

\sphinxlineitem{Returns}
\sphinxAtStartPar
The results object for the pipeline task is returned.

\end{description}\end{quote}
\subsubsection*{Examples}
\begin{enumerate}
\sphinxsetlistlabels{\arabic}{enumi}{enumii}{}{.}%
\item {} 
\sphinxAtStartPar
Locate known lines in any solar system object flux calibrators:

\end{enumerate}

\begin{sphinxVerbatim}[commandchars=\\\{\}]
\PYG{g+gp}{\PYGZgt{}\PYGZgt{}\PYGZgt{} }\PYG{n}{hifa\PYGZus{}fluxcalflag}\PYG{p}{(}\PYG{p}{)}
\end{sphinxVerbatim}

\end{fulllineitems}


\sphinxstepscope


\section{pipeline.hifa.cli.hifa\_gaincalsnr}
\label{\detokenize{_autosummary/pipeline.hifa.cli.hifa_gaincalsnr:pipeline-hifa-cli-hifa-gaincalsnr}}\label{\detokenize{_autosummary/pipeline.hifa.cli.hifa_gaincalsnr::doc}}\index{hifa\_gaincalsnr() (in module pipeline.hifa.cli)@\spxentry{hifa\_gaincalsnr()}\spxextra{in module pipeline.hifa.cli}}

\begin{fulllineitems}
\phantomsection\label{\detokenize{_autosummary/pipeline.hifa.cli.hifa_gaincalsnr:pipeline.hifa.cli.hifa_gaincalsnr}}
\pysigstartsignatures
\pysiglinewithargsret
{\sphinxbfcode{\sphinxupquote{hifa\_gaincalsnr}}}
{\sphinxparam{\DUrole{n}{vis}\DUrole{o}{=}\DUrole{default_value}{None}}\sphinxparamcomma \sphinxparam{\DUrole{n}{field}\DUrole{o}{=}\DUrole{default_value}{None}}\sphinxparamcomma \sphinxparam{\DUrole{n}{intent}\DUrole{o}{=}\DUrole{default_value}{None}}\sphinxparamcomma \sphinxparam{\DUrole{n}{spw}\DUrole{o}{=}\DUrole{default_value}{None}}\sphinxparamcomma \sphinxparam{\DUrole{n}{phasesnr}\DUrole{o}{=}\DUrole{default_value}{None}}\sphinxparamcomma \sphinxparam{\DUrole{n}{bwedgefrac}\DUrole{o}{=}\DUrole{default_value}{None}}\sphinxparamcomma \sphinxparam{\DUrole{n}{hm\_nantennas}\DUrole{o}{=}\DUrole{default_value}{None}}\sphinxparamcomma \sphinxparam{\DUrole{n}{maxfracflagged}\DUrole{o}{=}\DUrole{default_value}{None}}}
{}
\pysigstopsignatures
\sphinxAtStartPar
Compute gaincal signal\sphinxhyphen{}to\sphinxhyphen{}noise ratios per spw

\sphinxAtStartPar
The gaincal solution signal\sphinxhyphen{}to\sphinxhyphen{}noise is determined as follows:
\begin{itemize}
\item {} 
\sphinxAtStartPar
For each data set the list of source(s) to use for the per\sphinxhyphen{}scan gaincal
solution signal\sphinxhyphen{}to\sphinxhyphen{}noise estimation is compiled based on the values of the
field, intent, and spw parameters.

\item {} 
\sphinxAtStartPar
Source fluxes are determined for each spw and source combination.

\item {} 
\sphinxAtStartPar
Fluxes in Jy are derived from the pipeline context.

\item {} 
\sphinxAtStartPar
Pipeline context fluxes are derived from the online flux calibrator
catalog, the ASDM, or the user via the flux.csv file.

\item {} 
\sphinxAtStartPar
If no fluxes are available the task terminates.

\item {} 
\sphinxAtStartPar
Atmospheric calibration and observations scans are determined for each spw
and source combination.

\item {} 
\sphinxAtStartPar
If intent is set to ‘PHASE’ are there are no atmospheric scans associated
with the ‘PHASE’ calibrator, ‘TARGET’ atmospheric scans will be used
instead.

\item {} 
\sphinxAtStartPar
If atmospheric scans cannot be associated with any of the spw and source
combinations the task terminates.

\item {} 
\sphinxAtStartPar
Science spws are mapped to atmospheric spws for each science spw and
source combinations.

\item {} 
\sphinxAtStartPar
If mappings cannot be determined for any of the spws the task terminates.

\item {} 
\sphinxAtStartPar
The median Tsys value for each atmospheric spw and source combination is
determined from the SYSCAL table. Medians are computed first by channel,
then by antenna, in order to reduce sensitivity to deviant values.

\item {} 
\sphinxAtStartPar
The science spw parameters, exposure time(s), and integration time(s) are
determined.

\item {} 
\sphinxAtStartPar
The per scan sensitivity and signal\sphinxhyphen{}to\sphinxhyphen{}noise estimates are computed per
science spectral window. Nominal Tsys and sensitivity values per receiver
band provide by the ALMA project are used for this estimate.

\item {} 
\sphinxAtStartPar
The QA score is based on how many signal\sphinxhyphen{}to\sphinxhyphen{}noise estimates greater than
the requested signal\sphinxhyphen{}to\sphinxhyphen{}noise ratio can be computed.

\end{itemize}
\begin{quote}\begin{description}
\sphinxlineitem{Parameters}\begin{itemize}
\item {} 
\sphinxAtStartPar
\sphinxstylestrong{vis} \textendash{} The list of input MeasurementSets. Defaults to the list of
MeasurementSets specified in the pipeline context.
Example: vis={[}‘M82A.ms’, ‘M82B.ms’{]}

\item {} 
\sphinxAtStartPar
\sphinxstylestrong{field} \textendash{} The list of field names of sources to be used for signal to noise
estimation. Defaults to all fields with the standard intent.
Example: field=’3C279’

\item {} 
\sphinxAtStartPar
\sphinxstylestrong{intent} \textendash{} A string containing a comma delimited list of intents against which
the selected fields are matched. Defaults to ‘PHASE’.
Example: intent=’BANDPASS’

\item {} 
\sphinxAtStartPar
\sphinxstylestrong{spw} \textendash{} The list of spectral windows and channels for which gain solutions are
computed. Defaults to all the science spectral windows for which there are
both ‘intent’ and TARGET intents.
Example: spw=’13,15’

\item {} 
\sphinxAtStartPar
\sphinxstylestrong{phasesnr} \textendash{} The required gaincal solution signal to noise.
Example: phasesnr=20.0

\item {} 
\sphinxAtStartPar
\sphinxstylestrong{bwedgefrac} \textendash{} The fraction of the bandwidth edges that is flagged.
Example: bwedgefrac=0.0

\item {} 
\sphinxAtStartPar
\sphinxstylestrong{hm\_nantennas} \textendash{} The heuristics for determines the number of antennas to use
in the signal to noise estimate. The options are ‘all’ and ‘unflagged’.
The ‘unflagged’ options is not currently supported.
Example: hm\_nantennas=’unflagged’

\item {} 
\sphinxAtStartPar
\sphinxstylestrong{maxfracflagged} \textendash{} The maximum fraction of an antenna that can be flagged
before it is excluded from the signal to noise estimate.
Example: maxfracflagged=0.80

\end{itemize}

\sphinxlineitem{Returns}
\sphinxAtStartPar
The results object for the pipeline task is returned.

\end{description}\end{quote}
\subsubsection*{Examples}

\sphinxAtStartPar
1. Estimate the per scan gaincal solution sensitivities and signal to noise
ratios for all the science spectral windows:

\begin{sphinxVerbatim}[commandchars=\\\{\}]
\PYG{g+gp}{\PYGZgt{}\PYGZgt{}\PYGZgt{} }\PYG{n}{hifa\PYGZus{}gaincalsnr}\PYG{p}{(}\PYG{p}{)}
\end{sphinxVerbatim}

\end{fulllineitems}


\sphinxstepscope


\section{pipeline.hifa.cli.hifa\_gfluxscale}
\label{\detokenize{_autosummary/pipeline.hifa.cli.hifa_gfluxscale:pipeline-hifa-cli-hifa-gfluxscale}}\label{\detokenize{_autosummary/pipeline.hifa.cli.hifa_gfluxscale::doc}}\index{hifa\_gfluxscale() (in module pipeline.hifa.cli)@\spxentry{hifa\_gfluxscale()}\spxextra{in module pipeline.hifa.cli}}

\begin{fulllineitems}
\phantomsection\label{\detokenize{_autosummary/pipeline.hifa.cli.hifa_gfluxscale:pipeline.hifa.cli.hifa_gfluxscale}}
\pysigstartsignatures
\pysiglinewithargsret
{\sphinxbfcode{\sphinxupquote{hifa\_gfluxscale}}}
{\sphinxparam{\DUrole{n}{vis}\DUrole{o}{=}\DUrole{default_value}{None}}\sphinxparamcomma \sphinxparam{\DUrole{n}{reference}\DUrole{o}{=}\DUrole{default_value}{None}}\sphinxparamcomma \sphinxparam{\DUrole{n}{transfer}\DUrole{o}{=}\DUrole{default_value}{None}}\sphinxparamcomma \sphinxparam{\DUrole{n}{refintent}\DUrole{o}{=}\DUrole{default_value}{None}}\sphinxparamcomma \sphinxparam{\DUrole{n}{transintent}\DUrole{o}{=}\DUrole{default_value}{None}}\sphinxparamcomma \sphinxparam{\DUrole{n}{refspwmap}\DUrole{o}{=}\DUrole{default_value}{None}}\sphinxparamcomma \sphinxparam{\DUrole{n}{reffile}\DUrole{o}{=}\DUrole{default_value}{None}}\sphinxparamcomma \sphinxparam{\DUrole{n}{phaseupsolint}\DUrole{o}{=}\DUrole{default_value}{None}}\sphinxparamcomma \sphinxparam{\DUrole{n}{solint}\DUrole{o}{=}\DUrole{default_value}{None}}\sphinxparamcomma \sphinxparam{\DUrole{n}{minsnr}\DUrole{o}{=}\DUrole{default_value}{None}}\sphinxparamcomma \sphinxparam{\DUrole{n}{refant}\DUrole{o}{=}\DUrole{default_value}{None}}\sphinxparamcomma \sphinxparam{\DUrole{n}{hm\_resolvecals}\DUrole{o}{=}\DUrole{default_value}{None}}\sphinxparamcomma \sphinxparam{\DUrole{n}{antenna}\DUrole{o}{=}\DUrole{default_value}{None}}\sphinxparamcomma \sphinxparam{\DUrole{n}{peak\_fraction}\DUrole{o}{=}\DUrole{default_value}{None}}\sphinxparamcomma \sphinxparam{\DUrole{n}{amp\_outlier\_sigma}\DUrole{o}{=}\DUrole{default_value}{None}}}
{}
\pysigstopsignatures
\sphinxAtStartPar
Derive flux density scales from standard calibrators

\sphinxAtStartPar
Derive flux densities for point source transfer calibrators using flux models
for reference calibrators.

\sphinxAtStartPar
Flux values are determined by:
\begin{itemize}
\item {} 
\sphinxAtStartPar
computing phase only solutions for all the science spectral windows
using the calibrator data selected by the \sphinxcode{\sphinxupquote{reference}} and \sphinxcode{\sphinxupquote{refintent}}
parameters and the \sphinxcode{\sphinxupquote{transfer}} and \sphinxcode{\sphinxupquote{transintent}} parameters, the value
of the \sphinxcode{\sphinxupquote{phaseupsolint}} parameter, and any spw combination determined
in hifa\_spwphaseup.

\item {} 
\sphinxAtStartPar
computing complex amplitude only solutions for all the science spectral
windows using calibrator data selected with \sphinxcode{\sphinxupquote{reference}} and
\sphinxcode{\sphinxupquote{refintent}} parameters and the \sphinxcode{\sphinxupquote{transfer}} and \sphinxcode{\sphinxupquote{transintent}}
parameters, the value of the \sphinxcode{\sphinxupquote{solint}} parameter.

\item {} 
\sphinxAtStartPar
examining the amplitude\sphinxhyphen{}only solutions for obvious outliers and flagging
them in the caltable.

\item {} 
\sphinxAtStartPar
transferring the flux scale from the reference calibrators to the transfer
calibrators using \sphinxcode{\sphinxupquote{refspwmap}} for windows without data in the reference
calibrators.

\item {} 
\sphinxAtStartPar
inserting the computed flux density values into the MODEL\_DATA column.

\end{itemize}

\sphinxAtStartPar
Resolved calibrators are handled via antenna selection either automatically
(\sphinxcode{\sphinxupquote{hm\_resolvedcals}} = ‘automatic’) or manually (\sphinxcode{\sphinxupquote{hm\_resolvedcals}} =
‘manual’). In the former case, antennas closer to the reference antenna than
the uv distance where visibilities fall to \sphinxcode{\sphinxupquote{peak\_fraction}} of the peak are
used. In manual mode, the antennas specified in \sphinxcode{\sphinxupquote{antenna}} are used.

\sphinxAtStartPar
Note that the flux corrected calibration table computed internally is
not currently used in later pipeline apply calibration steps.
\begin{quote}\begin{description}
\sphinxlineitem{Parameters}\begin{itemize}
\item {} 
\sphinxAtStartPar
\sphinxstylestrong{vis} \textendash{} The list of input MeasurementSets. Defaults to the list of
MeasurementSets specified in the pipeline context.
Example: {[}‘M32A.ms’, ‘M32B.ms’{]}

\item {} 
\sphinxAtStartPar
\sphinxstylestrong{reference} \textendash{} A string containing a comma delimited list of field names
defining the reference calibrators. Defaults to field names with
intent ‘\sphinxtitleref{*AMP*}’.
Example: reference=’M82,3C273’

\item {} 
\sphinxAtStartPar
\sphinxstylestrong{transfer} \textendash{} A string containing a comma delimited list of field names
defining the transfer calibrators. Defaults to field names with
intent ‘\sphinxtitleref{*PHASE*}’.
Example: transfer=’J1328+041,J1206+30’

\item {} 
\sphinxAtStartPar
\sphinxstylestrong{refintent} \textendash{} A string containing a comma delimited list of intents
used to select the reference calibrators. Defaults to ‘AMPLITUDE’.
Example: refintent=’’, refintent=’AMPLITUDE’

\item {} 
\sphinxAtStartPar
\sphinxstylestrong{transintent} \textendash{} A string containing a comma delimited list of intents
defining the transfer calibrators. Defaults to
‘PHASE,BANDPASS,CHECK,POLARIZATION,POLANGLE,POLLEAKAGE’.
Example: transintent=’’, transintent=’PHASE,BANDPASS’

\item {} 
\sphinxAtStartPar
\sphinxstylestrong{refspwmap} \textendash{} Vector of spectral window ids enabling scaling across
spectral windows. Defaults to no scaling.
Example: refspwmap={[}1,1,3,3{]} \sphinxhyphen{} (4 spws, reference fields in 1 and 3, transfer
fields in 0,1,2,3

\item {} 
\sphinxAtStartPar
\sphinxstylestrong{reffile} \textendash{} Path to a file containing flux densities for calibrators.
Setjy will be run for any that have both reference and transfer intents.
Values given in this file will take precedence over MODEL column values
set by previous tasks. By default, the path is set to the CSV file created
by hifa\_importdata, consisting of catalogue fluxes extracted from the ASDM
and / or edited by the user.
example: reffile=’’, reffile=’working/flux.csv’

\item {} 
\sphinxAtStartPar
\sphinxstylestrong{phaseupsolint} \textendash{} Time solution intervals in CASA syntax for the phase solution.
example: phaseupsolint=’inf’, phaseupsolint=’int’, phaseupsolint=’100sec’

\item {} 
\sphinxAtStartPar
\sphinxstylestrong{solint} \textendash{} Time solution intervals in CASA syntax for the amplitude solution.
example: solint=’inf’, solint=’int’, solint=’100sec’

\item {} 
\sphinxAtStartPar
\sphinxstylestrong{minsnr} \textendash{} Minimum signal\sphinxhyphen{}to\sphinxhyphen{}noise ratio for gain calibration solutions.
example: minsnr=1.5, minsnr=0.0

\item {} 
\sphinxAtStartPar
\sphinxstylestrong{refant} \textendash{} A string specifying the reference antenna(s). By default,
this is read from the context.
Example: refant=’DV05’

\item {} 
\sphinxAtStartPar
\sphinxstylestrong{hm\_resolvedcals} \textendash{} Heuristics method for handling resolved calibrators. The
options are ‘automatic’ and ‘manual’. In automatic mode,
antennas closer to the reference antenna than the uv
distance where visibilities fall to \sphinxcode{\sphinxupquote{peak\_fraction}} of the
peak are used. In manual mode, the antennas specified in
\sphinxcode{\sphinxupquote{antenna}} are used.

\item {} 
\sphinxAtStartPar
\sphinxstylestrong{antenna} \textendash{} A comma delimited string specifying the antenna names or ids
to be used for the fluxscale determination. Used in
\sphinxcode{\sphinxupquote{hm\_resolvedcals}} = ‘manual’ mode.
Example: antenna=’DV16,DV07,DA12,DA08’

\item {} 
\sphinxAtStartPar
\sphinxstylestrong{peak\_fraction} \textendash{} The limiting UV distance from the reference antenna for
antennas to be included in the flux calibration. Defined as
the point where the calibrator visibilities have fallen to
\sphinxcode{\sphinxupquote{peak\_fraction}} of the peak value.

\item {} 
\sphinxAtStartPar
\sphinxstylestrong{amp\_outlier\_sigma} \textendash{} Sigma threshold used to identify outliers in the amplitude
caltable. Default: 50.0.
Example: amp\_outlier\_sigma=30.0

\end{itemize}

\sphinxlineitem{Returns}
\sphinxAtStartPar
The results object for the pipeline task is returned.

\end{description}\end{quote}
\subsubsection*{Examples}

\sphinxAtStartPar
1. Compute flux values for the phase calibrator using model data from
the amplitude calibrator:

\begin{sphinxVerbatim}[commandchars=\\\{\}]
\PYG{g+gp}{\PYGZgt{}\PYGZgt{}\PYGZgt{} }\PYG{n}{hifa\PYGZus{}gfluxscale}\PYG{p}{(}\PYG{p}{)}
\end{sphinxVerbatim}

\end{fulllineitems}


\sphinxstepscope


\section{pipeline.hifa.cli.hifa\_gfluxscaleflag}
\label{\detokenize{_autosummary/pipeline.hifa.cli.hifa_gfluxscaleflag:pipeline-hifa-cli-hifa-gfluxscaleflag}}\label{\detokenize{_autosummary/pipeline.hifa.cli.hifa_gfluxscaleflag::doc}}\index{hifa\_gfluxscaleflag() (in module pipeline.hifa.cli)@\spxentry{hifa\_gfluxscaleflag()}\spxextra{in module pipeline.hifa.cli}}

\begin{fulllineitems}
\phantomsection\label{\detokenize{_autosummary/pipeline.hifa.cli.hifa_gfluxscaleflag:pipeline.hifa.cli.hifa_gfluxscaleflag}}
\pysigstartsignatures
\pysiglinewithargsret
{\sphinxbfcode{\sphinxupquote{hifa\_gfluxscaleflag}}}
{\sphinxparam{\DUrole{n}{vis}\DUrole{o}{=}\DUrole{default_value}{None}}\sphinxparamcomma \sphinxparam{\DUrole{n}{intent}\DUrole{o}{=}\DUrole{default_value}{None}}\sphinxparamcomma \sphinxparam{\DUrole{n}{phaseupsolint}\DUrole{o}{=}\DUrole{default_value}{None}}\sphinxparamcomma \sphinxparam{\DUrole{n}{solint}\DUrole{o}{=}\DUrole{default_value}{None}}\sphinxparamcomma \sphinxparam{\DUrole{n}{minsnr}\DUrole{o}{=}\DUrole{default_value}{None}}\sphinxparamcomma \sphinxparam{\DUrole{n}{refant}\DUrole{o}{=}\DUrole{default_value}{None}}\sphinxparamcomma \sphinxparam{\DUrole{n}{antnegsig}\DUrole{o}{=}\DUrole{default_value}{None}}\sphinxparamcomma \sphinxparam{\DUrole{n}{antpossig}\DUrole{o}{=}\DUrole{default_value}{None}}\sphinxparamcomma \sphinxparam{\DUrole{n}{tmantint}\DUrole{o}{=}\DUrole{default_value}{None}}\sphinxparamcomma \sphinxparam{\DUrole{n}{tmint}\DUrole{o}{=}\DUrole{default_value}{None}}\sphinxparamcomma \sphinxparam{\DUrole{n}{tmbl}\DUrole{o}{=}\DUrole{default_value}{None}}\sphinxparamcomma \sphinxparam{\DUrole{n}{antblnegsig}\DUrole{o}{=}\DUrole{default_value}{None}}\sphinxparamcomma \sphinxparam{\DUrole{n}{antblpossig}\DUrole{o}{=}\DUrole{default_value}{None}}\sphinxparamcomma \sphinxparam{\DUrole{n}{relaxed\_factor}\DUrole{o}{=}\DUrole{default_value}{None}}\sphinxparamcomma \sphinxparam{\DUrole{n}{niter}\DUrole{o}{=}\DUrole{default_value}{None}}}
{}
\pysigstopsignatures
\sphinxAtStartPar
Flag the flux, diffgain, phase calibrators and check source

\sphinxAtStartPar
This task computes the flagging heuristics on the flux, diffgain, and phase
calibrators and the check source, by calling hif\_correctedampflag which
looks for outlier visibility points by statistically examining the scalar
difference of corrected amplitudes minus model amplitudes, and flags those
outliers. The philosophy is that only outlier data points that have remained
outliers after calibration will be flagged. The heuristic works equally well
on resolved calibrators and point sources because it is not performing a
vector difference, and thus is not sensitive to nulls in the flux density
vs. uvdistance domain. Note that the phase of the data is not assessed.

\sphinxAtStartPar
In further detail, the workflow is as follows: a snapshot of the flagging
state is preserved at the start, a preliminary phase and amplitude gaincal
solution is solved and applied, the flagging heuristics are run and
any outliers are marked for flagging, the flagging state is restored from the
snapshot. If any outliers were found, then these are flagged. Plots are
generated at two points in this workflow: after preliminary phase and
amplitude calibration but before flagging heuristics are run, and after
flagging heuristics have been run and applied. If no points were flagged,
the ‘after’ plots are not generated or displayed. The score for this stage
is the standard data flagging score, which depends on the fraction of data
flagged.
\begin{quote}\begin{description}
\sphinxlineitem{Parameters}\begin{itemize}
\item {} 
\sphinxAtStartPar
\sphinxstylestrong{vis} \textendash{} The list of input MeasurementSets. Defaults to the list of
MeasurementSets specified in the pipeline context.
Example: vis={[}‘M51.ms’{]}

\item {} 
\sphinxAtStartPar
\sphinxstylestrong{intent} \textendash{} A string containing a comma delimited list of intents against
which the selected fields are matched. If undefined (default), it
will select all data with the AMPLITUDE, PHASE, and CHECK intents,
except for one case: if one of the AMPLITUDE intent fields was also
used for BANDPASS, then this task will select only data with PHASE
and CHECK intents.
Example: intent=’\sphinxtitleref{*PHASE*}’

\item {} 
\sphinxAtStartPar
\sphinxstylestrong{phaseupsolint} \textendash{} The phase correction solution interval in CASA syntax.
Example: phaseupsolint=’300s’

\item {} 
\sphinxAtStartPar
\sphinxstylestrong{solint} \textendash{} Time and channel solution intervals in CASA syntax.
Example: solint=’inf,10ch’, solint=’inf’

\item {} 
\sphinxAtStartPar
\sphinxstylestrong{minsnr} \textendash{} Solutions below this SNR are rejected.

\item {} 
\sphinxAtStartPar
\sphinxstylestrong{refant} \textendash{} Reference antenna names. Defaults to the value(s) stored in the
pipeline context. If undefined in the pipeline context defaults to
the CASA reference antenna naming scheme.
Example: refant=’DV01’, refant=’DV06,DV07’

\item {} 
\sphinxAtStartPar
\sphinxstylestrong{antnegsig} \textendash{} Lower sigma threshold for identifying outliers as a result of
bad antennas within individual timestamps.
Example: antnegsig=4.0

\item {} 
\sphinxAtStartPar
\sphinxstylestrong{antpossig} \textendash{} Upper sigma threshold for identifying outliers as a result of
bad antennas within individual timestamps.
Example: antpossig=4.6

\item {} 
\sphinxAtStartPar
\sphinxstylestrong{tmantint} \textendash{} Threshold for maximum fraction of timestamps that are allowed to
contain outliers.
Example: tmantint=0.063

\item {} 
\sphinxAtStartPar
\sphinxstylestrong{tmint} \textendash{} Threshold for maximum fraction of “outlier timestamps” over
“total timestamps” that a baseline may be a part of.
Example: tmint=0.085

\item {} 
\sphinxAtStartPar
\sphinxstylestrong{tmbl} \textendash{} Initial threshold for maximum fraction of “bad baselines” over “all
baselines” that an antenna may be a part of.
Example: tmbl=0.175

\item {} 
\sphinxAtStartPar
\sphinxstylestrong{antblnegsig} \textendash{} Lower sigma threshold for identifying outliers as a result of
“bad baselines” and/or “bad antennas” within baselines, across all
timestamps.
Example: antblnegsig=3.4

\item {} 
\sphinxAtStartPar
\sphinxstylestrong{antblpossig} \textendash{} Threshold for identifying outliers as a result of
“bad baselines” and/or “bad antennas” within baselines, across all
timestamps.
Example: antblpossig=3.2

\item {} 
\sphinxAtStartPar
\sphinxstylestrong{relaxed\_factor} \textendash{} Relaxed value to set the threshold scaling factor to under
certain conditions (see task description).
Example: relaxed\_factor=2.0

\item {} 
\sphinxAtStartPar
\sphinxstylestrong{niter} \textendash{} Maximum number of times to iterate on evaluation of flagging
heuristics. If an iteration results in no new flags, then subsequent
iterations are skipped.
Example: niter=2

\end{itemize}

\sphinxlineitem{Returns}
\sphinxAtStartPar
The results object for the pipeline task is returned.

\end{description}\end{quote}
\subsubsection*{Examples}

\sphinxAtStartPar
1. run with recommended settings to create flux scale calibration with flagging
using recommended thresholds:

\begin{sphinxVerbatim}[commandchars=\\\{\}]
\PYG{g+gp}{\PYGZgt{}\PYGZgt{}\PYGZgt{} }\PYG{n}{hifa\PYGZus{}gfluxscaleflag}\PYG{p}{(}\PYG{p}{)}
\end{sphinxVerbatim}

\end{fulllineitems}


\sphinxstepscope


\section{pipeline.hifa.cli.hifa\_imageprecheck}
\label{\detokenize{_autosummary/pipeline.hifa.cli.hifa_imageprecheck:pipeline-hifa-cli-hifa-imageprecheck}}\label{\detokenize{_autosummary/pipeline.hifa.cli.hifa_imageprecheck::doc}}\index{hifa\_imageprecheck() (in module pipeline.hifa.cli)@\spxentry{hifa\_imageprecheck()}\spxextra{in module pipeline.hifa.cli}}

\begin{fulllineitems}
\phantomsection\label{\detokenize{_autosummary/pipeline.hifa.cli.hifa_imageprecheck:pipeline.hifa.cli.hifa_imageprecheck}}
\pysigstartsignatures
\pysiglinewithargsret
{\sphinxbfcode{\sphinxupquote{hifa\_imageprecheck}}}
{\sphinxparam{\DUrole{n}{vis}\DUrole{o}{=}\DUrole{default_value}{None}}\sphinxparamcomma \sphinxparam{\DUrole{n}{desired\_angular\_resolution}\DUrole{o}{=}\DUrole{default_value}{None}}\sphinxparamcomma \sphinxparam{\DUrole{n}{calcsb}\DUrole{o}{=}\DUrole{default_value}{None}}\sphinxparamcomma \sphinxparam{\DUrole{n}{parallel}\DUrole{o}{=}\DUrole{default_value}{None}}}
{}
\pysigstopsignatures
\sphinxAtStartPar
Calculates the best Briggs robust parameter to achieve sensitivity and angular resolution goals.

\sphinxAtStartPar
In this task, the representative source and the spw containing the representative
frequency selected by the PI in the OT are used to calculate the synthesized beam
and to make sensitivity estimates for the aggregate bandwidth and representative
bandwidth for several values of the Briggs robust parameter. This information is
reported in a table in the weblog. If no representative target/frequency information
is available, it defaults to the first target and center of first spw in the data
(e.g. pre\sphinxhyphen{}Cycle 5 data does not have this information available). The best Briggs
robust parameter to achieve the PI’s desired angular resolution is chosen
automatically. See the User’s guide for further details.
\begin{quote}\begin{description}
\sphinxlineitem{Parameters}\begin{itemize}
\item {} 
\sphinxAtStartPar
\sphinxstylestrong{vis} \textendash{} The list of input MeasurementSets. Defaults to the list of MeasurementSets specified in the h\_init or hif\_importdata task.
‘’: use all MeasurementSets in the context
Examples: ‘ngc5921.ms’, {[}‘ngc5921a.ms’, ngc5921b.ms’, ‘ngc5921c.ms’{]}

\item {} 
\sphinxAtStartPar
\sphinxstylestrong{desired\_angular\_resolution} \textendash{} User specified angular resolution goal string. When this parameter is set, uvtapering may be performed.
‘’: automatic from performance parameters (default).
Example: ‘1.0arcsec’

\item {} 
\sphinxAtStartPar
\sphinxstylestrong{calcsb} \textendash{} Force (re\sphinxhyphen{})calculation of sensitivities and beams; defaults to False

\item {} 
\sphinxAtStartPar
\sphinxstylestrong{parallel} \textendash{} Use MPI cluster where possible

\end{itemize}

\sphinxlineitem{Returns}
\sphinxAtStartPar
The results object for the pipeline task is returned.

\end{description}\end{quote}
\subsubsection*{Examples}
\begin{enumerate}
\sphinxsetlistlabels{\arabic}{enumi}{enumii}{}{.}%
\item {} 
\sphinxAtStartPar
run with recommended settings to perform checks prior to imaging:

\end{enumerate}

\begin{sphinxVerbatim}[commandchars=\\\{\}]
\PYG{g+gp}{\PYGZgt{}\PYGZgt{}\PYGZgt{} }\PYG{n}{hifa\PYGZus{}imageprecheck}\PYG{p}{(}\PYG{p}{)}
\end{sphinxVerbatim}

\sphinxAtStartPar
2. run to perform checks prior to imaging and force the re\sphinxhyphen{}calculation of
sensitivities and beams:

\begin{sphinxVerbatim}[commandchars=\\\{\}]
\PYG{g+gp}{\PYGZgt{}\PYGZgt{}\PYGZgt{} }\PYG{n}{hifa\PYGZus{}imageprecheck}\PYG{p}{(}\PYG{n}{calcsb}\PYG{o}{=}\PYG{k+kc}{True}\PYG{p}{)}
\end{sphinxVerbatim}

\end{fulllineitems}


\sphinxstepscope


\section{pipeline.hifa.cli.hifa\_importdata}
\label{\detokenize{_autosummary/pipeline.hifa.cli.hifa_importdata:pipeline-hifa-cli-hifa-importdata}}\label{\detokenize{_autosummary/pipeline.hifa.cli.hifa_importdata::doc}}\index{hifa\_importdata() (in module pipeline.hifa.cli)@\spxentry{hifa\_importdata()}\spxextra{in module pipeline.hifa.cli}}

\begin{fulllineitems}
\phantomsection\label{\detokenize{_autosummary/pipeline.hifa.cli.hifa_importdata:pipeline.hifa.cli.hifa_importdata}}
\pysigstartsignatures
\pysiglinewithargsret
{\sphinxbfcode{\sphinxupquote{hifa\_importdata}}}
{\sphinxparam{\DUrole{n}{vis}\DUrole{o}{=}\DUrole{default_value}{None}}\sphinxparamcomma \sphinxparam{\DUrole{n}{session}\DUrole{o}{=}\DUrole{default_value}{None}}\sphinxparamcomma \sphinxparam{\DUrole{n}{asis}\DUrole{o}{=}\DUrole{default_value}{None}}\sphinxparamcomma \sphinxparam{\DUrole{n}{process\_caldevice}\DUrole{o}{=}\DUrole{default_value}{None}}\sphinxparamcomma \sphinxparam{\DUrole{n}{overwrite}\DUrole{o}{=}\DUrole{default_value}{None}}\sphinxparamcomma \sphinxparam{\DUrole{n}{nocopy}\DUrole{o}{=}\DUrole{default_value}{None}}\sphinxparamcomma \sphinxparam{\DUrole{n}{bdfflags}\DUrole{o}{=}\DUrole{default_value}{None}}\sphinxparamcomma \sphinxparam{\DUrole{n}{datacolumns}\DUrole{o}{=}\DUrole{default_value}{None}}\sphinxparamcomma \sphinxparam{\DUrole{n}{lazy}\DUrole{o}{=}\DUrole{default_value}{None}}\sphinxparamcomma \sphinxparam{\DUrole{n}{dbservice}\DUrole{o}{=}\DUrole{default_value}{None}}\sphinxparamcomma \sphinxparam{\DUrole{n}{ocorr\_mode}\DUrole{o}{=}\DUrole{default_value}{None}}\sphinxparamcomma \sphinxparam{\DUrole{n}{createmms}\DUrole{o}{=}\DUrole{default_value}{None}}\sphinxparamcomma \sphinxparam{\DUrole{n}{minparang}\DUrole{o}{=}\DUrole{default_value}{None}}\sphinxparamcomma \sphinxparam{\DUrole{n}{parallel}\DUrole{o}{=}\DUrole{default_value}{None}}}
{}
\pysigstopsignatures
\sphinxAtStartPar
Imports data into the interferometry pipeline

\sphinxAtStartPar
The hifa\_importdata task loads the specified visibility data into the pipeline
context unpacking and / or converting it as necessary.

\sphinxAtStartPar
If the \sphinxcode{\sphinxupquote{overwrite}} input parameter is set to False and the task is asked
to convert an input ASDM input to an MS, then when the output MS already
exists in the output directory, the importasdm conversion step is skipped,
and the existing MS will be imported instead.
\begin{quote}\begin{description}
\sphinxlineitem{Parameters}\begin{itemize}
\item {} 
\sphinxAtStartPar
\sphinxstylestrong{vis} \textendash{} List of visibility data files. These may be ASDMs, tar
files of ASDMs, MSes, or tar files of MSes. If ASDM files
are specified, they will be converted to MS format.
Example: vis={[}‘X227.ms’, ‘asdms.tar.gz’{]}

\item {} 
\sphinxAtStartPar
\sphinxstylestrong{session} \textendash{} List of session names, one for each visibility dataset,
used to group the MSes into sessions.
Example: session={[}‘session\_1’, ‘session\_2’{]}

\item {} 
\sphinxAtStartPar
\sphinxstylestrong{asis} \textendash{} Creates verbatim copies of the ASDM tables in the output MS.
The value given to this option must be a list of table names
separated by space characters.

\item {} 
\sphinxAtStartPar
\sphinxstylestrong{process\_caldevice} \textendash{} Import the caldevice table from the ASDM.

\item {} 
\sphinxAtStartPar
\sphinxstylestrong{overwrite} \textendash{} Overwrite existing files on import; defaults to False.
When converting ASDM to MS, if overwrite=False and the MS
already exists in the output directory, then this existing
MS dataset will be used instead.
Example: overwrite=True

\item {} 
\sphinxAtStartPar
\sphinxstylestrong{nocopy} \textendash{} Disable copying of MS to working directory; defaults to
False.
Example: nocopy=True

\item {} 
\sphinxAtStartPar
\sphinxstylestrong{bdfflags} \textendash{} Apply BDF flags on import.

\item {} 
\sphinxAtStartPar
\sphinxstylestrong{datacolumns} \textendash{} Dictionary defining the data types of existing columns.
The format is:
\{‘data’: ‘data type 1’\}
or
\{‘data’: ‘data type 1’, ‘corrected’: ‘data type 2’\}.
For ASDMs the data type can only be RAW and one
can only specify it for the data column.
For MSes one can define two different data types
for the DATA and CORRECTED\_DATA columns and they
can be any of the known data types (RAW,
REGCAL\_CONTLINE\_ALL, REGCAL\_CONTLINE\_SCIENCE,
SELFCAL\_CONTLINE\_SCIENCE, REGCAL\_LINE\_SCIENCE,
SELFCAL\_LINE\_SCIENCE, BASELINED, ATMCORR). The
intent selection strings \_ALL or \_SCIENCE can be
skipped. In that case the task determines this
automatically by inspecting the existing intents
in the dataset.
Usually, a single datacolumns dictionary is used
for all datasets. If necessary, one can define a
list of dictionaries, one for each EB, with
different setups per EB.
If no types are specified,
\{‘data’:’raw’,’corrected’:’regcal\_contline’\}
or \{‘data’:’raw’\} will be assumed, depending on
whether the corrected column exists or not.

\item {} 
\sphinxAtStartPar
\sphinxstylestrong{lazy} \textendash{} Use the lazy filler import.

\item {} 
\sphinxAtStartPar
\sphinxstylestrong{dbservice} \textendash{} Use the online flux catalog.

\item {} 
\sphinxAtStartPar
\sphinxstylestrong{ocorr\_mode} \textendash{} ALMA default set to ca.

\item {} 
\sphinxAtStartPar
\sphinxstylestrong{createmms} \textendash{} Create an MMS.

\item {} 
\sphinxAtStartPar
\sphinxstylestrong{minparang} \textendash{} Minimum required parallactic angle range for polarisation
calibrator, in degrees. The default of 0.0 is used for
non\sphinxhyphen{}polarisation processing.

\item {} 
\sphinxAtStartPar
\sphinxstylestrong{parallel} \textendash{} Execute using CASA HPC functionality, if available.

\end{itemize}

\sphinxlineitem{Returns}
\sphinxAtStartPar
The results object for the pipeline task is returned.

\end{description}\end{quote}
\subsubsection*{Examples}
\begin{enumerate}
\sphinxsetlistlabels{\arabic}{enumi}{enumii}{}{.}%
\item {} 
\sphinxAtStartPar
Load an ASDM list in the ../rawdata subdirectory into the context:

\end{enumerate}

\begin{sphinxVerbatim}[commandchars=\\\{\}]
\PYG{g+gp}{\PYGZgt{}\PYGZgt{}\PYGZgt{} }\PYG{n}{hifa\PYGZus{}importdata}\PYG{p}{(}\PYG{n}{vis}\PYG{o}{=}\PYG{p}{[}\PYG{l+s+s1}{\PYGZsq{}}\PYG{l+s+s1}{../rawdata/uid\PYGZus{}\PYGZus{}\PYGZus{}A002\PYGZus{}X30a93d\PYGZus{}X43e}\PYG{l+s+s1}{\PYGZsq{}}\PYG{p}{,} \PYG{l+s+s1}{\PYGZsq{}}\PYG{l+s+s1}{../rawdata/uid\PYGZus{}A002\PYGZus{}x30a93d\PYGZus{}X44e}\PYG{l+s+s1}{\PYGZsq{}}\PYG{p}{]}\PYG{p}{)}
\end{sphinxVerbatim}
\begin{enumerate}
\sphinxsetlistlabels{\arabic}{enumi}{enumii}{}{.}%
\setcounter{enumi}{1}
\item {} 
\sphinxAtStartPar
Load an MS in the current directory into the context:

\end{enumerate}

\begin{sphinxVerbatim}[commandchars=\\\{\}]
\PYG{g+gp}{\PYGZgt{}\PYGZgt{}\PYGZgt{} }\PYG{n}{hifa\PYGZus{}importdata}\PYG{p}{(}\PYG{n}{vis}\PYG{o}{=}\PYG{p}{[}\PYG{l+s+s1}{\PYGZsq{}}\PYG{l+s+s1}{uid\PYGZus{}\PYGZus{}\PYGZus{}A002\PYGZus{}X30a93d\PYGZus{}X43e.ms}\PYG{l+s+s1}{\PYGZsq{}}\PYG{p}{]}\PYG{p}{)}
\end{sphinxVerbatim}
\begin{enumerate}
\sphinxsetlistlabels{\arabic}{enumi}{enumii}{}{.}%
\setcounter{enumi}{2}
\item {} 
\sphinxAtStartPar
Load a tarred ASDM in ../rawdata into the context:

\end{enumerate}

\begin{sphinxVerbatim}[commandchars=\\\{\}]
\PYG{g+gp}{\PYGZgt{}\PYGZgt{}\PYGZgt{} }\PYG{n}{hifa\PYGZus{}importdata}\PYG{p}{(}\PYG{n}{vis}\PYG{o}{=}\PYG{p}{[}\PYG{l+s+s1}{\PYGZsq{}}\PYG{l+s+s1}{../rawdata/uid\PYGZus{}\PYGZus{}\PYGZus{}A002\PYGZus{}X30a93d\PYGZus{}X43e.tar.gz}\PYG{l+s+s1}{\PYGZsq{}}\PYG{p}{]}\PYG{p}{)}
\end{sphinxVerbatim}
\begin{enumerate}
\sphinxsetlistlabels{\arabic}{enumi}{enumii}{}{.}%
\setcounter{enumi}{3}
\item {} 
\sphinxAtStartPar
Import a list of MeasurementSets:

\end{enumerate}

\begin{sphinxVerbatim}[commandchars=\\\{\}]
\PYG{g+gp}{\PYGZgt{}\PYGZgt{}\PYGZgt{} }\PYG{n}{myvislist} \PYG{o}{=} \PYG{p}{[}\PYG{l+s+s1}{\PYGZsq{}}\PYG{l+s+s1}{uid\PYGZus{}\PYGZus{}\PYGZus{}A002\PYGZus{}X30a93d\PYGZus{}X43e.ms}\PYG{l+s+s1}{\PYGZsq{}}\PYG{p}{,} \PYG{l+s+s1}{\PYGZsq{}}\PYG{l+s+s1}{uid\PYGZus{}A002\PYGZus{}x30a93d\PYGZus{}X44e.ms}\PYG{l+s+s1}{\PYGZsq{}}\PYG{p}{]}
\PYG{g+gp}{\PYGZgt{}\PYGZgt{}\PYGZgt{} }\PYG{n}{hifa\PYGZus{}importdata}\PYG{p}{(}\PYG{n}{vis}\PYG{o}{=}\PYG{n}{myvislist}\PYG{p}{)}
\end{sphinxVerbatim}
\begin{enumerate}
\sphinxsetlistlabels{\arabic}{enumi}{enumii}{}{.}%
\setcounter{enumi}{4}
\item {} 
\sphinxAtStartPar
Run with explicit setting of data column types:

\end{enumerate}

\begin{sphinxVerbatim}[commandchars=\\\{\}]
\PYG{g+gp}{\PYGZgt{}\PYGZgt{}\PYGZgt{} }\PYG{n}{hifa\PYGZus{}importdata}\PYG{p}{(}\PYG{n}{vis}\PYG{o}{=}\PYG{p}{[}\PYG{l+s+s1}{\PYGZsq{}}\PYG{l+s+s1}{uid\PYGZus{}\PYGZus{}\PYGZus{}A002\PYGZus{}X30a93d\PYGZus{}X43e\PYGZus{}targets.ms}\PYG{l+s+s1}{\PYGZsq{}}\PYG{p}{]}\PYG{p}{,} \PYG{n}{datacolumns}\PYG{o}{=}\PYG{p}{\PYGZob{}}\PYG{l+s+s1}{\PYGZsq{}}\PYG{l+s+s1}{data}\PYG{l+s+s1}{\PYGZsq{}}\PYG{p}{:} \PYG{l+s+s1}{\PYGZsq{}}\PYG{l+s+s1}{regcal\PYGZus{}contline}\PYG{l+s+s1}{\PYGZsq{}}\PYG{p}{\PYGZcb{}}\PYG{p}{)}
\PYG{g+gp}{\PYGZgt{}\PYGZgt{}\PYGZgt{} }\PYG{n}{hifa\PYGZus{}importdata}\PYG{p}{(}\PYG{n}{vis}\PYG{o}{=}\PYG{p}{[}\PYG{l+s+s1}{\PYGZsq{}}\PYG{l+s+s1}{uid\PYGZus{}\PYGZus{}\PYGZus{}A002\PYGZus{}X30a93d\PYGZus{}X43e\PYGZus{}targets\PYGZus{}line.ms}\PYG{l+s+s1}{\PYGZsq{}}\PYG{p}{]}\PYG{p}{,} \PYG{n}{datacolumns}\PYG{o}{=}\PYG{p}{\PYGZob{}}\PYG{l+s+s1}{\PYGZsq{}}\PYG{l+s+s1}{data}\PYG{l+s+s1}{\PYGZsq{}}\PYG{p}{:} \PYG{l+s+s1}{\PYGZsq{}}\PYG{l+s+s1}{regcal\PYGZus{}line}\PYG{l+s+s1}{\PYGZsq{}}\PYG{p}{,} \PYG{l+s+s1}{\PYGZsq{}}\PYG{l+s+s1}{corrected}\PYG{l+s+s1}{\PYGZsq{}}\PYG{p}{:} \PYG{l+s+s1}{\PYGZsq{}}\PYG{l+s+s1}{selfcal\PYGZus{}line}\PYG{l+s+s1}{\PYGZsq{}}\PYG{p}{\PYGZcb{}}\PYG{p}{)}
\end{sphinxVerbatim}

\end{fulllineitems}


\sphinxstepscope


\section{pipeline.hifa.cli.hifa\_lock\_refant}
\label{\detokenize{_autosummary/pipeline.hifa.cli.hifa_lock_refant:pipeline-hifa-cli-hifa-lock-refant}}\label{\detokenize{_autosummary/pipeline.hifa.cli.hifa_lock_refant::doc}}\index{hifa\_lock\_refant() (in module pipeline.hifa.cli)@\spxentry{hifa\_lock\_refant()}\spxextra{in module pipeline.hifa.cli}}

\begin{fulllineitems}
\phantomsection\label{\detokenize{_autosummary/pipeline.hifa.cli.hifa_lock_refant:pipeline.hifa.cli.hifa_lock_refant}}
\pysigstartsignatures
\pysiglinewithargsret
{\sphinxbfcode{\sphinxupquote{hifa\_lock\_refant}}}
{\sphinxparam{\DUrole{n}{vis}\DUrole{o}{=}\DUrole{default_value}{None}}}
{}
\pysigstopsignatures
\sphinxAtStartPar
Lock reference antenna list

\sphinxAtStartPar
hifa\_lock\_refant marks the reference antenna list as “locked” for specified
measurement sets, preventing modification of the refant list by subsequent
tasks.

\sphinxAtStartPar
After executing hifa\_lock\_refant, all subsequent gaincal calls will by
default be executed with refantmode=’strict’.

\sphinxAtStartPar
The refant list can be unlocked with the hifa\_unlock\_refant task.
\begin{quote}\begin{description}
\sphinxlineitem{Parameters}
\sphinxAtStartPar
\sphinxstylestrong{vis} \textendash{} List of input MeasurementSets. Defaults to the list of
MeasurementSets specified in the pipeline context.
Example: vis={[}‘ngc5921.ms’{]}

\sphinxlineitem{Returns}
\sphinxAtStartPar
The results object for the pipeline task is returned.

\end{description}\end{quote}
\subsubsection*{Examples}
\begin{enumerate}
\sphinxsetlistlabels{\arabic}{enumi}{enumii}{}{.}%
\item {} 
\sphinxAtStartPar
Lock the refant list for all MSes in pipeline context:

\end{enumerate}

\begin{sphinxVerbatim}[commandchars=\\\{\}]
\PYG{g+gp}{\PYGZgt{}\PYGZgt{}\PYGZgt{} }\PYG{n}{hifa\PYGZus{}lock\PYGZus{}refant}\PYG{p}{(}\PYG{p}{)}
\end{sphinxVerbatim}

\end{fulllineitems}


\sphinxstepscope


\section{pipeline.hifa.cli.hifa\_polcal}
\label{\detokenize{_autosummary/pipeline.hifa.cli.hifa_polcal:pipeline-hifa-cli-hifa-polcal}}\label{\detokenize{_autosummary/pipeline.hifa.cli.hifa_polcal::doc}}\index{hifa\_polcal() (in module pipeline.hifa.cli)@\spxentry{hifa\_polcal()}\spxextra{in module pipeline.hifa.cli}}

\begin{fulllineitems}
\phantomsection\label{\detokenize{_autosummary/pipeline.hifa.cli.hifa_polcal:pipeline.hifa.cli.hifa_polcal}}
\pysigstartsignatures
\pysiglinewithargsret
{\sphinxbfcode{\sphinxupquote{hifa\_polcal}}}
{\sphinxparam{\DUrole{n}{vis}\DUrole{o}{=}\DUrole{default_value}{None}}\sphinxparamcomma \sphinxparam{\DUrole{n}{minpacov}\DUrole{o}{=}\DUrole{default_value}{None}}\sphinxparamcomma \sphinxparam{\DUrole{n}{solint\_chavg}\DUrole{o}{=}\DUrole{default_value}{None}}\sphinxparamcomma \sphinxparam{\DUrole{n}{vs\_stats}\DUrole{o}{=}\DUrole{default_value}{None}}\sphinxparamcomma \sphinxparam{\DUrole{n}{vs\_thresh}\DUrole{o}{=}\DUrole{default_value}{None}}}
{}
\pysigstopsignatures
\sphinxAtStartPar
Derive instrumental polarization calibration for ALMA.

\sphinxAtStartPar
Derive the instrumental polarization calibrations for ALMA using the
polarization calibrators.
\begin{quote}\begin{description}
\sphinxlineitem{Parameters}\begin{itemize}
\item {} 
\sphinxAtStartPar
\sphinxstylestrong{vis} \textendash{} The list of input MeasurementSets. Defaults to the list of
MeasurementSets specified in the pipeline context.
Example: {[}‘M32A.ms’, ‘M32B.ms’{]}

\item {} 
\sphinxAtStartPar
\sphinxstylestrong{minpacov} \textendash{} Minimum Parallactic Angle Coverage (degrees) required for
Q, U estimation in task polfromgain. This enables avoiding
pathological cases of antennas with insufficient parallactic
angle coverage which can yield spurious source polarization
solutions or cause polfromgain to fail to find any
solutions.
Default: 30.0

\item {} 
\sphinxAtStartPar
\sphinxstylestrong{solint\_chavg} \textendash{} Channel averaging to include in solint for gaincal steps
producing cross\sphinxhyphen{}hand delay, cross\sphinxhyphen{}hand phase, and leakage
(D\sphinxhyphen{}terms) solutions.
Default: ‘5MHz’

\item {} 
\sphinxAtStartPar
\sphinxstylestrong{vs\_stats} \textendash{} List of visstat statistics to use for diagnostic comparison
between the concatenated session MS and individual MSes in
that session after applying polarization calibration tables.
Default: {[}‘min’,’max’,’mean’{]}

\item {} 
\sphinxAtStartPar
\sphinxstylestrong{vs\_thresh} \textendash{} Threshold to use in diagnostic comparison of visstat
statistics; relative differences larger than this threshold
are reported in the CASA log.
Default: 1e\sphinxhyphen{}3

\end{itemize}

\sphinxlineitem{Returns}
\sphinxAtStartPar
The results object for the pipeline task is returned.

\end{description}\end{quote}
\subsubsection*{Examples}
\begin{enumerate}
\sphinxsetlistlabels{\arabic}{enumi}{enumii}{}{.}%
\item {} 
\sphinxAtStartPar
Compute the polarization calibrations:

\end{enumerate}

\begin{sphinxVerbatim}[commandchars=\\\{\}]
\PYG{g+gp}{\PYGZgt{}\PYGZgt{}\PYGZgt{} }\PYG{n}{hifa\PYGZus{}polcal}\PYG{p}{(}\PYG{p}{)}
\end{sphinxVerbatim}

\end{fulllineitems}


\sphinxstepscope


\section{pipeline.hifa.cli.hifa\_polcalflag}
\label{\detokenize{_autosummary/pipeline.hifa.cli.hifa_polcalflag:pipeline-hifa-cli-hifa-polcalflag}}\label{\detokenize{_autosummary/pipeline.hifa.cli.hifa_polcalflag::doc}}\index{hifa\_polcalflag() (in module pipeline.hifa.cli)@\spxentry{hifa\_polcalflag()}\spxextra{in module pipeline.hifa.cli}}

\begin{fulllineitems}
\phantomsection\label{\detokenize{_autosummary/pipeline.hifa.cli.hifa_polcalflag:pipeline.hifa.cli.hifa_polcalflag}}
\pysigstartsignatures
\pysiglinewithargsret
{\sphinxbfcode{\sphinxupquote{hifa\_polcalflag}}}
{\sphinxparam{\DUrole{n}{vis}\DUrole{o}{=}\DUrole{default_value}{None}}}
{}
\pysigstopsignatures
\sphinxAtStartPar
Flag polarization calibrators

\sphinxAtStartPar
This task flags corrected visibility outliers in the polarization calibrator
data using the hif\_correctedampflag heuristics.
\begin{quote}\begin{description}
\sphinxlineitem{Parameters}
\sphinxAtStartPar
\sphinxstylestrong{vis} \textendash{} The list of input MeasurementSets. Defaults to the list of
MeasurementSets specified in the h\_init or hif\_importdata task.
‘’: use all MeasurementSets in the context
Examples: ‘ngc5921.ms’, {[}‘ngc5921a.ms’, ngc5921b.ms’, ‘ngc5921c.ms’{]}

\sphinxlineitem{Returns}
\sphinxAtStartPar
The results object for the pipeline task is returned.

\end{description}\end{quote}
\subsubsection*{Examples}

\sphinxAtStartPar
1. Run with recommended settings to flag visibility outliers in the
polarization calibrator data:

\begin{sphinxVerbatim}[commandchars=\\\{\}]
\PYG{g+gp}{\PYGZgt{}\PYGZgt{}\PYGZgt{} }\PYG{n}{hifa\PYGZus{}polcalflag}\PYG{p}{(}\PYG{p}{)}
\end{sphinxVerbatim}

\end{fulllineitems}


\sphinxstepscope


\section{pipeline.hifa.cli.hifa\_renorm}
\label{\detokenize{_autosummary/pipeline.hifa.cli.hifa_renorm:pipeline-hifa-cli-hifa-renorm}}\label{\detokenize{_autosummary/pipeline.hifa.cli.hifa_renorm::doc}}\index{hifa\_renorm() (in module pipeline.hifa.cli)@\spxentry{hifa\_renorm()}\spxextra{in module pipeline.hifa.cli}}

\begin{fulllineitems}
\phantomsection\label{\detokenize{_autosummary/pipeline.hifa.cli.hifa_renorm:pipeline.hifa.cli.hifa_renorm}}
\pysigstartsignatures
\pysiglinewithargsret
{\sphinxbfcode{\sphinxupquote{hifa\_renorm}}}
{\sphinxparam{\DUrole{n}{vis}\DUrole{o}{=}\DUrole{default_value}{None}}\sphinxparamcomma \sphinxparam{\DUrole{n}{createcaltable}\DUrole{o}{=}\DUrole{default_value}{None}}\sphinxparamcomma \sphinxparam{\DUrole{n}{threshold}\DUrole{o}{=}\DUrole{default_value}{None}}\sphinxparamcomma \sphinxparam{\DUrole{n}{spw}\DUrole{o}{=}\DUrole{default_value}{None}}\sphinxparamcomma \sphinxparam{\DUrole{n}{excludechan}\DUrole{o}{=}\DUrole{default_value}{None}}\sphinxparamcomma \sphinxparam{\DUrole{n}{atm\_auto\_exclude}\DUrole{o}{=}\DUrole{default_value}{None}}\sphinxparamcomma \sphinxparam{\DUrole{n}{bwthreshspw}\DUrole{o}{=}\DUrole{default_value}{None}}\sphinxparamcomma \sphinxparam{\DUrole{n}{parallel}\DUrole{o}{=}\DUrole{default_value}{None}}}
{}
\pysigstopsignatures
\sphinxAtStartPar
ALMA renormalization task

\sphinxAtStartPar
This task makes an assessment, and optionally applies a correction, to data
suffering from incorrect amplitude normalization caused by bright
astronomical lines detected in the autocorrelations of some target sources.

\sphinxAtStartPar
For a full description of the effects of bright emission lines and the
correction heuristics used in this task, please see the Pipeline User Guide.
\begin{quote}\begin{description}
\sphinxlineitem{Parameters}\begin{itemize}
\item {} 
\sphinxAtStartPar
\sphinxstylestrong{vis} \textendash{} List of input MeasurementSets. Defaults to the list of
MeasurementSets specified in the pipeline context.
Example: vis={[}‘ngc5921.ms’{]}

\item {} 
\sphinxAtStartPar
\sphinxstylestrong{createcaltable} \textendash{} Boolean to select whether to create the renormalization
correction cal table (True), or only run the assessment
(False, default).
Example: createcaltable=True

\item {} 
\sphinxAtStartPar
\sphinxstylestrong{threshold} \textendash{} Apply correction if max correction is above this threshold
value and \sphinxcode{\sphinxupquote{apply}} = True. Default is 1.02 (i.e. 2\%).
Example: threshold=1.02

\item {} 
\sphinxAtStartPar
\sphinxstylestrong{spw} \textendash{} The list of real (not virtual \sphinxhyphen{} i.e. the actual spwIDs in
the MS) spectral windows to evaluate.
Set to spw=’’ by default, which means the task will select
all relevant (science FDM) spectral windows.
Note that for data with multiple MSs, a list with the
correct spectral window selection for each MS can be
provided.

\sphinxAtStartPar
Examples:
\begin{itemize}
\item {} 
\sphinxAtStartPar
spw=”11,13,15,17”

\item {} 
\sphinxAtStartPar
spw={[}“11,13,15,17”, “5,7,11,13”{]}

\end{itemize}

\item {} 
\sphinxAtStartPar
\sphinxstylestrong{excludechan} \textendash{} Channels to exclude in either channel or frequency space
(TOPO, GHz), specifying the real (not virtual) spectral
window per selection.
Note that for data with multiple MSs, a list of
dictionaries with the correct selection for each MS can be
provided.
Examples:
\begin{itemize}
\item {} 
\sphinxAtStartPar
excludechan=\{‘22’:’100\textasciitilde{}150;800\textasciitilde{}850’, ‘24’:’100\textasciitilde{}200’\}

\item {} 
\sphinxAtStartPar
excludechan=\{‘22’:’230.1GHz\textasciitilde{}230.2GHz’\}

\item {} 
\sphinxAtStartPar
excludechan={[}\{‘22’:’100\textasciitilde{}150’\}, \{‘15’:’100\textasciitilde{}150’\}{]}

\end{itemize}

\item {} 
\sphinxAtStartPar
\sphinxstylestrong{atm\_auto\_exclude} \textendash{} Automatically find and exclude regions with atmospheric
features. Default is False

\item {} 
\sphinxAtStartPar
\sphinxstylestrong{bwthreshspw} \textendash{} Bandwidth beyond which a SPW is split into chunks to fit
separately. The default value for all SPWs is 120e6, and
this parameter allows one to override it for specific SPWs,
due to needing potentially various ‘nsegments’ when EBs
have very different SPW bandwidths.
Example: bwthreshspw=\{‘16: 64e6, ‘22: 64e6\}

\item {} 
\sphinxAtStartPar
\sphinxstylestrong{parallel} \textendash{} Execute using CASA HPC functionality, if available.

\end{itemize}

\sphinxlineitem{Returns}
\sphinxAtStartPar
The results object for the pipeline task is returned.

\end{description}\end{quote}
\subsubsection*{Examples}

\sphinxAtStartPar
1. Run with recommended settings to assess the need for an ALMA amplitude
renormalization correction.

\begin{sphinxVerbatim}[commandchars=\\\{\}]
\PYG{g+gp}{\PYGZgt{}\PYGZgt{}\PYGZgt{} }\PYG{n}{hifa\PYGZus{}renorm}\PYG{p}{(}\PYG{p}{)}
\end{sphinxVerbatim}

\sphinxAtStartPar
2. Run to assess the necessary ALMA amplitude renormalization correction,
and apply this correction if it exceeds a threshold of 3\% (1.03).

\begin{sphinxVerbatim}[commandchars=\\\{\}]
\PYG{g+gp}{\PYGZgt{}\PYGZgt{}\PYGZgt{} }\PYG{n}{hifa\PYGZus{}renorm}\PYG{p}{(}\PYG{n}{createcaltable}\PYG{o}{=}\PYG{k+kc}{True}\PYG{p}{,} \PYG{n}{threshold}\PYG{o}{=}\PYG{l+m+mf}{1.03}\PYG{p}{)}
\end{sphinxVerbatim}

\end{fulllineitems}


\sphinxstepscope


\section{pipeline.hifa.cli.hifa\_restoredata}
\label{\detokenize{_autosummary/pipeline.hifa.cli.hifa_restoredata:pipeline-hifa-cli-hifa-restoredata}}\label{\detokenize{_autosummary/pipeline.hifa.cli.hifa_restoredata::doc}}\index{hifa\_restoredata() (in module pipeline.hifa.cli)@\spxentry{hifa\_restoredata()}\spxextra{in module pipeline.hifa.cli}}

\begin{fulllineitems}
\phantomsection\label{\detokenize{_autosummary/pipeline.hifa.cli.hifa_restoredata:pipeline.hifa.cli.hifa_restoredata}}
\pysigstartsignatures
\pysiglinewithargsret
{\sphinxbfcode{\sphinxupquote{hifa\_restoredata}}}
{\sphinxparam{\DUrole{n}{vis}\DUrole{o}{=}\DUrole{default_value}{None}}\sphinxparamcomma \sphinxparam{\DUrole{n}{session}\DUrole{o}{=}\DUrole{default_value}{None}}\sphinxparamcomma \sphinxparam{\DUrole{n}{products\_dir}\DUrole{o}{=}\DUrole{default_value}{None}}\sphinxparamcomma \sphinxparam{\DUrole{n}{copytoraw}\DUrole{o}{=}\DUrole{default_value}{None}}\sphinxparamcomma \sphinxparam{\DUrole{n}{rawdata\_dir}\DUrole{o}{=}\DUrole{default_value}{None}}\sphinxparamcomma \sphinxparam{\DUrole{n}{lazy}\DUrole{o}{=}\DUrole{default_value}{None}}\sphinxparamcomma \sphinxparam{\DUrole{n}{bdfflags}\DUrole{o}{=}\DUrole{default_value}{None}}\sphinxparamcomma \sphinxparam{\DUrole{n}{ocorr\_mode}\DUrole{o}{=}\DUrole{default_value}{None}}\sphinxparamcomma \sphinxparam{\DUrole{n}{asis}\DUrole{o}{=}\DUrole{default_value}{None}}}
{}
\pysigstopsignatures
\sphinxAtStartPar
Restore flagged and calibration interferometry data from a pipeline run

\sphinxAtStartPar
The hifa\_restoredata task restores flagged and calibrated MeasurementSets
from archived ASDMs and pipeline flagging and calibration data products.

\sphinxAtStartPar
hifa\_restoredata assumes that the ASDMs to be restored are present in the
directory specified by the \sphinxcode{\sphinxupquote{rawdata\_dir}} (default: ‘../rawdata’).

\sphinxAtStartPar
By default (\sphinxcode{\sphinxupquote{copytoraw}} = True), hifa\_restoredata assumes that for each
ASDM in the input list, the corresponding pipeline flagging and calibration
data products (in the format produced by the hifa\_exportdata task) are
present in the directory specified by \sphinxcode{\sphinxupquote{products\_dir}} (default: ‘../products’).
At the start of the task, these products are copied from the \sphinxcode{\sphinxupquote{products\_dir}}
to the \sphinxcode{\sphinxupquote{rawdata\_dir}}.

\sphinxAtStartPar
If \sphinxcode{\sphinxupquote{copytoraw}} = False, hifa\_restoredata assumes that these products are
to be found in \sphinxcode{\sphinxupquote{rawdata\_dir}} along with the ASDMs.

\sphinxAtStartPar
The expected flagging and calibration products (for each ASDM) include:
\begin{itemize}
\item {} 
\sphinxAtStartPar
a compressed tar file of the final flagversions file, e.g.
uid\_\_\_A002\_X30a93d\_X43e.ms.flagversions.tar.gz

\item {} 
\sphinxAtStartPar
a text file containing the applycal instructions, e.g.
uid\_\_\_A002\_X30a93d\_X43e.ms.calapply.txt

\item {} 
\sphinxAtStartPar
a compressed tar file containing the caltables for the parent session,
e.g. uid\_\_\_A001\_X74\_X29.session\_3.caltables.tar.gz

\end{itemize}

\sphinxAtStartPar
hifa\_restoredata performs the following operations:
\begin{itemize}
\item {} 
\sphinxAtStartPar
imports the ASDM(s)

\item {} 
\sphinxAtStartPar
removes the default MS.flagversions directory created by the filler

\item {} 
\sphinxAtStartPar
restores the final MS.flagversions directory stored by the pipeline

\item {} 
\sphinxAtStartPar
restores the final set of pipeline flags to the MS

\item {} 
\sphinxAtStartPar
restores the final calibration state of the MS

\item {} 
\sphinxAtStartPar
restores the final calibration tables for each MS

\item {} 
\sphinxAtStartPar
applies the calibration tables to each MS

\end{itemize}

\sphinxAtStartPar
When importing the ASDM and converting it to a Measurement Set (MS), if the
output MS already exists in the output directory, then the importasdm
conversion step is skipped, and instead the existing MS will be imported.
\begin{quote}\begin{description}
\sphinxlineitem{Parameters}\begin{itemize}
\item {} 
\sphinxAtStartPar
\sphinxstylestrong{vis} \textendash{} List of raw visibility data files to be restored.
Assumed to be in the directory specified by rawdata\_dir.
Example: vis={[}‘uid\_\_\_A002\_X30a93d\_X43e’{]}

\item {} 
\sphinxAtStartPar
\sphinxstylestrong{session} \textendash{} List of sessions one per visibility file.
Example: session={[}‘session\_3’{]}

\item {} 
\sphinxAtStartPar
\sphinxstylestrong{products\_dir} \textendash{} Name of the data products directory to copy calibration
products from.
Default: ‘../products’
The parameter is effective only when \sphinxcode{\sphinxupquote{copytoraw}} = True.
When \sphinxcode{\sphinxupquote{copytoraw}} = False, calibration products in
\sphinxcode{\sphinxupquote{rawdata\_dir}} will be used.
Example: products\_dir=’myproductspath’

\item {} 
\sphinxAtStartPar
\sphinxstylestrong{copytoraw} \textendash{} Copy calibration and flagging tables from \sphinxcode{\sphinxupquote{products\_dir}} to
\sphinxcode{\sphinxupquote{rawdata\_dir}} directory.
Default: True.
Example: copytoraw=False

\item {} 
\sphinxAtStartPar
\sphinxstylestrong{rawdata\_dir} \textendash{} Name of the raw data directory.
Default: ‘../rawdata’.
Example: rawdata\_dir=’myrawdatapath’

\item {} 
\sphinxAtStartPar
\sphinxstylestrong{lazy} \textendash{} Use the lazy filler option.
Default: False.
Example: lazy=True

\item {} 
\sphinxAtStartPar
\sphinxstylestrong{bdfflags} \textendash{} Set the BDF flags.
Default: True.
Example: bdfflags=False

\item {} 
\sphinxAtStartPar
\sphinxstylestrong{ocorr\_mode} \textendash{} Set ocorr\_mode.
Default: ‘ca’.
Example: ocorr\_mode=’ca’

\item {} 
\sphinxAtStartPar
\sphinxstylestrong{asis} \textendash{} Creates verbatim copies of the ASDM tables in the output MS.
The value given to this option must be a string containing a
list of table names separated by whitespace characters.
Default: ‘SBSummary ExecBlock Antenna Annotation Station Receiver Source CalAtmosphere CalWVR CalPointing’.
Example: asis=’Source Receiver’

\end{itemize}

\sphinxlineitem{Returns}
\sphinxAtStartPar
The results object for the pipeline task is returned.

\end{description}\end{quote}
\subsubsection*{Examples}
\begin{enumerate}
\sphinxsetlistlabels{\arabic}{enumi}{enumii}{}{.}%
\item {} 
\sphinxAtStartPar
Restore the pipeline results for a single ASDM in a single session:

\end{enumerate}

\begin{sphinxVerbatim}[commandchars=\\\{\}]
\PYG{g+gp}{\PYGZgt{}\PYGZgt{}\PYGZgt{} }\PYG{n}{hifa\PYGZus{}restoredata}\PYG{p}{(}\PYG{n}{vis}\PYG{o}{=}\PYG{p}{[}\PYG{l+s+s1}{\PYGZsq{}}\PYG{l+s+s1}{uid\PYGZus{}\PYGZus{}\PYGZus{}A002\PYGZus{}X30a93d\PYGZus{}X43e}\PYG{l+s+s1}{\PYGZsq{}}\PYG{p}{]}\PYG{p}{,} \PYG{n}{session}\PYG{o}{=}\PYG{p}{[}\PYG{l+s+s1}{\PYGZsq{}}\PYG{l+s+s1}{session\PYGZus{}1}\PYG{l+s+s1}{\PYGZsq{}}\PYG{p}{]}\PYG{p}{,} \PYG{n}{ocorr\PYGZus{}mode}\PYG{o}{=}\PYG{l+s+s1}{\PYGZsq{}}\PYG{l+s+s1}{ca}\PYG{l+s+s1}{\PYGZsq{}}\PYG{p}{)}
\end{sphinxVerbatim}

\end{fulllineitems}


\sphinxstepscope


\section{pipeline.hifa.cli.hifa\_session\_refant}
\label{\detokenize{_autosummary/pipeline.hifa.cli.hifa_session_refant:pipeline-hifa-cli-hifa-session-refant}}\label{\detokenize{_autosummary/pipeline.hifa.cli.hifa_session_refant::doc}}\index{hifa\_session\_refant() (in module pipeline.hifa.cli)@\spxentry{hifa\_session\_refant()}\spxextra{in module pipeline.hifa.cli}}

\begin{fulllineitems}
\phantomsection\label{\detokenize{_autosummary/pipeline.hifa.cli.hifa_session_refant:pipeline.hifa.cli.hifa_session_refant}}
\pysigstartsignatures
\pysiglinewithargsret
{\sphinxbfcode{\sphinxupquote{hifa\_session\_refant}}}
{\sphinxparam{\DUrole{n}{vis}\DUrole{o}{=}\DUrole{default_value}{None}}\sphinxparamcomma \sphinxparam{\DUrole{n}{phase\_threshold}\DUrole{o}{=}\DUrole{default_value}{None}}}
{}
\pysigstopsignatures
\sphinxAtStartPar
Select best reference antenna for session(s)

\sphinxAtStartPar
This task re\sphinxhyphen{}evaluates the reference antenna lists from all measurement sets
within a session and combines these to select a single common reference
antenna (per session) that is to be used by any subsequent pipeline stages.
\begin{quote}\begin{description}
\sphinxlineitem{Parameters}\begin{itemize}
\item {} 
\sphinxAtStartPar
\sphinxstylestrong{vis} \textendash{} List of input MeasurementSets. Defaults to the list of
MeasurementSets specified in the pipeline context.
Example: vis={[}‘ngc5921.ms’{]}

\item {} 
\sphinxAtStartPar
\sphinxstylestrong{phase\_threshold} \textendash{} Threshold (in degrees) used to identify absolute phase
solution outliers in caltables.
Example: phase\_threshold=0.005

\end{itemize}

\sphinxlineitem{Returns}
\sphinxAtStartPar
The results object for the pipeline task is returned.

\end{description}\end{quote}
\subsubsection*{Examples}
\begin{enumerate}
\sphinxsetlistlabels{\arabic}{enumi}{enumii}{}{.}%
\item {} 
\sphinxAtStartPar
Compute a single common reference antenna per session:

\end{enumerate}

\begin{sphinxVerbatim}[commandchars=\\\{\}]
\PYG{g+gp}{\PYGZgt{}\PYGZgt{}\PYGZgt{} }\PYG{n}{hifa\PYGZus{}session\PYGZus{}refant}\PYG{p}{(}\PYG{p}{)}
\end{sphinxVerbatim}

\end{fulllineitems}


\sphinxstepscope


\section{pipeline.hifa.cli.hifa\_spwphaseup}
\label{\detokenize{_autosummary/pipeline.hifa.cli.hifa_spwphaseup:pipeline-hifa-cli-hifa-spwphaseup}}\label{\detokenize{_autosummary/pipeline.hifa.cli.hifa_spwphaseup::doc}}\index{hifa\_spwphaseup() (in module pipeline.hifa.cli)@\spxentry{hifa\_spwphaseup()}\spxextra{in module pipeline.hifa.cli}}

\begin{fulllineitems}
\phantomsection\label{\detokenize{_autosummary/pipeline.hifa.cli.hifa_spwphaseup:pipeline.hifa.cli.hifa_spwphaseup}}
\pysigstartsignatures
\pysiglinewithargsret
{\sphinxbfcode{\sphinxupquote{hifa\_spwphaseup}}}
{\sphinxparam{\DUrole{n}{vis}\DUrole{o}{=}\DUrole{default_value}{None}}\sphinxparamcomma \sphinxparam{\DUrole{n}{caltable}\DUrole{o}{=}\DUrole{default_value}{None}}\sphinxparamcomma \sphinxparam{\DUrole{n}{field}\DUrole{o}{=}\DUrole{default_value}{None}}\sphinxparamcomma \sphinxparam{\DUrole{n}{intent}\DUrole{o}{=}\DUrole{default_value}{None}}\sphinxparamcomma \sphinxparam{\DUrole{n}{spw}\DUrole{o}{=}\DUrole{default_value}{None}}\sphinxparamcomma \sphinxparam{\DUrole{n}{hm\_spwmapmode}\DUrole{o}{=}\DUrole{default_value}{None}}\sphinxparamcomma \sphinxparam{\DUrole{n}{maxnarrowbw}\DUrole{o}{=}\DUrole{default_value}{None}}\sphinxparamcomma \sphinxparam{\DUrole{n}{minfracmaxbw}\DUrole{o}{=}\DUrole{default_value}{None}}\sphinxparamcomma \sphinxparam{\DUrole{n}{samebb}\DUrole{o}{=}\DUrole{default_value}{None}}\sphinxparamcomma \sphinxparam{\DUrole{n}{phasesnr}\DUrole{o}{=}\DUrole{default_value}{None}}\sphinxparamcomma \sphinxparam{\DUrole{n}{bwedgefrac}\DUrole{o}{=}\DUrole{default_value}{None}}\sphinxparamcomma \sphinxparam{\DUrole{n}{hm\_nantennas}\DUrole{o}{=}\DUrole{default_value}{None}}\sphinxparamcomma \sphinxparam{\DUrole{n}{maxfracflagged}\DUrole{o}{=}\DUrole{default_value}{None}}\sphinxparamcomma \sphinxparam{\DUrole{n}{combine}\DUrole{o}{=}\DUrole{default_value}{None}}\sphinxparamcomma \sphinxparam{\DUrole{n}{refant}\DUrole{o}{=}\DUrole{default_value}{None}}\sphinxparamcomma \sphinxparam{\DUrole{n}{minblperant}\DUrole{o}{=}\DUrole{default_value}{None}}\sphinxparamcomma \sphinxparam{\DUrole{n}{minsnr}\DUrole{o}{=}\DUrole{default_value}{None}}\sphinxparamcomma \sphinxparam{\DUrole{n}{unregister\_existing}\DUrole{o}{=}\DUrole{default_value}{None}}}
{}
\pysigstopsignatures
\sphinxAtStartPar
Compute phase calibration spw map and per spw phase offsets

\sphinxAtStartPar
The spw map for phase calibration is computed. Phase offsets as a function of
spectral window are computed using high signal\sphinxhyphen{}to\sphinxhyphen{}noise calibration observations.

\sphinxAtStartPar
Previous calibrations are applied on the fly.

\sphinxAtStartPar
hifa\_spwphaseup performs two functions:
\begin{itemize}
\item {} 
\sphinxAtStartPar
Determines the spectral window mapping or combination mode, for each phase
and check source, to use when solving for phase as a function of time
(gfluxscale and timegaincal), and when applying those solutions to
targets.

\item {} 
\sphinxAtStartPar
Computes the per\sphinxhyphen{}spectral\sphinxhyphen{}window phase offset table that will be applied
to the data to remove mean phase differences between the spectral windows.

\end{itemize}

\sphinxAtStartPar
If \sphinxcode{\sphinxupquote{hm\_spwmapmode}} = ‘auto’, then the spectral window map is computed for
each SpectralSpec and each source with phase or check intent, using the
following algorithm:
\begin{itemize}
\item {} 
\sphinxAtStartPar
Estimate the per\sphinxhyphen{}spectral\sphinxhyphen{}window (spw) per\sphinxhyphen{}scan signal\sphinxhyphen{}to\sphinxhyphen{}noise ratio for
each phase and check source based on catalog flux densities, Tsys, number
of antennas, and integration scan time.

\item {} 
\sphinxAtStartPar
If the signal\sphinxhyphen{}to\sphinxhyphen{}noise of all spws is greater than \sphinxcode{\sphinxupquote{phasesnr}}, then
\sphinxcode{\sphinxupquote{hm\_spwmapmode}} = ‘default’ mapping is used in which each spw is used
to calibrate itself.

\item {} 
\sphinxAtStartPar
If the signal\sphinxhyphen{}to\sphinxhyphen{}noise of only some spws are greater than the value of
\sphinxcode{\sphinxupquote{phasesnr}}, then each lower\sphinxhyphen{}SNR spw is mapped to the highest SNR spw
in the same SpectralSpec.

\item {} 
\sphinxAtStartPar
If all spws have low SNR, or SNR cannot be computed for any reason (for
example, there is no flux information), then \sphinxcode{\sphinxupquote{hm\_spwmapmode}} = ‘combine’.

\end{itemize}

\sphinxAtStartPar
If \sphinxcode{\sphinxupquote{hm\_spwmapmode}} = ‘combine’, hifa\_spwphaseup maps all the science windows
to a single science spectral window. For example, if the list of science
spectral windows is {[}9, 11, 13, 15{]} then all the science spectral windows
in the data will be combined and mapped to the science window 9 in the
combined phase vs time calibration table.

\sphinxAtStartPar
If \sphinxcode{\sphinxupquote{hm\_spwmapmode}} = ‘simple’, a mapping from narrow science to wider
science spectral windows is computed using the following algorithm:
\begin{itemize}
\item {} 
\sphinxAtStartPar
Construct a list of the bandwidths of all the science spectral windows.

\item {} 
\sphinxAtStartPar
Determine the maximum bandwidth in this list as ‘maxbandwidth’

\item {} 
\sphinxAtStartPar
For each science spectral window with bandwidth less than ‘maxbandwidth’
construct a list of spectral windows with bandwidths greater than
\sphinxcode{\sphinxupquote{minfracmaxbw}} * ‘maxbandwidth’, then select the spectral window in this
list whose band center most closely matches the band center of the narrow
spectral window, and preferentially match within the same baseband if
\sphinxcode{\sphinxupquote{samebb}} = True.

\end{itemize}

\sphinxAtStartPar
If \sphinxcode{\sphinxupquote{hm\_spwmapmode}} = ‘default’ the spw mapping is assumed to be one to one.

\sphinxAtStartPar
Phase offsets per spectral window are determined by computing a phase only gain
calibration on the selected data, normally the high signal\sphinxhyphen{}to\sphinxhyphen{}noise bandpass
calibrator observations, using the solution interval ‘inf’.

\sphinxAtStartPar
At the end of the task the spectral window map and the phase offset calibration
table in the pipeline are stored in the context for use by later tasks.

\sphinxAtStartPar
Finally, the SNR of the calibration solutions are inspected and if the median value
on a per\sphinxhyphen{}spw basis does not reach specific thresholds, then issue a warning and reduced
QA score, with thresholds at \sphinxcode{\sphinxupquote{phasesnr}} {\color{red}\bfseries{}*}0.75 (blue), {\color{red}\bfseries{}*}0.5 (yellow) and {\color{red}\bfseries{}*}0.33 (red).
\begin{quote}\begin{description}
\sphinxlineitem{Parameters}\begin{itemize}
\item {} 
\sphinxAtStartPar
\sphinxstylestrong{vis} \textendash{} The list of input MeasurementSets. Defaults to the list of
MeasurementSets specified in the pipeline context.
Example: vis={[}‘M82A.ms’, ‘M82B.ms’{]}

\item {} 
\sphinxAtStartPar
\sphinxstylestrong{caltable} \textendash{} The list of output calibration tables. Defaults to the standard
pipeline naming convention.
Example: caltable={[}‘M82.gcal’, ‘M82B.gcal’{]}

\item {} 
\sphinxAtStartPar
\sphinxstylestrong{field} \textendash{} The list of field names or field ids for which phase offset solutions
are to be computed. Defaults to all fields with the default intent.
Example: field=’3C279’, field=’3C279, M82’

\item {} 
\sphinxAtStartPar
\sphinxstylestrong{intent} \textendash{} A string containing a comma delimited list of intents against
which the selected fields are matched. Defaults to the BANDPASS
observations.
Example: intent=’PHASE’

\item {} 
\sphinxAtStartPar
\sphinxstylestrong{spw} \textendash{} The list of spectral windows and channels for which gain solutions are
computed. Defaults to all the science spectral windows.
Example: spw=’13,15’

\item {} 
\sphinxAtStartPar
\sphinxstylestrong{hm\_spwmapmode} \textendash{} The spectral window mapping mode. The options are: ‘auto’,
‘combine’, ‘simple’, and ‘default’. In ‘auto’ mode hifa\_spwphaseup
estimates the SNR of the phase calibrator observations and uses these
estimates to choose between ‘combine’ mode (low SNR) and ‘default’ mode
(high SNR). In combine mode all spectral windows are combined and mapped to
one spectral window. In ‘simple’ mode narrow spectral windows are mapped to
wider ones using an algorithm defined by ‘maxnarrowbw’, ‘minfracmaxbw’, and
‘samebb’. In ‘default’ mode the spectral window map defaults to the
standard one to one mapping.
Example: hm\_spwmapmode=’combine’

\item {} 
\sphinxAtStartPar
\sphinxstylestrong{maxnarrowbw} \textendash{} The maximum bandwidth defining narrow spectral windows. Values
must be in CASA compatible frequency units.
Example: maxnarrowbw=’’

\item {} 
\sphinxAtStartPar
\sphinxstylestrong{minfracmaxbw} \textendash{} The minimum fraction of the maximum bandwidth in the set of
spws to use for matching.
Example: minfracmaxbw=0.75

\item {} 
\sphinxAtStartPar
\sphinxstylestrong{samebb} \textendash{} Match within the same baseband if possible.
Example: samebb=False

\item {} 
\sphinxAtStartPar
\sphinxstylestrong{phasesnr} \textendash{} The required gaincal solution signal\sphinxhyphen{}to\sphinxhyphen{}noise.
Example: phaseupsnr=20.0

\item {} 
\sphinxAtStartPar
\sphinxstylestrong{bwedgefrac} \textendash{} The fraction of the bandwidth edges that is flagged.
Example: bwedgefrac=0.0

\item {} 
\sphinxAtStartPar
\sphinxstylestrong{hm\_nantennas} \textendash{} The heuristics for determines the number of antennas to use
in the signal\sphinxhyphen{}to\sphinxhyphen{}noise estimate. The options are ‘all’ and ‘unflagged’.
The ‘unflagged’ options is not currently supported.
Example: hm\_nantennas=’unflagged’

\item {} 
\sphinxAtStartPar
\sphinxstylestrong{maxfracflagged} \textendash{} The maximum fraction of an antenna that can be flagged
before it is excluded from the signal\sphinxhyphen{}to\sphinxhyphen{}noise estimate.
Example: maxfracflagged=0.80

\item {} 
\sphinxAtStartPar
\sphinxstylestrong{combine} \textendash{} Data axes to combine for solving. Options are ‘’, ‘scan’, ‘spw’,
‘field’ or any comma\sphinxhyphen{}separated combination.
Example: combine=’’

\item {} 
\sphinxAtStartPar
\sphinxstylestrong{refant} \textendash{} Reference antenna name(s) in priority order. Defaults to most recent
values set in the pipeline context.  If no reference antenna is defined in
the pipeline context the CASA defaults are used.
Example: refant=’DV01’, refant=’DV05,DV07’

\item {} 
\sphinxAtStartPar
\sphinxstylestrong{minblperant} \textendash{} Minimum number of baselines required per antenna for each solve.
Antennas with fewer baselines are excluded from solutions.
Example: minblperant=2

\item {} 
\sphinxAtStartPar
\sphinxstylestrong{minsnr} \textendash{} Solutions below this SNR are rejected.

\item {} 
\sphinxAtStartPar
\sphinxstylestrong{unregister\_existing} \textendash{} Unregister previous spwphaseup calibrations from the pipeline context
before registering the new calibrations from this task.

\end{itemize}

\sphinxlineitem{Returns}
\sphinxAtStartPar
The results object for the pipeline task is returned.

\end{description}\end{quote}
\subsubsection*{Examples}

\sphinxAtStartPar
1. Compute the default spectral window map and the per spectral window phase
offsets:

\begin{sphinxVerbatim}[commandchars=\\\{\}]
\PYG{g+gp}{\PYGZgt{}\PYGZgt{}\PYGZgt{} }\PYG{n}{hifa\PYGZus{}spwphaseup}\PYG{p}{(}\PYG{p}{)}
\end{sphinxVerbatim}

\sphinxAtStartPar
2. Compute the default spectral window map and the per spectral window phase
offsets set the spectral window mapping mode to ‘simple’:

\begin{sphinxVerbatim}[commandchars=\\\{\}]
\PYG{g+gp}{\PYGZgt{}\PYGZgt{}\PYGZgt{} }\PYG{n}{hifa\PYGZus{}spwphaseup}\PYG{p}{(}\PYG{n}{hm\PYGZus{}spwmapmode}\PYG{o}{=}\PYG{l+s+s1}{\PYGZsq{}}\PYG{l+s+s1}{simple}\PYG{l+s+s1}{\PYGZsq{}}\PYG{p}{)}
\end{sphinxVerbatim}

\end{fulllineitems}


\sphinxstepscope


\section{pipeline.hifa.cli.hifa\_targetflag}
\label{\detokenize{_autosummary/pipeline.hifa.cli.hifa_targetflag:pipeline-hifa-cli-hifa-targetflag}}\label{\detokenize{_autosummary/pipeline.hifa.cli.hifa_targetflag::doc}}\index{hifa\_targetflag() (in module pipeline.hifa.cli)@\spxentry{hifa\_targetflag()}\spxextra{in module pipeline.hifa.cli}}

\begin{fulllineitems}
\phantomsection\label{\detokenize{_autosummary/pipeline.hifa.cli.hifa_targetflag:pipeline.hifa.cli.hifa_targetflag}}
\pysigstartsignatures
\pysiglinewithargsret
{\sphinxbfcode{\sphinxupquote{hifa\_targetflag}}}
{\sphinxparam{\DUrole{n}{vis}\DUrole{o}{=}\DUrole{default_value}{None}}}
{}
\pysigstopsignatures
\sphinxAtStartPar
Flag target source outliers

\sphinxAtStartPar
This task flags very obvious target source outliers. The calibration tables and
flags accumulated in the cal library up to this point are pre\sphinxhyphen{}applied, then
hif\_correctedampflag is called for just the TARGET intent. Any resulting
flags are applied and the calibration library is restored to the state before
calling this task.

\sphinxAtStartPar
Because science targets are generally not point sources, the flagging algorithm
needs to be more clever than for point source calibrators. The algorithm identifies
outliers by examining statistics within successive overlapping radial uv bins,
allowing it to adapt to an arbitrary uv structure. Outliers must appear to be a
potential outlier in two bins in order to be declared an outlier.  To further avoid
overflagging of good data, only the highest threshold levels are used (+12/\sphinxhyphen{}13 sigma).
This stage does can add significant processing time, particularly in making the plots.
So to save time, the amp vs. time plots are created only if flags are generated, and
the amp vs. uv distance plots are made for only those spws that generated flags.
Also, to avoid confusion in mosaics and single field surveys, the amp vs. uv distance
plots only show field IDs with new flags.
\begin{quote}\begin{description}
\sphinxlineitem{Parameters}
\sphinxAtStartPar
\sphinxstylestrong{vis} \textendash{} The list of input MeasurementSets. Defaults to the list of
MeasurementSets specified in the h\_init or hif\_importdata task.
‘’: use all MeasurementSets in the context
Examples: ‘ngc5921.ms’, {[}‘ngc5921a.ms’, ngc5921b.ms’, ‘ngc5921c.ms’{]}

\sphinxlineitem{Returns}
\sphinxAtStartPar
The results object for the pipeline task is returned.

\end{description}\end{quote}
\subsubsection*{Examples}
\begin{enumerate}
\sphinxsetlistlabels{\arabic}{enumi}{enumii}{}{.}%
\item {} 
\sphinxAtStartPar
Run with recommended settings to flag outliers in science target(s):

\end{enumerate}

\begin{sphinxVerbatim}[commandchars=\\\{\}]
\PYG{g+gp}{\PYGZgt{}\PYGZgt{}\PYGZgt{} }\PYG{n}{hifa\PYGZus{}targetflag}\PYG{p}{(}\PYG{p}{)}
\end{sphinxVerbatim}

\end{fulllineitems}


\sphinxstepscope


\section{pipeline.hifa.cli.hifa\_timegaincal}
\label{\detokenize{_autosummary/pipeline.hifa.cli.hifa_timegaincal:pipeline-hifa-cli-hifa-timegaincal}}\label{\detokenize{_autosummary/pipeline.hifa.cli.hifa_timegaincal::doc}}\index{hifa\_timegaincal() (in module pipeline.hifa.cli)@\spxentry{hifa\_timegaincal()}\spxextra{in module pipeline.hifa.cli}}

\begin{fulllineitems}
\phantomsection\label{\detokenize{_autosummary/pipeline.hifa.cli.hifa_timegaincal:pipeline.hifa.cli.hifa_timegaincal}}
\pysigstartsignatures
\pysiglinewithargsret
{\sphinxbfcode{\sphinxupquote{hifa\_timegaincal}}}
{\sphinxparam{\DUrole{n}{vis}\DUrole{o}{=}\DUrole{default_value}{None}}\sphinxparamcomma \sphinxparam{\DUrole{n}{calamptable}\DUrole{o}{=}\DUrole{default_value}{None}}\sphinxparamcomma \sphinxparam{\DUrole{n}{calphasetable}\DUrole{o}{=}\DUrole{default_value}{None}}\sphinxparamcomma \sphinxparam{\DUrole{n}{offsetstable}\DUrole{o}{=}\DUrole{default_value}{None}}\sphinxparamcomma \sphinxparam{\DUrole{n}{targetphasetable}\DUrole{o}{=}\DUrole{default_value}{None}}\sphinxparamcomma \sphinxparam{\DUrole{n}{amptable}\DUrole{o}{=}\DUrole{default_value}{None}}\sphinxparamcomma \sphinxparam{\DUrole{n}{field}\DUrole{o}{=}\DUrole{default_value}{None}}\sphinxparamcomma \sphinxparam{\DUrole{n}{spw}\DUrole{o}{=}\DUrole{default_value}{None}}\sphinxparamcomma \sphinxparam{\DUrole{n}{antenna}\DUrole{o}{=}\DUrole{default_value}{None}}\sphinxparamcomma \sphinxparam{\DUrole{n}{calsolint}\DUrole{o}{=}\DUrole{default_value}{None}}\sphinxparamcomma \sphinxparam{\DUrole{n}{targetsolint}\DUrole{o}{=}\DUrole{default_value}{None}}\sphinxparamcomma \sphinxparam{\DUrole{n}{refant}\DUrole{o}{=}\DUrole{default_value}{None}}\sphinxparamcomma \sphinxparam{\DUrole{n}{refantmode}\DUrole{o}{=}\DUrole{default_value}{None}}\sphinxparamcomma \sphinxparam{\DUrole{n}{solnorm}\DUrole{o}{=}\DUrole{default_value}{None}}\sphinxparamcomma \sphinxparam{\DUrole{n}{minblperant}\DUrole{o}{=}\DUrole{default_value}{None}}\sphinxparamcomma \sphinxparam{\DUrole{n}{calminsnr}\DUrole{o}{=}\DUrole{default_value}{None}}\sphinxparamcomma \sphinxparam{\DUrole{n}{targetminsnr}\DUrole{o}{=}\DUrole{default_value}{None}}\sphinxparamcomma \sphinxparam{\DUrole{n}{smodel}\DUrole{o}{=}\DUrole{default_value}{None}}}
{}
\pysigstopsignatures
\sphinxAtStartPar
Determine temporal gains from calibrator observations

\sphinxAtStartPar
The time\sphinxhyphen{}dependent complex gains for each antenna/spwid are determined from
the raw data (DATA column) divided by the model (MODEL column), for the
specified fields. The gains are computed according to the spw\sphinxhyphen{}combination
model determined in hifa\_spwphaseup.

\sphinxAtStartPar
Previous calibrations are applied on the fly.

\sphinxAtStartPar
The complex gains are derived from the data column (raw data) divided by the
model column (usually set with hif\_setjy). The gains are obtained for the
specified solution intervals, spw combination and field combination. One
gain solution is computed for the science targets and one for the calibrator
targets.

\sphinxAtStartPar
Good candidate reference antennas can be determined using the hif\_refant
task.

\sphinxAtStartPar
Previous calibrations that have been stored in the pipeline context are
applied on the fly. Users can interact with these calibrations via the
h\_export\_calstate and h\_import\_calstate tasks.
\begin{quote}\begin{description}
\sphinxlineitem{Parameters}\begin{itemize}
\item {} 
\sphinxAtStartPar
\sphinxstylestrong{vis} \textendash{} The list of input MeasurementSets. Defaults to the list of
MeasurementSets specified in the pipeline context.
Example: vis={[}‘M82A.ms’, ‘M82B.ms’{]}

\item {} 
\sphinxAtStartPar
\sphinxstylestrong{calamptable} \textendash{} The list of output diagnostic calibration amplitude tables for
the calibration targets. Defaults to the standard pipeline naming
convention.
Example: calamptable={[}‘M82.gacal’, ‘M82B.gacal’{]}

\item {} 
\sphinxAtStartPar
\sphinxstylestrong{offsetstable} \textendash{} The list of output diagnostic phase offset tables for the
calibration targets. Defaults to the standard pipeline naming convention.
Example: offsetstable={[}‘M82.offsets.gacal’, ‘M82B.offsets.gacal’{]}

\item {} 
\sphinxAtStartPar
\sphinxstylestrong{calphasetable} \textendash{} The list of output calibration phase tables for the
calibration targets. Defaults to the standard pipeline naming convention.
Example: calphasetable={[}‘M82.gpcal’, ‘M82B.gpcal’{]}

\item {} 
\sphinxAtStartPar
\sphinxstylestrong{targetphasetable} \textendash{} The list of output phase calibration tables for the science
targets. Defaults to the standard pipeline naming convention.
Example: targetphasetable={[}‘M82.gpcal’, ‘M82B.gpcal’{]}

\item {} 
\sphinxAtStartPar
\sphinxstylestrong{amptable} \textendash{} The list of output calibration amplitude tables for the
calibration and science targets.
Defaults to the standard pipeline naming convention.
Example: amptable={[}‘M82.gacal’, ‘M82B.gacal’{]}

\item {} 
\sphinxAtStartPar
\sphinxstylestrong{field} \textendash{} The list of field names or field ids for which gain solutions are to
be computed. Defaults to all fields with the standard intent.
Example: field=’3C279’, field=’3C279, M82’

\item {} 
\sphinxAtStartPar
\sphinxstylestrong{spw} \textendash{} The list of spectral windows and channels for which gain solutions are
computed. Defaults to all science spectral windows.
Example: spw=’11’, spw=’11, 13’

\item {} 
\sphinxAtStartPar
\sphinxstylestrong{antenna} \textendash{} The selection of antennas for which gains are computed. Defaults to all.

\item {} 
\sphinxAtStartPar
\sphinxstylestrong{calsolint} \textendash{} Time solution interval in CASA syntax for calibrator source
solutions.
Example: calsolint=’inf’, calsolint=’int’, calsolint=’100sec’

\item {} 
\sphinxAtStartPar
\sphinxstylestrong{targetsolint} \textendash{} Time solution interval in CASA syntax for target source
solutions.
Example: targetsolint=’inf’, targetsolint=’int’, targetsolint=’100sec’

\item {} 
\sphinxAtStartPar
\sphinxstylestrong{refant} \textendash{} Reference antenna name(s) in priority order. Defaults to most recent
values set in the pipeline context. If no reference antenna is defined in
the pipeline context use the CASA defaults.
example: refant=’DV01’, refant=’DV05,DV07’

\item {} 
\sphinxAtStartPar
\sphinxstylestrong{refantmode} \textendash{} Controls how the refant is applied. Currently available
choices are ‘flex’, ‘strict’, and the default value of ‘’.
Setting to ‘’ allows the pipeline to select the appropriate
mode based on the state of the reference antenna list.
Examples: refantmode=’strict’, refantmode=’’

\item {} 
\sphinxAtStartPar
\sphinxstylestrong{solnorm} \textendash{} Normalise the gain solutions.

\item {} 
\sphinxAtStartPar
\sphinxstylestrong{minblperant} \textendash{} Minimum number of baselines required per antenna for each solve.
Antennas with fewer baselines are excluded from solutions.
Example: minblperant=2

\item {} 
\sphinxAtStartPar
\sphinxstylestrong{calminsnr} \textendash{} Solutions below this SNR are rejected for calibrator solutions.

\item {} 
\sphinxAtStartPar
\sphinxstylestrong{targetminsnr} \textendash{} Solutions below this SNR are rejected for science target
solutions.

\item {} 
\sphinxAtStartPar
\sphinxstylestrong{smodel} \textendash{} Point source Stokes parameters for source model (experimental)
Defaults to using standard MODEL\_DATA column data.
Example: smodel={[}1,0,0,0{]}  \sphinxhyphen{} (I=1, unpolarized)

\end{itemize}

\sphinxlineitem{Returns}
\sphinxAtStartPar
The results object for the pipeline task is returned.

\end{description}\end{quote}
\subsubsection*{Examples}

\sphinxAtStartPar
1. Compute standard per scan gain solutions that will be used to calibrate
the target:

\begin{sphinxVerbatim}[commandchars=\\\{\}]
\PYG{g+gp}{\PYGZgt{}\PYGZgt{}\PYGZgt{} }\PYG{n}{hifa\PYGZus{}timegaincal}\PYG{p}{(}\PYG{p}{)}
\end{sphinxVerbatim}

\end{fulllineitems}


\sphinxstepscope


\section{pipeline.hifa.cli.hifa\_tsysflag}
\label{\detokenize{_autosummary/pipeline.hifa.cli.hifa_tsysflag:pipeline-hifa-cli-hifa-tsysflag}}\label{\detokenize{_autosummary/pipeline.hifa.cli.hifa_tsysflag::doc}}\index{hifa\_tsysflag() (in module pipeline.hifa.cli)@\spxentry{hifa\_tsysflag()}\spxextra{in module pipeline.hifa.cli}}

\begin{fulllineitems}
\phantomsection\label{\detokenize{_autosummary/pipeline.hifa.cli.hifa_tsysflag:pipeline.hifa.cli.hifa_tsysflag}}
\pysigstartsignatures
\pysiglinewithargsret
{\sphinxbfcode{\sphinxupquote{hifa\_tsysflag}}}
{\sphinxparam{\DUrole{n}{vis}\DUrole{o}{=}\DUrole{default_value}{None}}\sphinxparamcomma \sphinxparam{\DUrole{n}{caltable}\DUrole{o}{=}\DUrole{default_value}{None}}\sphinxparamcomma \sphinxparam{\DUrole{n}{flag\_nmedian}\DUrole{o}{=}\DUrole{default_value}{None}}\sphinxparamcomma \sphinxparam{\DUrole{n}{fnm\_limit}\DUrole{o}{=}\DUrole{default_value}{None}}\sphinxparamcomma \sphinxparam{\DUrole{n}{fnm\_byfield}\DUrole{o}{=}\DUrole{default_value}{None}}\sphinxparamcomma \sphinxparam{\DUrole{n}{flag\_derivative}\DUrole{o}{=}\DUrole{default_value}{None}}\sphinxparamcomma \sphinxparam{\DUrole{n}{fd\_max\_limit}\DUrole{o}{=}\DUrole{default_value}{None}}\sphinxparamcomma \sphinxparam{\DUrole{n}{flag\_edgechans}\DUrole{o}{=}\DUrole{default_value}{None}}\sphinxparamcomma \sphinxparam{\DUrole{n}{fe\_edge\_limit}\DUrole{o}{=}\DUrole{default_value}{None}}\sphinxparamcomma \sphinxparam{\DUrole{n}{flag\_fieldshape}\DUrole{o}{=}\DUrole{default_value}{None}}\sphinxparamcomma \sphinxparam{\DUrole{n}{ff\_refintent}\DUrole{o}{=}\DUrole{default_value}{None}}\sphinxparamcomma \sphinxparam{\DUrole{n}{ff\_max\_limit}\DUrole{o}{=}\DUrole{default_value}{None}}\sphinxparamcomma \sphinxparam{\DUrole{n}{flag\_birdies}\DUrole{o}{=}\DUrole{default_value}{None}}\sphinxparamcomma \sphinxparam{\DUrole{n}{fb\_sharps\_limit}\DUrole{o}{=}\DUrole{default_value}{None}}\sphinxparamcomma \sphinxparam{\DUrole{n}{flag\_toomany}\DUrole{o}{=}\DUrole{default_value}{None}}\sphinxparamcomma \sphinxparam{\DUrole{n}{tmf1\_limit}\DUrole{o}{=}\DUrole{default_value}{None}}\sphinxparamcomma \sphinxparam{\DUrole{n}{tmef1\_limit}\DUrole{o}{=}\DUrole{default_value}{None}}\sphinxparamcomma \sphinxparam{\DUrole{n}{metric\_order}\DUrole{o}{=}\DUrole{default_value}{None}}\sphinxparamcomma \sphinxparam{\DUrole{n}{normalize\_tsys}\DUrole{o}{=}\DUrole{default_value}{None}}\sphinxparamcomma \sphinxparam{\DUrole{n}{filetemplate}\DUrole{o}{=}\DUrole{default_value}{None}}}
{}
\pysigstopsignatures
\sphinxAtStartPar
Flag deviant system temperatures for ALMA interferometry measurements.

\sphinxAtStartPar
This task flags all deviant system temperature measurements in the system
temperature calibration table by running a sequence of flagging tests, each
designed to look for a different type of possible error.

\sphinxAtStartPar
If a file with manual Tsys flags is provided with the \sphinxcode{\sphinxupquote{filetemplate}}
parameter, then these flags are applied prior to the evaluation of the
flagging heuristics listed below.

\sphinxAtStartPar
The tests are:
\begin{enumerate}
\sphinxsetlistlabels{\arabic}{enumi}{enumii}{}{.}%
\item {} 
\sphinxAtStartPar
Flag Tsys spectra with high median values

\item {} 
\sphinxAtStartPar
Flag Tsys spectra with high median derivatives. This is meant to spot
spectra that are ‘ringing’.

\item {} 
\sphinxAtStartPar
Flag the edge channels of the Tsys spectra in each SpW.

\item {} 
\sphinxAtStartPar
Flag Tsys spectra whose shape is different from that associated with
the BANDPASS intent.

\item {} 
\sphinxAtStartPar
Flag ‘birdies’.

\item {} 
\sphinxAtStartPar
Flag the Tsys spectra of all antennas in a timestamp and spw if
proportion of antennas already flagged in this timestamp and spw exceeds
a threshold, and flag Tsys spectra for all antennas and all timestamps
in a spw, if proportion of antennas that are already entirely flagged
in all timestamps exceeds a threshold.

\end{enumerate}
\begin{quote}\begin{description}
\sphinxlineitem{Parameters}\begin{itemize}
\item {} 
\sphinxAtStartPar
\sphinxstylestrong{vis} \textendash{} List of input MeasurementSets (Not used).

\item {} 
\sphinxAtStartPar
\sphinxstylestrong{caltable} \textendash{} List of input Tsys calibration tables.
Default: {[}{]} \sphinxhyphen{} Use the table currently stored in the pipeline context.
Example: caltable={[}‘X132.ms.tsys.s2.tbl’{]}

\item {} 
\sphinxAtStartPar
\sphinxstylestrong{flag\_nmedian} \textendash{} True to flag Tsys spectra with high median value.

\item {} 
\sphinxAtStartPar
\sphinxstylestrong{fnm\_limit} \textendash{} Flag spectra with median value higher than fnm\_limit * median
of this measure over all spectra.

\item {} 
\sphinxAtStartPar
\sphinxstylestrong{fnm\_byfield} \textendash{} Evaluate the nmedian metric separately for each field.

\item {} 
\sphinxAtStartPar
\sphinxstylestrong{flag\_derivative} \textendash{} True to flag Tsys spectra with high median derivative.

\item {} 
\sphinxAtStartPar
\sphinxstylestrong{fd\_max\_limit} \textendash{} Flag spectra with median derivative higher than
fd\_max\_limit * median of this measure over all spectra.

\item {} 
\sphinxAtStartPar
\sphinxstylestrong{flag\_edgechans} \textendash{} True to flag edges of Tsys spectra.

\item {} 
\sphinxAtStartPar
\sphinxstylestrong{fe\_edge\_limit} \textendash{} Flag channels whose channel to channel difference \textgreater{}
fe\_edge\_limit * median across spectrum.

\item {} 
\sphinxAtStartPar
\sphinxstylestrong{flag\_fieldshape} \textendash{} True to flag Tsys spectra with a radically different
shape to those of the ff\_refintent.

\item {} 
\sphinxAtStartPar
\sphinxstylestrong{ff\_refintent} \textendash{} Data intent that provides the reference shape for
‘flag\_fieldshape’.

\item {} 
\sphinxAtStartPar
\sphinxstylestrong{ff\_max\_limit} \textendash{} Flag Tsys spectra with ‘fieldshape’ metric values \textgreater{}
ff\_max\_limit.

\item {} 
\sphinxAtStartPar
\sphinxstylestrong{flag\_birdies} \textendash{} True to flag channels covering sharp spectral features.

\item {} 
\sphinxAtStartPar
\sphinxstylestrong{fb\_sharps\_limit} \textendash{} Flag channels bracketing a channel to channel
difference \textgreater{} fb\_sharps\_limit.

\item {} 
\sphinxAtStartPar
\sphinxstylestrong{flag\_toomany} \textendash{} True to flag Tsys spectra for which a proportion of
antennas for given timestamp and/or proportion of antennas that are
entirely flagged in all timestamps exceeds their respective thresholds.

\item {} 
\sphinxAtStartPar
\sphinxstylestrong{tmf1\_limit} \textendash{} Flag Tsys spectra for all antennas in a timestamp and spw if
proportion of antennas already flagged in this timestamp and spw exceeds
tmf1\_limit.

\item {} 
\sphinxAtStartPar
\sphinxstylestrong{tmef1\_limit} \textendash{} Flag Tsys spectra for all antennas and all timestamps
in a spw, if proportion of antennas that are already entirely flagged
in all timestamps exceeds tmef1\_limit.

\item {} 
\sphinxAtStartPar
\sphinxstylestrong{metric\_order} \textendash{} Order in which to evaluate the flagging metrics that are
enabled. Disabled metrics are skipped.

\item {} 
\sphinxAtStartPar
\sphinxstylestrong{normalize\_tsys} \textendash{} True to create a normalized Tsys table that is used to
evaluate the Tsys flagging metrics. All newly found flags are also applied
to the original Tsys caltable that continues to be used for subsequent
calibration.

\item {} 
\sphinxAtStartPar
\sphinxstylestrong{filetemplate} \textendash{} The name of a text file that contains the manual Tsys flagging
template. If the template flags file is undefined, a name of the form
‘msname.flagtsystemplate.txt’ is assumed.

\end{itemize}

\sphinxlineitem{Returns}
\sphinxAtStartPar
The results object for the pipeline task is returned.

\end{description}\end{quote}
\subsubsection*{Examples}
\begin{enumerate}
\sphinxsetlistlabels{\arabic}{enumi}{enumii}{}{.}%
\item {} 
\sphinxAtStartPar
Flag Tsys measurements using currently recommended tests:

\end{enumerate}

\begin{sphinxVerbatim}[commandchars=\\\{\}]
\PYG{g+gp}{\PYGZgt{}\PYGZgt{}\PYGZgt{} }\PYG{n}{hifa\PYGZus{}tsysflag}\PYG{p}{(}\PYG{p}{)}
\end{sphinxVerbatim}

\sphinxAtStartPar
2. Flag Tsys measurements using all recommended tests apart from that
using the ‘fieldshape’ metric:

\begin{sphinxVerbatim}[commandchars=\\\{\}]
\PYG{g+gp}{\PYGZgt{}\PYGZgt{}\PYGZgt{} }\PYG{n}{hifa\PYGZus{}tsysflag}\PYG{p}{(}\PYG{n}{flag\PYGZus{}fieldshape}\PYG{o}{=}\PYG{k+kc}{False}\PYG{p}{)}
\end{sphinxVerbatim}

\end{fulllineitems}


\sphinxstepscope


\section{pipeline.hifa.cli.hifa\_tsysflagcontamination}
\label{\detokenize{_autosummary/pipeline.hifa.cli.hifa_tsysflagcontamination:pipeline-hifa-cli-hifa-tsysflagcontamination}}\label{\detokenize{_autosummary/pipeline.hifa.cli.hifa_tsysflagcontamination::doc}}\index{hifa\_tsysflagcontamination() (in module pipeline.hifa.cli)@\spxentry{hifa\_tsysflagcontamination()}\spxextra{in module pipeline.hifa.cli}}

\begin{fulllineitems}
\phantomsection\label{\detokenize{_autosummary/pipeline.hifa.cli.hifa_tsysflagcontamination:pipeline.hifa.cli.hifa_tsysflagcontamination}}
\pysigstartsignatures
\pysiglinewithargsret
{\sphinxbfcode{\sphinxupquote{hifa\_tsysflagcontamination}}}
{\sphinxparam{\DUrole{n}{vis}\DUrole{o}{=}\DUrole{default_value}{None}}\sphinxparamcomma \sphinxparam{\DUrole{n}{caltable}\DUrole{o}{=}\DUrole{default_value}{None}}\sphinxparamcomma \sphinxparam{\DUrole{n}{filetemplate}\DUrole{o}{=}\DUrole{default_value}{None}}\sphinxparamcomma \sphinxparam{\DUrole{n}{logpath}\DUrole{o}{=}\DUrole{default_value}{None}}\sphinxparamcomma \sphinxparam{\DUrole{n}{remove\_n\_extreme}\DUrole{o}{=}\DUrole{default_value}{None}}\sphinxparamcomma \sphinxparam{\DUrole{n}{relative\_detection\_factor}\DUrole{o}{=}\DUrole{default_value}{None}}\sphinxparamcomma \sphinxparam{\DUrole{n}{diagnostic\_plots}\DUrole{o}{=}\DUrole{default_value}{None}}\sphinxparamcomma \sphinxparam{\DUrole{n}{continue\_on\_failure}\DUrole{o}{=}\DUrole{default_value}{None}}}
{}
\pysigstopsignatures
\sphinxAtStartPar
Flag line contamination in ALMA interferometric Tsys caltables

\sphinxAtStartPar
This task flags all line contamination detected through an analysis of the
Tsys and bandpass caltables.

\sphinxAtStartPar
The general idea for the detection algorithm is to discern features which
appear in the Tsys calibration tables of the scans taken in the vicinity
of the source field in comparison with the Tsys calibration tables of the
scans taken toward the bandpass. The bandpass scan should be clean of
astrophysical line features.
\begin{quote}\begin{description}
\sphinxlineitem{Parameters}\begin{itemize}
\item {} 
\sphinxAtStartPar
\sphinxstylestrong{vis} \textendash{} List of input MeasurementSets (Not used).

\item {} 
\sphinxAtStartPar
\sphinxstylestrong{caltable} \textendash{} List of input Tsys calibration tables.
Default: {[}{]} \sphinxhyphen{} Use the table currently stored in the pipeline context.
Example: caltable={[}‘X132.ms.tsys.s2.tbl’{]}

\item {} 
\sphinxAtStartPar
\sphinxstylestrong{filetemplate} \textendash{} output file to which regions to flag will be written

\item {} 
\sphinxAtStartPar
\sphinxstylestrong{logpath} \textendash{} output file to which heuristic log statements will be
written

\item {} 
\sphinxAtStartPar
\sphinxstylestrong{remove\_n\_extreme} \textendash{} expert parameter for contamination heuristic
Default: 2

\item {} 
\sphinxAtStartPar
\sphinxstylestrong{relative\_detection\_factor} \textendash{} expert parameter for contamination detection heuristic
Default: 0.005

\item {} 
\sphinxAtStartPar
\sphinxstylestrong{diagnostic\_plots} \textendash{} create diagnostic plots for the line contamination heuristic
Default: True

\item {} 
\sphinxAtStartPar
\sphinxstylestrong{continue\_on\_failure} \textendash{} controls whether pipeline execution continues if a failure
occurs in the underlying contamination detection heuristic.
Default: True

\end{itemize}

\sphinxlineitem{Returns}
\sphinxAtStartPar
The results object for the pipeline task is returned.

\end{description}\end{quote}
\subsubsection*{Examples}
\begin{enumerate}
\sphinxsetlistlabels{\arabic}{enumi}{enumii}{}{.}%
\item {} 
\sphinxAtStartPar
Flag Tsys line contamination using currently recommended parameters:

\end{enumerate}

\begin{sphinxVerbatim}[commandchars=\\\{\}]
\PYG{g+gp}{\PYGZgt{}\PYGZgt{}\PYGZgt{} }\PYG{n}{hifa\PYGZus{}tsysflagcontamination}\PYG{p}{(}\PYG{p}{)}
\end{sphinxVerbatim}
\begin{enumerate}
\sphinxsetlistlabels{\arabic}{enumi}{enumii}{}{.}%
\setcounter{enumi}{1}
\item {} 
\sphinxAtStartPar
Halt pipeline execution if a failure occurs in the underlying heuristic:

\end{enumerate}

\begin{sphinxVerbatim}[commandchars=\\\{\}]
\PYG{g+gp}{\PYGZgt{}\PYGZgt{}\PYGZgt{} }\PYG{n}{hifa\PYGZus{}tsysflagcontamination}\PYG{p}{(}\PYG{n}{continue\PYGZus{}on\PYGZus{}failure}\PYG{o}{=}\PYG{k+kc}{False}\PYG{p}{)}
\end{sphinxVerbatim}

\end{fulllineitems}


\sphinxstepscope


\section{pipeline.hifa.cli.hifa\_unlock\_refant}
\label{\detokenize{_autosummary/pipeline.hifa.cli.hifa_unlock_refant:pipeline-hifa-cli-hifa-unlock-refant}}\label{\detokenize{_autosummary/pipeline.hifa.cli.hifa_unlock_refant::doc}}\index{hifa\_unlock\_refant() (in module pipeline.hifa.cli)@\spxentry{hifa\_unlock\_refant()}\spxextra{in module pipeline.hifa.cli}}

\begin{fulllineitems}
\phantomsection\label{\detokenize{_autosummary/pipeline.hifa.cli.hifa_unlock_refant:pipeline.hifa.cli.hifa_unlock_refant}}
\pysigstartsignatures
\pysiglinewithargsret
{\sphinxbfcode{\sphinxupquote{hifa\_unlock\_refant}}}
{\sphinxparam{\DUrole{n}{vis}\DUrole{o}{=}\DUrole{default_value}{None}}}
{}
\pysigstopsignatures
\sphinxAtStartPar
Unlock reference antenna list

\sphinxAtStartPar
hifa\_unlock\_refant marks the reference antenna list as “unlocked” for
specified measurement sets, allowing the list to be modified by subsequent
tasks.

\sphinxAtStartPar
After executing hifa\_unlock\_refant, all subsequent gaincal calls will by
default be executed with refantmode=’flex’.

\sphinxAtStartPar
The refant list can be locked with the hifa\_lock\_refant task.
\begin{quote}\begin{description}
\sphinxlineitem{Parameters}
\sphinxAtStartPar
\sphinxstylestrong{vis} \textendash{} List of input MeasurementSets. Defaults to the list of
MeasurementSets specified in the pipeline context.
Example: vis={[}‘ngc5921.ms’{]}

\sphinxlineitem{Returns}
\sphinxAtStartPar
The results object for the pipeline task is returned.

\end{description}\end{quote}
\subsubsection*{Examples}
\begin{enumerate}
\sphinxsetlistlabels{\arabic}{enumi}{enumii}{}{.}%
\item {} 
\sphinxAtStartPar
Unlock the refant list for all MSes in pipeline context:

\end{enumerate}

\begin{sphinxVerbatim}[commandchars=\\\{\}]
\PYG{g+gp}{\PYGZgt{}\PYGZgt{}\PYGZgt{} }\PYG{n}{hifa\PYGZus{}unlock\PYGZus{}refant}\PYG{p}{(}\PYG{p}{)}
\end{sphinxVerbatim}

\end{fulllineitems}


\sphinxstepscope


\section{pipeline.hifa.cli.hifa\_wvrgcal}
\label{\detokenize{_autosummary/pipeline.hifa.cli.hifa_wvrgcal:pipeline-hifa-cli-hifa-wvrgcal}}\label{\detokenize{_autosummary/pipeline.hifa.cli.hifa_wvrgcal::doc}}\index{hifa\_wvrgcal() (in module pipeline.hifa.cli)@\spxentry{hifa\_wvrgcal()}\spxextra{in module pipeline.hifa.cli}}

\begin{fulllineitems}
\phantomsection\label{\detokenize{_autosummary/pipeline.hifa.cli.hifa_wvrgcal:pipeline.hifa.cli.hifa_wvrgcal}}
\pysigstartsignatures
\pysiglinewithargsret
{\sphinxbfcode{\sphinxupquote{hifa\_wvrgcal}}}
{\sphinxparam{\DUrole{n}{vis}\DUrole{o}{=}\DUrole{default_value}{None}}\sphinxparamcomma \sphinxparam{\DUrole{n}{caltable}\DUrole{o}{=}\DUrole{default_value}{None}}\sphinxparamcomma \sphinxparam{\DUrole{n}{offsetstable}\DUrole{o}{=}\DUrole{default_value}{None}}\sphinxparamcomma \sphinxparam{\DUrole{n}{hm\_toffset}\DUrole{o}{=}\DUrole{default_value}{None}}\sphinxparamcomma \sphinxparam{\DUrole{n}{toffset}\DUrole{o}{=}\DUrole{default_value}{None}}\sphinxparamcomma \sphinxparam{\DUrole{n}{segsource}\DUrole{o}{=}\DUrole{default_value}{None}}\sphinxparamcomma \sphinxparam{\DUrole{n}{sourceflag}\DUrole{o}{=}\DUrole{default_value}{None}}\sphinxparamcomma \sphinxparam{\DUrole{n}{hm\_tie}\DUrole{o}{=}\DUrole{default_value}{None}}\sphinxparamcomma \sphinxparam{\DUrole{n}{tie}\DUrole{o}{=}\DUrole{default_value}{None}}\sphinxparamcomma \sphinxparam{\DUrole{n}{nsol}\DUrole{o}{=}\DUrole{default_value}{None}}\sphinxparamcomma \sphinxparam{\DUrole{n}{disperse}\DUrole{o}{=}\DUrole{default_value}{None}}\sphinxparamcomma \sphinxparam{\DUrole{n}{wvrflag}\DUrole{o}{=}\DUrole{default_value}{None}}\sphinxparamcomma \sphinxparam{\DUrole{n}{hm\_smooth}\DUrole{o}{=}\DUrole{default_value}{None}}\sphinxparamcomma \sphinxparam{\DUrole{n}{smooth}\DUrole{o}{=}\DUrole{default_value}{None}}\sphinxparamcomma \sphinxparam{\DUrole{n}{scale}\DUrole{o}{=}\DUrole{default_value}{None}}\sphinxparamcomma \sphinxparam{\DUrole{n}{maxdistm}\DUrole{o}{=}\DUrole{default_value}{None}}\sphinxparamcomma \sphinxparam{\DUrole{n}{minnumants}\DUrole{o}{=}\DUrole{default_value}{None}}\sphinxparamcomma \sphinxparam{\DUrole{n}{mingoodfrac}\DUrole{o}{=}\DUrole{default_value}{None}}\sphinxparamcomma \sphinxparam{\DUrole{n}{refant}\DUrole{o}{=}\DUrole{default_value}{None}}\sphinxparamcomma \sphinxparam{\DUrole{n}{qa\_intent}\DUrole{o}{=}\DUrole{default_value}{None}}\sphinxparamcomma \sphinxparam{\DUrole{n}{qa\_bandpass\_intent}\DUrole{o}{=}\DUrole{default_value}{None}}\sphinxparamcomma \sphinxparam{\DUrole{n}{qa\_spw}\DUrole{o}{=}\DUrole{default_value}{None}}\sphinxparamcomma \sphinxparam{\DUrole{n}{accept\_threshold}\DUrole{o}{=}\DUrole{default_value}{None}}}
{}
\pysigstopsignatures
\sphinxAtStartPar
Generate a gain table based on Water Vapor Radiometer data, and calculate
a QA score based on its effect on the interferometric data.

\sphinxAtStartPar
Generate a gain table based on the Water Vapor Radiometer data in each vis
file. By applying the wvr calibration to the data specified by \sphinxcode{\sphinxupquote{qa\_intent}}
and \sphinxcode{\sphinxupquote{qa\_spw}}, calculate a QA score to indicate its effect on
interferometric data; a score \textgreater{} 1 implies that the phase noise is improved,
a score \textless{} 1 implies that it is made worse. If the score is less than
\sphinxcode{\sphinxupquote{accept\_threshold}} then the wvr gain table is not accepted into the
context for subsequent use.
\begin{quote}\begin{description}
\sphinxlineitem{Parameters}\begin{itemize}
\item {} 
\sphinxAtStartPar
\sphinxstylestrong{vis} \textendash{} List of input visibility files.
Default: none, in which case the vis files to be used
will be read from the context.
Example: vis={[}‘ngc5921.ms’{]}

\item {} 
\sphinxAtStartPar
\sphinxstylestrong{caltable} \textendash{} List of output gain calibration tables.
Default: none, in which case the names of the caltables
will be generated automatically.
Example: caltable=’ngc5921.wvr’

\item {} 
\sphinxAtStartPar
\sphinxstylestrong{offsetstable} \textendash{} List of input temperature offsets table files to subtract
from WVR measurements before calculating phase corrections.
Default: none, in which case no offsets are applied.
Example: offsetstable={[}‘ngc5921.cloud\_offsets’{]}

\item {} 
\sphinxAtStartPar
\sphinxstylestrong{hm\_toffset} \textendash{} If ‘manual’, set the \sphinxcode{\sphinxupquote{toffset}} parameter to the user\sphinxhyphen{}specified value.
If ‘automatic’, set the \sphinxcode{\sphinxupquote{toffset}} parameter according to the
date of the MeasurementSet; \sphinxcode{\sphinxupquote{toffset}} = \sphinxhyphen{}1 if before 2013\sphinxhyphen{}01\sphinxhyphen{}21T00:00:00
\sphinxcode{\sphinxupquote{toffset}} = 0 otherwise.

\item {} 
\sphinxAtStartPar
\sphinxstylestrong{toffset} \textendash{} Time offset (sec) between interferometric and WVR data.

\item {} 
\sphinxAtStartPar
\sphinxstylestrong{segsource} \textendash{} If True calculate new atmospheric phase correction coefficients
for each source, subject to the constraints of the \sphinxcode{\sphinxupquote{tie}} parameter.
‘segsource’ is forced to be True if the \sphinxcode{\sphinxupquote{tie}} parameter is set to a
non\sphinxhyphen{}empty value by the user or by the automatic heuristic.

\item {} 
\sphinxAtStartPar
\sphinxstylestrong{sourceflag} \textendash{} Flag the WVR data for these source(s) as bad and do not
produce corrections for it. Requires \sphinxcode{\sphinxupquote{segsource}} = True.
Example: {[}‘3C273’{]}

\item {} 
\sphinxAtStartPar
\sphinxstylestrong{hm\_tie} \textendash{} If ‘manual’, set the \sphinxcode{\sphinxupquote{tie}} parameter to the
user\sphinxhyphen{}specified value. If ‘automatic’, set the \sphinxcode{\sphinxupquote{tie}}
parameter to include with the target all calibrators
that are within 15 degrees of it: if no calibrators are
that close then \sphinxcode{\sphinxupquote{tie}} is left empty.

\item {} 
\sphinxAtStartPar
\sphinxstylestrong{tie} \textendash{} Use the same atmospheric phase correction coefficients
when calculating the WVR correction for all sources in
the \sphinxcode{\sphinxupquote{tie}}. If \sphinxcode{\sphinxupquote{tie}} is not empty then \sphinxcode{\sphinxupquote{segsource}}
is forced to be True. Ignored unless \sphinxcode{\sphinxupquote{hm\_tie}} = ‘manual’.
Example: tie={[}‘3C273,NGC253’, ‘IC433,3C279’{]}

\item {} 
\sphinxAtStartPar
\sphinxstylestrong{nsol} \textendash{} Number of solutions for phase correction coefficients during this
observation, evenly distributed in time throughout the observation. It
is used only if \sphinxcode{\sphinxupquote{segsource}} = False because if \sphinxcode{\sphinxupquote{segsource}} = True then the
coefficients are recomputed whenever the telescope moves to a new
source (within the limits imposed by \sphinxcode{\sphinxupquote{tie}}).

\item {} 
\sphinxAtStartPar
\sphinxstylestrong{disperse} \textendash{} Apply correction for dispersion. (Deprecated; will be removed)

\item {} 
\sphinxAtStartPar
\sphinxstylestrong{wvrflag} \textendash{} Flag the WVR data for the listed antennas as bad and replace
their data with values interpolated from the 3 nearest antennas with
unflagged data.
Example: {[}‘DV03’,’DA05’,’PM02’{]}

\item {} 
\sphinxAtStartPar
\sphinxstylestrong{hm\_smooth} \textendash{} If ‘manual’ set the \sphinxcode{\sphinxupquote{smooth}} parameter to the user\sphinxhyphen{}specified value.
If ‘automatic’, run the wvrgcal task with the range of \sphinxcode{\sphinxupquote{smooth}} parameters
required to match the integration time of the wvr data to that of the
interferometric data in each spectral window.

\item {} 
\sphinxAtStartPar
\sphinxstylestrong{smooth} \textendash{} Smooth WVR data on this timescale before calculating the correction.
Ignored unless hm\_smooth=’manual’.

\item {} 
\sphinxAtStartPar
\sphinxstylestrong{scale} \textendash{} Scale the entire phase correction by this factor.

\item {} 
\sphinxAtStartPar
\sphinxstylestrong{maxdistm} \textendash{} Maximum distance in meters of an antenna used for interpolation
from a flagged antenna.
Default: \sphinxhyphen{}1  (automatically set to 100m if \textgreater{}50\% of
antennas are 7m antennas without WVR and otherwise set to
500m).
Example: maxdistm=550

\item {} 
\sphinxAtStartPar
\sphinxstylestrong{minnumants} \textendash{} Minimum number of nearby antennas (up to 3) used for
interpolation from a flagged antenna.
Example: minnumants=3

\item {} 
\sphinxAtStartPar
\sphinxstylestrong{mingoodfrac} \textendash{} Minimum fraction of good data per antenna.

\item {} 
\sphinxAtStartPar
\sphinxstylestrong{refant} \textendash{} Ranked comma delimited list of reference antennas.
Example: refant=’DV01,DV02’

\item {} 
\sphinxAtStartPar
\sphinxstylestrong{qa\_intent} \textendash{} The list of data intents on which the wvr correction is to be
tried as a means of estimating its effectiveness.
A QA ‘view’ will be calculated for each specified intent, in each spectral
window in each vis file.
Each QA ‘view’ will consist of a pair of 2\sphinxhyphen{}d images with dimensions
{[}‘ANTENNA’, ‘TIME’{]}, one showing the data phase\sphinxhyphen{}noise before the
wvr application, the second showing the phase noise after (both ‘before’
and ‘after’ images have a bandpass calibration applied as well).
An overall QA score is calculated for each vis file, by dividing the
‘before’ images by the ‘after’ and taking the median of the result. An
overall score of 1 would correspond to no change in the phase noise,
a score \textgreater{} 1 implies an improvement.
If the overall score for a vis file is less than the value in
‘accept\_threshold’ then the wvr calibration file is not made available
for merging into the context for use in the subsequent reduction.
If you do not want any QA calculations then set qa\_intent=’’.
example: qa\_intent=’PHASE’

\item {} 
\sphinxAtStartPar
\sphinxstylestrong{qa\_bandpass\_intent} \textendash{} The data intent to use for the bandpass calibration in
the qa calculation. The default is blank to allow the underlying bandpass
task to select a sensible intent if the dataset lacks BANDPASS data.

\item {} 
\sphinxAtStartPar
\sphinxstylestrong{qa\_spw} \textendash{} The SpW(s) to use for the qa calculation, in the order that they
should be tried. Input as a comma\sphinxhyphen{}separated list. The default is blank, in
which case the task will try SpWs in order of decreasing median sky
opacity.

\item {} 
\sphinxAtStartPar
\sphinxstylestrong{accept\_threshold} \textendash{} The phase\sphinxhyphen{}rms improvement ratio
(rms without wvr / rms with wvr) above which the wrvg file will be
accepted into the context for subsequent application.

\end{itemize}

\sphinxlineitem{Returns}
\sphinxAtStartPar
The results object for the pipeline task is returned.

\end{description}\end{quote}
\subsubsection*{Examples}
\begin{enumerate}
\sphinxsetlistlabels{\arabic}{enumi}{enumii}{}{.}%
\item {} 
\sphinxAtStartPar
Compute the WVR calibration for all the MeasurementSets:

\end{enumerate}

\begin{sphinxVerbatim}[commandchars=\\\{\}]
\PYG{g+gp}{\PYGZgt{}\PYGZgt{}\PYGZgt{} }\PYG{n}{hifa\PYGZus{}wvrgcal}\PYG{p}{(}\PYG{n}{hm\PYGZus{}tie}\PYG{o}{=}\PYG{l+s+s1}{\PYGZsq{}}\PYG{l+s+s1}{automatic}\PYG{l+s+s1}{\PYGZsq{}}\PYG{p}{)}
\end{sphinxVerbatim}

\end{fulllineitems}


\sphinxstepscope


\section{pipeline.hifa.cli.hifa\_wvrgcalflag}
\label{\detokenize{_autosummary/pipeline.hifa.cli.hifa_wvrgcalflag:pipeline-hifa-cli-hifa-wvrgcalflag}}\label{\detokenize{_autosummary/pipeline.hifa.cli.hifa_wvrgcalflag::doc}}\index{hifa\_wvrgcalflag() (in module pipeline.hifa.cli)@\spxentry{hifa\_wvrgcalflag()}\spxextra{in module pipeline.hifa.cli}}

\begin{fulllineitems}
\phantomsection\label{\detokenize{_autosummary/pipeline.hifa.cli.hifa_wvrgcalflag:pipeline.hifa.cli.hifa_wvrgcalflag}}
\pysigstartsignatures
\pysiglinewithargsret
{\sphinxbfcode{\sphinxupquote{hifa\_wvrgcalflag}}}
{\sphinxparam{\DUrole{n}{vis}\DUrole{o}{=}\DUrole{default_value}{None}}\sphinxparamcomma \sphinxparam{\DUrole{n}{caltable}\DUrole{o}{=}\DUrole{default_value}{None}}\sphinxparamcomma \sphinxparam{\DUrole{n}{offsetstable}\DUrole{o}{=}\DUrole{default_value}{None}}\sphinxparamcomma \sphinxparam{\DUrole{n}{hm\_toffset}\DUrole{o}{=}\DUrole{default_value}{None}}\sphinxparamcomma \sphinxparam{\DUrole{n}{toffset}\DUrole{o}{=}\DUrole{default_value}{None}}\sphinxparamcomma \sphinxparam{\DUrole{n}{segsource}\DUrole{o}{=}\DUrole{default_value}{None}}\sphinxparamcomma \sphinxparam{\DUrole{n}{sourceflag}\DUrole{o}{=}\DUrole{default_value}{None}}\sphinxparamcomma \sphinxparam{\DUrole{n}{hm\_tie}\DUrole{o}{=}\DUrole{default_value}{None}}\sphinxparamcomma \sphinxparam{\DUrole{n}{tie}\DUrole{o}{=}\DUrole{default_value}{None}}\sphinxparamcomma \sphinxparam{\DUrole{n}{nsol}\DUrole{o}{=}\DUrole{default_value}{None}}\sphinxparamcomma \sphinxparam{\DUrole{n}{disperse}\DUrole{o}{=}\DUrole{default_value}{None}}\sphinxparamcomma \sphinxparam{\DUrole{n}{wvrflag}\DUrole{o}{=}\DUrole{default_value}{None}}\sphinxparamcomma \sphinxparam{\DUrole{n}{hm\_smooth}\DUrole{o}{=}\DUrole{default_value}{None}}\sphinxparamcomma \sphinxparam{\DUrole{n}{smooth}\DUrole{o}{=}\DUrole{default_value}{None}}\sphinxparamcomma \sphinxparam{\DUrole{n}{scale}\DUrole{o}{=}\DUrole{default_value}{None}}\sphinxparamcomma \sphinxparam{\DUrole{n}{maxdistm}\DUrole{o}{=}\DUrole{default_value}{None}}\sphinxparamcomma \sphinxparam{\DUrole{n}{minnumants}\DUrole{o}{=}\DUrole{default_value}{None}}\sphinxparamcomma \sphinxparam{\DUrole{n}{mingoodfrac}\DUrole{o}{=}\DUrole{default_value}{None}}\sphinxparamcomma \sphinxparam{\DUrole{n}{refant}\DUrole{o}{=}\DUrole{default_value}{None}}\sphinxparamcomma \sphinxparam{\DUrole{n}{flag\_intent}\DUrole{o}{=}\DUrole{default_value}{None}}\sphinxparamcomma \sphinxparam{\DUrole{n}{qa\_intent}\DUrole{o}{=}\DUrole{default_value}{None}}\sphinxparamcomma \sphinxparam{\DUrole{n}{qa\_bandpass\_intent}\DUrole{o}{=}\DUrole{default_value}{None}}\sphinxparamcomma \sphinxparam{\DUrole{n}{accept\_threshold}\DUrole{o}{=}\DUrole{default_value}{None}}\sphinxparamcomma \sphinxparam{\DUrole{n}{flag\_hi}\DUrole{o}{=}\DUrole{default_value}{None}}\sphinxparamcomma \sphinxparam{\DUrole{n}{fhi\_limit}\DUrole{o}{=}\DUrole{default_value}{None}}\sphinxparamcomma \sphinxparam{\DUrole{n}{fhi\_minsample}\DUrole{o}{=}\DUrole{default_value}{None}}\sphinxparamcomma \sphinxparam{\DUrole{n}{ants\_with\_wvr\_thresh}\DUrole{o}{=}\DUrole{default_value}{None}}}
{}
\pysigstopsignatures
\sphinxAtStartPar
Generate a gain table based on Water Vapor Radiometer data, interpolating over
antennas with bad radiometers.

\sphinxAtStartPar
This task will first identify for each vis whether it includes at least 3
antennas with Water Vapor Radiometer (WVR) data, and that the fraction of
WVR antennas / all antennas exceeds the minimum threshold
(ants\_with\_wvr\_thresh).

\sphinxAtStartPar
If there are not enough WVR antennas by number and/or fraction, then no WVR
caltable is created and no WVR calibration will be applied to the corresponding
vis. If there are enough WVR antennas, then the task proceeds as follows for
each valid vis:

\sphinxAtStartPar
First, generate a gain table based on the Water Vapor Radiometer data for
each vis.

\sphinxAtStartPar
Second, apply the WVR calibration to the data specified by ‘flag\_intent’,
calculate flagging ‘views’ showing the ratio
‘phase\sphinxhyphen{}rms with WVR / phase\sphinxhyphen{}rms without WVR’ for each scan. A ratio \textless{} 1
implies that the phase noise is improved, a ratio \textgreater{} 1 implies that it
is made worse.

\sphinxAtStartPar
Third, search the flagging views for antennas with anomalous high values.
If any are found then recalculate the WVR calibration with the ‘wvrflag’
parameter set to ignore their data and interpolate results from other
antennas according to ‘maxdistm’ and ‘minnumants’.

\sphinxAtStartPar
Fourth, after flagging, if the remaining unflagged antennas with WVR number
fewer than 3, or represent a smaller fraction of antennas than the minimum
threshold (ants\_with\_wvr\_thresh), then the WVR calibration file is rejected
and will not be merged into the context, i.e. not be used in subsequent
calibration.

\sphinxAtStartPar
Fifth, if the overall QA score for the final WVR correction of a vis file
is greater than the value in ‘accept\_threshold’ then make available the
wvr calibration file for merging into the context and use in the
subsequent reduction.
\begin{quote}\begin{description}
\sphinxlineitem{Parameters}\begin{itemize}
\item {} 
\sphinxAtStartPar
\sphinxstylestrong{vis} \textendash{} List of input visibility files.
Default: none, in which case the vis files to be used
will be read from the context.
Example: vis={[}‘ngc5921.ms’{]}

\item {} 
\sphinxAtStartPar
\sphinxstylestrong{caltable} \textendash{} List of output gain calibration tables.
Default: none, in which case the names of the caltables
will be generated automatically.
Example: caltable=’ngc5921.wvr’

\item {} 
\sphinxAtStartPar
\sphinxstylestrong{offsetstable} \textendash{} List of input temperature offsets table files to
subtract from WVR measurements before calculating phase
corrections.
Default: none, in which case no offsets are applied.
Example: offsetstable={[}‘ngc5921.cloud\_offsets’{]}

\item {} 
\sphinxAtStartPar
\sphinxstylestrong{hm\_toffset} \textendash{} If ‘manual’, set the ‘toffset’ parameter to the user\sphinxhyphen{}specified
value. If ‘automatic’, set the ‘toffset’ parameter according to the
date of the MeasurementSet; toffset=\sphinxhyphen{}1 if before 2013\sphinxhyphen{}01\sphinxhyphen{}21T00:00:00
toffset=0 otherwise.

\item {} 
\sphinxAtStartPar
\sphinxstylestrong{toffset} \textendash{} Time offset (sec) between interferometric and WVR data.

\item {} 
\sphinxAtStartPar
\sphinxstylestrong{segsource} \textendash{} If True calculate new atmospheric phase correction
coefficients for each source, subject to the
constraints of the \sphinxcode{\sphinxupquote{tie}} parameter. \sphinxcode{\sphinxupquote{segsource}} is
forced to be True if the \sphinxcode{\sphinxupquote{tie}} parameter is set to a
non\sphinxhyphen{}empty value by the user or by the automatic
heuristic.

\item {} 
\sphinxAtStartPar
\sphinxstylestrong{sourceflag} \textendash{} Flag the WVR data for these source(s) as bad and do not
produce corrections for it. Requires
\sphinxcode{\sphinxupquote{segsource}} = True.
Example: sourceflag={[}‘3C273’{]}

\item {} 
\sphinxAtStartPar
\sphinxstylestrong{hm\_tie} \textendash{} If ‘manual’, set the \sphinxcode{\sphinxupquote{tie}} parameter to the
user\sphinxhyphen{}specified value. If ‘automatic’, set the \sphinxcode{\sphinxupquote{tie}}
parameter to include with the target all calibrators
that are within 15 degrees of it: if no calibrators are
that close then \sphinxcode{\sphinxupquote{tie}} is left empty.

\item {} 
\sphinxAtStartPar
\sphinxstylestrong{tie} \textendash{} Use the same atmospheric phase correction coefficients
when calculating the WVR correction for all sources in
the \sphinxcode{\sphinxupquote{tie}}. If \sphinxcode{\sphinxupquote{tie}} is not empty then \sphinxcode{\sphinxupquote{segsource}}
is forced to be True. Ignored unless \sphinxcode{\sphinxupquote{hm\_tie}} = ‘manual’.
Example: tie={[}‘3C273,NGC253’, ‘IC433,3C279’{]}

\item {} 
\sphinxAtStartPar
\sphinxstylestrong{nsol} \textendash{} Number of solutions for phase correction coefficients during this
observation, evenly distributed in time throughout the observation. It
is used only if segsource=False because if segsource=True then the
coefficients are recomputed whenever the telescope moves to a new
source (within the limits imposed by ‘tie’).

\item {} 
\sphinxAtStartPar
\sphinxstylestrong{disperse} \textendash{} Apply correction for dispersion.
(Deprecated; will be removed)

\item {} 
\sphinxAtStartPar
\sphinxstylestrong{wvrflag} \textendash{} Flag the WVR data for these antenna(s) as bad and replace its data
with interpolated values.
Example: wvrflag={[}‘DV03’,’DA05’,’PM02’{]}

\item {} 
\sphinxAtStartPar
\sphinxstylestrong{hm\_smooth} \textendash{} If ‘manual’ set the ‘smooth’ parameter to the user\sphinxhyphen{}specified value.
If ‘automatic’, run the wvrgcal task with the range of ‘smooth’ parameters
required to match the integration time of the WVR data to that of the
interferometric data in each spectral window.

\item {} 
\sphinxAtStartPar
\sphinxstylestrong{smooth} \textendash{} Smooth WVR data on this timescale before calculating the correction.
Ignored unless hm\_smooth=’manual’.

\item {} 
\sphinxAtStartPar
\sphinxstylestrong{scale} \textendash{} Scale the entire phase correction by this factor.

\item {} 
\sphinxAtStartPar
\sphinxstylestrong{maxdistm} \textendash{} Distance in meters of an antenna used for interpolation
from a flagged antenna.
Default: \sphinxhyphen{}1 (automatically set to 100m if \textgreater{}50\% of
antennas are 7m antennas without WVR and otherwise set
to 500m).
Example: maxdistm=550

\item {} 
\sphinxAtStartPar
\sphinxstylestrong{minnumants} \textendash{} Minimum number of nearby antennas (up to 3) used for
interpolation from a flagged antenna.
Example: minnumants=3

\item {} 
\sphinxAtStartPar
\sphinxstylestrong{mingoodfrac} \textendash{} Minimum fraction of good data per antenna.
Example: mingoodfrac=0.7

\item {} 
\sphinxAtStartPar
\sphinxstylestrong{refant} \textendash{} Ranked comma delimited list of reference antennas.
Example: refant=’DV02,DV06’

\item {} 
\sphinxAtStartPar
\sphinxstylestrong{flag\_intent} \textendash{} The data intent(s) on whose WVR correction results the
search for bad WVR antennas is to be based.
A ‘flagging view’ will be calculated for each specified
intent, in each spectral window in each vis file.
Each ‘flagging view’ will consist of a 2\sphinxhyphen{}d image with dimensions
{[}‘ANTENNA’, ‘TIME’{]}, showing the phase noise after the WVR
correction has been applied.
If flag\_intent is left blank, the default, the flagging views will be
derived from data with the default bandpass calibration intent i.e.
the first in the list BANDPASS, PHASE, AMPLITUDE for which the
MeasurementSet has data.

\item {} 
\sphinxAtStartPar
\sphinxstylestrong{qa\_intent} \textendash{} The list of data intents on which the WVR correction is to be
tried as a means of estimating its effectiveness.
A QA ‘view’ will be calculated for each specified intent, in each spectral
window in each vis file.
Each QA ‘view’ will consist of a pair of 2\sphinxhyphen{}d images with dimensions
{[}‘ANTENNA’, ‘TIME’{]}, one showing the data phase\sphinxhyphen{}noise before the
WVR application, the second showing the phase noise after (both ‘before’
and ‘after’ images have a bandpass calibration applied as well).
An overall QA score is calculated for each vis file, by dividing the
‘before’ images by the ‘after’ and taking the median of the result. An
overall score of 1 would correspond to no change in the phase noise,
a score \textgreater{} 1 implies an improvement.
If the overall score for a vis file is less than the value in
‘accept\_threshold’ then the WVR calibration file is not made available for
merging into the context for use in the subsequent reduction.

\item {} 
\sphinxAtStartPar
\sphinxstylestrong{qa\_bandpass\_intent} \textendash{} The data intent to use for the bandpass calibration in
the qa calculation. The default is blank to allow the underlying bandpass
task to select a sensible intent if the dataset lacks BANDPASS data.

\item {} 
\sphinxAtStartPar
\sphinxstylestrong{accept\_threshold} \textendash{} The phase\sphinxhyphen{}rms improvement ratio
(rms without WVR / rms with WVR) above which the wrvg file will be
accepted into the context for subsequent application.

\item {} 
\sphinxAtStartPar
\sphinxstylestrong{flag\_hi} \textendash{} True to flag high figure of merit outliers.

\item {} 
\sphinxAtStartPar
\sphinxstylestrong{fhi\_limit} \textendash{} Flag figure of merit values higher than limit * MAD.

\item {} 
\sphinxAtStartPar
\sphinxstylestrong{fhi\_minsample} \textendash{} Minimum number of samples for valid MAD estimate.

\item {} 
\sphinxAtStartPar
\sphinxstylestrong{ants\_with\_wvr\_thresh} \textendash{} This threshold sets the minimum fraction of antennas
that should have WVR data for WVR calibration and flagging to proceed; the
same threshold is used to determine, after flagging, whether there remain
enough unflagged antennas with WVR data for the WVR calibration to be
applied.
Example: ants\_with\_wvr\_thresh=0.5

\end{itemize}

\sphinxlineitem{Returns}
\sphinxAtStartPar
The results object for the pipeline task is returned.

\end{description}\end{quote}
\subsubsection*{Examples}
\begin{enumerate}
\sphinxsetlistlabels{\arabic}{enumi}{enumii}{}{.}%
\item {} 
\sphinxAtStartPar
Compute the WVR calibration for all the MeasurementSets:

\end{enumerate}

\begin{sphinxVerbatim}[commandchars=\\\{\}]
\PYG{g+gp}{\PYGZgt{}\PYGZgt{}\PYGZgt{} }\PYG{n}{hifa\PYGZus{}wvrgcalflag}\PYG{p}{(}\PYG{n}{hm\PYGZus{}tie}\PYG{o}{=}\PYG{l+s+s1}{\PYGZsq{}}\PYG{l+s+s1}{automatic}\PYG{l+s+s1}{\PYGZsq{}}\PYG{p}{)}
\end{sphinxVerbatim}

\end{fulllineitems}


\sphinxstepscope


\chapter{pipeline.hifv.cli}
\label{\detokenize{_autosummary/pipeline.hifv.cli:module-pipeline.hifv.cli}}\label{\detokenize{_autosummary/pipeline.hifv.cli:pipeline-hifv-cli}}\label{\detokenize{_autosummary/pipeline.hifv.cli::doc}}\index{module@\spxentry{module}!pipeline.hifv.cli@\spxentry{pipeline.hifv.cli}}\index{pipeline.hifv.cli@\spxentry{pipeline.hifv.cli}!module@\spxentry{module}}
\sphinxAtStartPar
Interferometry VLA Tasks
\subsubsection*{Functions}


\begin{savenotes}
\sphinxatlongtablestart
\sphinxthistablewithglobalstyle
\sphinxthistablewithnovlinesstyle
\makeatletter
  \LTleft \@totalleftmargin plus1fill
  \LTright\dimexpr\columnwidth-\@totalleftmargin-\linewidth\relax plus1fill
\makeatother
\begin{longtable}{\X{1}{2}\X{1}{2}}
\sphinxtoprule
\endfirsthead

\multicolumn{2}{c}{\sphinxnorowcolor
    \makebox[0pt]{\sphinxtablecontinued{\tablename\ \thetable{} \textendash{} continued from previous page}}%
}\\
\sphinxtoprule
\endhead

\sphinxbottomrule
\multicolumn{2}{r}{\sphinxnorowcolor
    \makebox[0pt][r]{\sphinxtablecontinued{continues on next page}}%
}\\
\endfoot

\endlastfoot
\sphinxtableatstartofbodyhook

\sphinxAtStartPar
{\hyperref[\detokenize{_autosummary/pipeline.hifv.cli.hifv_analyzestokescubes:pipeline.hifv.cli.hifv_analyzestokescubes}]{\sphinxcrossref{\sphinxcode{\sphinxupquote{hifv\_analyzestokescubes}}}}}
&
\sphinxAtStartPar
Characterize stokes IQUV flux densities as a function of frequency for VLASS Coarse Cube (CC) images.
\\
\sphinxhline
\sphinxAtStartPar
{\hyperref[\detokenize{_autosummary/pipeline.hifv.cli.hifv_applycals:pipeline.hifv.cli.hifv_applycals}]{\sphinxcrossref{\sphinxcode{\sphinxupquote{hifv\_applycals}}}}}
&
\sphinxAtStartPar
Apply calibration tables to input MeasurementSets.
\\
\sphinxhline
\sphinxAtStartPar
{\hyperref[\detokenize{_autosummary/pipeline.hifv.cli.hifv_checkflag:pipeline.hifv.cli.hifv_checkflag}]{\sphinxcrossref{\sphinxcode{\sphinxupquote{hifv\_checkflag}}}}}
&
\sphinxAtStartPar
Run RFI flagging using flagdata in various modes.
\\
\sphinxhline
\sphinxAtStartPar
{\hyperref[\detokenize{_autosummary/pipeline.hifv.cli.hifv_circfeedpolcal:pipeline.hifv.cli.hifv_circfeedpolcal}]{\sphinxcrossref{\sphinxcode{\sphinxupquote{hifv\_circfeedpolcal}}}}}
&
\sphinxAtStartPar
Perform polarization calibration for VLA circular feeds.
\\
\sphinxhline
\sphinxAtStartPar
{\hyperref[\detokenize{_autosummary/pipeline.hifv.cli.hifv_exportdata:pipeline.hifv.cli.hifv_exportdata}]{\sphinxcrossref{\sphinxcode{\sphinxupquote{hifv\_exportdata}}}}}
&
\sphinxAtStartPar
Prepare and export interferometry and imaging data.
\\
\sphinxhline
\sphinxAtStartPar
{\hyperref[\detokenize{_autosummary/pipeline.hifv.cli.hifv_exportvlassdata:pipeline.hifv.cli.hifv_exportvlassdata}]{\sphinxcrossref{\sphinxcode{\sphinxupquote{hifv\_exportvlassdata}}}}}
&
\sphinxAtStartPar
Export Image data from QL, SE, and Coarse Cube modes of VLASS Survey.
\\
\sphinxhline
\sphinxAtStartPar
{\hyperref[\detokenize{_autosummary/pipeline.hifv.cli.hifv_finalcals:pipeline.hifv.cli.hifv_finalcals}]{\sphinxcrossref{\sphinxcode{\sphinxupquote{hifv\_finalcals}}}}}
&
\sphinxAtStartPar
Compute final gain calibration tables.
\\
\sphinxhline
\sphinxAtStartPar
{\hyperref[\detokenize{_autosummary/pipeline.hifv.cli.hifv_fixpointing:pipeline.hifv.cli.hifv_fixpointing}]{\sphinxcrossref{\sphinxcode{\sphinxupquote{hifv\_fixpointing}}}}}
&
\sphinxAtStartPar
Base fixpointing task
\\
\sphinxhline
\sphinxAtStartPar
{\hyperref[\detokenize{_autosummary/pipeline.hifv.cli.hifv_flagcal:pipeline.hifv.cli.hifv_flagcal}]{\sphinxcrossref{\sphinxcode{\sphinxupquote{hifv\_flagcal}}}}}
&
\sphinxAtStartPar
Flagcal task.
\\
\sphinxhline
\sphinxAtStartPar
{\hyperref[\detokenize{_autosummary/pipeline.hifv.cli.hifv_flagdata:pipeline.hifv.cli.hifv_flagdata}]{\sphinxcrossref{\sphinxcode{\sphinxupquote{hifv\_flagdata}}}}}
&
\sphinxAtStartPar
Do basic deterministic flagging of a list of MeasurementSets.
\\
\sphinxhline
\sphinxAtStartPar
{\hyperref[\detokenize{_autosummary/pipeline.hifv.cli.hifv_flagtargetsdata:pipeline.hifv.cli.hifv_flagtargetsdata}]{\sphinxcrossref{\sphinxcode{\sphinxupquote{hifv\_flagtargetsdata}}}}}
&
\sphinxAtStartPar
Apply a flagtemplate to target data prior to running imaging pipeline tasks.
\\
\sphinxhline
\sphinxAtStartPar
{\hyperref[\detokenize{_autosummary/pipeline.hifv.cli.hifv_fluxboot:pipeline.hifv.cli.hifv_fluxboot}]{\sphinxcrossref{\sphinxcode{\sphinxupquote{hifv\_fluxboot}}}}}
&
\sphinxAtStartPar
Fluxboot
\\
\sphinxhline
\sphinxAtStartPar
{\hyperref[\detokenize{_autosummary/pipeline.hifv.cli.hifv_gaincurves:pipeline.hifv.cli.hifv_gaincurves}]{\sphinxcrossref{\sphinxcode{\sphinxupquote{hifv\_gaincurves}}}}}
&
\sphinxAtStartPar
Runs gencal in gc mode.
\\
\sphinxhline
\sphinxAtStartPar
{\hyperref[\detokenize{_autosummary/pipeline.hifv.cli.hifv_hanning:pipeline.hifv.cli.hifv_hanning}]{\sphinxcrossref{\sphinxcode{\sphinxupquote{hifv\_hanning}}}}}
&
\sphinxAtStartPar
Hanning smoothing on a dataset.
\\
\sphinxhline
\sphinxAtStartPar
{\hyperref[\detokenize{_autosummary/pipeline.hifv.cli.hifv_importdata:pipeline.hifv.cli.hifv_importdata}]{\sphinxcrossref{\sphinxcode{\sphinxupquote{hifv\_importdata}}}}}
&
\sphinxAtStartPar
Imports data into the VLA pipeline.
\\
\sphinxhline
\sphinxAtStartPar
{\hyperref[\detokenize{_autosummary/pipeline.hifv.cli.hifv_mstransform:pipeline.hifv.cli.hifv_mstransform}]{\sphinxcrossref{\sphinxcode{\sphinxupquote{hifv\_mstransform}}}}}
&
\sphinxAtStartPar
Create new MeasurementSets for science target imaging
\\
\sphinxhline
\sphinxAtStartPar
{\hyperref[\detokenize{_autosummary/pipeline.hifv.cli.hifv_opcal:pipeline.hifv.cli.hifv_opcal}]{\sphinxcrossref{\sphinxcode{\sphinxupquote{hifv\_opcal}}}}}
&
\sphinxAtStartPar
Runs gencal in opac mode.
\\
\sphinxhline
\sphinxAtStartPar
{\hyperref[\detokenize{_autosummary/pipeline.hifv.cli.hifv_pbcor:pipeline.hifv.cli.hifv_pbcor}]{\sphinxcrossref{\sphinxcode{\sphinxupquote{hifv\_pbcor}}}}}
&
\sphinxAtStartPar
Apply primary beam correction to VLA and VLASS images.
\\
\sphinxhline
\sphinxAtStartPar
{\hyperref[\detokenize{_autosummary/pipeline.hifv.cli.hifv_plotsummary:pipeline.hifv.cli.hifv_plotsummary}]{\sphinxcrossref{\sphinxcode{\sphinxupquote{hifv\_plotsummary}}}}}
&
\sphinxAtStartPar
Create pipeline summary plots.
\\
\sphinxhline
\sphinxAtStartPar
{\hyperref[\detokenize{_autosummary/pipeline.hifv.cli.hifv_priorcals:pipeline.hifv.cli.hifv_priorcals}]{\sphinxcrossref{\sphinxcode{\sphinxupquote{hifv\_priorcals}}}}}
&
\sphinxAtStartPar
Runs gaincurves, opacities, requantizer gains, antenna position corrections, tec\_maps, switched power.
\\
\sphinxhline
\sphinxAtStartPar
{\hyperref[\detokenize{_autosummary/pipeline.hifv.cli.hifv_restoredata:pipeline.hifv.cli.hifv_restoredata}]{\sphinxcrossref{\sphinxcode{\sphinxupquote{hifv\_restoredata}}}}}
&
\sphinxAtStartPar
Restore flagged and calibration interferometry data from a pipeline run.
\\
\sphinxhline
\sphinxAtStartPar
{\hyperref[\detokenize{_autosummary/pipeline.hifv.cli.hifv_restorepims:pipeline.hifv.cli.hifv_restorepims}]{\sphinxcrossref{\sphinxcode{\sphinxupquote{hifv\_restorepims}}}}}
&
\sphinxAtStartPar
Restore VLASS SE per\sphinxhyphen{}image measurement set data, resetting flagging, weights, and applying self\sphinxhyphen{}calibration.
\\
\sphinxhline
\sphinxAtStartPar
{\hyperref[\detokenize{_autosummary/pipeline.hifv.cli.hifv_rqcal:pipeline.hifv.cli.hifv_rqcal}]{\sphinxcrossref{\sphinxcode{\sphinxupquote{hifv\_rqcal}}}}}
&
\sphinxAtStartPar
Runs gencal in rq mode.
\\
\sphinxhline
\sphinxAtStartPar
{\hyperref[\detokenize{_autosummary/pipeline.hifv.cli.hifv_selfcal:pipeline.hifv.cli.hifv_selfcal}]{\sphinxcrossref{\sphinxcode{\sphinxupquote{hifv\_selfcal}}}}}
&
\sphinxAtStartPar
Perform phase\sphinxhyphen{}only self\sphinxhyphen{}calibration, per scan row, on VLASS SE images.
\\
\sphinxhline
\sphinxAtStartPar
{\hyperref[\detokenize{_autosummary/pipeline.hifv.cli.hifv_semiFinalBPdcals:pipeline.hifv.cli.hifv_semiFinalBPdcals}]{\sphinxcrossref{\sphinxcode{\sphinxupquote{hifv\_semiFinalBPdcals}}}}}
&
\sphinxAtStartPar
Runs a second delay and bandpass calibration and applies to calibrators to setup for RFI flagging.
\\
\sphinxhline
\sphinxAtStartPar
{\hyperref[\detokenize{_autosummary/pipeline.hifv.cli.hifv_solint:pipeline.hifv.cli.hifv_solint}]{\sphinxcrossref{\sphinxcode{\sphinxupquote{hifv\_solint}}}}}
&
\sphinxAtStartPar
Determines different solution intervals.
\\
\sphinxhline
\sphinxAtStartPar
{\hyperref[\detokenize{_autosummary/pipeline.hifv.cli.hifv_statwt:pipeline.hifv.cli.hifv_statwt}]{\sphinxcrossref{\sphinxcode{\sphinxupquote{hifv\_statwt}}}}}
&
\sphinxAtStartPar
Compute statistical weights and write them to measurement set.
\\
\sphinxhline
\sphinxAtStartPar
{\hyperref[\detokenize{_autosummary/pipeline.hifv.cli.hifv_swpowcal:pipeline.hifv.cli.hifv_swpowcal}]{\sphinxcrossref{\sphinxcode{\sphinxupquote{hifv\_swpowcal}}}}}
&
\sphinxAtStartPar
Runs gencal in swpow mode.
\\
\sphinxhline
\sphinxAtStartPar
{\hyperref[\detokenize{_autosummary/pipeline.hifv.cli.hifv_syspower:pipeline.hifv.cli.hifv_syspower}]{\sphinxcrossref{\sphinxcode{\sphinxupquote{hifv\_syspower}}}}}
&
\sphinxAtStartPar
Determine amount of gain compression affecting VLA data below Ku\sphinxhyphen{}band.
\\
\sphinxhline
\sphinxAtStartPar
{\hyperref[\detokenize{_autosummary/pipeline.hifv.cli.hifv_targetflag:pipeline.hifv.cli.hifv_targetflag}]{\sphinxcrossref{\sphinxcode{\sphinxupquote{hifv\_targetflag}}}}}
&
\sphinxAtStartPar
Targetflag
\\
\sphinxhline
\sphinxAtStartPar
{\hyperref[\detokenize{_autosummary/pipeline.hifv.cli.hifv_tecmaps:pipeline.hifv.cli.hifv_tecmaps}]{\sphinxcrossref{\sphinxcode{\sphinxupquote{hifv\_tecmaps}}}}}
&
\sphinxAtStartPar
Base tecmaps task
\\
\sphinxhline
\sphinxAtStartPar
{\hyperref[\detokenize{_autosummary/pipeline.hifv.cli.hifv_testBPdcals:pipeline.hifv.cli.hifv_testBPdcals}]{\sphinxcrossref{\sphinxcode{\sphinxupquote{hifv\_testBPdcals}}}}}
&
\sphinxAtStartPar
Runs initial delay and bandpass calibration to setup for RFI flagging.
\\
\sphinxhline
\sphinxAtStartPar
{\hyperref[\detokenize{_autosummary/pipeline.hifv.cli.hifv_vlasetjy:pipeline.hifv.cli.hifv_vlasetjy}]{\sphinxcrossref{\sphinxcode{\sphinxupquote{hifv\_vlasetjy}}}}}
&
\sphinxAtStartPar
Sets flux density scale and fills calibrator model to measurement set.
\\
\sphinxhline
\sphinxAtStartPar
{\hyperref[\detokenize{_autosummary/pipeline.hifv.cli.hifv_vlassmasking:pipeline.hifv.cli.hifv_vlassmasking}]{\sphinxcrossref{\sphinxcode{\sphinxupquote{hifv\_vlassmasking}}}}}
&
\sphinxAtStartPar
Create clean masks for VLASS Single Epoch (SE) images.
\\
\sphinxbottomrule
\end{longtable}
\sphinxtableafterendhook
\sphinxatlongtableend
\end{savenotes}

\sphinxstepscope


\section{pipeline.hifv.cli.hifv\_analyzestokescubes}
\label{\detokenize{_autosummary/pipeline.hifv.cli.hifv_analyzestokescubes:pipeline-hifv-cli-hifv-analyzestokescubes}}\label{\detokenize{_autosummary/pipeline.hifv.cli.hifv_analyzestokescubes::doc}}\index{hifv\_analyzestokescubes() (in module pipeline.hifv.cli)@\spxentry{hifv\_analyzestokescubes()}\spxextra{in module pipeline.hifv.cli}}

\begin{fulllineitems}
\phantomsection\label{\detokenize{_autosummary/pipeline.hifv.cli.hifv_analyzestokescubes:pipeline.hifv.cli.hifv_analyzestokescubes}}
\pysigstartsignatures
\pysiglinewithargsret
{\sphinxbfcode{\sphinxupquote{hifv\_analyzestokescubes}}}
{\sphinxparam{\DUrole{n}{vis}\DUrole{o}{=}\DUrole{default_value}{None}}}
{}
\pysigstopsignatures
\sphinxAtStartPar
Characterize stokes IQUV flux densities as a function of frequency for VLASS Coarse Cube (CC) images.
\begin{quote}\begin{description}
\sphinxlineitem{Parameters}
\sphinxAtStartPar
\sphinxstylestrong{vis} \textendash{} The list of input MeasurementSets. Defaults to the list of MeasurementSets specified in the h\_init or hifv\_importdata task.

\sphinxlineitem{Returns}
\sphinxAtStartPar
The results object for the pipeline task is returned.

\end{description}\end{quote}
\subsubsection*{Examples}
\begin{enumerate}
\sphinxsetlistlabels{\arabic}{enumi}{enumii}{}{.}%
\item {} 
\sphinxAtStartPar
Basic analyzestokescubes task

\end{enumerate}

\begin{sphinxVerbatim}[commandchars=\\\{\}]
\PYG{g+gp}{\PYGZgt{}\PYGZgt{}\PYGZgt{} }\PYG{n}{hifv\PYGZus{}analyzestokescubes}\PYG{p}{(}\PYG{p}{)}
\end{sphinxVerbatim}

\end{fulllineitems}


\sphinxstepscope


\section{pipeline.hifv.cli.hifv\_applycals}
\label{\detokenize{_autosummary/pipeline.hifv.cli.hifv_applycals:pipeline-hifv-cli-hifv-applycals}}\label{\detokenize{_autosummary/pipeline.hifv.cli.hifv_applycals::doc}}\index{hifv\_applycals() (in module pipeline.hifv.cli)@\spxentry{hifv\_applycals()}\spxextra{in module pipeline.hifv.cli}}

\begin{fulllineitems}
\phantomsection\label{\detokenize{_autosummary/pipeline.hifv.cli.hifv_applycals:pipeline.hifv.cli.hifv_applycals}}
\pysigstartsignatures
\pysiglinewithargsret
{\sphinxbfcode{\sphinxupquote{hifv\_applycals}}}
{\sphinxparam{\DUrole{n}{vis}\DUrole{o}{=}\DUrole{default_value}{None}}\sphinxparamcomma \sphinxparam{\DUrole{n}{field}\DUrole{o}{=}\DUrole{default_value}{None}}\sphinxparamcomma \sphinxparam{\DUrole{n}{intent}\DUrole{o}{=}\DUrole{default_value}{None}}\sphinxparamcomma \sphinxparam{\DUrole{n}{spw}\DUrole{o}{=}\DUrole{default_value}{None}}\sphinxparamcomma \sphinxparam{\DUrole{n}{antenna}\DUrole{o}{=}\DUrole{default_value}{None}}\sphinxparamcomma \sphinxparam{\DUrole{n}{applymode}\DUrole{o}{=}\DUrole{default_value}{None}}\sphinxparamcomma \sphinxparam{\DUrole{n}{flagbackup}\DUrole{o}{=}\DUrole{default_value}{None}}\sphinxparamcomma \sphinxparam{\DUrole{n}{flagsum}\DUrole{o}{=}\DUrole{default_value}{None}}\sphinxparamcomma \sphinxparam{\DUrole{n}{flagdetailedsum}\DUrole{o}{=}\DUrole{default_value}{None}}\sphinxparamcomma \sphinxparam{\DUrole{n}{gainmap}\DUrole{o}{=}\DUrole{default_value}{None}}}
{}
\pysigstopsignatures
\sphinxAtStartPar
Apply calibration tables to input MeasurementSets.

\sphinxAtStartPar
hifv\_applycals applies the precomputed calibration tables stored in the pipeline
context to the set of visibility files using predetermined field and
spectral window maps and default values for the interpolation schemes.

\sphinxAtStartPar
Users can interact with the pipeline calibration state using the tasks
h\_export\_calstate and h\_import\_calstate.
\begin{quote}\begin{description}
\sphinxlineitem{Parameters}\begin{itemize}
\item {} 
\sphinxAtStartPar
\sphinxstylestrong{vis} \textendash{} The list of input MeasurementSets. Defaults to the list of MeasurementSets specified in the h\_init or hifv\_importdata task.

\item {} 
\sphinxAtStartPar
\sphinxstylestrong{field} \textendash{} A string containing the list of field names or field ids to which the calibration will be applied. Defaults to all fields in the pipeline
context.
example: ‘3C279’, ‘3C279, M82’

\item {} 
\sphinxAtStartPar
\sphinxstylestrong{intent} \textendash{} A string containing the list of intents against which the selected fields will be matched. Defaults to all supported intents
in the pipeline context.
example: \sphinxtitleref{‘*TARGET*’}

\item {} 
\sphinxAtStartPar
\sphinxstylestrong{spw} \textendash{} The list of spectral windows and channels to which the calibration will be applied. Defaults to all science windows in the pipeline.
example: ‘17’, ‘11, 15’

\item {} 
\sphinxAtStartPar
\sphinxstylestrong{antenna} \textendash{} The selection of antennas to which the calibration will be applied. Defaults to all antennas. Not currently supported.

\item {} 
\sphinxAtStartPar
\sphinxstylestrong{applymode} \textendash{} Calibration apply mode.
\begin{itemize}
\item {} 
\sphinxAtStartPar
‘calflag’: calibrate data and apply flags from solutions

\item {} 
\sphinxAtStartPar
‘calflagstrict’: same as above except flag spws for which calibration is
unavailable in one or more tables (instead of allowing them to pass
uncalibrated and unflagged)

\item {} 
\sphinxAtStartPar
‘trial’: report on flags from solutions, dataset entirely unchanged

\item {} 
\sphinxAtStartPar
‘flagonly’: apply flags from solutions only, data not calibrated

\item {} 
\sphinxAtStartPar
‘flagonlystrict’: same as above except flag spws for which calibration is
unavailable in one or more tables

\item {} 
\sphinxAtStartPar
‘calonly’: calibrate data only, flags from solutions NOT applied

\end{itemize}

\item {} 
\sphinxAtStartPar
\sphinxstylestrong{flagbackup} \textendash{} Backup the flags before the apply.

\item {} 
\sphinxAtStartPar
\sphinxstylestrong{flagsum} \textendash{} Compute before and after flagging summary statistics.

\item {} 
\sphinxAtStartPar
\sphinxstylestrong{flagdetailedsum} \textendash{} Compute detailed flagging statistics.

\item {} 
\sphinxAtStartPar
\sphinxstylestrong{gainmap} \textendash{} Mode to map gainfields to scans.

\end{itemize}

\sphinxlineitem{Returns}
\sphinxAtStartPar
The results object for the pipeline task is returned.

\end{description}\end{quote}
\subsubsection*{Examples}
\begin{enumerate}
\sphinxsetlistlabels{\arabic}{enumi}{enumii}{}{.}%
\item {} 
\sphinxAtStartPar
Run the final applycals stage of the VLA CASA pipeline:

\end{enumerate}

\begin{sphinxVerbatim}[commandchars=\\\{\}]
\PYG{g+gp}{\PYGZgt{}\PYGZgt{}\PYGZgt{} }\PYG{n}{hifv\PYGZus{}applycals}\PYG{p}{(}\PYG{p}{)}
\end{sphinxVerbatim}

\end{fulllineitems}


\sphinxstepscope


\section{pipeline.hifv.cli.hifv\_checkflag}
\label{\detokenize{_autosummary/pipeline.hifv.cli.hifv_checkflag:pipeline-hifv-cli-hifv-checkflag}}\label{\detokenize{_autosummary/pipeline.hifv.cli.hifv_checkflag::doc}}\index{hifv\_checkflag() (in module pipeline.hifv.cli)@\spxentry{hifv\_checkflag()}\spxextra{in module pipeline.hifv.cli}}

\begin{fulllineitems}
\phantomsection\label{\detokenize{_autosummary/pipeline.hifv.cli.hifv_checkflag:pipeline.hifv.cli.hifv_checkflag}}
\pysigstartsignatures
\pysiglinewithargsret
{\sphinxbfcode{\sphinxupquote{hifv\_checkflag}}}
{\sphinxparam{\DUrole{n}{vis}\DUrole{o}{=}\DUrole{default_value}{None}}\sphinxparamcomma \sphinxparam{\DUrole{n}{checkflagmode}\DUrole{o}{=}\DUrole{default_value}{None}}\sphinxparamcomma \sphinxparam{\DUrole{n}{growflags}\DUrole{o}{=}\DUrole{default_value}{None}}\sphinxparamcomma \sphinxparam{\DUrole{n}{overwrite\_modelcol}\DUrole{o}{=}\DUrole{default_value}{None}}}
{}
\pysigstopsignatures
\sphinxAtStartPar
Run RFI flagging using flagdata in various modes.
\begin{quote}\begin{description}
\sphinxlineitem{Parameters}\begin{itemize}
\item {} 
\sphinxAtStartPar
\sphinxstylestrong{vis} \textendash{} The list of input MeasurementSets. Defaults to the list of MeasurementSets specified in the h\_init or hifv\_importdata task.

\item {} 
\sphinxAtStartPar
\sphinxstylestrong{checkflagmode} \textendash{}
\begin{itemize}
\item {} 
\sphinxAtStartPar
Standard VLA modes with improved RFI flagging heuristics: ‘bpd\sphinxhyphen{}vla’, ‘allcals\sphinxhyphen{}vla’, ‘target\sphinxhyphen{}vla’

\item {} 
\sphinxAtStartPar
blank string default use of rflag on bandpass and delay calibrators

\item {} 
\sphinxAtStartPar
use string ‘semi’ after hifv\_semiFinalBPdcals() for executing rflag on calibrators

\item {} 
\sphinxAtStartPar
use string ‘bpd’, for the bandpass and delay calibrators:
execute rflag on all calibrated cross\sphinxhyphen{}hand corrected data;
extend flags to all correlations
execute rflag on all calibrated parallel\sphinxhyphen{}hand residual data;
extend flags to all correlations
execute tfcrop on all calibrated cross\sphinxhyphen{}hand corrected data,
per visibility; extend flags to all correlations
execute tfcrop on all calibrated parallel\sphinxhyphen{}hand corrected data,
per visibility; extend flags to all correlations

\item {} 
\sphinxAtStartPar
use string ‘allcals’, for all the other calibrators, with delays and BPcal applied:
similar procedure as ‘bpd’ mode, but uses corrected data throughout

\item {} 
\sphinxAtStartPar
use string ‘target’, for the target data:
similar procedure as ‘allcals’ mode, but with a higher SNR cutoff
for rflag to avoid flagging data due to source structure, and
with an additional series of tfcrop executions to make up for
the higher SNR cutoff in rflag

\item {} 
\sphinxAtStartPar
VLASS specific modes include ‘bpd\sphinxhyphen{}vlass’, ‘allcals\sphinxhyphen{}vlass’, and ‘target\sphinxhyphen{}vlass’
which calculate thresholds to use per spw/field/scan (action=’calculate’, then,
per baseband/field/scan, replace all spw thresholds above the median with the median,
before re\sphinxhyphen{}running rflag with the new thresholds.  This has the effect of
lowering the thresholds for spws with RFI to be closer to the RFI\sphinxhyphen{}free
thresholds, and catches more of the RFI.

\item {} 
\sphinxAtStartPar
Mode ‘vlass\sphinxhyphen{}imaging’ is similar to ‘target\sphinxhyphen{}vlass’, except that it executes on the split off target
data, intent=’{\color{red}\bfseries{}*}TARGET’, datacolumn=’data’ and uses a timedevscale of 4.0.

\end{itemize}

\item {} 
\sphinxAtStartPar
\sphinxstylestrong{growflags} \textendash{} Grow flags in time at the end of the following checkflagmodes:
\begin{itemize}
\item {} 
\sphinxAtStartPar
default=True, for ‘bpd\sphinxhyphen{}vla’, ‘allcals\sphinxhyphen{}vla’, ‘bpd’, and ‘allcals.’

\item {} 
\sphinxAtStartPar
default=False, for ‘’ and ‘semi’

\end{itemize}

\item {} 
\sphinxAtStartPar
\sphinxstylestrong{overwrite\_modelcol} \textendash{} Always write the model column, even if it already exists.

\end{itemize}

\sphinxlineitem{Returns}
\sphinxAtStartPar
The results object for the pipeline task is returned.

\end{description}\end{quote}
\subsubsection*{Examples}
\begin{enumerate}
\sphinxsetlistlabels{\arabic}{enumi}{enumii}{}{.}%
\item {} 
\sphinxAtStartPar
Run RFLAG with associated heuristics in the VLA CASA pipeline:

\end{enumerate}

\begin{sphinxVerbatim}[commandchars=\\\{\}]
\PYG{g+gp}{\PYGZgt{}\PYGZgt{}\PYGZgt{} }\PYG{n}{hifv\PYGZus{}checkflag}\PYG{p}{(}\PYG{p}{)}
\end{sphinxVerbatim}

\end{fulllineitems}


\sphinxstepscope


\section{pipeline.hifv.cli.hifv\_circfeedpolcal}
\label{\detokenize{_autosummary/pipeline.hifv.cli.hifv_circfeedpolcal:pipeline-hifv-cli-hifv-circfeedpolcal}}\label{\detokenize{_autosummary/pipeline.hifv.cli.hifv_circfeedpolcal::doc}}\index{hifv\_circfeedpolcal() (in module pipeline.hifv.cli)@\spxentry{hifv\_circfeedpolcal()}\spxextra{in module pipeline.hifv.cli}}

\begin{fulllineitems}
\phantomsection\label{\detokenize{_autosummary/pipeline.hifv.cli.hifv_circfeedpolcal:pipeline.hifv.cli.hifv_circfeedpolcal}}
\pysigstartsignatures
\pysiglinewithargsret
{\sphinxbfcode{\sphinxupquote{hifv\_circfeedpolcal}}}
{\sphinxparam{\DUrole{n}{vis}\DUrole{o}{=}\DUrole{default_value}{None}}\sphinxparamcomma \sphinxparam{\DUrole{n}{Dterm\_solint}\DUrole{o}{=}\DUrole{default_value}{None}}\sphinxparamcomma \sphinxparam{\DUrole{n}{refantignore}\DUrole{o}{=}\DUrole{default_value}{None}}\sphinxparamcomma \sphinxparam{\DUrole{n}{leakage\_poltype}\DUrole{o}{=}\DUrole{default_value}{None}}\sphinxparamcomma \sphinxparam{\DUrole{n}{mbdkcross}\DUrole{o}{=}\DUrole{default_value}{None}}\sphinxparamcomma \sphinxparam{\DUrole{n}{clipminmax}\DUrole{o}{=}\DUrole{default_value}{None}}\sphinxparamcomma \sphinxparam{\DUrole{n}{refant}\DUrole{o}{=}\DUrole{default_value}{None}}\sphinxparamcomma \sphinxparam{\DUrole{n}{run\_setjy}\DUrole{o}{=}\DUrole{default_value}{None}}}
{}
\pysigstopsignatures
\sphinxAtStartPar
Perform polarization calibration for VLA circular feeds.

\sphinxAtStartPar
Only validated for VLA sky survey data in S\sphinxhyphen{}band continuum mode with 3C138
or 3C286 as polarization angle. Requires that all polarization intents are
properly set during observation.
\begin{quote}\begin{description}
\sphinxlineitem{Parameters}\begin{itemize}
\item {} 
\sphinxAtStartPar
\sphinxstylestrong{vis} \textendash{} List of input visibility data.

\item {} 
\sphinxAtStartPar
\sphinxstylestrong{Dterm\_solint} \textendash{} D\sphinxhyphen{}terms spectral averaging.  Example:  refantignore=’ea02,ea03’.

\item {} 
\sphinxAtStartPar
\sphinxstylestrong{refantignore} \textendash{} String list of antennas to ignore.

\item {} 
\sphinxAtStartPar
\sphinxstylestrong{leakage\_poltype} \textendash{} poltype to use in first polcal execution \sphinxhyphen{} blank string means use default heuristics.

\item {} 
\sphinxAtStartPar
\sphinxstylestrong{mbdkcross} \textendash{} Run gaincal KCROSS grouped by baseband.

\item {} 
\sphinxAtStartPar
\sphinxstylestrong{clipminmax} \textendash{} Acceptable range for leakage amplitudes, values outside will be flagged.

\item {} 
\sphinxAtStartPar
\sphinxstylestrong{refant} \textendash{} A csv string of reference antenna(s). When used, disables \sphinxcode{\sphinxupquote{refantignore}}. Example: refant = ‘ea01, ea02’

\item {} 
\sphinxAtStartPar
\sphinxstylestrong{run\_setjy} \textendash{} Run setjy for amplitude/flux calibrator, default set to True.

\end{itemize}

\sphinxlineitem{Returns}
\sphinxAtStartPar
The results object for the pipeline task is returned.

\end{description}\end{quote}
\subsubsection*{Examples}
\begin{enumerate}
\sphinxsetlistlabels{\arabic}{enumi}{enumii}{}{.}%
\item {} 
\sphinxAtStartPar
Basic circfeedpolcal task:

\end{enumerate}

\begin{sphinxVerbatim}[commandchars=\\\{\}]
\PYG{g+gp}{\PYGZgt{}\PYGZgt{}\PYGZgt{} }\PYG{n}{hifv\PYGZus{}circfeedpolcal}\PYG{p}{(}\PYG{p}{)}
\end{sphinxVerbatim}

\end{fulllineitems}


\sphinxstepscope


\section{pipeline.hifv.cli.hifv\_exportdata}
\label{\detokenize{_autosummary/pipeline.hifv.cli.hifv_exportdata:pipeline-hifv-cli-hifv-exportdata}}\label{\detokenize{_autosummary/pipeline.hifv.cli.hifv_exportdata::doc}}\index{hifv\_exportdata() (in module pipeline.hifv.cli)@\spxentry{hifv\_exportdata()}\spxextra{in module pipeline.hifv.cli}}

\begin{fulllineitems}
\phantomsection\label{\detokenize{_autosummary/pipeline.hifv.cli.hifv_exportdata:pipeline.hifv.cli.hifv_exportdata}}
\pysigstartsignatures
\pysiglinewithargsret
{\sphinxbfcode{\sphinxupquote{hifv\_exportdata}}}
{\sphinxparam{\DUrole{n}{vis}\DUrole{o}{=}\DUrole{default_value}{None}}\sphinxparamcomma \sphinxparam{\DUrole{n}{session}\DUrole{o}{=}\DUrole{default_value}{None}}\sphinxparamcomma \sphinxparam{\DUrole{n}{imaging\_products\_only}\DUrole{o}{=}\DUrole{default_value}{None}}\sphinxparamcomma \sphinxparam{\DUrole{n}{exportmses}\DUrole{o}{=}\DUrole{default_value}{None}}\sphinxparamcomma \sphinxparam{\DUrole{n}{tarms}\DUrole{o}{=}\DUrole{default_value}{None}}\sphinxparamcomma \sphinxparam{\DUrole{n}{exportcalprods}\DUrole{o}{=}\DUrole{default_value}{None}}\sphinxparamcomma \sphinxparam{\DUrole{n}{pprfile}\DUrole{o}{=}\DUrole{default_value}{None}}\sphinxparamcomma \sphinxparam{\DUrole{n}{calintents}\DUrole{o}{=}\DUrole{default_value}{None}}\sphinxparamcomma \sphinxparam{\DUrole{n}{calimages}\DUrole{o}{=}\DUrole{default_value}{None}}\sphinxparamcomma \sphinxparam{\DUrole{n}{targetimages}\DUrole{o}{=}\DUrole{default_value}{None}}\sphinxparamcomma \sphinxparam{\DUrole{n}{products\_dir}\DUrole{o}{=}\DUrole{default_value}{None}}\sphinxparamcomma \sphinxparam{\DUrole{n}{gainmap}\DUrole{o}{=}\DUrole{default_value}{None}}}
{}
\pysigstopsignatures
\sphinxAtStartPar
Prepare and export interferometry and imaging data.

\sphinxAtStartPar
The hifv\_exportdata task for the VLA CASA pipeline exports the data defined
in the pipeline context and exports it to the data products directory,
converting and or packing it as necessary.

\sphinxAtStartPar
The current version of the task exports the following products
\begin{itemize}
\item {} 
\sphinxAtStartPar
an XML file containing the pipeline processing request

\item {} 
\sphinxAtStartPar
a tar file per ASDM / MS containing the final flags version OR the MS if tarms is False

\item {} 
\sphinxAtStartPar
a text file per ASDM / MS containing the final calibration apply list

\item {} 
\sphinxAtStartPar
a FITS image for each selected calibrator source image

\item {} 
\sphinxAtStartPar
a FITS image for each selected science target source image

\item {} 
\sphinxAtStartPar
a tar file per session containing the caltables for that session

\item {} 
\sphinxAtStartPar
a tar file containing the file web log

\item {} 
\sphinxAtStartPar
a text file containing the final list of CASA commands

\end{itemize}
\begin{quote}\begin{description}
\sphinxlineitem{Parameters}\begin{itemize}
\item {} 
\sphinxAtStartPar
\sphinxstylestrong{vis} \textendash{} List of visibility data files for which flagging and calibration information will be exported. Defaults to the list maintained in the
pipeline context.
example: vis={[}‘X227.ms’, ‘X228.ms’{]}

\item {} 
\sphinxAtStartPar
\sphinxstylestrong{session} \textendash{} List of sessions one per visibility file. Currently defaults to a single virtual session containing all the visibility files in vis.
In the future, this will default to the set of observing sessions defined
in the context.
example: session={[}‘session1’, ‘session2’{]}

\item {} 
\sphinxAtStartPar
\sphinxstylestrong{imaging\_products\_only} \textendash{} Export science target imaging products only

\item {} 
\sphinxAtStartPar
\sphinxstylestrong{exportmses} \textendash{} Export the final MeasurementSets instead of the final flags, calibration tables, and calibration instructions.

\item {} 
\sphinxAtStartPar
\sphinxstylestrong{tarms} \textendash{} Tar final MeasurementSets

\item {} 
\sphinxAtStartPar
\sphinxstylestrong{exportcalprods} \textendash{} Export flags and caltables in addition to MeasurementSets. this parameter is only valid when exportmses = True.

\item {} 
\sphinxAtStartPar
\sphinxstylestrong{pprfile} \textendash{} Name of the pipeline processing request to be exported. Defaults to a file matching the template ‘PPR\_*.xml’.
example: pprfile={[}‘PPR\_GRB021004.xml’{]}

\item {} 
\sphinxAtStartPar
\sphinxstylestrong{calintents} \textendash{} List of calibrator image types to be exported. Defaults to all standard calibrator intents, ‘BANDPASS’, ‘PHASE’, ‘FLUX’.
example: ‘PHASE’

\item {} 
\sphinxAtStartPar
\sphinxstylestrong{calimages} \textendash{} List of calibrator images to be exported. Defaults to all calibrator images recorded in the pipeline context.
example: calimages={[}‘3C454.3.bandpass’, ‘3C279.phase’{]}

\item {} 
\sphinxAtStartPar
\sphinxstylestrong{targetimages} \textendash{} List of science target images to be exported. Defaults to all science target images recorded in the pipeline context.
example: targetimages={[}‘NGC3256.band3’, ‘NGC3256.band6’{]}

\item {} 
\sphinxAtStartPar
\sphinxstylestrong{products\_dir} \textendash{} Name of the data products subdirectory. Defaults to ‘./’ example: ‘../products’

\item {} 
\sphinxAtStartPar
\sphinxstylestrong{gainmap} \textendash{} The value of \sphinxcode{\sphinxupquote{gainmap}} parameter in hifv\_restoredata task put in casa\_piperestorescript.py

\end{itemize}

\sphinxlineitem{Returns}
\sphinxAtStartPar
The results object for the pipeline task is returned.

\end{description}\end{quote}
\subsubsection*{Examples}

\sphinxAtStartPar
1. Export the pipeline results for a single session to the data products
directory:

\begin{sphinxVerbatim}[commandchars=\\\{\}]
\PYGZgt{}\PYGZgt{}\PYGZgt{} !mkdir ../products
\PYGZgt{}\PYGZgt{}\PYGZgt{} hifv\PYGZus{}exportdata (products\PYGZus{}dir=\PYGZsq{}../products\PYGZsq{})
\end{sphinxVerbatim}

\sphinxAtStartPar
2. Export the pipeline results to the data products directory specify that
only the gain calibrator images be saved:

\begin{sphinxVerbatim}[commandchars=\\\{\}]
\PYGZgt{}\PYGZgt{}\PYGZgt{} !mkdir ../products
\PYGZgt{}\PYGZgt{}\PYGZgt{} hifv\PYGZus{}exportdata (products\PYGZus{}dir=\PYGZsq{}../products\PYGZsq{}, calintents=\PYGZsq{}*PHASE*\PYGZsq{})
\end{sphinxVerbatim}

\end{fulllineitems}


\sphinxstepscope


\section{pipeline.hifv.cli.hifv\_exportvlassdata}
\label{\detokenize{_autosummary/pipeline.hifv.cli.hifv_exportvlassdata:pipeline-hifv-cli-hifv-exportvlassdata}}\label{\detokenize{_autosummary/pipeline.hifv.cli.hifv_exportvlassdata::doc}}\index{hifv\_exportvlassdata() (in module pipeline.hifv.cli)@\spxentry{hifv\_exportvlassdata()}\spxextra{in module pipeline.hifv.cli}}

\begin{fulllineitems}
\phantomsection\label{\detokenize{_autosummary/pipeline.hifv.cli.hifv_exportvlassdata:pipeline.hifv.cli.hifv_exportvlassdata}}
\pysigstartsignatures
\pysiglinewithargsret
{\sphinxbfcode{\sphinxupquote{hifv\_exportvlassdata}}}
{\sphinxparam{\DUrole{n}{vis}\DUrole{o}{=}\DUrole{default_value}{None}}}
{}
\pysigstopsignatures
\sphinxAtStartPar
Export Image data from QL, SE, and Coarse Cube modes of VLASS Survey.
\begin{quote}\begin{description}
\sphinxlineitem{Parameters}
\sphinxAtStartPar
\sphinxstylestrong{vis} \textendash{} The list of input MeasurementSets. Defaults to the list of MeasurementSets specified in the h\_init or hifv\_importdata task.

\sphinxlineitem{Returns}
\sphinxAtStartPar
The results object for the pipeline task is returned.

\end{description}\end{quote}
\subsubsection*{Examples}
\begin{enumerate}
\sphinxsetlistlabels{\arabic}{enumi}{enumii}{}{.}%
\item {} 
\sphinxAtStartPar
Basic exportvlassdata task:

\end{enumerate}

\begin{sphinxVerbatim}[commandchars=\\\{\}]
\PYG{g+gp}{\PYGZgt{}\PYGZgt{}\PYGZgt{} }\PYG{n}{hifv\PYGZus{}exportvlassdata}\PYG{p}{(}\PYG{p}{)}
\end{sphinxVerbatim}

\end{fulllineitems}


\sphinxstepscope


\section{pipeline.hifv.cli.hifv\_finalcals}
\label{\detokenize{_autosummary/pipeline.hifv.cli.hifv_finalcals:pipeline-hifv-cli-hifv-finalcals}}\label{\detokenize{_autosummary/pipeline.hifv.cli.hifv_finalcals::doc}}\index{hifv\_finalcals() (in module pipeline.hifv.cli)@\spxentry{hifv\_finalcals()}\spxextra{in module pipeline.hifv.cli}}

\begin{fulllineitems}
\phantomsection\label{\detokenize{_autosummary/pipeline.hifv.cli.hifv_finalcals:pipeline.hifv.cli.hifv_finalcals}}
\pysigstartsignatures
\pysiglinewithargsret
{\sphinxbfcode{\sphinxupquote{hifv\_finalcals}}}
{\sphinxparam{\DUrole{n}{vis}\DUrole{o}{=}\DUrole{default_value}{None}}\sphinxparamcomma \sphinxparam{\DUrole{n}{weakbp}\DUrole{o}{=}\DUrole{default_value}{None}}\sphinxparamcomma \sphinxparam{\DUrole{n}{refantignore}\DUrole{o}{=}\DUrole{default_value}{None}}\sphinxparamcomma \sphinxparam{\DUrole{n}{refant}\DUrole{o}{=}\DUrole{default_value}{None}}}
{}
\pysigstopsignatures
\sphinxAtStartPar
Compute final gain calibration tables.
\begin{quote}\begin{description}
\sphinxlineitem{Parameters}\begin{itemize}
\item {} 
\sphinxAtStartPar
\sphinxstylestrong{vis} \textendash{} The list of input MeasurementSets. Defaults to the list of MeasurementSets specified in the h\_init or hifv\_importdata task.

\item {} 
\sphinxAtStartPar
\sphinxstylestrong{weakbp} \textendash{} Activate weak bandpass heuristics

\item {} 
\sphinxAtStartPar
\sphinxstylestrong{refantignore} \textendash{} String list of antennas to ignore.

\item {} 
\sphinxAtStartPar
\sphinxstylestrong{refant} \textendash{} A csv string of reference antenna(s). When used, disables \sphinxcode{\sphinxupquote{refantignore}}. Example: refant = ‘ea01, ea02’

\end{itemize}

\sphinxlineitem{Returns}
\sphinxAtStartPar
The results object for the pipeline task is returned.

\end{description}\end{quote}
\subsubsection*{Examples}
\begin{enumerate}
\sphinxsetlistlabels{\arabic}{enumi}{enumii}{}{.}%
\item {} 
\sphinxAtStartPar
Create the final calibration tables to be applied to the data in the VLA CASA pipeline:

\end{enumerate}

\begin{sphinxVerbatim}[commandchars=\\\{\}]
\PYG{g+gp}{\PYGZgt{}\PYGZgt{}\PYGZgt{} }\PYG{n}{hifv\PYGZus{}finalcals}\PYG{p}{(}\PYG{p}{)}
\end{sphinxVerbatim}

\end{fulllineitems}


\sphinxstepscope


\section{pipeline.hifv.cli.hifv\_fixpointing}
\label{\detokenize{_autosummary/pipeline.hifv.cli.hifv_fixpointing:pipeline-hifv-cli-hifv-fixpointing}}\label{\detokenize{_autosummary/pipeline.hifv.cli.hifv_fixpointing::doc}}\index{hifv\_fixpointing() (in module pipeline.hifv.cli)@\spxentry{hifv\_fixpointing()}\spxextra{in module pipeline.hifv.cli}}

\begin{fulllineitems}
\phantomsection\label{\detokenize{_autosummary/pipeline.hifv.cli.hifv_fixpointing:pipeline.hifv.cli.hifv_fixpointing}}
\pysigstartsignatures
\pysiglinewithargsret
{\sphinxbfcode{\sphinxupquote{hifv\_fixpointing}}}
{\sphinxparam{\DUrole{n}{vis}\DUrole{o}{=}\DUrole{default_value}{None}}}
{}
\pysigstopsignatures
\sphinxAtStartPar
Base fixpointing task
\begin{quote}\begin{description}
\sphinxlineitem{Parameters}
\sphinxAtStartPar
\sphinxstylestrong{vis} \textendash{} The list of input MeasurementSets. Defaults to the list of MeasurementSets specified in the h\_init or hifv\_importdata task.

\sphinxlineitem{Returns}
\sphinxAtStartPar
The results object for the pipeline task is returned.

\end{description}\end{quote}
\subsubsection*{Examples}
\begin{enumerate}
\sphinxsetlistlabels{\arabic}{enumi}{enumii}{}{.}%
\item {} 
\sphinxAtStartPar
Basic fixpointing task:

\end{enumerate}

\begin{sphinxVerbatim}[commandchars=\\\{\}]
\PYG{g+gp}{\PYGZgt{}\PYGZgt{}\PYGZgt{} }\PYG{n}{hifv\PYGZus{}fixpointing}\PYG{p}{(}\PYG{p}{)}
\end{sphinxVerbatim}

\end{fulllineitems}


\sphinxstepscope


\section{pipeline.hifv.cli.hifv\_flagcal}
\label{\detokenize{_autosummary/pipeline.hifv.cli.hifv_flagcal:pipeline-hifv-cli-hifv-flagcal}}\label{\detokenize{_autosummary/pipeline.hifv.cli.hifv_flagcal::doc}}\index{hifv\_flagcal() (in module pipeline.hifv.cli)@\spxentry{hifv\_flagcal()}\spxextra{in module pipeline.hifv.cli}}

\begin{fulllineitems}
\phantomsection\label{\detokenize{_autosummary/pipeline.hifv.cli.hifv_flagcal:pipeline.hifv.cli.hifv_flagcal}}
\pysigstartsignatures
\pysiglinewithargsret
{\sphinxbfcode{\sphinxupquote{hifv\_flagcal}}}
{\sphinxparam{\DUrole{n}{vis}\DUrole{o}{=}\DUrole{default_value}{None}}\sphinxparamcomma \sphinxparam{\DUrole{n}{caltable}\DUrole{o}{=}\DUrole{default_value}{None}}\sphinxparamcomma \sphinxparam{\DUrole{n}{clipminmax}\DUrole{o}{=}\DUrole{default_value}{None}}}
{}
\pysigstopsignatures
\sphinxAtStartPar
Flagcal task.
\begin{quote}\begin{description}
\sphinxlineitem{Parameters}\begin{itemize}
\item {} 
\sphinxAtStartPar
\sphinxstylestrong{vis} \textendash{} List of input visibility data.

\item {} 
\sphinxAtStartPar
\sphinxstylestrong{caltable} \textendash{} String name of the caltable.

\item {} 
\sphinxAtStartPar
\sphinxstylestrong{clipminmax} \textendash{} Range to use for clipping.

\end{itemize}

\sphinxlineitem{Returns}
\sphinxAtStartPar
The results object for the pipeline task is returned.

\end{description}\end{quote}
\subsubsection*{Examples}
\begin{enumerate}
\sphinxsetlistlabels{\arabic}{enumi}{enumii}{}{.}%
\item {} 
\sphinxAtStartPar
Flag existing caltable:

\end{enumerate}

\begin{sphinxVerbatim}[commandchars=\\\{\}]
\PYG{g+gp}{\PYGZgt{}\PYGZgt{}\PYGZgt{} }\PYG{n}{hifv\PYGZus{}flagcal}\PYG{p}{(}\PYG{p}{)}
\end{sphinxVerbatim}

\end{fulllineitems}


\sphinxstepscope


\section{pipeline.hifv.cli.hifv\_flagdata}
\label{\detokenize{_autosummary/pipeline.hifv.cli.hifv_flagdata:pipeline-hifv-cli-hifv-flagdata}}\label{\detokenize{_autosummary/pipeline.hifv.cli.hifv_flagdata::doc}}\index{hifv\_flagdata() (in module pipeline.hifv.cli)@\spxentry{hifv\_flagdata()}\spxextra{in module pipeline.hifv.cli}}

\begin{fulllineitems}
\phantomsection\label{\detokenize{_autosummary/pipeline.hifv.cli.hifv_flagdata:pipeline.hifv.cli.hifv_flagdata}}
\pysigstartsignatures
\pysiglinewithargsret
{\sphinxbfcode{\sphinxupquote{hifv\_flagdata}}}
{\sphinxparam{\DUrole{n}{vis}\DUrole{o}{=}\DUrole{default_value}{None}}\sphinxparamcomma \sphinxparam{\DUrole{n}{autocorr}\DUrole{o}{=}\DUrole{default_value}{None}}\sphinxparamcomma \sphinxparam{\DUrole{n}{shadow}\DUrole{o}{=}\DUrole{default_value}{None}}\sphinxparamcomma \sphinxparam{\DUrole{n}{scan}\DUrole{o}{=}\DUrole{default_value}{None}}\sphinxparamcomma \sphinxparam{\DUrole{n}{scannumber}\DUrole{o}{=}\DUrole{default_value}{None}}\sphinxparamcomma \sphinxparam{\DUrole{n}{quack}\DUrole{o}{=}\DUrole{default_value}{None}}\sphinxparamcomma \sphinxparam{\DUrole{n}{clip}\DUrole{o}{=}\DUrole{default_value}{None}}\sphinxparamcomma \sphinxparam{\DUrole{n}{baseband}\DUrole{o}{=}\DUrole{default_value}{None}}\sphinxparamcomma \sphinxparam{\DUrole{n}{intents}\DUrole{o}{=}\DUrole{default_value}{None}}\sphinxparamcomma \sphinxparam{\DUrole{n}{edgespw}\DUrole{o}{=}\DUrole{default_value}{None}}\sphinxparamcomma \sphinxparam{\DUrole{n}{fracspw}\DUrole{o}{=}\DUrole{default_value}{None}}\sphinxparamcomma \sphinxparam{\DUrole{n}{online}\DUrole{o}{=}\DUrole{default_value}{None}}\sphinxparamcomma \sphinxparam{\DUrole{n}{fileonline}\DUrole{o}{=}\DUrole{default_value}{None}}\sphinxparamcomma \sphinxparam{\DUrole{n}{template}\DUrole{o}{=}\DUrole{default_value}{None}}\sphinxparamcomma \sphinxparam{\DUrole{n}{filetemplate}\DUrole{o}{=}\DUrole{default_value}{None}}\sphinxparamcomma \sphinxparam{\DUrole{n}{hm\_tbuff}\DUrole{o}{=}\DUrole{default_value}{None}}\sphinxparamcomma \sphinxparam{\DUrole{n}{tbuff}\DUrole{o}{=}\DUrole{default_value}{None}}\sphinxparamcomma \sphinxparam{\DUrole{n}{flagbackup}\DUrole{o}{=}\DUrole{default_value}{None}}}
{}
\pysigstopsignatures
\sphinxAtStartPar
Do basic deterministic flagging of a list of MeasurementSets.

\sphinxAtStartPar
The hifv\_flagdata task performs basic flagging operations on a list of MeasurementSets including:
\begin{itemize}
\item {} 
\sphinxAtStartPar
autocorrelation data flagging

\item {} 
\sphinxAtStartPar
shadowed antenna data flagging

\item {} 
\sphinxAtStartPar
scan based flagging

\item {} 
\sphinxAtStartPar
edge channel flagging

\item {} 
\sphinxAtStartPar
baseband edge flagging

\item {} 
\sphinxAtStartPar
applying online flags

\item {} 
\sphinxAtStartPar
applying a flagging template

\item {} 
\sphinxAtStartPar
quack, shadow, and basebands

\item {} 
\sphinxAtStartPar
Antenna not\sphinxhyphen{}on\sphinxhyphen{}source (ANOS)

\end{itemize}
\begin{quote}\begin{description}
\sphinxlineitem{Parameters}\begin{itemize}
\item {} 
\sphinxAtStartPar
\sphinxstylestrong{vis} \textendash{} The list of input MeasurementSets. Defaults to the list of MeasurementSets specified in the h\_init or hifv\_importdata task.

\item {} 
\sphinxAtStartPar
\sphinxstylestrong{autocorr} \textendash{} Flag autocorrelation data

\item {} 
\sphinxAtStartPar
\sphinxstylestrong{shadow} \textendash{} Flag shadowed antennas

\item {} 
\sphinxAtStartPar
\sphinxstylestrong{scan} \textendash{} Flag specified scans

\item {} 
\sphinxAtStartPar
\sphinxstylestrong{scannumber} \textendash{} A string containing a  comma delimited list of scans to be flagged. example: ‘3,5,6’

\item {} 
\sphinxAtStartPar
\sphinxstylestrong{quack} \textendash{} Quack scans

\item {} 
\sphinxAtStartPar
\sphinxstylestrong{clip} \textendash{} Clip mode

\item {} 
\sphinxAtStartPar
\sphinxstylestrong{baseband} \textendash{} Flag 20MHz of each edge of basebands

\item {} 
\sphinxAtStartPar
\sphinxstylestrong{intents} \textendash{} A string containing a comma delimited list of intents against which the scans to be flagged are matched.
example: \sphinxtitleref{‘*BANDPASS*’}

\item {} 
\sphinxAtStartPar
\sphinxstylestrong{edgespw} \textendash{} Fraction of the baseline correlator TDM edge channels to be flagged.

\item {} 
\sphinxAtStartPar
\sphinxstylestrong{fracspw} \textendash{} Fraction of baseline correlator edge channels to be flagged.

\item {} 
\sphinxAtStartPar
\sphinxstylestrong{online} \textendash{} Apply the online flags.

\item {} 
\sphinxAtStartPar
\sphinxstylestrong{fileonline} \textendash{} File containing the online flags. These are computed by the h\_init or hif\_importdata data tasks. If the online flags files
are undefined a name of the form ‘msname.flagonline.txt’ is assumed.

\item {} 
\sphinxAtStartPar
\sphinxstylestrong{template} \textendash{} Apply a flagging template.

\item {} 
\sphinxAtStartPar
\sphinxstylestrong{filetemplate} \textendash{} The name of a text file that contains the flagging template for RFI, birdies, telluric lines, etc.  If the template flags files
is undefined a name of the form ‘msname.flagtemplate.txt’ is assumed.

\item {} 
\sphinxAtStartPar
\sphinxstylestrong{hm\_tbuff} \textendash{} The time buffer computation heuristic

\item {} 
\sphinxAtStartPar
\sphinxstylestrong{tbuff} \textendash{} List of time buffers (sec) to pad timerange in flag commands.

\item {} 
\sphinxAtStartPar
\sphinxstylestrong{flagbackup} \textendash{} Backup pre\sphinxhyphen{}existing flags before applying new ones.

\end{itemize}

\sphinxlineitem{Returns}
\sphinxAtStartPar
The results object for the pipeline task is returned.

\end{description}\end{quote}
\subsubsection*{Examples}
\begin{enumerate}
\sphinxsetlistlabels{\arabic}{enumi}{enumii}{}{.}%
\item {} 
\sphinxAtStartPar
Do basic flagging on a MeasurementSet:

\end{enumerate}

\begin{sphinxVerbatim}[commandchars=\\\{\}]
\PYG{g+gp}{\PYGZgt{}\PYGZgt{}\PYGZgt{} }\PYG{n}{hifv\PYGZus{}flagdata}\PYG{p}{(}\PYG{p}{)}
\end{sphinxVerbatim}

\sphinxAtStartPar
2. Do basic flagging on a MeasurementSet as well as flag pointing and
atmosphere data:

\begin{sphinxVerbatim}[commandchars=\\\{\}]
\PYG{g+gp}{\PYGZgt{}\PYGZgt{}\PYGZgt{} }\PYG{n}{hifv\PYGZus{}flagdata}\PYG{p}{(}\PYG{n}{scan}\PYG{o}{=}\PYG{k+kc}{True} \PYG{n}{intent}\PYG{o}{=}\PYG{l+s+s1}{\PYGZsq{}}\PYG{l+s+s1}{*BANDPASS*}\PYG{l+s+s1}{\PYGZsq{}}\PYG{p}{)}
\end{sphinxVerbatim}

\end{fulllineitems}


\sphinxstepscope


\section{pipeline.hifv.cli.hifv\_flagtargetsdata}
\label{\detokenize{_autosummary/pipeline.hifv.cli.hifv_flagtargetsdata:pipeline-hifv-cli-hifv-flagtargetsdata}}\label{\detokenize{_autosummary/pipeline.hifv.cli.hifv_flagtargetsdata::doc}}\index{hifv\_flagtargetsdata() (in module pipeline.hifv.cli)@\spxentry{hifv\_flagtargetsdata()}\spxextra{in module pipeline.hifv.cli}}

\begin{fulllineitems}
\phantomsection\label{\detokenize{_autosummary/pipeline.hifv.cli.hifv_flagtargetsdata:pipeline.hifv.cli.hifv_flagtargetsdata}}
\pysigstartsignatures
\pysiglinewithargsret
{\sphinxbfcode{\sphinxupquote{hifv\_flagtargetsdata}}}
{\sphinxparam{\DUrole{n}{vis}\DUrole{o}{=}\DUrole{default_value}{None}}\sphinxparamcomma \sphinxparam{\DUrole{n}{template}\DUrole{o}{=}\DUrole{default_value}{None}}\sphinxparamcomma \sphinxparam{\DUrole{n}{filetemplate}\DUrole{o}{=}\DUrole{default_value}{None}}\sphinxparamcomma \sphinxparam{\DUrole{n}{flagbackup}\DUrole{o}{=}\DUrole{default_value}{None}}}
{}
\pysigstopsignatures
\sphinxAtStartPar
Apply a flagtemplate to target data prior to running imaging pipeline tasks.
\begin{quote}\begin{description}
\sphinxlineitem{Parameters}\begin{itemize}
\item {} 
\sphinxAtStartPar
\sphinxstylestrong{vis} \textendash{} The list of input MeasurementSets. Defaults to the list of MeasurementSets defined in the pipeline context.

\item {} 
\sphinxAtStartPar
\sphinxstylestrong{template} \textendash{} Apply flagging templates.

\item {} 
\sphinxAtStartPar
\sphinxstylestrong{filetemplate} \textendash{} The name of a text file that contains the flagging template for issues with the science target data etc.
If the template flags files is undefined a name of the
form ‘msname\_flagtargetstemplate.txt’ is assumed.

\item {} 
\sphinxAtStartPar
\sphinxstylestrong{flagbackup} \textendash{} Back up any pre\sphinxhyphen{}existing flags.

\end{itemize}

\sphinxlineitem{Returns}
\sphinxAtStartPar
The results object for the pipeline task is returned.

\end{description}\end{quote}
\subsubsection*{Examples}
\begin{enumerate}
\sphinxsetlistlabels{\arabic}{enumi}{enumii}{}{.}%
\item {} 
\sphinxAtStartPar
Basic flagtargetsdata task:

\end{enumerate}

\begin{sphinxVerbatim}[commandchars=\\\{\}]
\PYG{g+gp}{\PYGZgt{}\PYGZgt{}\PYGZgt{} }\PYG{n}{hifv\PYGZus{}flagtargetsdata}\PYG{p}{(}\PYG{p}{)}
\end{sphinxVerbatim}

\end{fulllineitems}


\sphinxstepscope


\section{pipeline.hifv.cli.hifv\_fluxboot}
\label{\detokenize{_autosummary/pipeline.hifv.cli.hifv_fluxboot:pipeline-hifv-cli-hifv-fluxboot}}\label{\detokenize{_autosummary/pipeline.hifv.cli.hifv_fluxboot::doc}}\index{hifv\_fluxboot() (in module pipeline.hifv.cli)@\spxentry{hifv\_fluxboot()}\spxextra{in module pipeline.hifv.cli}}

\begin{fulllineitems}
\phantomsection\label{\detokenize{_autosummary/pipeline.hifv.cli.hifv_fluxboot:pipeline.hifv.cli.hifv_fluxboot}}
\pysigstartsignatures
\pysiglinewithargsret
{\sphinxbfcode{\sphinxupquote{hifv\_fluxboot}}}
{\sphinxparam{\DUrole{n}{vis}\DUrole{o}{=}\DUrole{default_value}{None}}\sphinxparamcomma \sphinxparam{\DUrole{n}{caltable}\DUrole{o}{=}\DUrole{default_value}{None}}\sphinxparamcomma \sphinxparam{\DUrole{n}{fitorder}\DUrole{o}{=}\DUrole{default_value}{None}}\sphinxparamcomma \sphinxparam{\DUrole{n}{refantignore}\DUrole{o}{=}\DUrole{default_value}{None}}\sphinxparamcomma \sphinxparam{\DUrole{n}{refant}\DUrole{o}{=}\DUrole{default_value}{None}}}
{}
\pysigstopsignatures
\sphinxAtStartPar
Fluxboot

\sphinxAtStartPar
Determine flux density bootstrapping for gain calibrators relative to flux calibrator.
\begin{quote}\begin{description}
\sphinxlineitem{Parameters}\begin{itemize}
\item {} 
\sphinxAtStartPar
\sphinxstylestrong{vis} \textendash{} The list of input MeasurementSets. Defaults to the list of MeasurementSets specified in the h\_init or hifv\_importdata task.

\item {} 
\sphinxAtStartPar
\sphinxstylestrong{caltable} \textendash{} String name of the flagged caltable.

\item {} 
\sphinxAtStartPar
\sphinxstylestrong{fitorder} \textendash{} Polynomial order of the spectral fitting for valid flux densities with multiple spws.  The default value of \sphinxhyphen{}1 means that the heuristics determine the fit order based on
fractional bandwidth and receiver bands present in the observation.
An override value of 1,2,3 or 4 may be specified by the user.
Spectral index (1) and, if applicable, curvature (2) are reported in the weblog.
If no determination can be made by the heuristics, a fitorder of 1 will be used.

\item {} 
\sphinxAtStartPar
\sphinxstylestrong{refantignore} \textendash{} String list of antennas to ignore Example:  refantignore=’ea02, ea03’

\item {} 
\sphinxAtStartPar
\sphinxstylestrong{refant} \textendash{} A csv string of reference antenna(s). When used, disables \sphinxcode{\sphinxupquote{refantignore}}.
Example: refant = ‘ea01, ea02’

\end{itemize}

\sphinxlineitem{Returns}
\sphinxAtStartPar
The results object for the pipeline task is returned.

\end{description}\end{quote}
\subsubsection*{Examples}
\begin{enumerate}
\sphinxsetlistlabels{\arabic}{enumi}{enumii}{}{.}%
\item {} 
\sphinxAtStartPar
VLA CASA pipeline flux density bootstrapping:

\end{enumerate}

\begin{sphinxVerbatim}[commandchars=\\\{\}]
\PYG{g+gp}{\PYGZgt{}\PYGZgt{}\PYGZgt{} }\PYG{n}{hifv\PYGZus{}fluxboot}\PYG{p}{(}\PYG{p}{)}
\end{sphinxVerbatim}

\end{fulllineitems}


\sphinxstepscope


\section{pipeline.hifv.cli.hifv\_gaincurves}
\label{\detokenize{_autosummary/pipeline.hifv.cli.hifv_gaincurves:pipeline-hifv-cli-hifv-gaincurves}}\label{\detokenize{_autosummary/pipeline.hifv.cli.hifv_gaincurves::doc}}\index{hifv\_gaincurves() (in module pipeline.hifv.cli)@\spxentry{hifv\_gaincurves()}\spxextra{in module pipeline.hifv.cli}}

\begin{fulllineitems}
\phantomsection\label{\detokenize{_autosummary/pipeline.hifv.cli.hifv_gaincurves:pipeline.hifv.cli.hifv_gaincurves}}
\pysigstartsignatures
\pysiglinewithargsret
{\sphinxbfcode{\sphinxupquote{hifv\_gaincurves}}}
{\sphinxparam{\DUrole{n}{vis}\DUrole{o}{=}\DUrole{default_value}{None}}\sphinxparamcomma \sphinxparam{\DUrole{n}{caltable}\DUrole{o}{=}\DUrole{default_value}{None}}}
{}
\pysigstopsignatures
\sphinxAtStartPar
Runs gencal in gc mode.
\begin{quote}\begin{description}
\sphinxlineitem{Parameters}\begin{itemize}
\item {} 
\sphinxAtStartPar
\sphinxstylestrong{vis} \textendash{} List of input visibility data.

\item {} 
\sphinxAtStartPar
\sphinxstylestrong{caltable} \textendash{} String name of caltable.

\end{itemize}

\sphinxlineitem{Returns}
\sphinxAtStartPar
The results object for the pipeline task is returned.

\end{description}\end{quote}
\subsubsection*{Examples}
\begin{enumerate}
\sphinxsetlistlabels{\arabic}{enumi}{enumii}{}{.}%
\item {} 
\sphinxAtStartPar
Load an ASDM list in the ../rawdata subdirectory into the context:

\end{enumerate}

\begin{sphinxVerbatim}[commandchars=\\\{\}]
\PYG{g+gp}{\PYGZgt{}\PYGZgt{}\PYGZgt{} }\PYG{n}{hifv\PYGZus{}gaincurves}\PYG{p}{(}\PYG{p}{)}
\end{sphinxVerbatim}

\end{fulllineitems}


\sphinxstepscope


\section{pipeline.hifv.cli.hifv\_hanning}
\label{\detokenize{_autosummary/pipeline.hifv.cli.hifv_hanning:pipeline-hifv-cli-hifv-hanning}}\label{\detokenize{_autosummary/pipeline.hifv.cli.hifv_hanning::doc}}\index{hifv\_hanning() (in module pipeline.hifv.cli)@\spxentry{hifv\_hanning()}\spxextra{in module pipeline.hifv.cli}}

\begin{fulllineitems}
\phantomsection\label{\detokenize{_autosummary/pipeline.hifv.cli.hifv_hanning:pipeline.hifv.cli.hifv_hanning}}
\pysigstartsignatures
\pysiglinewithargsret
{\sphinxbfcode{\sphinxupquote{hifv\_hanning}}}
{\sphinxparam{\DUrole{n}{vis}\DUrole{o}{=}\DUrole{default_value}{None}}\sphinxparamcomma \sphinxparam{\DUrole{n}{maser\_detection}\DUrole{o}{=}\DUrole{default_value}{None}}}
{}
\pysigstopsignatures
\sphinxAtStartPar
Hanning smoothing on a dataset.

\sphinxAtStartPar
The hifv\_hanning task will hanning smooth a VLA dataset.
\begin{quote}\begin{description}
\sphinxlineitem{Parameters}\begin{itemize}
\item {} 
\sphinxAtStartPar
\sphinxstylestrong{vis} \textendash{} The list of input MeasurementSets. Defaults to the list of MeasurementSets specified in the h\_init or hifv\_importdata task.

\item {} 
\sphinxAtStartPar
\sphinxstylestrong{maser\_detection} \textendash{} Run maser detect algorithm on spectral line windows. Defaults to True.

\end{itemize}

\sphinxlineitem{Returns}
\sphinxAtStartPar
The results object for the pipeline task is returned.

\end{description}\end{quote}
\subsubsection*{Examples}
\begin{enumerate}
\sphinxsetlistlabels{\arabic}{enumi}{enumii}{}{.}%
\item {} 
\sphinxAtStartPar
Run the task to execute hanning smoothing on a VLA CASA pipeline loaded MeasurementSet:

\end{enumerate}

\begin{sphinxVerbatim}[commandchars=\\\{\}]
\PYG{g+gp}{\PYGZgt{}\PYGZgt{}\PYGZgt{} }\PYG{n}{hifv\PYGZus{}hanning}\PYG{p}{(}\PYG{p}{)}
\end{sphinxVerbatim}

\end{fulllineitems}


\sphinxstepscope


\section{pipeline.hifv.cli.hifv\_importdata}
\label{\detokenize{_autosummary/pipeline.hifv.cli.hifv_importdata:pipeline-hifv-cli-hifv-importdata}}\label{\detokenize{_autosummary/pipeline.hifv.cli.hifv_importdata::doc}}\index{hifv\_importdata() (in module pipeline.hifv.cli)@\spxentry{hifv\_importdata()}\spxextra{in module pipeline.hifv.cli}}

\begin{fulllineitems}
\phantomsection\label{\detokenize{_autosummary/pipeline.hifv.cli.hifv_importdata:pipeline.hifv.cli.hifv_importdata}}
\pysigstartsignatures
\pysiglinewithargsret
{\sphinxbfcode{\sphinxupquote{hifv\_importdata}}}
{\sphinxparam{\DUrole{n}{vis}\DUrole{o}{=}\DUrole{default_value}{None}}\sphinxparamcomma \sphinxparam{\DUrole{n}{session}\DUrole{o}{=}\DUrole{default_value}{None}}\sphinxparamcomma \sphinxparam{\DUrole{n}{asis}\DUrole{o}{=}\DUrole{default_value}{None}}\sphinxparamcomma \sphinxparam{\DUrole{n}{overwrite}\DUrole{o}{=}\DUrole{default_value}{None}}\sphinxparamcomma \sphinxparam{\DUrole{n}{nocopy}\DUrole{o}{=}\DUrole{default_value}{None}}\sphinxparamcomma \sphinxparam{\DUrole{n}{createmms}\DUrole{o}{=}\DUrole{default_value}{None}}\sphinxparamcomma \sphinxparam{\DUrole{n}{ocorr\_mode}\DUrole{o}{=}\DUrole{default_value}{None}}\sphinxparamcomma \sphinxparam{\DUrole{n}{datacolumns}\DUrole{o}{=}\DUrole{default_value}{None}}\sphinxparamcomma \sphinxparam{\DUrole{n}{specline\_spws}\DUrole{o}{=}\DUrole{default_value}{None}}\sphinxparamcomma \sphinxparam{\DUrole{n}{parallel}\DUrole{o}{=}\DUrole{default_value}{None}}}
{}
\pysigstopsignatures
\sphinxAtStartPar
Imports data into the VLA pipeline.

\sphinxAtStartPar
The hifv\_importdata task loads the specified visibility data into the pipeline
context unpacking and / or converting it as necessary.
\begin{quote}\begin{description}
\sphinxlineitem{Parameters}\begin{itemize}
\item {} 
\sphinxAtStartPar
\sphinxstylestrong{vis} \textendash{} List of visibility data files. These may be ASDMs, tar files of ASDMs, MSes, or tar files of MSes, If ASDM files are specified, they will be
converted  to MS format.
example: vis={[}‘X227.ms’, ‘asdms.tar.gz’{]}

\item {} 
\sphinxAtStartPar
\sphinxstylestrong{session} \textendash{} List of sessions to which the visibility files belong. Defaults to a single session containing all the visibility files, otherwise
a session must be assigned to each vis file.
example: session={[}‘Session\_1’, ‘Sessions\_2’{]}

\item {} 
\sphinxAtStartPar
\sphinxstylestrong{asis} \textendash{} Creates verbatim copies of the ASDM tables in the output MS. The value given to this option must be a list of table names
separated by space characters.
examples: ‘Receiver CalAtmosphere’
‘Receiver’, ‘’

\item {} 
\sphinxAtStartPar
\sphinxstylestrong{overwrite} \textendash{} Overwrite existing files on import.

\item {} 
\sphinxAtStartPar
\sphinxstylestrong{nocopy} \textendash{} When importing an MS, disable copying of the MS to the working directory.

\item {} 
\sphinxAtStartPar
\sphinxstylestrong{createmms} \textendash{} Create a multi\sphinxhyphen{}MeasurementSet (‘true’) ready for parallel processing, or a standard MeasurementSet (‘false’). The default setting
(‘automatic’) creates an MMS if running in a cluster environment.

\item {} 
\sphinxAtStartPar
\sphinxstylestrong{ocorr\_mode} \textendash{} Read in cross\sphinxhyphen{} and auto\sphinxhyphen{}correlation data(ca), cross\sphinxhyphen{} correlation data only (co), or autocorrelation data only (ao).

\item {} 
\sphinxAtStartPar
\sphinxstylestrong{datacolumns} \textendash{} Dictionary defining the data types of existing columns. The format is:

\sphinxAtStartPar
\{‘data’: ‘data type 1’\}

\sphinxAtStartPar
or

\sphinxAtStartPar
\{‘data’: ‘data type 1’, ‘corrected’: ‘data type 2’\}

\sphinxAtStartPar
For ASDMs the data type can only be RAW and one
can only specify it for the data column.
For MSes one can define two different data types
for the DATA and CORRECTED\_DATA columns and they
can be any of the known data types (RAW,
REGCAL\_CONTLINE\_ALL, REGCAL\_CONTLINE\_SCIENCE,
SELFCAL\_CONTLINE\_SCIENCE, REGCAL\_LINE\_SCIENCE,
SELFCAL\_LINE\_SCIENCE, BASELINED, ATMCORR). The
intent selection strings \_ALL or \_SCIENCE can be
skipped. In that case the task determines this
automatically by inspecting the existing intents
in the dataset.
Usually, a single datacolumns dictionary is used
for all datasets. If necessary, one can define a
list of dictionaries, one for each EB, with
different setups per EB.
If no types are specified,
\{‘data’:’raw’,’corrected’:’regcal\_contline’\}
or \{‘data’:’raw’\} will be assumed, depending on
whether the corrected column exists or not.

\item {} 
\sphinxAtStartPar
\sphinxstylestrong{specline\_spws} \textendash{} String indicating how the pipeline should determine whether a spw should be processed as a spectral line window or continuum. The default setting of
‘auto’ will use defined heuristics to determine this definition. Accepted
values are ‘auto’, ‘none’ (no spws will be defined as spectral line), or
a string of spw definitions in the CASA format.
example: specline\_spws=’2, 3, 4\textasciitilde{}9, 23’

\item {} 
\sphinxAtStartPar
\sphinxstylestrong{parallel} \textendash{} Execute using CASA HPC functionality, if available.

\end{itemize}

\sphinxlineitem{Returns}
\sphinxAtStartPar
The results object for the pipeline task is returned.

\end{description}\end{quote}
\subsubsection*{Examples}
\begin{enumerate}
\sphinxsetlistlabels{\arabic}{enumi}{enumii}{}{.}%
\item {} 
\sphinxAtStartPar
Load an ASDM list in the ../rawdata subdirectory into the context:

\end{enumerate}

\begin{sphinxVerbatim}[commandchars=\\\{\}]
\PYG{g+gp}{\PYGZgt{}\PYGZgt{}\PYGZgt{} }\PYG{n}{hifv\PYGZus{}importdata} \PYG{p}{(}\PYG{n}{vis}\PYG{o}{=}\PYG{p}{[}\PYG{l+s+s1}{\PYGZsq{}}\PYG{l+s+s1}{../rawdata/uid\PYGZus{}\PYGZus{}\PYGZus{}A002\PYGZus{}X30a93d\PYGZus{}X43e}\PYG{l+s+s1}{\PYGZsq{}}\PYG{p}{,} \PYG{l+s+s1}{\PYGZsq{}}\PYG{l+s+s1}{../rawdata/uid\PYGZus{}A002\PYGZus{}x30a93d\PYGZus{}X44e}\PYG{l+s+s1}{\PYGZsq{}}\PYG{p}{]}\PYG{p}{)}
\end{sphinxVerbatim}
\begin{enumerate}
\sphinxsetlistlabels{\arabic}{enumi}{enumii}{}{.}%
\setcounter{enumi}{1}
\item {} 
\sphinxAtStartPar
Load an MS in the current directory into the context:

\end{enumerate}

\begin{sphinxVerbatim}[commandchars=\\\{\}]
\PYG{g+gp}{\PYGZgt{}\PYGZgt{}\PYGZgt{} }\PYG{n}{hifv\PYGZus{}importdata} \PYG{p}{(}\PYG{n}{vis}\PYG{o}{=}\PYG{p}{[}\PYG{l+s+s1}{\PYGZsq{}}\PYG{l+s+s1}{uid\PYGZus{}\PYGZus{}\PYGZus{}A002\PYGZus{}X30a93d\PYGZus{}X43e.ms}\PYG{l+s+s1}{\PYGZsq{}}\PYG{p}{]}\PYG{p}{)}
\end{sphinxVerbatim}
\begin{enumerate}
\sphinxsetlistlabels{\arabic}{enumi}{enumii}{}{.}%
\setcounter{enumi}{2}
\item {} 
\sphinxAtStartPar
Load a tarred ASDM in ../rawdata into the context:

\end{enumerate}

\begin{sphinxVerbatim}[commandchars=\\\{\}]
\PYG{g+gp}{\PYGZgt{}\PYGZgt{}\PYGZgt{} }\PYG{n}{hifv\PYGZus{}importdata} \PYG{p}{(}\PYG{n}{vis}\PYG{o}{=}\PYG{p}{[}\PYG{l+s+s1}{\PYGZsq{}}\PYG{l+s+s1}{../rawdata/uid\PYGZus{}\PYGZus{}\PYGZus{}A002\PYGZus{}X30a93d\PYGZus{}X43e.tar.gz}\PYG{l+s+s1}{\PYGZsq{}}\PYG{p}{]}\PYG{p}{)}
\end{sphinxVerbatim}
\begin{enumerate}
\sphinxsetlistlabels{\arabic}{enumi}{enumii}{}{.}%
\setcounter{enumi}{3}
\item {} 
\sphinxAtStartPar
Check the hifv\_importdata inputs, then import the data:

\end{enumerate}

\begin{sphinxVerbatim}[commandchars=\\\{\}]
\PYG{g+gp}{\PYGZgt{}\PYGZgt{}\PYGZgt{} }\PYG{n}{myvislist} \PYG{o}{=} \PYG{p}{[}\PYG{l+s+s1}{\PYGZsq{}}\PYG{l+s+s1}{uid\PYGZus{}\PYGZus{}\PYGZus{}A002\PYGZus{}X30a93d\PYGZus{}X43e.ms}\PYG{l+s+s1}{\PYGZsq{}}\PYG{p}{,} \PYG{l+s+s1}{\PYGZsq{}}\PYG{l+s+s1}{uid\PYGZus{}A002\PYGZus{}x30a93d\PYGZus{}X44e.ms}\PYG{l+s+s1}{\PYGZsq{}}\PYG{p}{]}
\PYG{g+gp}{\PYGZgt{}\PYGZgt{}\PYGZgt{} }\PYG{n}{hifv\PYGZus{}importdata}\PYG{p}{(}\PYG{n}{vis}\PYG{o}{=}\PYG{n}{myvislist}\PYG{p}{)}
\end{sphinxVerbatim}
\begin{enumerate}
\sphinxsetlistlabels{\arabic}{enumi}{enumii}{}{.}%
\setcounter{enumi}{4}
\item {} 
\sphinxAtStartPar
Run with explicit setting of data column types:

\end{enumerate}

\begin{sphinxVerbatim}[commandchars=\\\{\}]
\PYG{g+gp}{\PYGZgt{}\PYGZgt{}\PYGZgt{} }\PYG{n}{hifv\PYGZus{}importdata}\PYG{p}{(}\PYG{n}{vis}\PYG{o}{=}\PYG{p}{[}\PYG{l+s+s1}{\PYGZsq{}}\PYG{l+s+s1}{uid\PYGZus{}\PYGZus{}\PYGZus{}A002\PYGZus{}X30a93d\PYGZus{}X43e\PYGZus{}targets.ms}\PYG{l+s+s1}{\PYGZsq{}}\PYG{p}{]}\PYG{p}{,} \PYG{n}{datacolumns}\PYG{o}{=}\PYG{p}{\PYGZob{}}\PYG{l+s+s1}{\PYGZsq{}}\PYG{l+s+s1}{data}\PYG{l+s+s1}{\PYGZsq{}}\PYG{p}{:} \PYG{l+s+s1}{\PYGZsq{}}\PYG{l+s+s1}{regcal\PYGZus{}contline}\PYG{l+s+s1}{\PYGZsq{}}\PYG{p}{\PYGZcb{}}\PYG{p}{)}
\PYG{g+gp}{\PYGZgt{}\PYGZgt{}\PYGZgt{} }\PYG{n}{hifv\PYGZus{}importdata}\PYG{p}{(}\PYG{n}{vis}\PYG{o}{=}\PYG{p}{[}\PYG{l+s+s1}{\PYGZsq{}}\PYG{l+s+s1}{uid\PYGZus{}\PYGZus{}\PYGZus{}A002\PYGZus{}X30a93d\PYGZus{}X43e\PYGZus{}targets\PYGZus{}line.ms}\PYG{l+s+s1}{\PYGZsq{}}\PYG{p}{]}\PYG{p}{,} \PYG{n}{datacolumns}\PYG{o}{=}\PYG{p}{\PYGZob{}}\PYG{l+s+s1}{\PYGZsq{}}\PYG{l+s+s1}{data}\PYG{l+s+s1}{\PYGZsq{}}\PYG{p}{:} \PYG{l+s+s1}{\PYGZsq{}}\PYG{l+s+s1}{regcal\PYGZus{}line}\PYG{l+s+s1}{\PYGZsq{}}\PYG{p}{,} \PYG{l+s+s1}{\PYGZsq{}}\PYG{l+s+s1}{corrected}\PYG{l+s+s1}{\PYGZsq{}}\PYG{p}{:} \PYG{l+s+s1}{\PYGZsq{}}\PYG{l+s+s1}{selfcal\PYGZus{}line}\PYG{l+s+s1}{\PYGZsq{}}\PYG{p}{\PYGZcb{}}\PYG{p}{)}
\end{sphinxVerbatim}

\end{fulllineitems}


\sphinxstepscope


\section{pipeline.hifv.cli.hifv\_mstransform}
\label{\detokenize{_autosummary/pipeline.hifv.cli.hifv_mstransform:pipeline-hifv-cli-hifv-mstransform}}\label{\detokenize{_autosummary/pipeline.hifv.cli.hifv_mstransform::doc}}\index{hifv\_mstransform() (in module pipeline.hifv.cli)@\spxentry{hifv\_mstransform()}\spxextra{in module pipeline.hifv.cli}}

\begin{fulllineitems}
\phantomsection\label{\detokenize{_autosummary/pipeline.hifv.cli.hifv_mstransform:pipeline.hifv.cli.hifv_mstransform}}
\pysigstartsignatures
\pysiglinewithargsret
{\sphinxbfcode{\sphinxupquote{hifv\_mstransform}}}
{\sphinxparam{\DUrole{n}{vis}\DUrole{o}{=}\DUrole{default_value}{None}}\sphinxparamcomma \sphinxparam{\DUrole{n}{outputvis}\DUrole{o}{=}\DUrole{default_value}{None}}\sphinxparamcomma \sphinxparam{\DUrole{n}{outputvis\_for\_line}\DUrole{o}{=}\DUrole{default_value}{None}}\sphinxparamcomma \sphinxparam{\DUrole{n}{field}\DUrole{o}{=}\DUrole{default_value}{None}}\sphinxparamcomma \sphinxparam{\DUrole{n}{intent}\DUrole{o}{=}\DUrole{default_value}{None}}\sphinxparamcomma \sphinxparam{\DUrole{n}{spw}\DUrole{o}{=}\DUrole{default_value}{None}}\sphinxparamcomma \sphinxparam{\DUrole{n}{spw\_line}\DUrole{o}{=}\DUrole{default_value}{None}}\sphinxparamcomma \sphinxparam{\DUrole{n}{chanbin}\DUrole{o}{=}\DUrole{default_value}{None}}\sphinxparamcomma \sphinxparam{\DUrole{n}{timebin}\DUrole{o}{=}\DUrole{default_value}{None}}\sphinxparamcomma \sphinxparam{\DUrole{n}{omit\_contline\_ms}\DUrole{o}{=}\DUrole{default_value}{None}}}
{}
\pysigstopsignatures
\sphinxAtStartPar
Create new MeasurementSets for science target imaging

\sphinxAtStartPar
Create new MeasurementSets for imaging from the corrected column of the input
MeasurementSet via calling mstransform with all data selection parameters.
By default, all science target data is copied to the new MS(s). The
new MeasurementSet is not re\sphinxhyphen{}indexed to the selected data and the new MS will
have the same source, field, and spw names and ids as it does in the parent MS.

\sphinxAtStartPar
The first MeasurementSet that is produced is intended for continuum imaging and
will end in targets\_cont.ms. If there are spws that have been detected or specified as
spectral line spws in the input MeasurementSet, an MS for science target line imaging
will also be produced, which will end in \_targets.ms.
\begin{quote}\begin{description}
\sphinxlineitem{Parameters}\begin{itemize}
\item {} 
\sphinxAtStartPar
\sphinxstylestrong{vis} \textendash{} The list of input MeasurementSets. Defaults to the list of MeasurementSets specified in the h\_init or hif\_importdata task.
‘’: use all MeasurementSets in the context
Examples: ‘ngc5921.ms’, {[}‘ngc5921a.ms’, ngc5921b.ms’, ‘ngc5921c.ms’{]}

\item {} 
\sphinxAtStartPar
\sphinxstylestrong{outputvis} \textendash{} The list of output transformed MeasurementSets to be used for continuum imaging. The output list must be the same length as the input
list. The default output name defaults to
\textless{}msrootname\textgreater{}\_targets\_cont.ms
Examples: ‘ngc5921.ms’,
{[}‘ngc5921a.ms’, ngc5921b.ms’, ‘ngc5921c.ms’{]}

\item {} 
\sphinxAtStartPar
\sphinxstylestrong{outputvis\_for\_line} \textendash{} The list of output transformed MeasurementSets to be used for line imaging. The output list must be the same length as the input
list. The default output name defaults to
\textless{}msrootname\textgreater{}\_targets.ms
Examples: ‘ngc5921.ms’,
{[}‘ngc5921a.ms’, ngc5921b.ms’, ‘ngc5921c.ms’{]}

\item {} 
\sphinxAtStartPar
\sphinxstylestrong{field} \textendash{} Select fields name(s) or id(s) to transform. Only fields with data matching the intent will be selected.
Examples: ‘3C279’, ‘Centaurus*’, ‘3C279,J1427\sphinxhyphen{}421’

\item {} 
\sphinxAtStartPar
\sphinxstylestrong{intent} \textendash{} Select intents for which associated fields will be imaged. By default only TARGET data is selected.
Examples: ‘PHASE, BANDPASS’

\item {} 
\sphinxAtStartPar
\sphinxstylestrong{spw} \textendash{} Select spectral window/channels to include for continuum imaging. By default all science spws for which the specified intent is valid are
selected.

\item {} 
\sphinxAtStartPar
\sphinxstylestrong{spw\_line} \textendash{} Select spectral window/channels to include for line imaging. If specified, these will override the default, which
is to use the spws identified as specline\_windows in hifv\_importdata
or hifv\_restoredata.

\item {} 
\sphinxAtStartPar
\sphinxstylestrong{chanbin} \textendash{} Width (bin) of input channels to average to form an output channel. If chanbin \textgreater{} 1 then chanaverage is automatically
switched to True.

\item {} 
\sphinxAtStartPar
\sphinxstylestrong{timebin} \textendash{} Bin width for time averaging. If timebin \textgreater{} 0s then timeaverage is automatically switched to True.

\item {} 
\sphinxAtStartPar
\sphinxstylestrong{omit\_contline\_ms} \textendash{} If True, don’t make the contline ms (\_targets.ms). Only make cont MS (\_targets\_cont.ms). Default is False.

\end{itemize}

\sphinxlineitem{Returns}
\sphinxAtStartPar
The results object for the pipeline task is returned.

\end{description}\end{quote}
\subsubsection*{Examples}
\begin{enumerate}
\sphinxsetlistlabels{\arabic}{enumi}{enumii}{}{.}%
\item {} 
\sphinxAtStartPar
Create a science target MS from the corrected column in the input MS:

\end{enumerate}

\begin{sphinxVerbatim}[commandchars=\\\{\}]
\PYG{g+gp}{\PYGZgt{}\PYGZgt{}\PYGZgt{} }\PYG{n}{hifv\PYGZus{}mstransform}\PYG{p}{(}\PYG{p}{)}
\end{sphinxVerbatim}

\sphinxAtStartPar
2. Make a phase and bandpass calibrator targets MS from the corrected
column in the input MS:

\begin{sphinxVerbatim}[commandchars=\\\{\}]
\PYG{g+gp}{\PYGZgt{}\PYGZgt{}\PYGZgt{} }\PYG{n}{hifv\PYGZus{}mstransform}\PYG{p}{(}\PYG{n}{intent}\PYG{o}{=}\PYG{l+s+s1}{\PYGZsq{}}\PYG{l+s+s1}{PHASE,BANDPASS}\PYG{l+s+s1}{\PYGZsq{}}\PYG{p}{)}
\end{sphinxVerbatim}

\end{fulllineitems}


\sphinxstepscope


\section{pipeline.hifv.cli.hifv\_opcal}
\label{\detokenize{_autosummary/pipeline.hifv.cli.hifv_opcal:pipeline-hifv-cli-hifv-opcal}}\label{\detokenize{_autosummary/pipeline.hifv.cli.hifv_opcal::doc}}\index{hifv\_opcal() (in module pipeline.hifv.cli)@\spxentry{hifv\_opcal()}\spxextra{in module pipeline.hifv.cli}}

\begin{fulllineitems}
\phantomsection\label{\detokenize{_autosummary/pipeline.hifv.cli.hifv_opcal:pipeline.hifv.cli.hifv_opcal}}
\pysigstartsignatures
\pysiglinewithargsret
{\sphinxbfcode{\sphinxupquote{hifv\_opcal}}}
{\sphinxparam{\DUrole{n}{vis}\DUrole{o}{=}\DUrole{default_value}{None}}\sphinxparamcomma \sphinxparam{\DUrole{n}{caltable}\DUrole{o}{=}\DUrole{default_value}{None}}}
{}
\pysigstopsignatures
\sphinxAtStartPar
Runs gencal in opac mode.
\begin{quote}\begin{description}
\sphinxlineitem{Parameters}
\sphinxAtStartPar
\sphinxstylestrong{vis} \textendash{} List of input visibility data.

\sphinxlineitem{Returns}
\sphinxAtStartPar
The results object for the pipeline task is returned.

\end{description}\end{quote}
\subsubsection*{Examples}
\begin{enumerate}
\sphinxsetlistlabels{\arabic}{enumi}{enumii}{}{.}%
\item {} 
\sphinxAtStartPar
Load an ASDM list in the ../rawdata subdirectory into the context:

\end{enumerate}

\begin{sphinxVerbatim}[commandchars=\\\{\}]
\PYG{g+gp}{\PYGZgt{}\PYGZgt{}\PYGZgt{} }\PYG{n}{hifv\PYGZus{}opcal}\PYG{p}{(}\PYG{p}{)}
\end{sphinxVerbatim}

\end{fulllineitems}


\sphinxstepscope


\section{pipeline.hifv.cli.hifv\_pbcor}
\label{\detokenize{_autosummary/pipeline.hifv.cli.hifv_pbcor:pipeline-hifv-cli-hifv-pbcor}}\label{\detokenize{_autosummary/pipeline.hifv.cli.hifv_pbcor::doc}}\index{hifv\_pbcor() (in module pipeline.hifv.cli)@\spxentry{hifv\_pbcor()}\spxextra{in module pipeline.hifv.cli}}

\begin{fulllineitems}
\phantomsection\label{\detokenize{_autosummary/pipeline.hifv.cli.hifv_pbcor:pipeline.hifv.cli.hifv_pbcor}}
\pysigstartsignatures
\pysiglinewithargsret
{\sphinxbfcode{\sphinxupquote{hifv\_pbcor}}}
{\sphinxparam{\DUrole{n}{vis}\DUrole{o}{=}\DUrole{default_value}{None}}}
{}
\pysigstopsignatures
\sphinxAtStartPar
Apply primary beam correction to VLA and VLASS images.
\begin{quote}\begin{description}
\sphinxlineitem{Parameters}
\sphinxAtStartPar
\sphinxstylestrong{vis} \textendash{} List of input visibility data.

\sphinxlineitem{Returns}
\sphinxAtStartPar
The results object for the pipeline task is returned.

\end{description}\end{quote}
\subsubsection*{Examples}
\begin{enumerate}
\sphinxsetlistlabels{\arabic}{enumi}{enumii}{}{.}%
\item {} 
\sphinxAtStartPar
Basic pbcor task:

\end{enumerate}

\begin{sphinxVerbatim}[commandchars=\\\{\}]
\PYG{g+gp}{\PYGZgt{}\PYGZgt{}\PYGZgt{} }\PYG{n}{hifv\PYGZus{}pbcor}\PYG{p}{(}\PYG{p}{)}
\end{sphinxVerbatim}

\end{fulllineitems}


\sphinxstepscope


\section{pipeline.hifv.cli.hifv\_plotsummary}
\label{\detokenize{_autosummary/pipeline.hifv.cli.hifv_plotsummary:pipeline-hifv-cli-hifv-plotsummary}}\label{\detokenize{_autosummary/pipeline.hifv.cli.hifv_plotsummary::doc}}\index{hifv\_plotsummary() (in module pipeline.hifv.cli)@\spxentry{hifv\_plotsummary()}\spxextra{in module pipeline.hifv.cli}}

\begin{fulllineitems}
\phantomsection\label{\detokenize{_autosummary/pipeline.hifv.cli.hifv_plotsummary:pipeline.hifv.cli.hifv_plotsummary}}
\pysigstartsignatures
\pysiglinewithargsret
{\sphinxbfcode{\sphinxupquote{hifv\_plotsummary}}}
{\sphinxparam{\DUrole{n}{vis}\DUrole{o}{=}\DUrole{default_value}{None}}}
{}
\pysigstopsignatures
\sphinxAtStartPar
Create pipeline summary plots.
\begin{quote}\begin{description}
\sphinxlineitem{Parameters}
\sphinxAtStartPar
\sphinxstylestrong{vis} \textendash{} List of input visibility data.

\sphinxlineitem{Returns}
\sphinxAtStartPar
The results object for the pipeline task is returned.

\end{description}\end{quote}
\subsubsection*{Examples}
\begin{enumerate}
\sphinxsetlistlabels{\arabic}{enumi}{enumii}{}{.}%
\item {} 
\sphinxAtStartPar
Execute the pipeline plotting task:

\end{enumerate}

\begin{sphinxVerbatim}[commandchars=\\\{\}]
\PYG{g+gp}{\PYGZgt{}\PYGZgt{}\PYGZgt{} }\PYG{n}{hifv\PYGZus{}plotsummary}\PYG{p}{(}\PYG{p}{)}
\end{sphinxVerbatim}

\end{fulllineitems}


\sphinxstepscope


\section{pipeline.hifv.cli.hifv\_priorcals}
\label{\detokenize{_autosummary/pipeline.hifv.cli.hifv_priorcals:pipeline-hifv-cli-hifv-priorcals}}\label{\detokenize{_autosummary/pipeline.hifv.cli.hifv_priorcals::doc}}\index{hifv\_priorcals() (in module pipeline.hifv.cli)@\spxentry{hifv\_priorcals()}\spxextra{in module pipeline.hifv.cli}}

\begin{fulllineitems}
\phantomsection\label{\detokenize{_autosummary/pipeline.hifv.cli.hifv_priorcals:pipeline.hifv.cli.hifv_priorcals}}
\pysigstartsignatures
\pysiglinewithargsret
{\sphinxbfcode{\sphinxupquote{hifv\_priorcals}}}
{\sphinxparam{\DUrole{n}{vis}\DUrole{o}{=}\DUrole{default_value}{None}}\sphinxparamcomma \sphinxparam{\DUrole{n}{show\_tec\_maps}\DUrole{o}{=}\DUrole{default_value}{None}}\sphinxparamcomma \sphinxparam{\DUrole{n}{apply\_tec\_correction}\DUrole{o}{=}\DUrole{default_value}{None}}\sphinxparamcomma \sphinxparam{\DUrole{n}{apply\_gaincurves}\DUrole{o}{=}\DUrole{default_value}{None}}\sphinxparamcomma \sphinxparam{\DUrole{n}{apply\_opcal}\DUrole{o}{=}\DUrole{default_value}{None}}\sphinxparamcomma \sphinxparam{\DUrole{n}{apply\_rqcal}\DUrole{o}{=}\DUrole{default_value}{None}}\sphinxparamcomma \sphinxparam{\DUrole{n}{apply\_antpos}\DUrole{o}{=}\DUrole{default_value}{None}}\sphinxparamcomma \sphinxparam{\DUrole{n}{apply\_swpowcal}\DUrole{o}{=}\DUrole{default_value}{None}}\sphinxparamcomma \sphinxparam{\DUrole{n}{swpow\_spw}\DUrole{o}{=}\DUrole{default_value}{None}}\sphinxparamcomma \sphinxparam{\DUrole{n}{ant\_pos\_time\_limit}\DUrole{o}{=}\DUrole{default_value}{None}}}
{}
\pysigstopsignatures
\sphinxAtStartPar
Runs gaincurves, opacities, requantizer gains, antenna position corrections, tec\_maps, switched power.
\begin{quote}\begin{description}
\sphinxlineitem{Parameters}\begin{itemize}
\item {} 
\sphinxAtStartPar
\sphinxstylestrong{vis} \textendash{} List of visibility data files. These may be ASDMs, tar files of ASDMs, MSes, or tar files of MSes, If ASDM files are specified, they will be
converted  to MS format.
example: vis={[}‘X227.ms’, ‘asdms.tar.gz’{]}

\item {} 
\sphinxAtStartPar
\sphinxstylestrong{show\_tec\_maps} \textendash{} Plot tec maps.

\item {} 
\sphinxAtStartPar
\sphinxstylestrong{apply\_tec\_correction} \textendash{} Apply tec correction.

\item {} 
\sphinxAtStartPar
\sphinxstylestrong{apply\_gaincurves} \textendash{} Apply gain curves correction, default True.

\item {} 
\sphinxAtStartPar
\sphinxstylestrong{apply\_opcal} \textendash{} Apply opacities correction, default True.

\item {} 
\sphinxAtStartPar
\sphinxstylestrong{apply\_rqcal} \textendash{} Apply requantizer gains correction, default True.

\item {} 
\sphinxAtStartPar
\sphinxstylestrong{apply\_antpos} \textendash{} Apply antenna position corrections, default True.

\item {} 
\sphinxAtStartPar
\sphinxstylestrong{apply\_swpowcal} \textendash{} Apply switched power table, default False.  If set True, \sphinxcode{\sphinxupquote{apply\_rqcal}} is ignored and no requantizer gain correction will be applied.

\item {} 
\sphinxAtStartPar
\sphinxstylestrong{swpow\_spw} \textendash{} Spectral\sphinxhyphen{}window(s) for plotting: “” ==\textgreater{}all, spw=”6, 14”

\item {} 
\sphinxAtStartPar
\sphinxstylestrong{ant\_pos\_time\_limit} \textendash{} Antenna position time limit in days, default to 150 days.

\end{itemize}

\sphinxlineitem{Returns}
\sphinxAtStartPar
The results object for the pipeline task is returned.

\end{description}\end{quote}
\subsubsection*{Examples}
\begin{enumerate}
\sphinxsetlistlabels{\arabic}{enumi}{enumii}{}{.}%
\item {} 
\sphinxAtStartPar
Run gaincurves, opacities, requantizer gains and antenna position corrections:

\end{enumerate}

\begin{sphinxVerbatim}[commandchars=\\\{\}]
\PYG{g+gp}{\PYGZgt{}\PYGZgt{}\PYGZgt{} }\PYG{n}{hifv\PYGZus{}priorcals}\PYG{p}{(}\PYG{p}{)}
\end{sphinxVerbatim}

\end{fulllineitems}


\sphinxstepscope


\section{pipeline.hifv.cli.hifv\_restoredata}
\label{\detokenize{_autosummary/pipeline.hifv.cli.hifv_restoredata:pipeline-hifv-cli-hifv-restoredata}}\label{\detokenize{_autosummary/pipeline.hifv.cli.hifv_restoredata::doc}}\index{hifv\_restoredata() (in module pipeline.hifv.cli)@\spxentry{hifv\_restoredata()}\spxextra{in module pipeline.hifv.cli}}

\begin{fulllineitems}
\phantomsection\label{\detokenize{_autosummary/pipeline.hifv.cli.hifv_restoredata:pipeline.hifv.cli.hifv_restoredata}}
\pysigstartsignatures
\pysiglinewithargsret
{\sphinxbfcode{\sphinxupquote{hifv\_restoredata}}}
{\sphinxparam{\DUrole{n}{vis}\DUrole{o}{=}\DUrole{default_value}{None}}\sphinxparamcomma \sphinxparam{\DUrole{n}{session}\DUrole{o}{=}\DUrole{default_value}{None}}\sphinxparamcomma \sphinxparam{\DUrole{n}{products\_dir}\DUrole{o}{=}\DUrole{default_value}{None}}\sphinxparamcomma \sphinxparam{\DUrole{n}{copytoraw}\DUrole{o}{=}\DUrole{default_value}{None}}\sphinxparamcomma \sphinxparam{\DUrole{n}{rawdata\_dir}\DUrole{o}{=}\DUrole{default_value}{None}}\sphinxparamcomma \sphinxparam{\DUrole{n}{lazy}\DUrole{o}{=}\DUrole{default_value}{None}}\sphinxparamcomma \sphinxparam{\DUrole{n}{bdfflags}\DUrole{o}{=}\DUrole{default_value}{None}}\sphinxparamcomma \sphinxparam{\DUrole{n}{ocorr\_mode}\DUrole{o}{=}\DUrole{default_value}{None}}\sphinxparamcomma \sphinxparam{\DUrole{n}{gainmap}\DUrole{o}{=}\DUrole{default_value}{None}}\sphinxparamcomma \sphinxparam{\DUrole{n}{asis}\DUrole{o}{=}\DUrole{default_value}{None}}}
{}
\pysigstopsignatures
\sphinxAtStartPar
Restore flagged and calibration interferometry data from a pipeline run.

\sphinxAtStartPar
The hifv\_restoredata restores flagged and calibrated data from archived
ASDMs and pipeline flagging and calibration data products.

\sphinxAtStartPar
hifv\_restoredata assumes that the ASDMs to be restored are present in the
directory specified by the \sphinxcode{\sphinxupquote{rawdata\_dir}} (default: ‘../rawdata’).

\sphinxAtStartPar
By default (\sphinxcode{\sphinxupquote{copytoraw}} = True), hifv\_restoredata assumes that for each
ASDM in the input list, the corresponding pipeline flagging and calibration
data products (in the format produced by the hifv\_exportdata task) are
present in the directory specified by \sphinxcode{\sphinxupquote{products\_dir}} (default: ‘../products’).
At the start of the task, these products are copied from the \sphinxcode{\sphinxupquote{products\_dir}}
to the \sphinxcode{\sphinxupquote{rawdata\_dir}}.

\sphinxAtStartPar
If \sphinxcode{\sphinxupquote{copytoraw}} = False, hifv\_restoredata assumes that these products are
to be found in \sphinxcode{\sphinxupquote{rawdata\_dir}} along with the ASDMs.

\sphinxAtStartPar
The expected flagging and calibration products (for each ASDM) include:
\begin{itemize}
\item {} 
\sphinxAtStartPar
a compressed tar file of the final flagversions file, e.g.
uid\_\_\_A002\_X30a93d\_X43e.ms.flagversions.tar.gz

\item {} 
\sphinxAtStartPar
a text file containing the applycal instructions, e.g.
uid\_\_\_A002\_X30a93d\_X43e.ms.calapply.txt

\item {} 
\sphinxAtStartPar
a compressed tar file containing the caltables for the parent session,
e.g. uid\_\_\_A001\_X74\_X29.session\_3.caltables.tar.gz

\end{itemize}

\sphinxAtStartPar
hifv\_restoredata performs the following operations:
\begin{itemize}
\item {} 
\sphinxAtStartPar
imports the ASDM(s)

\item {} 
\sphinxAtStartPar
runs the hanning smoothing task

\item {} 
\sphinxAtStartPar
removes the default MS.flagversions directory created by the filler

\item {} 
\sphinxAtStartPar
restores the final MS.flagversions directory stored by the pipeline

\item {} 
\sphinxAtStartPar
restores the final set of pipeline flags to the MS

\item {} 
\sphinxAtStartPar
restores the final calibration state of the MS

\item {} 
\sphinxAtStartPar
restores the final calibration tables for each MS

\item {} 
\sphinxAtStartPar
applies the calibration tables to each MS

\end{itemize}
\begin{quote}\begin{description}
\sphinxlineitem{Parameters}\begin{itemize}
\item {} 
\sphinxAtStartPar
\sphinxstylestrong{vis} \textendash{} List of visibility data files. These may be ASDMs, tar files of ASDMs, MSes, or tar files of MSes, If ASDM files are specified, they will be
converted  to MS format.
example: vis={[}‘X227.ms’, ‘asdms.tar.gz’{]}

\item {} 
\sphinxAtStartPar
\sphinxstylestrong{session} \textendash{} List of sessions one per visibility file. Example: session={[}‘session\_3’{]}

\item {} 
\sphinxAtStartPar
\sphinxstylestrong{products\_dir} \textendash{} Name of the data products directory to copy calibration products from.
Default: ‘../products’
The parameter is effective only when \sphinxcode{\sphinxupquote{copytoraw}} = True.
When \sphinxcode{\sphinxupquote{copytoraw}} = False, calibration products in
\sphinxcode{\sphinxupquote{rawdata\_dir}} will be used.
example: products\_dir=’myproductspath’

\item {} 
\sphinxAtStartPar
\sphinxstylestrong{copytoraw} \textendash{} Copy calibration and flagging tables from \sphinxcode{\sphinxupquote{products\_dir}} to \sphinxcode{\sphinxupquote{rawdata\_dir}} directory.
Default: True
Example: copytoraw=False.

\item {} 
\sphinxAtStartPar
\sphinxstylestrong{rawdata\_dir} \textendash{} Name of the raw data directory. Default: ‘../rawdata’
Example: rawdata\_dir=’myrawdatapath’

\item {} 
\sphinxAtStartPar
\sphinxstylestrong{lazy} \textendash{} Use the lazy filler option. Default: False

\item {} 
\sphinxAtStartPar
\sphinxstylestrong{bdfflags} \textendash{} Set the BDF flags. Default: False

\item {} 
\sphinxAtStartPar
\sphinxstylestrong{ocorr\_mode} \textendash{} Correlation import mode. Default: ‘co’

\item {} 
\sphinxAtStartPar
\sphinxstylestrong{gainmap} \textendash{} If True, map gainfields to a particular list of scans when applying calibration tables.
Default: False

\item {} 
\sphinxAtStartPar
\sphinxstylestrong{asis} \textendash{} List of tables to import asis. Default: ‘Receiver CalAtmosphere’

\end{itemize}

\sphinxlineitem{Returns}
\sphinxAtStartPar
The results object for the pipeline task is returned.

\end{description}\end{quote}
\subsubsection*{Examples}
\begin{enumerate}
\sphinxsetlistlabels{\arabic}{enumi}{enumii}{}{.}%
\item {} 
\sphinxAtStartPar
Restore the pipeline results for a single ASDM in a single session:

\end{enumerate}

\begin{sphinxVerbatim}[commandchars=\\\{\}]
\PYG{g+gp}{\PYGZgt{}\PYGZgt{}\PYGZgt{} }\PYG{n}{hifv\PYGZus{}restoredata} \PYG{p}{(}\PYG{n}{vis}\PYG{o}{=}\PYG{p}{[}\PYG{l+s+s1}{\PYGZsq{}}\PYG{l+s+s1}{myVLAsdm}\PYG{l+s+s1}{\PYGZsq{}}\PYG{p}{]}\PYG{p}{,} \PYG{n}{session}\PYG{o}{=}\PYG{p}{[}\PYG{l+s+s1}{\PYGZsq{}}\PYG{l+s+s1}{session\PYGZus{}1}\PYG{l+s+s1}{\PYGZsq{}}\PYG{p}{]}\PYG{p}{,} \PYG{n}{ocorr\PYGZus{}mode}\PYG{o}{=}\PYG{l+s+s1}{\PYGZsq{}}\PYG{l+s+s1}{ca}\PYG{l+s+s1}{\PYGZsq{}}\PYG{p}{)}
\end{sphinxVerbatim}

\end{fulllineitems}


\sphinxstepscope


\section{pipeline.hifv.cli.hifv\_restorepims}
\label{\detokenize{_autosummary/pipeline.hifv.cli.hifv_restorepims:pipeline-hifv-cli-hifv-restorepims}}\label{\detokenize{_autosummary/pipeline.hifv.cli.hifv_restorepims::doc}}\index{hifv\_restorepims() (in module pipeline.hifv.cli)@\spxentry{hifv\_restorepims()}\spxextra{in module pipeline.hifv.cli}}

\begin{fulllineitems}
\phantomsection\label{\detokenize{_autosummary/pipeline.hifv.cli.hifv_restorepims:pipeline.hifv.cli.hifv_restorepims}}
\pysigstartsignatures
\pysiglinewithargsret
{\sphinxbfcode{\sphinxupquote{hifv\_restorepims}}}
{\sphinxparam{\DUrole{n}{vis}\DUrole{o}{=}\DUrole{default_value}{None}}\sphinxparamcomma \sphinxparam{\DUrole{n}{reimaging\_resources}\DUrole{o}{=}\DUrole{default_value}{None}}}
{}
\pysigstopsignatures
\sphinxAtStartPar
Restore VLASS SE per\sphinxhyphen{}image measurement set data, resetting flagging, weights, and applying self\sphinxhyphen{}calibration.
\begin{quote}\begin{description}
\sphinxlineitem{Parameters}\begin{itemize}
\item {} 
\sphinxAtStartPar
\sphinxstylestrong{vis} \textendash{} List of input visibility data.

\item {} 
\sphinxAtStartPar
\sphinxstylestrong{reimaging\_resources} \textendash{} file path of reimaging\_resources.tgz from the SE imaging product.

\end{itemize}

\sphinxlineitem{Returns}
\sphinxAtStartPar
The results object for the pipeline task is returned.

\end{description}\end{quote}
\subsubsection*{Examples}
\begin{enumerate}
\sphinxsetlistlabels{\arabic}{enumi}{enumii}{}{.}%
\item {} 
\sphinxAtStartPar
Basic restorepims task:

\end{enumerate}

\begin{sphinxVerbatim}[commandchars=\\\{\}]
\PYG{g+gp}{\PYGZgt{}\PYGZgt{}\PYGZgt{} }\PYG{n}{hifv\PYGZus{}restorepims}\PYG{p}{(}\PYG{p}{)}
\end{sphinxVerbatim}

\end{fulllineitems}


\sphinxstepscope


\section{pipeline.hifv.cli.hifv\_rqcal}
\label{\detokenize{_autosummary/pipeline.hifv.cli.hifv_rqcal:pipeline-hifv-cli-hifv-rqcal}}\label{\detokenize{_autosummary/pipeline.hifv.cli.hifv_rqcal::doc}}\index{hifv\_rqcal() (in module pipeline.hifv.cli)@\spxentry{hifv\_rqcal()}\spxextra{in module pipeline.hifv.cli}}

\begin{fulllineitems}
\phantomsection\label{\detokenize{_autosummary/pipeline.hifv.cli.hifv_rqcal:pipeline.hifv.cli.hifv_rqcal}}
\pysigstartsignatures
\pysiglinewithargsret
{\sphinxbfcode{\sphinxupquote{hifv\_rqcal}}}
{\sphinxparam{\DUrole{n}{vis}\DUrole{o}{=}\DUrole{default_value}{None}}\sphinxparamcomma \sphinxparam{\DUrole{n}{caltable}\DUrole{o}{=}\DUrole{default_value}{None}}}
{}
\pysigstopsignatures
\sphinxAtStartPar
Runs gencal in rq mode.
\begin{quote}\begin{description}
\sphinxlineitem{Parameters}\begin{itemize}
\item {} 
\sphinxAtStartPar
\sphinxstylestrong{vis} \textendash{} List of input visibility data.

\item {} 
\sphinxAtStartPar
\sphinxstylestrong{caltable} \textendash{} String name of caltable.

\end{itemize}

\sphinxlineitem{Returns}
\sphinxAtStartPar
The results object for the pipeline task is returned.

\end{description}\end{quote}
\subsubsection*{Examples}
\begin{enumerate}
\sphinxsetlistlabels{\arabic}{enumi}{enumii}{}{.}%
\item {} 
\sphinxAtStartPar
Load an ASDM list in the ../rawdata subdirectory into the context:

\end{enumerate}

\begin{sphinxVerbatim}[commandchars=\\\{\}]
\PYG{g+gp}{\PYGZgt{}\PYGZgt{}\PYGZgt{} }\PYG{n}{hifv\PYGZus{}rqcal}\PYG{p}{(}\PYG{p}{)}
\end{sphinxVerbatim}

\end{fulllineitems}


\sphinxstepscope


\section{pipeline.hifv.cli.hifv\_selfcal}
\label{\detokenize{_autosummary/pipeline.hifv.cli.hifv_selfcal:pipeline-hifv-cli-hifv-selfcal}}\label{\detokenize{_autosummary/pipeline.hifv.cli.hifv_selfcal::doc}}\index{hifv\_selfcal() (in module pipeline.hifv.cli)@\spxentry{hifv\_selfcal()}\spxextra{in module pipeline.hifv.cli}}

\begin{fulllineitems}
\phantomsection\label{\detokenize{_autosummary/pipeline.hifv.cli.hifv_selfcal:pipeline.hifv.cli.hifv_selfcal}}
\pysigstartsignatures
\pysiglinewithargsret
{\sphinxbfcode{\sphinxupquote{hifv\_selfcal}}}
{\sphinxparam{\DUrole{n}{vis}\DUrole{o}{=}\DUrole{default_value}{None}}\sphinxparamcomma \sphinxparam{\DUrole{n}{refantignore}\DUrole{o}{=}\DUrole{default_value}{None}}\sphinxparamcomma \sphinxparam{\DUrole{n}{combine}\DUrole{o}{=}\DUrole{default_value}{None}}\sphinxparamcomma \sphinxparam{\DUrole{n}{selfcalmode}\DUrole{o}{=}\DUrole{default_value}{None}}\sphinxparamcomma \sphinxparam{\DUrole{n}{refantmode}\DUrole{o}{=}\DUrole{default_value}{None}}\sphinxparamcomma \sphinxparam{\DUrole{n}{overwrite\_modelcol}\DUrole{o}{=}\DUrole{default_value}{None}}}
{}
\pysigstopsignatures
\sphinxAtStartPar
Perform phase\sphinxhyphen{}only self\sphinxhyphen{}calibration, per scan row, on VLASS SE images.
\begin{quote}\begin{description}
\sphinxlineitem{Parameters}\begin{itemize}
\item {} 
\sphinxAtStartPar
\sphinxstylestrong{vis} \textendash{} The list of input MeasurementSets. Defaults to the list of MeasurementSets specified in the h\_init or hifv\_importdata task.

\item {} 
\sphinxAtStartPar
\sphinxstylestrong{refantignore} \textendash{} String list of antennas to ignore.

\item {} 
\sphinxAtStartPar
\sphinxstylestrong{combine} \textendash{} Data axes which to combine for solve.
Options: ‘’,’obs’,’scan’,’spw’,field’, or any
comma\sphinxhyphen{}separated combination in a single string
Example: combine=’scan,spw’ \sphinxhyphen{} Extend solutions
over scan boundaries (up to the solint), and
combine spws for solving.
In selfcalmode=’VLASS\sphinxhyphen{}SE’ use the default value.

\item {} 
\sphinxAtStartPar
\sphinxstylestrong{selfcalmode} \textendash{} Heuristics mode selection. Known modes are ‘VLASS’ and ‘VLASS\sphinxhyphen{}SE’. Default value is ‘VLASS’.

\item {} 
\sphinxAtStartPar
\sphinxstylestrong{refantmode} \textendash{} Reference antenna mode.

\item {} 
\sphinxAtStartPar
\sphinxstylestrong{overwrite\_modelcol} \textendash{} Always write the model column, even if it already exists.

\end{itemize}

\sphinxlineitem{Returns}
\sphinxAtStartPar
The results object for the pipeline task is returned.

\end{description}\end{quote}
\subsubsection*{Examples}
\begin{enumerate}
\sphinxsetlistlabels{\arabic}{enumi}{enumii}{}{.}%
\item {} 
\sphinxAtStartPar
Basic selfcal task:

\end{enumerate}

\begin{sphinxVerbatim}[commandchars=\\\{\}]
\PYG{g+gp}{\PYGZgt{}\PYGZgt{}\PYGZgt{} }\PYG{n}{hifv\PYGZus{}selfcal}\PYG{p}{(}\PYG{p}{)}
\end{sphinxVerbatim}
\begin{enumerate}
\sphinxsetlistlabels{\arabic}{enumi}{enumii}{}{.}%
\setcounter{enumi}{1}
\item {} 
\sphinxAtStartPar
VLASS\sphinxhyphen{}SE selfcal usage:

\end{enumerate}

\begin{sphinxVerbatim}[commandchars=\\\{\}]
\PYG{g+gp}{\PYGZgt{}\PYGZgt{}\PYGZgt{} }\PYG{n}{hifv\PYGZus{}selfcal}\PYG{p}{(}\PYG{n}{selfcalmode}\PYG{o}{=}\PYG{l+s+s1}{\PYGZsq{}}\PYG{l+s+s1}{VLASS\PYGZhy{}SE}\PYG{l+s+s1}{\PYGZsq{}}\PYG{p}{,} \PYG{n}{combine}\PYG{o}{=}\PYG{l+s+s1}{\PYGZsq{}}\PYG{l+s+s1}{field, spw}\PYG{l+s+s1}{\PYGZsq{}}\PYG{p}{)}
\end{sphinxVerbatim}

\end{fulllineitems}


\sphinxstepscope


\section{pipeline.hifv.cli.hifv\_semiFinalBPdcals}
\label{\detokenize{_autosummary/pipeline.hifv.cli.hifv_semiFinalBPdcals:pipeline-hifv-cli-hifv-semifinalbpdcals}}\label{\detokenize{_autosummary/pipeline.hifv.cli.hifv_semiFinalBPdcals::doc}}\index{hifv\_semiFinalBPdcals() (in module pipeline.hifv.cli)@\spxentry{hifv\_semiFinalBPdcals()}\spxextra{in module pipeline.hifv.cli}}

\begin{fulllineitems}
\phantomsection\label{\detokenize{_autosummary/pipeline.hifv.cli.hifv_semiFinalBPdcals:pipeline.hifv.cli.hifv_semiFinalBPdcals}}
\pysigstartsignatures
\pysiglinewithargsret
{\sphinxbfcode{\sphinxupquote{hifv\_semiFinalBPdcals}}}
{\sphinxparam{\DUrole{n}{vis}\DUrole{o}{=}\DUrole{default_value}{None}}\sphinxparamcomma \sphinxparam{\DUrole{n}{weakbp}\DUrole{o}{=}\DUrole{default_value}{None}}\sphinxparamcomma \sphinxparam{\DUrole{n}{refantignore}\DUrole{o}{=}\DUrole{default_value}{None}}\sphinxparamcomma \sphinxparam{\DUrole{n}{refant}\DUrole{o}{=}\DUrole{default_value}{None}}}
{}
\pysigstopsignatures
\sphinxAtStartPar
Runs a second delay and bandpass calibration and applies to calibrators to setup for RFI flagging.
\begin{quote}\begin{description}
\sphinxlineitem{Parameters}\begin{itemize}
\item {} 
\sphinxAtStartPar
\sphinxstylestrong{vis} \textendash{} The list of input MeasurementSets. Defaults to the list of MeasurementSets specified in the h\_init or hifv\_importdata task.

\item {} 
\sphinxAtStartPar
\sphinxstylestrong{weakbp} \textendash{} Activate weak bandpass heuristics.

\item {} 
\sphinxAtStartPar
\sphinxstylestrong{refantignore} \textendash{} String list of antennas to ignore.

\item {} 
\sphinxAtStartPar
\sphinxstylestrong{refant} \textendash{} A csv string of reference antenna(s). When used, disables \sphinxcode{\sphinxupquote{refantignore}}. Example: refant = ‘ea01, ea02’

\end{itemize}

\sphinxlineitem{Returns}
\sphinxAtStartPar
The results object for the pipeline task is returned.

\end{description}\end{quote}
\subsubsection*{Examples}
\begin{enumerate}
\sphinxsetlistlabels{\arabic}{enumi}{enumii}{}{.}%
\item {} 
\sphinxAtStartPar
Heuristic flagging:

\end{enumerate}

\begin{sphinxVerbatim}[commandchars=\\\{\}]
\PYG{g+gp}{\PYGZgt{}\PYGZgt{}\PYGZgt{} }\PYG{n}{hifv\PYGZus{}semiFinalBPdcals}\PYG{p}{(}\PYG{p}{)}
\end{sphinxVerbatim}

\end{fulllineitems}


\sphinxstepscope


\section{pipeline.hifv.cli.hifv\_solint}
\label{\detokenize{_autosummary/pipeline.hifv.cli.hifv_solint:pipeline-hifv-cli-hifv-solint}}\label{\detokenize{_autosummary/pipeline.hifv.cli.hifv_solint::doc}}\index{hifv\_solint() (in module pipeline.hifv.cli)@\spxentry{hifv\_solint()}\spxextra{in module pipeline.hifv.cli}}

\begin{fulllineitems}
\phantomsection\label{\detokenize{_autosummary/pipeline.hifv.cli.hifv_solint:pipeline.hifv.cli.hifv_solint}}
\pysigstartsignatures
\pysiglinewithargsret
{\sphinxbfcode{\sphinxupquote{hifv\_solint}}}
{\sphinxparam{\DUrole{n}{vis}\DUrole{o}{=}\DUrole{default_value}{None}}\sphinxparamcomma \sphinxparam{\DUrole{n}{limit\_short\_solint}\DUrole{o}{=}\DUrole{default_value}{None}}\sphinxparamcomma \sphinxparam{\DUrole{n}{refantignore}\DUrole{o}{=}\DUrole{default_value}{None}}\sphinxparamcomma \sphinxparam{\DUrole{n}{refant}\DUrole{o}{=}\DUrole{default_value}{None}}}
{}
\pysigstopsignatures
\sphinxAtStartPar
Determines different solution intervals.

\sphinxAtStartPar
The hifv\_solint task determines different solution intervals. Note that the short solint value is switched to ‘int’ when
the minimum solution interval corresponds to one integration.
\begin{quote}\begin{description}
\sphinxlineitem{Parameters}\begin{itemize}
\item {} 
\sphinxAtStartPar
\sphinxstylestrong{vis} \textendash{} The list of input MeasurementSets. Defaults to the list of MeasurementSets specified in the h\_init or hifv\_importdata task.

\item {} 
\sphinxAtStartPar
\sphinxstylestrong{limit\_short\_solint} \textendash{} Keyword argument in units of seconds to limit the short solution interval. Can be a string or float numerical value in units of seconds of ‘0.45’ or 0.45.
Can be set to a string value of ‘int’.

\item {} 
\sphinxAtStartPar
\sphinxstylestrong{refantignore} \textendash{} String list of antennas to ignore Example:  refantignore=’ea02, ea03’

\item {} 
\sphinxAtStartPar
\sphinxstylestrong{refant} \textendash{} A csv string of reference antenna(s). When used, disables \sphinxcode{\sphinxupquote{refantignore}}. Example: refant = ‘ea01, ea02’

\end{itemize}

\sphinxlineitem{Returns}
\sphinxAtStartPar
The results object for the pipeline task is returned.

\end{description}\end{quote}
\subsubsection*{Examples}
\begin{enumerate}
\sphinxsetlistlabels{\arabic}{enumi}{enumii}{}{.}%
\item {} 
\sphinxAtStartPar
Determines different solution intervals:

\end{enumerate}

\begin{sphinxVerbatim}[commandchars=\\\{\}]
\PYG{g+gp}{\PYGZgt{}\PYGZgt{}\PYGZgt{} }\PYG{n}{hifv\PYGZus{}solint}\PYG{p}{(}\PYG{p}{)}
\end{sphinxVerbatim}

\end{fulllineitems}


\sphinxstepscope


\section{pipeline.hifv.cli.hifv\_statwt}
\label{\detokenize{_autosummary/pipeline.hifv.cli.hifv_statwt:pipeline-hifv-cli-hifv-statwt}}\label{\detokenize{_autosummary/pipeline.hifv.cli.hifv_statwt::doc}}\index{hifv\_statwt() (in module pipeline.hifv.cli)@\spxentry{hifv\_statwt()}\spxextra{in module pipeline.hifv.cli}}

\begin{fulllineitems}
\phantomsection\label{\detokenize{_autosummary/pipeline.hifv.cli.hifv_statwt:pipeline.hifv.cli.hifv_statwt}}
\pysigstartsignatures
\pysiglinewithargsret
{\sphinxbfcode{\sphinxupquote{hifv\_statwt}}}
{\sphinxparam{\DUrole{n}{vis}\DUrole{o}{=}\DUrole{default_value}{None}}\sphinxparamcomma \sphinxparam{\DUrole{n}{datacolumn}\DUrole{o}{=}\DUrole{default_value}{None}}\sphinxparamcomma \sphinxparam{\DUrole{n}{overwrite\_modelcol}\DUrole{o}{=}\DUrole{default_value}{None}}\sphinxparamcomma \sphinxparam{\DUrole{n}{statwtmode}\DUrole{o}{=}\DUrole{default_value}{None}}}
{}
\pysigstopsignatures
\sphinxAtStartPar
Compute statistical weights and write them to measurement set.
\begin{quote}\begin{description}
\sphinxlineitem{Parameters}\begin{itemize}
\item {} 
\sphinxAtStartPar
\sphinxstylestrong{vis} \textendash{} The list of input MeasurementSets. Defaults to the list of MeasurementSets specified in the h\_init or hifv\_importdata task.

\item {} 
\sphinxAtStartPar
\sphinxstylestrong{datacolumn} \textendash{} Data column used to compute weights. Supported values are “data”, “corrected”, “residual”, and “residual\_data”
(case insensitive, minimum match supported).

\item {} 
\sphinxAtStartPar
\sphinxstylestrong{overwrite\_modelcol} \textendash{} Always write the model column, even if it already exists.

\item {} 
\sphinxAtStartPar
\sphinxstylestrong{statwtmode} \textendash{} Sets the weighting parameters for general VLA (‘VLA’) or VLASS Single Epoch (‘VLASS\sphinxhyphen{}SE’) use case. Note that the ‘VLASS\sphinxhyphen{}SE’
mode is meant to be used with datacolumn=’residual\_data’.
Default is ‘VLA’.

\end{itemize}

\sphinxlineitem{Returns}
\sphinxAtStartPar
The results object for the pipeline task is returned.

\end{description}\end{quote}
\subsubsection*{Examples}
\begin{enumerate}
\sphinxsetlistlabels{\arabic}{enumi}{enumii}{}{.}%
\item {} 
\sphinxAtStartPar
Statistical weighting of the visibilities:

\end{enumerate}

\begin{sphinxVerbatim}[commandchars=\\\{\}]
\PYG{g+gp}{\PYGZgt{}\PYGZgt{}\PYGZgt{} }\PYG{n}{hifv\PYGZus{}statwt}\PYG{p}{(}\PYG{p}{)}
\end{sphinxVerbatim}
\begin{enumerate}
\sphinxsetlistlabels{\arabic}{enumi}{enumii}{}{.}%
\setcounter{enumi}{1}
\item {} 
\sphinxAtStartPar
Statistical weighting of the visibilities in the Very Large Array Sky Survey Single Epoch use case:

\end{enumerate}

\begin{sphinxVerbatim}[commandchars=\\\{\}]
\PYG{g+gp}{\PYGZgt{}\PYGZgt{}\PYGZgt{} }\PYG{n}{hifv\PYGZus{}statwt}\PYG{p}{(}\PYG{n}{mode}\PYG{o}{=}\PYG{l+s+s1}{\PYGZsq{}}\PYG{l+s+s1}{vlass\PYGZhy{}se}\PYG{l+s+s1}{\PYGZsq{}}\PYG{p}{,} \PYG{n}{datacolumn}\PYG{o}{=}\PYG{l+s+s1}{\PYGZsq{}}\PYG{l+s+s1}{residual\PYGZus{}data}\PYG{l+s+s1}{\PYGZsq{}}\PYG{p}{)}
\end{sphinxVerbatim}

\end{fulllineitems}


\sphinxstepscope


\section{pipeline.hifv.cli.hifv\_swpowcal}
\label{\detokenize{_autosummary/pipeline.hifv.cli.hifv_swpowcal:pipeline-hifv-cli-hifv-swpowcal}}\label{\detokenize{_autosummary/pipeline.hifv.cli.hifv_swpowcal::doc}}\index{hifv\_swpowcal() (in module pipeline.hifv.cli)@\spxentry{hifv\_swpowcal()}\spxextra{in module pipeline.hifv.cli}}

\begin{fulllineitems}
\phantomsection\label{\detokenize{_autosummary/pipeline.hifv.cli.hifv_swpowcal:pipeline.hifv.cli.hifv_swpowcal}}
\pysigstartsignatures
\pysiglinewithargsret
{\sphinxbfcode{\sphinxupquote{hifv\_swpowcal}}}
{\sphinxparam{\DUrole{n}{vis}\DUrole{o}{=}\DUrole{default_value}{None}}\sphinxparamcomma \sphinxparam{\DUrole{n}{caltable}\DUrole{o}{=}\DUrole{default_value}{None}}\sphinxparamcomma \sphinxparam{\DUrole{n}{spw}\DUrole{o}{=}\DUrole{default_value}{None}}}
{}
\pysigstopsignatures
\sphinxAtStartPar
Runs gencal in swpow mode.
\begin{quote}\begin{description}
\sphinxlineitem{Parameters}\begin{itemize}
\item {} 
\sphinxAtStartPar
\sphinxstylestrong{vis} \textendash{} List of input visibility data.

\item {} 
\sphinxAtStartPar
\sphinxstylestrong{caltable} \textendash{} String name of caltable.

\item {} 
\sphinxAtStartPar
\sphinxstylestrong{spw} \textendash{} Spectral\sphinxhyphen{}window/frequency/channel: ‘’ ==\textgreater{} all, spw=”0:17\textasciitilde{}19”

\end{itemize}

\sphinxlineitem{Returns}
\sphinxAtStartPar
The results object for the pipeline task is returned.

\end{description}\end{quote}
\subsubsection*{Examples}
\begin{enumerate}
\sphinxsetlistlabels{\arabic}{enumi}{enumii}{}{.}%
\item {} 
\sphinxAtStartPar
Load an ASDM list in the ../rawdata subdirectory into the context:

\end{enumerate}

\begin{sphinxVerbatim}[commandchars=\\\{\}]
\PYG{g+gp}{\PYGZgt{}\PYGZgt{}\PYGZgt{} }\PYG{n}{hifv\PYGZus{}swpowcal}\PYG{p}{(}\PYG{p}{)}
\end{sphinxVerbatim}

\end{fulllineitems}


\sphinxstepscope


\section{pipeline.hifv.cli.hifv\_syspower}
\label{\detokenize{_autosummary/pipeline.hifv.cli.hifv_syspower:pipeline-hifv-cli-hifv-syspower}}\label{\detokenize{_autosummary/pipeline.hifv.cli.hifv_syspower::doc}}\index{hifv\_syspower() (in module pipeline.hifv.cli)@\spxentry{hifv\_syspower()}\spxextra{in module pipeline.hifv.cli}}

\begin{fulllineitems}
\phantomsection\label{\detokenize{_autosummary/pipeline.hifv.cli.hifv_syspower:pipeline.hifv.cli.hifv_syspower}}
\pysigstartsignatures
\pysiglinewithargsret
{\sphinxbfcode{\sphinxupquote{hifv\_syspower}}}
{\sphinxparam{\DUrole{n}{vis}\DUrole{o}{=}\DUrole{default_value}{None}}\sphinxparamcomma \sphinxparam{\DUrole{n}{clip\_sp\_template}\DUrole{o}{=}\DUrole{default_value}{None}}\sphinxparamcomma \sphinxparam{\DUrole{n}{antexclude}\DUrole{o}{=}\DUrole{default_value}{None}}\sphinxparamcomma \sphinxparam{\DUrole{n}{apply}\DUrole{o}{=}\DUrole{default_value}{None}}\sphinxparamcomma \sphinxparam{\DUrole{n}{do\_not\_apply}\DUrole{o}{=}\DUrole{default_value}{None}}}
{}
\pysigstopsignatures
\sphinxAtStartPar
Determine amount of gain compression affecting VLA data below Ku\sphinxhyphen{}band.
\begin{quote}\begin{description}
\sphinxlineitem{Parameters}\begin{itemize}
\item {} 
\sphinxAtStartPar
\sphinxstylestrong{vis} \textendash{} List of input visibility data.

\item {} 
\sphinxAtStartPar
\sphinxstylestrong{clip\_sp\_template} \textendash{} Acceptable range for Pdiff data; data are clipped outside this range and flagged.

\item {} 
\sphinxAtStartPar
\sphinxstylestrong{antexclude} \textendash{} dictionary in the format of:

\sphinxAtStartPar
\{‘L’: \{‘ea02’: \{‘usemedian’: True\}, ‘ea03’: \{‘usemedian’: False\}\},
‘X’: \{‘ea02’: \{‘usemedian’: True\}, ‘ea03’: \{‘usemedian’: False\}\},
‘S’: \{‘ea12’: \{‘usemedian’: False\}, ‘ea22’: \{‘usemedian’: False\}\}\}

\sphinxAtStartPar
If antexclude is specified with ‘usemedian’: False, the template values are replaced with 1.0.
If ‘usemedian’: True, the template values are replaced with the median of the good antennas.

\item {} 
\sphinxAtStartPar
\sphinxstylestrong{apply} \textendash{} Apply task results to RQ table.

\item {} 
\sphinxAtStartPar
\sphinxstylestrong{do\_not\_apply} \textendash{} csv string of band names to not apply. Example: ‘L,X,S’

\end{itemize}

\sphinxlineitem{Returns}
\sphinxAtStartPar
The results object for the pipeline task is returned.

\end{description}\end{quote}
\subsubsection*{Examples}
\begin{enumerate}
\sphinxsetlistlabels{\arabic}{enumi}{enumii}{}{.}%
\item {} 
\sphinxAtStartPar
Basic syspower task:

\end{enumerate}

\begin{sphinxVerbatim}[commandchars=\\\{\}]
\PYG{g+gp}{\PYGZgt{}\PYGZgt{}\PYGZgt{} }\PYG{n}{hifv\PYGZus{}syspower}\PYG{p}{(}\PYG{p}{)}
\end{sphinxVerbatim}

\end{fulllineitems}


\sphinxstepscope


\section{pipeline.hifv.cli.hifv\_targetflag}
\label{\detokenize{_autosummary/pipeline.hifv.cli.hifv_targetflag:pipeline-hifv-cli-hifv-targetflag}}\label{\detokenize{_autosummary/pipeline.hifv.cli.hifv_targetflag::doc}}\index{hifv\_targetflag() (in module pipeline.hifv.cli)@\spxentry{hifv\_targetflag()}\spxextra{in module pipeline.hifv.cli}}

\begin{fulllineitems}
\phantomsection\label{\detokenize{_autosummary/pipeline.hifv.cli.hifv_targetflag:pipeline.hifv.cli.hifv_targetflag}}
\pysigstartsignatures
\pysiglinewithargsret
{\sphinxbfcode{\sphinxupquote{hifv\_targetflag}}}
{\sphinxparam{\DUrole{n}{vis}\DUrole{o}{=}\DUrole{default_value}{None}}\sphinxparamcomma \sphinxparam{\DUrole{n}{intents}\DUrole{o}{=}\DUrole{default_value}{None}}}
{}
\pysigstopsignatures
\sphinxAtStartPar
Targetflag
\begin{quote}\begin{description}
\sphinxlineitem{Parameters}\begin{itemize}
\item {} 
\sphinxAtStartPar
\sphinxstylestrong{vis} \textendash{} The list of input MeasurementSets. Defaults to the list of MeasurementSets specified in the h\_init or hifv\_importdata task.

\item {} 
\sphinxAtStartPar
\sphinxstylestrong{intents} \textendash{} List of intents of scans to be flagged.

\end{itemize}

\sphinxlineitem{Returns}
\sphinxAtStartPar
The results object for the pipeline task is returned.

\end{description}\end{quote}
\subsubsection*{Examples}
\begin{enumerate}
\sphinxsetlistlabels{\arabic}{enumi}{enumii}{}{.}%
\item {} 
\sphinxAtStartPar
Run rflag on both the science targets and calibrators:

\end{enumerate}

\begin{sphinxVerbatim}[commandchars=\\\{\}]
\PYG{g+gp}{\PYGZgt{}\PYGZgt{}\PYGZgt{} }\PYG{n}{hifv\PYGZus{}targetflag}\PYG{p}{(}\PYG{p}{)}
\end{sphinxVerbatim}

\end{fulllineitems}


\sphinxstepscope


\section{pipeline.hifv.cli.hifv\_tecmaps}
\label{\detokenize{_autosummary/pipeline.hifv.cli.hifv_tecmaps:pipeline-hifv-cli-hifv-tecmaps}}\label{\detokenize{_autosummary/pipeline.hifv.cli.hifv_tecmaps::doc}}\index{hifv\_tecmaps() (in module pipeline.hifv.cli)@\spxentry{hifv\_tecmaps()}\spxextra{in module pipeline.hifv.cli}}

\begin{fulllineitems}
\phantomsection\label{\detokenize{_autosummary/pipeline.hifv.cli.hifv_tecmaps:pipeline.hifv.cli.hifv_tecmaps}}
\pysigstartsignatures
\pysiglinewithargsret
{\sphinxbfcode{\sphinxupquote{hifv\_tecmaps}}}
{\sphinxparam{\DUrole{n}{vis}\DUrole{o}{=}\DUrole{default_value}{None}}}
{}
\pysigstopsignatures
\sphinxAtStartPar
Base tecmaps task
\begin{quote}\begin{description}
\sphinxlineitem{Parameters}
\sphinxAtStartPar
\sphinxstylestrong{vis} \textendash{} The list of input MeasurementSets. Defaults to the list of MeasurementSets specified in the h\_init or hifv\_importdata task.

\sphinxlineitem{Returns}
\sphinxAtStartPar
The results object for the pipeline task is returned.

\end{description}\end{quote}
\subsubsection*{Examples}
\begin{enumerate}
\sphinxsetlistlabels{\arabic}{enumi}{enumii}{}{.}%
\item {} 
\sphinxAtStartPar
Basic tecmaps task:

\end{enumerate}

\begin{sphinxVerbatim}[commandchars=\\\{\}]
\PYG{g+gp}{\PYGZgt{}\PYGZgt{}\PYGZgt{} }\PYG{n}{hifv\PYGZus{}tecmaps}\PYG{p}{(}\PYG{p}{)}
\end{sphinxVerbatim}

\end{fulllineitems}


\sphinxstepscope


\section{pipeline.hifv.cli.hifv\_testBPdcals}
\label{\detokenize{_autosummary/pipeline.hifv.cli.hifv_testBPdcals:pipeline-hifv-cli-hifv-testbpdcals}}\label{\detokenize{_autosummary/pipeline.hifv.cli.hifv_testBPdcals::doc}}\index{hifv\_testBPdcals() (in module pipeline.hifv.cli)@\spxentry{hifv\_testBPdcals()}\spxextra{in module pipeline.hifv.cli}}

\begin{fulllineitems}
\phantomsection\label{\detokenize{_autosummary/pipeline.hifv.cli.hifv_testBPdcals:pipeline.hifv.cli.hifv_testBPdcals}}
\pysigstartsignatures
\pysiglinewithargsret
{\sphinxbfcode{\sphinxupquote{hifv\_testBPdcals}}}
{\sphinxparam{\DUrole{n}{vis}\DUrole{o}{=}\DUrole{default_value}{None}}\sphinxparamcomma \sphinxparam{\DUrole{n}{weakbp}\DUrole{o}{=}\DUrole{default_value}{None}}\sphinxparamcomma \sphinxparam{\DUrole{n}{refantignore}\DUrole{o}{=}\DUrole{default_value}{None}}\sphinxparamcomma \sphinxparam{\DUrole{n}{doflagundernspwlimit}\DUrole{o}{=}\DUrole{default_value}{None}}\sphinxparamcomma \sphinxparam{\DUrole{n}{refant}\DUrole{o}{=}\DUrole{default_value}{None}}}
{}
\pysigstopsignatures
\sphinxAtStartPar
Runs initial delay and bandpass calibration to setup for RFI flagging.
\begin{quote}\begin{description}
\sphinxlineitem{Parameters}\begin{itemize}
\item {} 
\sphinxAtStartPar
\sphinxstylestrong{vis} \textendash{} The list of input MeasurementSets. Defaults to the list of MeasurementSets specified in the h\_init or hifv\_importdata task.

\item {} 
\sphinxAtStartPar
\sphinxstylestrong{weakbp} \textendash{} Activate weak bandpass heuristics.

\item {} 
\sphinxAtStartPar
\sphinxstylestrong{refantignore} \textendash{} String list of antennas to ignore Example:  refantignore=’ea02, ea03’

\item {} 
\sphinxAtStartPar
\sphinxstylestrong{doflagundernspwlimit} \textendash{} If the number of bad spws is greater than zero, and the keyword is True, then spws are flagged individually.

\item {} 
\sphinxAtStartPar
\sphinxstylestrong{refant} \textendash{} A csv string of reference antenna(s). When used, disables \sphinxcode{\sphinxupquote{refantignore}}. Example: refant = ‘ea01, ea02’

\end{itemize}

\sphinxlineitem{Returns}
\sphinxAtStartPar
The results object for the pipeline task is returned.

\end{description}\end{quote}
\subsubsection*{Examples}
\begin{enumerate}
\sphinxsetlistlabels{\arabic}{enumi}{enumii}{}{.}%
\item {} 
\sphinxAtStartPar
Initial delay calibration to set up heuristic flagging:

\end{enumerate}

\begin{sphinxVerbatim}[commandchars=\\\{\}]
\PYG{g+gp}{\PYGZgt{}\PYGZgt{}\PYGZgt{} }\PYG{n}{hifv\PYGZus{}testBPdcals}\PYG{p}{(}\PYG{p}{)}
\end{sphinxVerbatim}

\end{fulllineitems}


\sphinxstepscope


\section{pipeline.hifv.cli.hifv\_vlasetjy}
\label{\detokenize{_autosummary/pipeline.hifv.cli.hifv_vlasetjy:pipeline-hifv-cli-hifv-vlasetjy}}\label{\detokenize{_autosummary/pipeline.hifv.cli.hifv_vlasetjy::doc}}\index{hifv\_vlasetjy() (in module pipeline.hifv.cli)@\spxentry{hifv\_vlasetjy()}\spxextra{in module pipeline.hifv.cli}}

\begin{fulllineitems}
\phantomsection\label{\detokenize{_autosummary/pipeline.hifv.cli.hifv_vlasetjy:pipeline.hifv.cli.hifv_vlasetjy}}
\pysigstartsignatures
\pysiglinewithargsret
{\sphinxbfcode{\sphinxupquote{hifv\_vlasetjy}}}
{\sphinxparam{\DUrole{n}{vis}\DUrole{o}{=}\DUrole{default_value}{None}}\sphinxparamcomma \sphinxparam{\DUrole{n}{field}\DUrole{o}{=}\DUrole{default_value}{None}}\sphinxparamcomma \sphinxparam{\DUrole{n}{intent}\DUrole{o}{=}\DUrole{default_value}{None}}\sphinxparamcomma \sphinxparam{\DUrole{n}{spw}\DUrole{o}{=}\DUrole{default_value}{None}}\sphinxparamcomma \sphinxparam{\DUrole{n}{model}\DUrole{o}{=}\DUrole{default_value}{None}}\sphinxparamcomma \sphinxparam{\DUrole{n}{reffile}\DUrole{o}{=}\DUrole{default_value}{None}}\sphinxparamcomma \sphinxparam{\DUrole{n}{fluxdensity}\DUrole{o}{=}\DUrole{default_value}{None}}\sphinxparamcomma \sphinxparam{\DUrole{n}{spix}\DUrole{o}{=}\DUrole{default_value}{None}}\sphinxparamcomma \sphinxparam{\DUrole{n}{reffreq}\DUrole{o}{=}\DUrole{default_value}{None}}\sphinxparamcomma \sphinxparam{\DUrole{n}{scalebychan}\DUrole{o}{=}\DUrole{default_value}{None}}\sphinxparamcomma \sphinxparam{\DUrole{n}{standard}\DUrole{o}{=}\DUrole{default_value}{None}}}
{}
\pysigstopsignatures
\sphinxAtStartPar
Sets flux density scale and fills calibrator model to measurement set.

\sphinxAtStartPar
The hifv\_vlasetjy task does an initial run of setjy on the vis.
\begin{quote}\begin{description}
\sphinxlineitem{Parameters}\begin{itemize}
\item {} 
\sphinxAtStartPar
\sphinxstylestrong{vis} \textendash{} The list of input MeasurementSets. Defaults to the list of MeasurementSets specified in the h\_init or hifv\_importdata task.

\item {} 
\sphinxAtStartPar
\sphinxstylestrong{field} \textendash{} List of field names or ids.

\item {} 
\sphinxAtStartPar
\sphinxstylestrong{intent} \textendash{} Observing intent of flux calibrators.

\item {} 
\sphinxAtStartPar
\sphinxstylestrong{spw} \textendash{} List of spectral window ids.

\item {} 
\sphinxAtStartPar
\sphinxstylestrong{model} \textendash{} File location for field model.

\item {} 
\sphinxAtStartPar
\sphinxstylestrong{reffile} \textendash{} Path to file with fluxes for non\sphinxhyphen{}solar system calibrators.

\item {} 
\sphinxAtStartPar
\sphinxstylestrong{fluxdensity} \textendash{} Specified flux density {[}I,Q,U,V{]}; \sphinxhyphen{}1 will lookup values.

\item {} 
\sphinxAtStartPar
\sphinxstylestrong{spix} \textendash{} Spectral index of fluxdensity.  Can be set when fluxdensity is not \sphinxhyphen{}1.

\item {} 
\sphinxAtStartPar
\sphinxstylestrong{reffreq} \textendash{} Reference frequency for spix.  Can be set when fluxdensity is not \sphinxhyphen{}1.

\item {} 
\sphinxAtStartPar
\sphinxstylestrong{scalebychan} \textendash{} Scale the flux density on a per channel basis or else on a per spw basis.

\item {} 
\sphinxAtStartPar
\sphinxstylestrong{standard} \textendash{} Flux density standard.

\end{itemize}

\sphinxlineitem{Returns}
\sphinxAtStartPar
The results object for the pipeline task is returned.

\end{description}\end{quote}
\subsubsection*{Examples}
\begin{enumerate}
\sphinxsetlistlabels{\arabic}{enumi}{enumii}{}{.}%
\item {} 
\sphinxAtStartPar
Initial run of setjy:

\end{enumerate}

\begin{sphinxVerbatim}[commandchars=\\\{\}]
\PYG{g+gp}{\PYGZgt{}\PYGZgt{}\PYGZgt{} }\PYG{n}{hifv\PYGZus{}vlasetjy}\PYG{p}{(}\PYG{p}{)}
\end{sphinxVerbatim}

\end{fulllineitems}


\sphinxstepscope


\section{pipeline.hifv.cli.hifv\_vlassmasking}
\label{\detokenize{_autosummary/pipeline.hifv.cli.hifv_vlassmasking:pipeline-hifv-cli-hifv-vlassmasking}}\label{\detokenize{_autosummary/pipeline.hifv.cli.hifv_vlassmasking::doc}}\index{hifv\_vlassmasking() (in module pipeline.hifv.cli)@\spxentry{hifv\_vlassmasking()}\spxextra{in module pipeline.hifv.cli}}

\begin{fulllineitems}
\phantomsection\label{\detokenize{_autosummary/pipeline.hifv.cli.hifv_vlassmasking:pipeline.hifv.cli.hifv_vlassmasking}}
\pysigstartsignatures
\pysiglinewithargsret
{\sphinxbfcode{\sphinxupquote{hifv\_vlassmasking}}}
{\sphinxparam{\DUrole{n}{vis}\DUrole{o}{=}\DUrole{default_value}{None}}\sphinxparamcomma \sphinxparam{\DUrole{n}{vlass\_ql\_database}\DUrole{o}{=}\DUrole{default_value}{None}}\sphinxparamcomma \sphinxparam{\DUrole{n}{maskingmode}\DUrole{o}{=}\DUrole{default_value}{None}}\sphinxparamcomma \sphinxparam{\DUrole{n}{catalog\_search\_size}\DUrole{o}{=}\DUrole{default_value}{None}}}
{}
\pysigstopsignatures
\sphinxAtStartPar
Create clean masks for VLASS Single Epoch (SE) images.
\begin{quote}\begin{description}
\sphinxlineitem{Parameters}\begin{itemize}
\item {} 
\sphinxAtStartPar
\sphinxstylestrong{vis} \textendash{} The list of input MeasurementSets. Defaults to the list of MeasurementSets specified in the h\_init or hifv\_importdata task.

\item {} 
\sphinxAtStartPar
\sphinxstylestrong{vlass\_ql\_database} \textendash{} vlass\_ql\_database \sphinxhyphen{} usage in Socorro: /home/vlass/packages/VLASS1Q.fits

\item {} 
\sphinxAtStartPar
\sphinxstylestrong{maskingmode} \textendash{} maskingmode options are vlass\sphinxhyphen{}se\sphinxhyphen{}tier\sphinxhyphen{}1 or vlass\sphinxhyphen{}se\sphinxhyphen{}tier\sphinxhyphen{}2.

\item {} 
\sphinxAtStartPar
\sphinxstylestrong{catalog\_search\_size} \textendash{} catalog\_search\_size in units of degrees

\end{itemize}

\sphinxlineitem{Returns}
\sphinxAtStartPar
The results object for the pipeline task is returned.

\end{description}\end{quote}
\subsubsection*{Examples}
\begin{enumerate}
\sphinxsetlistlabels{\arabic}{enumi}{enumii}{}{.}%
\item {} 
\sphinxAtStartPar
Basic vlassmasking task:

\end{enumerate}

\begin{sphinxVerbatim}[commandchars=\\\{\}]
\PYG{g+gp}{\PYGZgt{}\PYGZgt{}\PYGZgt{} }\PYG{n}{hifv\PYGZus{}vlassmasking}\PYG{p}{(}\PYG{p}{)}
\end{sphinxVerbatim}

\end{fulllineitems}


\sphinxstepscope


\chapter{pipeline.hsd.cli}
\label{\detokenize{_autosummary/pipeline.hsd.cli:module-pipeline.hsd.cli}}\label{\detokenize{_autosummary/pipeline.hsd.cli:pipeline-hsd-cli}}\label{\detokenize{_autosummary/pipeline.hsd.cli::doc}}\index{module@\spxentry{module}!pipeline.hsd.cli@\spxentry{pipeline.hsd.cli}}\index{pipeline.hsd.cli@\spxentry{pipeline.hsd.cli}!module@\spxentry{module}}
\sphinxAtStartPar
ALMA Single Dish Tasks
\subsubsection*{Functions}


\begin{savenotes}\sphinxattablestart
\sphinxthistablewithglobalstyle
\sphinxthistablewithnovlinesstyle
\centering
\begin{tabulary}{\linewidth}[t]{\X{1}{2}\X{1}{2}}
\sphinxtoprule
\sphinxtableatstartofbodyhook
\sphinxAtStartPar
{\hyperref[\detokenize{_autosummary/pipeline.hsd.cli.hsd_applycal:pipeline.hsd.cli.hsd_applycal}]{\sphinxcrossref{\sphinxcode{\sphinxupquote{hsd\_applycal}}}}}
&
\sphinxAtStartPar
Apply the calibration(s) to the data.
\\
\sphinxhline
\sphinxAtStartPar
{\hyperref[\detokenize{_autosummary/pipeline.hsd.cli.hsd_atmcor:pipeline.hsd.cli.hsd_atmcor}]{\sphinxcrossref{\sphinxcode{\sphinxupquote{hsd\_atmcor}}}}}
&
\sphinxAtStartPar
Apply offline ATM correction to the data.
\\
\sphinxhline
\sphinxAtStartPar
{\hyperref[\detokenize{_autosummary/pipeline.hsd.cli.hsd_baseline:pipeline.hsd.cli.hsd_baseline}]{\sphinxcrossref{\sphinxcode{\sphinxupquote{hsd\_baseline}}}}}
&
\sphinxAtStartPar
Detect and validate spectral lines, subtract baseline by masking detected lines.
\\
\sphinxhline
\sphinxAtStartPar
{\hyperref[\detokenize{_autosummary/pipeline.hsd.cli.hsd_blflag:pipeline.hsd.cli.hsd_blflag}]{\sphinxcrossref{\sphinxcode{\sphinxupquote{hsd\_blflag}}}}}
&
\sphinxAtStartPar
Flag spectra based on predefined criteria of single dish pipeline.
\\
\sphinxhline
\sphinxAtStartPar
{\hyperref[\detokenize{_autosummary/pipeline.hsd.cli.hsd_exportdata:pipeline.hsd.cli.hsd_exportdata}]{\sphinxcrossref{\sphinxcode{\sphinxupquote{hsd\_exportdata}}}}}
&
\sphinxAtStartPar
Prepare single dish data for export.
\\
\sphinxhline
\sphinxAtStartPar
{\hyperref[\detokenize{_autosummary/pipeline.hsd.cli.hsd_flagdata:pipeline.hsd.cli.hsd_flagdata}]{\sphinxcrossref{\sphinxcode{\sphinxupquote{hsd\_flagdata}}}}}
&
\sphinxAtStartPar
Do basic flagging of a list of MeasurementSets.
\\
\sphinxhline
\sphinxAtStartPar
{\hyperref[\detokenize{_autosummary/pipeline.hsd.cli.hsd_imaging:pipeline.hsd.cli.hsd_imaging}]{\sphinxcrossref{\sphinxcode{\sphinxupquote{hsd\_imaging}}}}}
&
\sphinxAtStartPar
Generate single dish images.
\\
\sphinxhline
\sphinxAtStartPar
{\hyperref[\detokenize{_autosummary/pipeline.hsd.cli.hsd_importdata:pipeline.hsd.cli.hsd_importdata}]{\sphinxcrossref{\sphinxcode{\sphinxupquote{hsd\_importdata}}}}}
&
\sphinxAtStartPar
Imports data into the single dish pipeline.
\\
\sphinxhline
\sphinxAtStartPar
{\hyperref[\detokenize{_autosummary/pipeline.hsd.cli.hsd_k2jycal:pipeline.hsd.cli.hsd_k2jycal}]{\sphinxcrossref{\sphinxcode{\sphinxupquote{hsd\_k2jycal}}}}}
&
\sphinxAtStartPar
Derive Kelvin to Jy calibration tables.
\\
\sphinxhline
\sphinxAtStartPar
{\hyperref[\detokenize{_autosummary/pipeline.hsd.cli.hsd_restoredata:pipeline.hsd.cli.hsd_restoredata}]{\sphinxcrossref{\sphinxcode{\sphinxupquote{hsd\_restoredata}}}}}
&
\sphinxAtStartPar
Restore flagged and calibration single dish data from a pipeline run.
\\
\sphinxhline
\sphinxAtStartPar
{\hyperref[\detokenize{_autosummary/pipeline.hsd.cli.hsd_skycal:pipeline.hsd.cli.hsd_skycal}]{\sphinxcrossref{\sphinxcode{\sphinxupquote{hsd\_skycal}}}}}
&
\sphinxAtStartPar
Calibrate data.
\\
\sphinxhline
\sphinxAtStartPar
{\hyperref[\detokenize{_autosummary/pipeline.hsd.cli.hsd_tsysflag:pipeline.hsd.cli.hsd_tsysflag}]{\sphinxcrossref{\sphinxcode{\sphinxupquote{hsd\_tsysflag}}}}}
&
\sphinxAtStartPar
Flag deviant system temperature measurements.
\\
\sphinxbottomrule
\end{tabulary}
\sphinxtableafterendhook\par
\sphinxattableend\end{savenotes}

\sphinxstepscope


\section{pipeline.hsd.cli.hsd\_applycal}
\label{\detokenize{_autosummary/pipeline.hsd.cli.hsd_applycal:pipeline-hsd-cli-hsd-applycal}}\label{\detokenize{_autosummary/pipeline.hsd.cli.hsd_applycal::doc}}\index{hsd\_applycal() (in module pipeline.hsd.cli)@\spxentry{hsd\_applycal()}\spxextra{in module pipeline.hsd.cli}}

\begin{fulllineitems}
\phantomsection\label{\detokenize{_autosummary/pipeline.hsd.cli.hsd_applycal:pipeline.hsd.cli.hsd_applycal}}
\pysigstartsignatures
\pysiglinewithargsret
{\sphinxbfcode{\sphinxupquote{hsd\_applycal}}}
{\sphinxparam{\DUrole{n}{vis}\DUrole{o}{=}\DUrole{default_value}{None}}\sphinxparamcomma \sphinxparam{\DUrole{n}{field}\DUrole{o}{=}\DUrole{default_value}{None}}\sphinxparamcomma \sphinxparam{\DUrole{n}{intent}\DUrole{o}{=}\DUrole{default_value}{None}}\sphinxparamcomma \sphinxparam{\DUrole{n}{spw}\DUrole{o}{=}\DUrole{default_value}{None}}\sphinxparamcomma \sphinxparam{\DUrole{n}{antenna}\DUrole{o}{=}\DUrole{default_value}{None}}\sphinxparamcomma \sphinxparam{\DUrole{n}{applymode}\DUrole{o}{=}\DUrole{default_value}{None}}\sphinxparamcomma \sphinxparam{\DUrole{n}{calwt}\DUrole{o}{=}\DUrole{default_value}{None}}\sphinxparamcomma \sphinxparam{\DUrole{n}{flagbackup}\DUrole{o}{=}\DUrole{default_value}{None}}\sphinxparamcomma \sphinxparam{\DUrole{n}{parallel}\DUrole{o}{=}\DUrole{default_value}{None}}}
{}
\pysigstopsignatures
\sphinxAtStartPar
Apply the calibration(s) to the data.

\sphinxAtStartPar
hsd\_applycal applies the precomputed calibration tables stored in the pipeline
context to the set of visibility files using predetermined field and
spectral window maps and default values for the interpolation schemes.

\sphinxAtStartPar
Users can interact with the pipeline calibration state using the tasks
h\_export\_calstate and h\_import\_calstate.

\begin{sphinxadmonition}{note}{Note:}
\sphinxAtStartPar
There is some discussion about the appropriate values of calwt. Given
properly scaled data, the correct value should be the CASA default of True.
However at the current time ALMA is suggesting that calwt be set to True for
applying observatory calibrations, e.g. antenna positions, WVR, and system
temperature corrections, and to False for applying instrument calibrations,
e.g. bandpass, gain, and flux.
\end{sphinxadmonition}
\begin{quote}\begin{description}
\sphinxlineitem{Parameters}\begin{itemize}
\item {} 
\sphinxAtStartPar
\sphinxstylestrong{vis} \textendash{} The list of input MeasurementSets. Defaults to the list of MeasurementSets in the pipeline context.
example: {[}‘X227.ms’{]}

\item {} 
\sphinxAtStartPar
\sphinxstylestrong{field} \textendash{} A string containing the list of field names or field ids to which the calibration will be applied. Defaults to all fields in the pipeline context.
example: ‘3C279’, ‘3C279, M82’

\item {} 
\sphinxAtStartPar
\sphinxstylestrong{intent} \textendash{} A string containing the list of intents against which the selected fields will be matched. Defaults to all supported intents
in the pipeline context.
example: \sphinxtitleref{‘*TARGET*’}

\item {} 
\sphinxAtStartPar
\sphinxstylestrong{spw} \textendash{} The list of spectral windows and channels to which the calibration will be applied. Defaults to all science windows in the pipeline context.
example: ‘17’, ‘11, 15’

\item {} 
\sphinxAtStartPar
\sphinxstylestrong{antenna} \textendash{} The selection of antennas to which the calibration will be applied. Defaults to all antennas. Not currently supported.

\item {} 
\sphinxAtStartPar
\sphinxstylestrong{applymode} \textendash{} Calibration apply mode.
\begin{itemize}
\item {} 
\sphinxAtStartPar
‘calflag’: calibrate data and apply flags from solutions.

\item {} 
\sphinxAtStartPar
‘calflagstrict’: same as above except flag spws for which calibration is
unavailable in one or more tables (instead of allowing them to pass
uncalibrated and unflagged).

\item {} 
\sphinxAtStartPar
‘trial’: report on flags from solutions, dataset entirely unchanged.

\item {} 
\sphinxAtStartPar
‘flagonly’: apply flags from solutions only, data not calibrated.

\item {} 
\sphinxAtStartPar
‘flagonlystrict’: same as above except flag spws for which calibration is
unavailable in one or more tables.

\item {} 
\sphinxAtStartPar
‘calonly’: calibrate data only, flags from solutions NOT applied.

\end{itemize}

\item {} 
\sphinxAtStartPar
\sphinxstylestrong{calwt} \textendash{} Calibrate the weights as well as the data.

\item {} 
\sphinxAtStartPar
\sphinxstylestrong{flagbackup} \textendash{} Backup the flags before the apply.

\item {} 
\sphinxAtStartPar
\sphinxstylestrong{parallel} \textendash{} Execute using CASA HPC functionality, if available. Options: ‘automatic’, ‘true’, ‘false’, True, False
default: None (equivalent to ‘automatic’)

\end{itemize}

\sphinxlineitem{Returns}
\sphinxAtStartPar
The results object for the pipeline task is returned.

\end{description}\end{quote}
\subsubsection*{Examples}
\begin{enumerate}
\sphinxsetlistlabels{\arabic}{enumi}{enumii}{}{.}%
\item {} 
\sphinxAtStartPar
Apply the calibration to the target data

\end{enumerate}

\begin{sphinxVerbatim}[commandchars=\\\{\}]
\PYG{g+gp}{\PYGZgt{}\PYGZgt{}\PYGZgt{} }\PYG{n}{hsd\PYGZus{}applycal} \PYG{p}{(}\PYG{n}{intent}\PYG{o}{=}\PYG{l+s+s1}{\PYGZsq{}}\PYG{l+s+s1}{TARGET}\PYG{l+s+s1}{\PYGZsq{}}\PYG{p}{)}
\end{sphinxVerbatim}
\begin{enumerate}
\sphinxsetlistlabels{\arabic}{enumi}{enumii}{}{.}%
\setcounter{enumi}{1}
\item {} 
\sphinxAtStartPar
Specify fields and spectral windows

\end{enumerate}

\begin{sphinxVerbatim}[commandchars=\\\{\}]
\PYG{g+gp}{\PYGZgt{}\PYGZgt{}\PYGZgt{} }\PYG{n}{hsd\PYGZus{}applycal}\PYG{p}{(}\PYG{n}{field}\PYG{o}{=}\PYG{l+s+s1}{\PYGZsq{}}\PYG{l+s+s1}{3C279, M82}\PYG{l+s+s1}{\PYGZsq{}}\PYG{p}{,} \PYG{n}{spw}\PYG{o}{=}\PYG{l+s+s1}{\PYGZsq{}}\PYG{l+s+s1}{17}\PYG{l+s+s1}{\PYGZsq{}}\PYG{p}{,} \PYG{n}{intent}\PYG{o}{=}\PYG{l+s+s1}{\PYGZsq{}}\PYG{l+s+s1}{TARGET}\PYG{l+s+s1}{\PYGZsq{}}\PYG{p}{)}
\end{sphinxVerbatim}

\end{fulllineitems}


\sphinxstepscope


\section{pipeline.hsd.cli.hsd\_atmcor}
\label{\detokenize{_autosummary/pipeline.hsd.cli.hsd_atmcor:pipeline-hsd-cli-hsd-atmcor}}\label{\detokenize{_autosummary/pipeline.hsd.cli.hsd_atmcor::doc}}\index{hsd\_atmcor() (in module pipeline.hsd.cli)@\spxentry{hsd\_atmcor()}\spxextra{in module pipeline.hsd.cli}}

\begin{fulllineitems}
\phantomsection\label{\detokenize{_autosummary/pipeline.hsd.cli.hsd_atmcor:pipeline.hsd.cli.hsd_atmcor}}
\pysigstartsignatures
\pysiglinewithargsret
{\sphinxbfcode{\sphinxupquote{hsd\_atmcor}}}
{\sphinxparam{\DUrole{n}{atmtype}\DUrole{o}{=}\DUrole{default_value}{None}}\sphinxparamcomma \sphinxparam{\DUrole{n}{dtem\_dh}\DUrole{o}{=}\DUrole{default_value}{None}}\sphinxparamcomma \sphinxparam{\DUrole{n}{h0}\DUrole{o}{=}\DUrole{default_value}{None}}\sphinxparamcomma \sphinxparam{\DUrole{n}{infiles}\DUrole{o}{=}\DUrole{default_value}{None}}\sphinxparamcomma \sphinxparam{\DUrole{n}{antenna}\DUrole{o}{=}\DUrole{default_value}{None}}\sphinxparamcomma \sphinxparam{\DUrole{n}{field}\DUrole{o}{=}\DUrole{default_value}{None}}\sphinxparamcomma \sphinxparam{\DUrole{n}{spw}\DUrole{o}{=}\DUrole{default_value}{None}}\sphinxparamcomma \sphinxparam{\DUrole{n}{pol}\DUrole{o}{=}\DUrole{default_value}{None}}}
{}
\pysigstopsignatures
\sphinxAtStartPar
Apply offline ATM correction to the data.

\sphinxAtStartPar
The hsd\_atmcor task provides the capability of offline correction of
residual atmospheric features in the calibrated single\sphinxhyphen{}dish spectra
originated from incomplete calibration mainly due to a difference of
elevation angles between ON\_SOURCE and OFF\_SOURCE measurements.

\sphinxAtStartPar
Optimal atmospheric model is automatically determined by default
(atmtype = ‘auto’). You may specify desired atmospheric model by giving
either single integer (apply to all EBs) or a list of integers (models
per EB) to atmtype parameter. Please see parameter description for the
meanings of integer values.
\begin{quote}\begin{description}
\sphinxlineitem{Parameters}\begin{itemize}
\item {} 
\sphinxAtStartPar
\sphinxstylestrong{atmtype} \textendash{} Type of atmospheric transmission model represented as an integer. Available options are as follows. Integer values can be given as
either integer or string, i.e. both 1 and ‘1’ are acceptable.
\begin{itemize}
\item {} 
\sphinxAtStartPar
‘auto’: perform heuristics to choose best model (default).

\item {} 
\sphinxAtStartPar
1: tropical.

\item {} 
\sphinxAtStartPar
2: mid latitude summer.

\item {} 
\sphinxAtStartPar
3: mid latitude winter.

\item {} 
\sphinxAtStartPar
4: subarctic summer.

\item {} 
\sphinxAtStartPar
5: subarctic winter.

\end{itemize}

\sphinxAtStartPar
If list of integer is given, it also performs heuristics using the
provided values instead of default, {[}1, 2, 3, 4{]}, which is used
when ‘auto’ is provided. List input should not contain ‘auto’.

\sphinxAtStartPar
Default: ‘auto’

\item {} 
\sphinxAtStartPar
\sphinxstylestrong{dtem\_dh} \textendash{} Temperature gradient {[}K/km{]}, e.g. \sphinxhyphen{}5.6. (”” = Tool default) The value is directly passed to initialization method for ATM model.
Float and string types are acceptable. Float value is interpreted as
the value in K/km. String value should be the numeric value with unit
such as ‘\sphinxhyphen{}5.6K/km’. When list of values are given, it will
trigger heuristics to choose best model from the provided value.
Default: ‘’ (tool default, \sphinxhyphen{}5.6K/km, is used)

\item {} 
\sphinxAtStartPar
\sphinxstylestrong{h0} \textendash{} Scale height for water {[}km{]}, e.g. 2.0. (”” = Tool default) The value is directly passed to initialization method for ATM model.
Float and string types are acceptable. Float value is interpreted as
the value in kilometer. String value should be the numeric value with
unit compatible with length, such as ‘2km’ or ‘2000m’.
When list of values are given, it will trigger heuristics to
choose best model from the provided value.
Default: ‘’ (tool default, 2.0km, is used)

\item {} 
\sphinxAtStartPar
\sphinxstylestrong{infiles} \textendash{} ASDM or MS files to be processed. This parameter behaves as data selection parameter. The name specified by
infiles must be registered to context before you run
hsd\_atmcor.

\item {} 
\sphinxAtStartPar
\sphinxstylestrong{antenna} \textendash{} Data selection by antenna names or ids. example: ‘PM03,PM04’,
‘’ (all antennas)

\item {} 
\sphinxAtStartPar
\sphinxstylestrong{field} \textendash{} Data selection by field names or ids. example: ‘\sphinxtitleref{*Sgr*,M100}’,
‘’ (all fields)

\item {} 
\sphinxAtStartPar
\sphinxstylestrong{spw} \textendash{} Data selection by spw ids. example: ‘3,4’ (spw 3 and 4),
‘’ (all spws)

\item {} 
\sphinxAtStartPar
\sphinxstylestrong{pol} \textendash{} Data selection by polarizations. example: ‘XX,YY’ (correlation XX and YY),
‘’ (all polarizations)

\end{itemize}

\sphinxlineitem{Returns}
\sphinxAtStartPar
The results object for the pipeline task is returned.

\end{description}\end{quote}
\subsubsection*{Examples}
\begin{enumerate}
\sphinxsetlistlabels{\arabic}{enumi}{enumii}{}{.}%
\item {} 
\sphinxAtStartPar
Basic usage

\end{enumerate}

\begin{sphinxVerbatim}[commandchars=\\\{\}]
\PYG{g+gp}{\PYGZgt{}\PYGZgt{}\PYGZgt{} }\PYG{n}{hsd\PYGZus{}atmcor}\PYG{p}{(}\PYG{p}{)}
\end{sphinxVerbatim}
\begin{enumerate}
\sphinxsetlistlabels{\arabic}{enumi}{enumii}{}{.}%
\setcounter{enumi}{1}
\item {} 
\sphinxAtStartPar
Specify atmospheric model and data selection

\end{enumerate}

\begin{sphinxVerbatim}[commandchars=\\\{\}]
\PYG{g+gp}{\PYGZgt{}\PYGZgt{}\PYGZgt{} }\PYG{n}{hsd\PYGZus{}atmcor}\PYG{p}{(}\PYG{n}{atmtype}\PYG{o}{=}\PYG{l+m+mi}{1}\PYG{p}{,} \PYG{n}{antenna}\PYG{o}{=}\PYG{l+s+s1}{\PYGZsq{}}\PYG{l+s+s1}{PM03,PM04}\PYG{l+s+s1}{\PYGZsq{}}\PYG{p}{,} \PYG{n}{field}\PYG{o}{=}\PYG{l+s+s1}{\PYGZsq{}}\PYG{l+s+s1}{*Sgr*,M100}\PYG{l+s+s1}{\PYGZsq{}}\PYG{p}{)}
\end{sphinxVerbatim}
\begin{enumerate}
\sphinxsetlistlabels{\arabic}{enumi}{enumii}{}{.}%
\setcounter{enumi}{2}
\item {} 
\sphinxAtStartPar
Specify atmospheric model per EB (atmtype 1 for 1st EB, 2 for 2nd EB)

\end{enumerate}

\begin{sphinxVerbatim}[commandchars=\\\{\}]
\PYG{g+gp}{\PYGZgt{}\PYGZgt{}\PYGZgt{} }\PYG{n}{hsd\PYGZus{}atmcor}\PYG{p}{(}\PYG{n}{atmtype}\PYG{o}{=}\PYG{p}{[}\PYG{l+m+mi}{1}\PYG{p}{,} \PYG{l+m+mi}{2}\PYG{p}{]}\PYG{p}{)}
\end{sphinxVerbatim}

\end{fulllineitems}


\sphinxstepscope


\section{pipeline.hsd.cli.hsd\_baseline}
\label{\detokenize{_autosummary/pipeline.hsd.cli.hsd_baseline:pipeline-hsd-cli-hsd-baseline}}\label{\detokenize{_autosummary/pipeline.hsd.cli.hsd_baseline::doc}}\index{hsd\_baseline() (in module pipeline.hsd.cli)@\spxentry{hsd\_baseline()}\spxextra{in module pipeline.hsd.cli}}

\begin{fulllineitems}
\phantomsection\label{\detokenize{_autosummary/pipeline.hsd.cli.hsd_baseline:pipeline.hsd.cli.hsd_baseline}}
\pysigstartsignatures
\pysiglinewithargsret
{\sphinxbfcode{\sphinxupquote{hsd\_baseline}}}
{\sphinxparam{\DUrole{n}{fitfunc}\DUrole{o}{=}\DUrole{default_value}{None}}\sphinxparamcomma \sphinxparam{\DUrole{n}{fitorder}\DUrole{o}{=}\DUrole{default_value}{None}}\sphinxparamcomma \sphinxparam{\DUrole{n}{switchpoly}\DUrole{o}{=}\DUrole{default_value}{None}}\sphinxparamcomma \sphinxparam{\DUrole{n}{linewindow}\DUrole{o}{=}\DUrole{default_value}{None}}\sphinxparamcomma \sphinxparam{\DUrole{n}{linewindowmode}\DUrole{o}{=}\DUrole{default_value}{None}}\sphinxparamcomma \sphinxparam{\DUrole{n}{edge}\DUrole{o}{=}\DUrole{default_value}{None}}\sphinxparamcomma \sphinxparam{\DUrole{n}{broadline}\DUrole{o}{=}\DUrole{default_value}{None}}\sphinxparamcomma \sphinxparam{\DUrole{n}{clusteringalgorithm}\DUrole{o}{=}\DUrole{default_value}{None}}\sphinxparamcomma \sphinxparam{\DUrole{n}{deviationmask}\DUrole{o}{=}\DUrole{default_value}{None}}\sphinxparamcomma \sphinxparam{\DUrole{n}{parallel}\DUrole{o}{=}\DUrole{default_value}{None}}\sphinxparamcomma \sphinxparam{\DUrole{n}{infiles}\DUrole{o}{=}\DUrole{default_value}{None}}\sphinxparamcomma \sphinxparam{\DUrole{n}{field}\DUrole{o}{=}\DUrole{default_value}{None}}\sphinxparamcomma \sphinxparam{\DUrole{n}{antenna}\DUrole{o}{=}\DUrole{default_value}{None}}\sphinxparamcomma \sphinxparam{\DUrole{n}{spw}\DUrole{o}{=}\DUrole{default_value}{None}}\sphinxparamcomma \sphinxparam{\DUrole{n}{pol}\DUrole{o}{=}\DUrole{default_value}{None}}}
{}
\pysigstopsignatures
\sphinxAtStartPar
Detect and validate spectral lines, subtract baseline by masking detected lines.

\sphinxAtStartPar
The hsd\_baseline task subtracts baseline from calibrated spectra.
By default, the task tries to find spectral line feature using
line detection and validation algorithms. Then, the task puts a
mask on detected lines and perform baseline subtraction.
The user is able to turn off automatic line masking by setting
linewindow parameter, which specifies pre\sphinxhyphen{}defined line window.

\sphinxAtStartPar
Fitting order is automatically determined by default. It can be
disabled by specifying fitorder as non\sphinxhyphen{}negative value. In this
case, the value specified by fitorder will be used.

\sphinxAtStartPar
\sphinxstylestrong{*WARNING*}
Currently, hsd\_baseline overwrites the result obtained by the
previous run. Due to this behavior, users need to be careful
about an order of the task execution when they run hsd\_baseline
multiple times with different data selection. Suppose there are
two spectral windows (0 and 1) and hsd\_baseline is executed
separately for each spw as below,

\begin{sphinxVerbatim}[commandchars=\\\{\}]
\PYG{g+gp}{\PYGZgt{}\PYGZgt{}\PYGZgt{} }\PYG{n}{hsd\PYGZus{}baseline}\PYG{p}{(}\PYG{n}{spw}\PYG{o}{=}\PYG{l+s+s1}{\PYGZsq{}}\PYG{l+s+s1}{0}\PYG{l+s+s1}{\PYGZsq{}}\PYG{p}{)}
\PYG{g+gp}{\PYGZgt{}\PYGZgt{}\PYGZgt{} }\PYG{n}{hsd\PYGZus{}baseline}\PYG{p}{(}\PYG{n}{spw}\PYG{o}{=}\PYG{l+s+s1}{\PYGZsq{}}\PYG{l+s+s1}{1}\PYG{l+s+s1}{\PYGZsq{}}\PYG{p}{)}
\PYG{g+gp}{\PYGZgt{}\PYGZgt{}\PYGZgt{} }\PYG{n}{hsd\PYGZus{}blflag}\PYG{p}{(}\PYG{p}{)}
\PYG{g+gp}{\PYGZgt{}\PYGZgt{}\PYGZgt{} }\PYG{n}{hsd\PYGZus{}imaging}\PYG{p}{(}\PYG{p}{)}
\end{sphinxVerbatim}

\sphinxAtStartPar
Since the second run of hsd\_baseline overwrites the result for
spw 0 with the data before baseline subtraction, this will not
produce correct result for spw 0. Proper sequence for this use
case is to process each spw to the imaging stage separately,
which looks like as follows:

\begin{sphinxVerbatim}[commandchars=\\\{\}]
\PYG{g+gp}{\PYGZgt{}\PYGZgt{}\PYGZgt{} }\PYG{n}{hsd\PYGZus{}baseline}\PYG{p}{(}\PYG{n}{spw}\PYG{o}{=}\PYG{l+s+s1}{\PYGZsq{}}\PYG{l+s+s1}{0}\PYG{l+s+s1}{\PYGZsq{}}\PYG{p}{)}
\PYG{g+gp}{\PYGZgt{}\PYGZgt{}\PYGZgt{} }\PYG{n}{hsd\PYGZus{}blflag}\PYG{p}{(}\PYG{n}{spw}\PYG{o}{=}\PYG{l+s+s1}{\PYGZsq{}}\PYG{l+s+s1}{0}\PYG{l+s+s1}{\PYGZsq{}}\PYG{p}{)}
\PYG{g+gp}{\PYGZgt{}\PYGZgt{}\PYGZgt{} }\PYG{n}{hsd\PYGZus{}imaging}\PYG{p}{(}\PYG{n}{spw}\PYG{o}{=}\PYG{l+s+s1}{\PYGZsq{}}\PYG{l+s+s1}{0}\PYG{l+s+s1}{\PYGZsq{}}\PYG{p}{)}\PYG{p}{)}
\PYG{g+gp}{\PYGZgt{}\PYGZgt{}\PYGZgt{} }\PYG{n}{hsd\PYGZus{}baseline}\PYG{p}{(}\PYG{n}{spw}\PYG{o}{=}\PYG{l+s+s1}{\PYGZsq{}}\PYG{l+s+s1}{1}\PYG{l+s+s1}{\PYGZsq{}}\PYG{p}{)}
\PYG{g+gp}{\PYGZgt{}\PYGZgt{}\PYGZgt{} }\PYG{n}{hsd\PYGZus{}blflag}\PYG{p}{(}\PYG{n}{spw}\PYG{o}{=}\PYG{l+s+s1}{\PYGZsq{}}\PYG{l+s+s1}{1}\PYG{l+s+s1}{\PYGZsq{}}\PYG{p}{)}
\PYG{g+gp}{\PYGZgt{}\PYGZgt{}\PYGZgt{} }\PYG{n}{hsd\PYGZus{}imaging}\PYG{p}{(}\PYG{n}{spw}\PYG{o}{=}\PYG{l+s+s1}{\PYGZsq{}}\PYG{l+s+s1}{1}\PYG{l+s+s1}{\PYGZsq{}}\PYG{p}{)}
\end{sphinxVerbatim}
\begin{quote}\begin{description}
\sphinxlineitem{Parameters}\begin{itemize}
\item {} 
\sphinxAtStartPar
\sphinxstylestrong{fitfunc} \textendash{} Fitting function for baseline subtraction. You can choose either cubic spline (‘spline’ or ‘cspline’) or polynomial
(‘poly’ or ‘polynomial’). Default is ‘cspline’.

\item {} 
\sphinxAtStartPar
\sphinxstylestrong{fitorder} \textendash{} Fitting order for polynomial. For cubic spline, it is used to determine how much the spectrum is segmented into.
Default (\sphinxhyphen{}1) is to determine the order automatically.

\item {} 
\sphinxAtStartPar
\sphinxstylestrong{switchpoly} \textendash{} Whether to fall back the fits from cubic spline to 1st or 2nd order polynomial when large masks exist at the edges
of the spw. Condition for switching is as follows:
if nmask \textgreater{} nchan/2      =\textgreater{} 1st order polynomial
else if nmask \textgreater{} nchan/4 =\textgreater{} 2nd order polynomial
else                    =\textgreater{} use fitfunc and fitorder
where nmask is a number of channels for mask at edge while
nchan is a number of channels of entire spectral window.

\item {} 
\sphinxAtStartPar
\sphinxstylestrong{linewindow} \textendash{} Pre\sphinxhyphen{}defined line window. If this is set, specified line windows are used as a line mask for baseline subtraction
instead to determine masks based on line detection and
validation stage. Several types of format are acceptable.
One is channel\sphinxhyphen{}based window,
{[}min\_chan, max\_chan{]}
where min\_chan and max\_chan should be an integer. For
multiple  windows, nested list is also acceptable,
{[}{[}min\_chan0, max\_chan0{]}, {[}min\_chan1, max\_chan1{]}, …{]}
Another way is frequency\sphinxhyphen{}based window,
{[}min\_freq, max\_freq{]}
where min\_freq and max\_freq should be either a float or
a string. If float value is given, it is interpreted as
a frequency in Hz. String should be a quantity consisting
of “value” and “unit”,
e.g., ‘100GHz’. Multiple windows are also supported.
{[}{[}min\_freq0, max\_freq0{]}, {[}min\_freq1, max\_freq1{]}, …{]}
Note that the specified frequencies are assumed to be
the value in LSRK frame. Note also that there is a
limitation when multiple MSes are processed.
If native frequency frame of the data is not LSRK
(e.g. TOPO), frequencies need to be converted to that
frame. As a result, corresponding channel range may vary
between MSes. However, current implementation is not
able to handle such case. Frequencies are converted to
desired frame using representative MS (time, position,
direction).
In the above cases, specified line windows are applied
to all science spws. In case when line windows vary with
spw, line windows can be specified by a dictionary whose
key is spw id while value is line window.
For example, the following dictionary gives different
line windows to spws 17 and 19. Other spws, if available,
will have an empty line window.
\{17: {[}{[}100, 200{]}, {[}1200, 1400{]}{]}, 19: {[}‘112115MHz’, ‘112116MHz’{]}\}
Furthermore, linewindow accepts MS selection string.
The following string gives {[}{[}100,200{]},{[}1200,1400{]}{]} for
spw 17 while {[}1000,1500{]} for spw 21.
“17:100\textasciitilde{}200;1200\textasciitilde{}1400,21:1000\textasciitilde{}1500”
The string also accepts frequency with units. Note,
however, that frequency reference frame in this case is
not fixed to LSRK. Instead, the frame will be taken from
the MS (typically TOPO for ALMA).
Thus, the following two frequency\sphinxhyphen{}based line windows
result different channel selections.
\{19: {[}‘112115MHz’, ‘112116MHz’{]}\} \# frequency frame is LSRK
“19:11215MHz\textasciitilde{}11216MHz” \# frequency frame is taken from the data
\# (TOPO for ALMA)
None is allowed as a value of dictionary input to indicate that
no line detection/validation is required even if manually specified
line window does not exist. When None is given as a value and if
\sphinxcode{\sphinxupquote{linewindowmode}} is ‘replace’, line detection/validation is not performed
for the corresponding spw. For example, suppose the following parameters
are given for the data with four science spws, 17, 19, 21, and 23.
linewindow=\{17: {[}112.1e9, 112.2e9{]}, 19: {[}113.1e9, 113.15e9{]}, 21: None\}
linewindowmode=’replace’
The task will use given line window for 17 and 19 while the task performs
line deteciton/validation for spw 23 because no line window is set.
On the other hand, line detection/validation is skipped for spw 21 due to
the effect of None.
example: {[}100,200{]} (channel), {[}115e9, 115.1e9{]} (frequency in Hz)
{[}‘115GHz’, ‘115.1GHz’{]}, see above for more examples

\item {} 
\sphinxAtStartPar
\sphinxstylestrong{linewindowmode} \textendash{} Merge or replace given manual line window with line detection/validation result. If ‘replace’ is given, line
detection and validation will not be performed.
On the other hand, when ‘merge’ is specified, line
detection/validation will be performed and manually
specified line windows are added to the result.
Note that this has no effect when linewindow for target
spw is an empty list. In that case, line detection/validation
will be performed regardless of the value of linewindowmode.
In case if no linewindow nor line detection/validation
are necessary, you should set linewindowmode to ‘replace’
and specify None as a value of the linewindow dictionary
for the spw to apply. See parameter description of \sphinxcode{\sphinxupquote{linewindow}}
for detail.

\item {} 
\sphinxAtStartPar
\sphinxstylestrong{edge} \textendash{} Number of edge channels to be dropped from baseline subtraction. The value must be a list with length of 2,
whose values specify left and right edge channels,
respectively.
example: {[}10,10{]}

\item {} 
\sphinxAtStartPar
\sphinxstylestrong{broadline} \textendash{} Try to detect broad component of spectral line if True.

\item {} 
\sphinxAtStartPar
\sphinxstylestrong{clusteringalgorithm} \textendash{} Selection of the algorithm used in the clustering analysis to check the validity of detected line features.
The ‘kmean’ algorithm, hierarchical clustering algorithm,
‘hierarchy’, and their combination (‘both’) are so far
implemented.

\item {} 
\sphinxAtStartPar
\sphinxstylestrong{deviationmask} \textendash{} Apply deviation mask in addition to masks determined by the automatic line detection.

\item {} 
\sphinxAtStartPar
\sphinxstylestrong{parallel} \textendash{} Execute using CASA HPC functionality, if available. Options: ‘automatic’, ‘true’, ‘false’, True, False.
default: None (equivalent to ‘automatic’).

\item {} 
\sphinxAtStartPar
\sphinxstylestrong{infiles} \textendash{} List of data files. These must be a name of MeasurementSets that are registered to context via
hsd\_importdata task.
example: vis={[}‘X227.ms’, ‘X228.ms’{]}

\item {} 
\sphinxAtStartPar
\sphinxstylestrong{field} \textendash{} Data selection by field. example: ‘1’ (select by FIELD\_ID),
‘M100*’ (select by field name),
‘’ (all fields).

\item {} 
\sphinxAtStartPar
\sphinxstylestrong{antenna} \textendash{} Data selection by antenna. example: ‘1’ (select by ANTENNA\_ID),
‘PM03’ (select by antenna name),
‘’ (all antennas).

\item {} 
\sphinxAtStartPar
\sphinxstylestrong{spw} \textendash{} Data selection by spw. example: ‘3,4’ (generate caltable for spw 3 and 4),
{[}‘0’,’2’{]} (spw 0 for first data, 2 for second),
‘’ (all spws).

\item {} 
\sphinxAtStartPar
\sphinxstylestrong{pol} \textendash{} Data selection by polarizations. example: ‘0’ (generate caltable for pol 0),
{[}‘0\textasciitilde{}1’,’0’{]} (pol 0 and 1 for first data, only 0 for second),
‘’ (all polarizations).

\end{itemize}

\sphinxlineitem{Returns}
\sphinxAtStartPar
The results object for the pipeline task is returned.

\end{description}\end{quote}
\subsubsection*{Examples}
\begin{enumerate}
\sphinxsetlistlabels{\arabic}{enumi}{enumii}{}{.}%
\item {} 
\sphinxAtStartPar
Basic usage with automatic line detection and validation

\end{enumerate}

\begin{sphinxVerbatim}[commandchars=\\\{\}]
\PYG{g+gp}{\PYGZgt{}\PYGZgt{}\PYGZgt{} }\PYG{n}{hsd\PYGZus{}baseline}\PYG{p}{(}\PYG{n}{antenna}\PYG{o}{=}\PYG{l+s+s1}{\PYGZsq{}}\PYG{l+s+s1}{PM03}\PYG{l+s+s1}{\PYGZsq{}}\PYG{p}{,} \PYG{n}{spw}\PYG{o}{=}\PYG{l+s+s1}{\PYGZsq{}}\PYG{l+s+s1}{17,19}\PYG{l+s+s1}{\PYGZsq{}}\PYG{p}{)}
\end{sphinxVerbatim}
\begin{enumerate}
\sphinxsetlistlabels{\arabic}{enumi}{enumii}{}{.}%
\setcounter{enumi}{1}
\item {} 
\sphinxAtStartPar
Using pre\sphinxhyphen{}defined line windows without automatic line detection
and edge channels

\end{enumerate}

\begin{sphinxVerbatim}[commandchars=\\\{\}]
\PYG{g+gp}{\PYGZgt{}\PYGZgt{}\PYGZgt{} }\PYG{n}{hsd\PYGZus{}baseline}\PYG{p}{(}\PYG{n}{linewindow}\PYG{o}{=}\PYG{p}{[}\PYG{p}{[}\PYG{l+m+mi}{100}\PYG{p}{,} \PYG{l+m+mi}{200}\PYG{p}{]}\PYG{p}{,} \PYG{p}{[}\PYG{l+m+mi}{1200}\PYG{p}{,} \PYG{l+m+mi}{1400}\PYG{p}{]}\PYG{p}{]}\PYG{p}{,}
\PYG{g+go}{                 linewindowmode=\PYGZsq{}replace\PYGZsq{}, edge=[10, 10])}
\end{sphinxVerbatim}
\begin{enumerate}
\sphinxsetlistlabels{\arabic}{enumi}{enumii}{}{.}%
\setcounter{enumi}{2}
\item {} 
\sphinxAtStartPar
Using per spw pre\sphinxhyphen{}defined line windows with automatic line detection

\end{enumerate}

\begin{sphinxVerbatim}[commandchars=\\\{\}]
\PYG{g+gp}{\PYGZgt{}\PYGZgt{}\PYGZgt{} }\PYG{n}{hsd\PYGZus{}baseline}\PYG{p}{(}\PYG{n}{linewindow}\PYG{o}{=}\PYG{p}{\PYGZob{}}\PYG{l+m+mi}{19}\PYG{p}{:} \PYG{p}{[}\PYG{p}{[}\PYG{l+m+mi}{390}\PYG{p}{,} \PYG{l+m+mi}{550}\PYG{p}{]}\PYG{p}{]}\PYG{p}{,} \PYG{l+m+mi}{23}\PYG{p}{:} \PYG{p}{[}\PYG{p}{[}\PYG{l+m+mi}{100}\PYG{p}{,} \PYG{l+m+mi}{200}\PYG{p}{]}\PYG{p}{,} \PYG{p}{[}\PYG{l+m+mi}{1200}\PYG{p}{,} \PYG{l+m+mi}{1400}\PYG{p}{]}\PYG{p}{]}\PYG{p}{\PYGZcb{}}\PYG{p}{,}
\PYG{g+go}{                 linewindowmode=\PYGZsq{}merge\PYGZsq{})}
\end{sphinxVerbatim}

\end{fulllineitems}


\sphinxstepscope


\section{pipeline.hsd.cli.hsd\_blflag}
\label{\detokenize{_autosummary/pipeline.hsd.cli.hsd_blflag:pipeline-hsd-cli-hsd-blflag}}\label{\detokenize{_autosummary/pipeline.hsd.cli.hsd_blflag::doc}}\index{hsd\_blflag() (in module pipeline.hsd.cli)@\spxentry{hsd\_blflag()}\spxextra{in module pipeline.hsd.cli}}

\begin{fulllineitems}
\phantomsection\label{\detokenize{_autosummary/pipeline.hsd.cli.hsd_blflag:pipeline.hsd.cli.hsd_blflag}}
\pysigstartsignatures
\pysiglinewithargsret
{\sphinxbfcode{\sphinxupquote{hsd\_blflag}}}
{\sphinxparam{\DUrole{n}{iteration}\DUrole{o}{=}\DUrole{default_value}{None}}\sphinxparamcomma \sphinxparam{\DUrole{n}{edge}\DUrole{o}{=}\DUrole{default_value}{None}}\sphinxparamcomma \sphinxparam{\DUrole{n}{flag\_tsys}\DUrole{o}{=}\DUrole{default_value}{None}}\sphinxparamcomma \sphinxparam{\DUrole{n}{tsys\_thresh}\DUrole{o}{=}\DUrole{default_value}{None}}\sphinxparamcomma \sphinxparam{\DUrole{n}{flag\_prfre}\DUrole{o}{=}\DUrole{default_value}{None}}\sphinxparamcomma \sphinxparam{\DUrole{n}{prfre\_thresh}\DUrole{o}{=}\DUrole{default_value}{None}}\sphinxparamcomma \sphinxparam{\DUrole{n}{flag\_pofre}\DUrole{o}{=}\DUrole{default_value}{None}}\sphinxparamcomma \sphinxparam{\DUrole{n}{pofre\_thresh}\DUrole{o}{=}\DUrole{default_value}{None}}\sphinxparamcomma \sphinxparam{\DUrole{n}{flag\_prfr}\DUrole{o}{=}\DUrole{default_value}{None}}\sphinxparamcomma \sphinxparam{\DUrole{n}{prfr\_thresh}\DUrole{o}{=}\DUrole{default_value}{None}}\sphinxparamcomma \sphinxparam{\DUrole{n}{flag\_pofr}\DUrole{o}{=}\DUrole{default_value}{None}}\sphinxparamcomma \sphinxparam{\DUrole{n}{pofr\_thresh}\DUrole{o}{=}\DUrole{default_value}{None}}\sphinxparamcomma \sphinxparam{\DUrole{n}{flag\_prfrm}\DUrole{o}{=}\DUrole{default_value}{None}}\sphinxparamcomma \sphinxparam{\DUrole{n}{prfrm\_thresh}\DUrole{o}{=}\DUrole{default_value}{None}}\sphinxparamcomma \sphinxparam{\DUrole{n}{prfrm\_nmean}\DUrole{o}{=}\DUrole{default_value}{None}}\sphinxparamcomma \sphinxparam{\DUrole{n}{flag\_pofrm}\DUrole{o}{=}\DUrole{default_value}{None}}\sphinxparamcomma \sphinxparam{\DUrole{n}{pofrm\_thresh}\DUrole{o}{=}\DUrole{default_value}{None}}\sphinxparamcomma \sphinxparam{\DUrole{n}{pofrm\_nmean}\DUrole{o}{=}\DUrole{default_value}{None}}\sphinxparamcomma \sphinxparam{\DUrole{n}{plotflag}\DUrole{o}{=}\DUrole{default_value}{None}}\sphinxparamcomma \sphinxparam{\DUrole{n}{parallel}\DUrole{o}{=}\DUrole{default_value}{None}}\sphinxparamcomma \sphinxparam{\DUrole{n}{infiles}\DUrole{o}{=}\DUrole{default_value}{None}}\sphinxparamcomma \sphinxparam{\DUrole{n}{antenna}\DUrole{o}{=}\DUrole{default_value}{None}}\sphinxparamcomma \sphinxparam{\DUrole{n}{field}\DUrole{o}{=}\DUrole{default_value}{None}}\sphinxparamcomma \sphinxparam{\DUrole{n}{spw}\DUrole{o}{=}\DUrole{default_value}{None}}\sphinxparamcomma \sphinxparam{\DUrole{n}{pol}\DUrole{o}{=}\DUrole{default_value}{None}}}
{}
\pysigstopsignatures
\sphinxAtStartPar
Flag spectra based on predefined criteria of single dish pipeline.

\sphinxAtStartPar
Data are flagged based on several flagging rules. Available rules are:
expected rms, calculated rms, and running mean of both pre\sphinxhyphen{}fit and
post\sphinxhyphen{}fit spectra. Tsys flagging is also available.

\sphinxAtStartPar
In addition, the heuristics script creates many plots for each stage.
Those plots are included in the weblog.
\begin{quote}\begin{description}
\sphinxlineitem{Parameters}\begin{itemize}
\item {} 
\sphinxAtStartPar
\sphinxstylestrong{iteration} \textendash{} Number of iterations to perform sigma clipping to calculate threshold value of flagging.

\item {} 
\sphinxAtStartPar
\sphinxstylestrong{edge} \textendash{} Number of channels to be dropped from the edge. The value must be a list of integer with length of one or
two. If list length is one, same number will be applied
both side of the band.
example: {[}10,20{]}, {[}10{]}

\item {} 
\sphinxAtStartPar
\sphinxstylestrong{flag\_tsys} \textendash{} Activate (True) or deactivate (False) Tsys flag.

\item {} 
\sphinxAtStartPar
\sphinxstylestrong{tsys\_thresh} \textendash{} Threshold value for Tsys flag.

\item {} 
\sphinxAtStartPar
\sphinxstylestrong{flag\_prfre} \textendash{} Activate (True) or deactivate (False) flag by expected rms of pre\sphinxhyphen{}fit spectra.

\item {} 
\sphinxAtStartPar
\sphinxstylestrong{prfre\_thresh} \textendash{} Threshold value for flag by expected rms of pre\sphinxhyphen{}fit spectra.

\item {} 
\sphinxAtStartPar
\sphinxstylestrong{flag\_pofre} \textendash{} Activate (True) or deactivate (False) flag by expected rms of post\sphinxhyphen{}fit spectra.

\item {} 
\sphinxAtStartPar
\sphinxstylestrong{pofre\_thresh} \textendash{} Threshold value for flag by expected rms of post\sphinxhyphen{}fit spectra.

\item {} 
\sphinxAtStartPar
\sphinxstylestrong{flag\_prfr} \textendash{} Activate (True) or deactivate (False) flag by rms of pre\sphinxhyphen{}fit spectra.

\item {} 
\sphinxAtStartPar
\sphinxstylestrong{prfr\_thresh} \textendash{} Threshold value for flag by rms of pre\sphinxhyphen{}fit spectra.

\item {} 
\sphinxAtStartPar
\sphinxstylestrong{flag\_pofr} \textendash{} Activate (True) or deactivate (False) flag by rms of post\sphinxhyphen{}fit spectra.

\item {} 
\sphinxAtStartPar
\sphinxstylestrong{pofr\_thresh} \textendash{} Threshold value for flag by rms of post\sphinxhyphen{}fit spectra.

\item {} 
\sphinxAtStartPar
\sphinxstylestrong{flag\_prfrm} \textendash{} Activate (True) or deactivate (False) flag by running mean of pre\sphinxhyphen{}fit spectra.

\item {} 
\sphinxAtStartPar
\sphinxstylestrong{prfrm\_thresh} \textendash{} Threshold value for flag by running mean of pre\sphinxhyphen{}fit spectra.

\item {} 
\sphinxAtStartPar
\sphinxstylestrong{prfrm\_nmean} \textendash{} Number of channels for running mean of pre\sphinxhyphen{}fit spectra.

\item {} 
\sphinxAtStartPar
\sphinxstylestrong{flag\_pofrm} \textendash{} Activate (True) or deactivate (False) flag by running mean of post\sphinxhyphen{}fit spectra.

\item {} 
\sphinxAtStartPar
\sphinxstylestrong{pofrm\_thresh} \textendash{} Threshold value for flag by running mean of post\sphinxhyphen{}fit spectra.

\item {} 
\sphinxAtStartPar
\sphinxstylestrong{pofrm\_nmean} \textendash{} Number of channels for running mean of post\sphinxhyphen{}fit spectra.

\item {} 
\sphinxAtStartPar
\sphinxstylestrong{plotflag} \textendash{} True to plot result of data flagging.

\item {} 
\sphinxAtStartPar
\sphinxstylestrong{parallel} \textendash{} Execute using CASA HPC functionality, if available. Options: ‘automatic’, ‘true’, ‘false’, True, False.
Default: None (equivalent to ‘automatic’)

\item {} 
\sphinxAtStartPar
\sphinxstylestrong{infiles} \textendash{} ASDM or MS files to be processed. This parameter behaves as data selection parameter. The name specified by
infiles must be registered to context before you run
hsd\_blflag.

\item {} 
\sphinxAtStartPar
\sphinxstylestrong{antenna} \textendash{} Data selection by antenna names or ids. example: ‘PM03,PM04’,
‘’ (all antennas).

\item {} 
\sphinxAtStartPar
\sphinxstylestrong{field} \textendash{} Data selection by field names or ids. example: ‘\sphinxtitleref{*Sgr*,M100}’,
‘’ (all fields).

\item {} 
\sphinxAtStartPar
\sphinxstylestrong{spw} \textendash{} Data selection by spw ids. example: ‘3,4’ (spw 3 and 4),
‘’ (all spws).

\item {} 
\sphinxAtStartPar
\sphinxstylestrong{pol} \textendash{} Data selection by polarizations. example: ‘XX,YY’ (correlation XX and YY),
‘’ (all polarizations).

\end{itemize}

\sphinxlineitem{Returns}
\sphinxAtStartPar
The results object for the pipeline task is returned.

\end{description}\end{quote}
\subsubsection*{Examples}
\begin{enumerate}
\sphinxsetlistlabels{\arabic}{enumi}{enumii}{}{.}%
\item {} 
\sphinxAtStartPar
flagging with all rules

\end{enumerate}

\begin{sphinxVerbatim}[commandchars=\\\{\}]
\PYG{g+gp}{\PYGZgt{}\PYGZgt{}\PYGZgt{} }\PYG{n}{hsd\PYGZus{}blflag}\PYG{p}{(}\PYG{p}{)}
\end{sphinxVerbatim}

\end{fulllineitems}


\sphinxstepscope


\section{pipeline.hsd.cli.hsd\_exportdata}
\label{\detokenize{_autosummary/pipeline.hsd.cli.hsd_exportdata:pipeline-hsd-cli-hsd-exportdata}}\label{\detokenize{_autosummary/pipeline.hsd.cli.hsd_exportdata::doc}}\index{hsd\_exportdata() (in module pipeline.hsd.cli)@\spxentry{hsd\_exportdata()}\spxextra{in module pipeline.hsd.cli}}

\begin{fulllineitems}
\phantomsection\label{\detokenize{_autosummary/pipeline.hsd.cli.hsd_exportdata:pipeline.hsd.cli.hsd_exportdata}}
\pysigstartsignatures
\pysiglinewithargsret
{\sphinxbfcode{\sphinxupquote{hsd\_exportdata}}}
{\sphinxparam{\DUrole{n}{pprfile}\DUrole{o}{=}\DUrole{default_value}{None}}\sphinxparamcomma \sphinxparam{\DUrole{n}{targetimages}\DUrole{o}{=}\DUrole{default_value}{None}}\sphinxparamcomma \sphinxparam{\DUrole{n}{products\_dir}\DUrole{o}{=}\DUrole{default_value}{None}}}
{}
\pysigstopsignatures
\sphinxAtStartPar
Prepare single dish data for export.

\sphinxAtStartPar
The hsd\_exportdata task exports the data defined in the pipeline context
and exports it to the data products directory, converting and or packing
it as necessary.

\sphinxAtStartPar
The current version of the task exports the following products
\begin{itemize}
\item {} 
\sphinxAtStartPar
a FITS image for each selected science target source image

\item {} 
\sphinxAtStartPar
a tar file per ASDM containing the final flags version and blparam

\item {} 
\sphinxAtStartPar
a tar file containing the file web log

\end{itemize}
\begin{quote}\begin{description}
\sphinxlineitem{Parameters}\begin{itemize}
\item {} 
\sphinxAtStartPar
\sphinxstylestrong{pprfile} \textendash{} Name of the pipeline processing request to be exported. Defaults to a file matching the template ‘PPR\_*.xml’.
example: pprfile={[}‘PPR\_GRB021004.xml’{]}

\item {} 
\sphinxAtStartPar
\sphinxstylestrong{targetimages} \textendash{} List of science target images to be exported. Defaults to all science target images recorded in the pipeline context.
example: targetimages={[}‘r\_aqr.CM02.spw5.line0.XXYY.sd.im’, ‘r\_aqr.CM02.spw5.XXYY.sd.cont.im’{]}

\item {} 
\sphinxAtStartPar
\sphinxstylestrong{products\_dir} \textendash{} Name of the data products subdirectory. Defaults to ‘./’. Example: products\_dir=’../products’.

\end{itemize}

\sphinxlineitem{Returns}
\sphinxAtStartPar
The results object for the pipeline task is returned.

\end{description}\end{quote}
\subsubsection*{Examples}

\sphinxAtStartPar
1. Export the pipeline results for a single session to the data products
directory

\begin{sphinxVerbatim}[commandchars=\\\{\}]
\PYGZgt{}\PYGZgt{}\PYGZgt{} !mkdir ../products
\PYGZgt{}\PYGZgt{}\PYGZgt{} hsd\PYGZus{}exportdata (products\PYGZus{}dir=\PYGZsq{}../products\PYGZsq{})
\end{sphinxVerbatim}

\end{fulllineitems}


\sphinxstepscope


\section{pipeline.hsd.cli.hsd\_flagdata}
\label{\detokenize{_autosummary/pipeline.hsd.cli.hsd_flagdata:pipeline-hsd-cli-hsd-flagdata}}\label{\detokenize{_autosummary/pipeline.hsd.cli.hsd_flagdata::doc}}\index{hsd\_flagdata() (in module pipeline.hsd.cli)@\spxentry{hsd\_flagdata()}\spxextra{in module pipeline.hsd.cli}}

\begin{fulllineitems}
\phantomsection\label{\detokenize{_autosummary/pipeline.hsd.cli.hsd_flagdata:pipeline.hsd.cli.hsd_flagdata}}
\pysigstartsignatures
\pysiglinewithargsret
{\sphinxbfcode{\sphinxupquote{hsd\_flagdata}}}
{\sphinxparam{\DUrole{n}{vis}\DUrole{o}{=}\DUrole{default_value}{None}}\sphinxparamcomma \sphinxparam{\DUrole{n}{autocorr}\DUrole{o}{=}\DUrole{default_value}{None}}\sphinxparamcomma \sphinxparam{\DUrole{n}{shadow}\DUrole{o}{=}\DUrole{default_value}{None}}\sphinxparamcomma \sphinxparam{\DUrole{n}{scan}\DUrole{o}{=}\DUrole{default_value}{None}}\sphinxparamcomma \sphinxparam{\DUrole{n}{scannumber}\DUrole{o}{=}\DUrole{default_value}{None}}\sphinxparamcomma \sphinxparam{\DUrole{n}{intents}\DUrole{o}{=}\DUrole{default_value}{None}}\sphinxparamcomma \sphinxparam{\DUrole{n}{edgespw}\DUrole{o}{=}\DUrole{default_value}{None}}\sphinxparamcomma \sphinxparam{\DUrole{n}{fracspw}\DUrole{o}{=}\DUrole{default_value}{None}}\sphinxparamcomma \sphinxparam{\DUrole{n}{fracspwfps}\DUrole{o}{=}\DUrole{default_value}{None}}\sphinxparamcomma \sphinxparam{\DUrole{n}{online}\DUrole{o}{=}\DUrole{default_value}{None}}\sphinxparamcomma \sphinxparam{\DUrole{n}{fileonline}\DUrole{o}{=}\DUrole{default_value}{None}}\sphinxparamcomma \sphinxparam{\DUrole{n}{template}\DUrole{o}{=}\DUrole{default_value}{None}}\sphinxparamcomma \sphinxparam{\DUrole{n}{filetemplate}\DUrole{o}{=}\DUrole{default_value}{None}}\sphinxparamcomma \sphinxparam{\DUrole{n}{pointing}\DUrole{o}{=}\DUrole{default_value}{None}}\sphinxparamcomma \sphinxparam{\DUrole{n}{filepointing}\DUrole{o}{=}\DUrole{default_value}{None}}\sphinxparamcomma \sphinxparam{\DUrole{n}{incompleteraster}\DUrole{o}{=}\DUrole{default_value}{None}}\sphinxparamcomma \sphinxparam{\DUrole{n}{hm\_tbuff}\DUrole{o}{=}\DUrole{default_value}{None}}\sphinxparamcomma \sphinxparam{\DUrole{n}{tbuff}\DUrole{o}{=}\DUrole{default_value}{None}}\sphinxparamcomma \sphinxparam{\DUrole{n}{qa0}\DUrole{o}{=}\DUrole{default_value}{None}}\sphinxparamcomma \sphinxparam{\DUrole{n}{qa2}\DUrole{o}{=}\DUrole{default_value}{None}}\sphinxparamcomma \sphinxparam{\DUrole{n}{parallel}\DUrole{o}{=}\DUrole{default_value}{None}}\sphinxparamcomma \sphinxparam{\DUrole{n}{flagbackup}\DUrole{o}{=}\DUrole{default_value}{None}}}
{}
\pysigstopsignatures
\sphinxAtStartPar
Do basic flagging of a list of MeasurementSets.

\sphinxAtStartPar
The hsd\_flagdata data performs basic flagging operations on a list of
MeasurementSets including:
\begin{itemize}
\item {} 
\sphinxAtStartPar
applying online flags

\item {} 
\sphinxAtStartPar
applying a flagging template

\item {} 
\sphinxAtStartPar
shadowed antenna data flagging

\item {} 
\sphinxAtStartPar
scan\sphinxhyphen{}based flagging by intent or scan number

\item {} 
\sphinxAtStartPar
edge channel flagging

\end{itemize}
\begin{quote}\begin{description}
\sphinxlineitem{Parameters}\begin{itemize}
\item {} 
\sphinxAtStartPar
\sphinxstylestrong{vis} \textendash{} The list of input MeasurementSets. Defaults to the list of MeasurementSets defined in the pipeline context.

\item {} 
\sphinxAtStartPar
\sphinxstylestrong{autocorr} \textendash{} Flag autocorrelation data.

\item {} 
\sphinxAtStartPar
\sphinxstylestrong{shadow} \textendash{} Flag shadowed antennas.

\item {} 
\sphinxAtStartPar
\sphinxstylestrong{scan} \textendash{} Flag a list of scans and intents specified by scannumber and intents.

\item {} 
\sphinxAtStartPar
\sphinxstylestrong{scannumber} \textendash{} A string containing a comma delimited list of scans to be flagged.

\item {} 
\sphinxAtStartPar
\sphinxstylestrong{intents} \textendash{} A string containing a comma delimited list of intents against which the scans to be flagged are matched.
Example: \sphinxtitleref{‘*BANDPASS*’}.

\item {} 
\sphinxAtStartPar
\sphinxstylestrong{edgespw} \textendash{} Flag the edge spectral window channels.

\item {} 
\sphinxAtStartPar
\sphinxstylestrong{fracspw} \textendash{} Fraction of the baseline correlator TDM edge channels to be flagged.

\item {} 
\sphinxAtStartPar
\sphinxstylestrong{fracspwfps} \textendash{} Fraction of the ACS correlator TDM edge channels to be flagged.

\item {} 
\sphinxAtStartPar
\sphinxstylestrong{online} \textendash{} Apply the online flags.

\item {} 
\sphinxAtStartPar
\sphinxstylestrong{fileonline} \textendash{} File containing the online flags. These are computed by the h\_init or hif\_importdata data tasks. If the online flags files
are undefined a name of the form ‘msname.flagonline.txt’ is assumed.

\item {} 
\sphinxAtStartPar
\sphinxstylestrong{template} \textendash{} Apply a flagging template.

\item {} 
\sphinxAtStartPar
\sphinxstylestrong{filetemplate} \textendash{} The name of a text file that contains the flagging template for RFI, birdies, telluric lines, etc.  If the template flags files
is undefined a name of the form ‘msname.flagtemplate.txt’ is assumed.

\item {} 
\sphinxAtStartPar
\sphinxstylestrong{pointing} \textendash{} Apply a flagging template for pointing flag.

\item {} 
\sphinxAtStartPar
\sphinxstylestrong{filepointing} \textendash{} The name of a text file that contains the flagging template for pointing flag. If the template flags files is undefined a name of
the form ‘msname.flagpointing.txt’ is assumed.

\item {} 
\sphinxAtStartPar
\sphinxstylestrong{incompleteraster} \textendash{} Apply commands to flag incomplete raster sequence. If this is False, relevant commands in filepointing are
simply commented out.

\item {} 
\sphinxAtStartPar
\sphinxstylestrong{hm\_tbuff} \textendash{} The heuristic for computing the default time interval padding parameter. The options are ‘halfint’ and ‘manual’. In ‘halfint’ mode tbuff is set to
half the maximum of the median integration time of the science and calibrator target
observations.

\item {} 
\sphinxAtStartPar
\sphinxstylestrong{tbuff} \textendash{} The time in seconds used to pad flagging command time intervals if hm\_tbuff=’manual’.

\item {} 
\sphinxAtStartPar
\sphinxstylestrong{qa0} \textendash{} QA0 flags

\item {} 
\sphinxAtStartPar
\sphinxstylestrong{qa2} \textendash{} QA2 flags

\item {} 
\sphinxAtStartPar
\sphinxstylestrong{parallel} \textendash{} Execute using CASA HPC functionality, if available. Options: ‘automatic’, ‘true’, ‘false’, True, False.
Default: None (equivalent to ‘automatic’).

\item {} 
\sphinxAtStartPar
\sphinxstylestrong{flagbackup} \textendash{} Back up any pre\sphinxhyphen{}existing flags before applying new ones.

\end{itemize}

\sphinxlineitem{Returns}
\sphinxAtStartPar
The results object for the pipeline task is returned.

\end{description}\end{quote}
\subsubsection*{Examples}
\begin{enumerate}
\sphinxsetlistlabels{\arabic}{enumi}{enumii}{}{.}%
\item {} 
\sphinxAtStartPar
Do basic flagging on a MeasurementSet

\end{enumerate}

\begin{sphinxVerbatim}[commandchars=\\\{\}]
\PYG{g+gp}{\PYGZgt{}\PYGZgt{}\PYGZgt{} }\PYG{n}{hsd\PYGZus{}flagdata}\PYG{p}{(}\PYG{p}{)}
\end{sphinxVerbatim}

\sphinxAtStartPar
2. Do basic flagging on a MeasurementSet flagging additional scans selected
by number as well.

\begin{sphinxVerbatim}[commandchars=\\\{\}]
\PYG{g+gp}{\PYGZgt{}\PYGZgt{}\PYGZgt{} }\PYG{n}{hsd\PYGZus{}flagdata}\PYG{p}{(}\PYG{n}{scannumber}\PYG{o}{=}\PYG{l+s+s1}{\PYGZsq{}}\PYG{l+s+s1}{13,18}\PYG{l+s+s1}{\PYGZsq{}}\PYG{p}{)}
\end{sphinxVerbatim}

\end{fulllineitems}


\sphinxstepscope


\section{pipeline.hsd.cli.hsd\_imaging}
\label{\detokenize{_autosummary/pipeline.hsd.cli.hsd_imaging:pipeline-hsd-cli-hsd-imaging}}\label{\detokenize{_autosummary/pipeline.hsd.cli.hsd_imaging::doc}}\index{hsd\_imaging() (in module pipeline.hsd.cli)@\spxentry{hsd\_imaging()}\spxextra{in module pipeline.hsd.cli}}

\begin{fulllineitems}
\phantomsection\label{\detokenize{_autosummary/pipeline.hsd.cli.hsd_imaging:pipeline.hsd.cli.hsd_imaging}}
\pysigstartsignatures
\pysiglinewithargsret
{\sphinxbfcode{\sphinxupquote{hsd\_imaging}}}
{\sphinxparam{\DUrole{n}{mode}\DUrole{o}{=}\DUrole{default_value}{None}}\sphinxparamcomma \sphinxparam{\DUrole{n}{restfreq}\DUrole{o}{=}\DUrole{default_value}{None}}\sphinxparamcomma \sphinxparam{\DUrole{n}{infiles}\DUrole{o}{=}\DUrole{default_value}{None}}\sphinxparamcomma \sphinxparam{\DUrole{n}{field}\DUrole{o}{=}\DUrole{default_value}{None}}\sphinxparamcomma \sphinxparam{\DUrole{n}{spw}\DUrole{o}{=}\DUrole{default_value}{None}}}
{}
\pysigstopsignatures
\sphinxAtStartPar
Generate single dish images.

\sphinxAtStartPar
The hsd\_imaging task generates single dish images per antenna as
well as combined image over whole antennas for each field and
spectral window. Image configuration (grid size, number of pixels,
etc.) is automatically determined based on meta data such as
antenna diameter, map extent, etc.

\sphinxAtStartPar
Note that generated images are always in LSRK frame.
\begin{quote}\begin{description}
\sphinxlineitem{Parameters}\begin{itemize}
\item {} 
\sphinxAtStartPar
\sphinxstylestrong{mode} \textendash{} Imaging mode controls imaging parameters in the task. Accepts either “line” (spectral line imaging) or “ampcal”
(image settings for amplitude calibrator).

\item {} 
\sphinxAtStartPar
\sphinxstylestrong{restfreq} \textendash{} Rest frequency

\item {} 
\sphinxAtStartPar
\sphinxstylestrong{infiles} \textendash{} List of data files. These must be a name of MeasurementSets that are registered to context via
hsd\_importdata task.
example: vis={[}‘uid\_\_\_A002\_X85c183\_X36f.ms’,
‘uid\_\_\_A002\_X85c183\_X60b.ms’{]}

\item {} 
\sphinxAtStartPar
\sphinxstylestrong{field} \textendash{} Data selection by field names or ids. example: “\sphinxtitleref{*Sgr*,M100}”

\item {} 
\sphinxAtStartPar
\sphinxstylestrong{spw} \textendash{} Data selection by spw ids. example: “3,4” (generate images for spw 3 and 4)

\end{itemize}

\sphinxlineitem{Returns}
\sphinxAtStartPar
The results object for the pipeline task is returned.

\end{description}\end{quote}
\subsubsection*{Examples}
\begin{enumerate}
\sphinxsetlistlabels{\arabic}{enumi}{enumii}{}{.}%
\item {} 
\sphinxAtStartPar
Generate images with default settings and context

\end{enumerate}

\begin{sphinxVerbatim}[commandchars=\\\{\}]
\PYG{g+gp}{\PYGZgt{}\PYGZgt{}\PYGZgt{} }\PYG{n}{hsd\PYGZus{}imaging}\PYG{p}{(}\PYG{p}{)}
\end{sphinxVerbatim}
\begin{enumerate}
\sphinxsetlistlabels{\arabic}{enumi}{enumii}{}{.}%
\setcounter{enumi}{1}
\item {} 
\sphinxAtStartPar
Generate images with amplitude calibrator and specific parameters

\end{enumerate}

\begin{sphinxVerbatim}[commandchars=\\\{\}]
\PYG{g+gp}{\PYGZgt{}\PYGZgt{}\PYGZgt{} }\PYG{n}{hsd\PYGZus{}imaging}\PYG{p}{(}\PYG{n}{mode}\PYG{o}{=}\PYG{l+s+s1}{\PYGZsq{}}\PYG{l+s+s1}{ampcal}\PYG{l+s+s1}{\PYGZsq{}}\PYG{p}{,} \PYG{n}{field}\PYG{o}{=}\PYG{l+s+s1}{\PYGZsq{}}\PYG{l+s+s1}{*Sgr*,M100}\PYG{l+s+s1}{\PYGZsq{}}\PYG{p}{,} \PYG{n}{spw}\PYG{o}{=}\PYG{l+s+s1}{\PYGZsq{}}\PYG{l+s+s1}{17,19}\PYG{l+s+s1}{\PYGZsq{}}\PYG{p}{)}
\end{sphinxVerbatim}

\end{fulllineitems}


\sphinxstepscope


\section{pipeline.hsd.cli.hsd\_importdata}
\label{\detokenize{_autosummary/pipeline.hsd.cli.hsd_importdata:pipeline-hsd-cli-hsd-importdata}}\label{\detokenize{_autosummary/pipeline.hsd.cli.hsd_importdata::doc}}\index{hsd\_importdata() (in module pipeline.hsd.cli)@\spxentry{hsd\_importdata()}\spxextra{in module pipeline.hsd.cli}}

\begin{fulllineitems}
\phantomsection\label{\detokenize{_autosummary/pipeline.hsd.cli.hsd_importdata:pipeline.hsd.cli.hsd_importdata}}
\pysigstartsignatures
\pysiglinewithargsret
{\sphinxbfcode{\sphinxupquote{hsd\_importdata}}}
{\sphinxparam{\DUrole{n}{vis}\DUrole{o}{=}\DUrole{default_value}{None}}\sphinxparamcomma \sphinxparam{\DUrole{n}{session}\DUrole{o}{=}\DUrole{default_value}{None}}\sphinxparamcomma \sphinxparam{\DUrole{n}{hm\_rasterscan}\DUrole{o}{=}\DUrole{default_value}{None}}\sphinxparamcomma \sphinxparam{\DUrole{n}{parallel}\DUrole{o}{=}\DUrole{default_value}{None}}\sphinxparamcomma \sphinxparam{\DUrole{n}{asis}\DUrole{o}{=}\DUrole{default_value}{None}}\sphinxparamcomma \sphinxparam{\DUrole{n}{process\_caldevice}\DUrole{o}{=}\DUrole{default_value}{None}}\sphinxparamcomma \sphinxparam{\DUrole{n}{overwrite}\DUrole{o}{=}\DUrole{default_value}{None}}\sphinxparamcomma \sphinxparam{\DUrole{n}{nocopy}\DUrole{o}{=}\DUrole{default_value}{None}}\sphinxparamcomma \sphinxparam{\DUrole{n}{bdfflags}\DUrole{o}{=}\DUrole{default_value}{None}}\sphinxparamcomma \sphinxparam{\DUrole{n}{datacolumns}\DUrole{o}{=}\DUrole{default_value}{None}}\sphinxparamcomma \sphinxparam{\DUrole{n}{lazy}\DUrole{o}{=}\DUrole{default_value}{None}}\sphinxparamcomma \sphinxparam{\DUrole{n}{with\_pointing\_correction}\DUrole{o}{=}\DUrole{default_value}{None}}\sphinxparamcomma \sphinxparam{\DUrole{n}{createmms}\DUrole{o}{=}\DUrole{default_value}{None}}}
{}
\pysigstopsignatures
\sphinxAtStartPar
Imports data into the single dish pipeline.

\sphinxAtStartPar
The hsd\_importdata task loads the specified visibility data into the pipeline
context unpacking and / or converting it as necessary.

\sphinxAtStartPar
If the \sphinxcode{\sphinxupquote{overwrite}} input parameter is set to False and the task is asked to
convert an input ASDM input to an MS, then when the output MS already exists in
the output directory, the importasdm conversion step is skipped, and the
existing MS will be imported instead.
\begin{quote}\begin{description}
\sphinxlineitem{Parameters}\begin{itemize}
\item {} 
\sphinxAtStartPar
\sphinxstylestrong{vis} \textendash{} List of visibility data files. These may be ASDMs, tar files of ASDMs, MSes, or tar files of MSes, If ASDM files are specified, they will be
converted to MS format.
example: vis={[}‘X227.ms’, ‘asdms.tar.gz’{]}

\item {} 
\sphinxAtStartPar
\sphinxstylestrong{session} \textendash{} List of sessions to which the visibility files belong. Defaults to a single session containing all the visibility files, otherwise
a session must be assigned to each vis file.
example: session={[}‘Session\_1’, ‘Sessions\_2’{]}

\item {} 
\sphinxAtStartPar
\sphinxstylestrong{hm\_rasterscan} \textendash{} Heuristics method for raster scan analysis. Two analysis modes, time\sphinxhyphen{}domain analysis (‘time’) and direction analysis (‘direction’), are available.
Default is ‘time’.

\item {} 
\sphinxAtStartPar
\sphinxstylestrong{parallel} \textendash{} Execute using CASA HPC functionality, if available. Options: ‘automatic’, ‘true’, ‘false’, True, False.
Default: None (equivalent to ‘automatic’).

\item {} 
\sphinxAtStartPar
\sphinxstylestrong{asis} \textendash{} Creates verbatim copies of the ASDM tables in the output MS. The value given to this option must be a list of table names
separated by space characters.
example: ‘Receiver’, ‘’

\item {} 
\sphinxAtStartPar
\sphinxstylestrong{process\_caldevice} \textendash{} Ingest the ASDM caldevice table. example: True

\item {} 
\sphinxAtStartPar
\sphinxstylestrong{overwrite} \textendash{} Overwrite existing files on import. When converting ASDM to MS, if overwrite=False and the MS already
exists in output directory, then this existing MS dataset will be used
instead.

\item {} 
\sphinxAtStartPar
\sphinxstylestrong{nocopy} \textendash{} Disable copying of MS to working directory

\item {} 
\sphinxAtStartPar
\sphinxstylestrong{bdfflags} \textendash{} Apply BDF flags on import.

\item {} 
\sphinxAtStartPar
\sphinxstylestrong{datacolumns} \textendash{} Dictionary defining the data types of existing columns. The format is:
\{‘data’: ‘data type 1’\}
or
\{‘data’: ‘data type 1’, ‘corrected’: ‘data type 2’\}.
For ASDMs the data type can only be RAW and one
can only specify it for the data column.
For MSes one can define two different data types
for the DATA and CORRECTED\_DATA columns and they
can be any of the known data types (RAW,
REGCAL\_CONTLINE\_ALL, REGCAL\_CONTLINE\_SCIENCE,
SELFCAL\_CONTLINE\_SCIENCE, REGCAL\_LINE\_SCIENCE,
SELFCAL\_LINE\_SCIENCE, BASELINED, ATMCORR). The
intent selection strings \_ALL or \_SCIENCE can be
skipped. In that case the task determines this
automatically by inspecting the existing intents
in the dataset.
Usually, a single datacolumns dictionary is used
for all datasets. If necessary, one can define a
list of dictionaries, one for each EB, with
different setups per EB.
If no type is specified, \{‘data’:’raw’\} will
be assumed.

\item {} 
\sphinxAtStartPar
\sphinxstylestrong{lazy} \textendash{} Use the lazy filter import

\item {} 
\sphinxAtStartPar
\sphinxstylestrong{with\_pointing\_correction} \textendash{} add (ASDM::Pointing::encoder \sphinxhyphen{} ASDM::Pointing::pointingDirection) to the value to be written in MS::Pointing::direction

\item {} 
\sphinxAtStartPar
\sphinxstylestrong{createmms} \textendash{} Create an MMS

\end{itemize}

\sphinxlineitem{Returns}
\sphinxAtStartPar
The results object for the pipeline task is returned.

\end{description}\end{quote}
\subsubsection*{Examples}
\begin{enumerate}
\sphinxsetlistlabels{\arabic}{enumi}{enumii}{}{.}%
\item {} 
\sphinxAtStartPar
Load an ASDM list in the ../rawdata subdirectory into the context.

\end{enumerate}

\begin{sphinxVerbatim}[commandchars=\\\{\}]
\PYG{g+gp}{\PYGZgt{}\PYGZgt{}\PYGZgt{} }\PYG{n}{hsd\PYGZus{}importdata} \PYG{p}{(}\PYG{n}{vis}\PYG{o}{=}\PYG{p}{[}\PYG{l+s+s1}{\PYGZsq{}}\PYG{l+s+s1}{../rawdata/uid\PYGZus{}\PYGZus{}\PYGZus{}A002\PYGZus{}X30a93d\PYGZus{}X43e}\PYG{l+s+s1}{\PYGZsq{}}\PYG{p}{,} \PYG{l+s+s1}{\PYGZsq{}}\PYG{l+s+s1}{../rawdata/uid\PYGZus{}A002\PYGZus{}x30a93d\PYGZus{}X44e}\PYG{l+s+s1}{\PYGZsq{}}\PYG{p}{]}\PYG{p}{)}
\end{sphinxVerbatim}
\begin{enumerate}
\sphinxsetlistlabels{\arabic}{enumi}{enumii}{}{.}%
\setcounter{enumi}{1}
\item {} 
\sphinxAtStartPar
Load an MS in the current directory into the context.

\end{enumerate}

\begin{sphinxVerbatim}[commandchars=\\\{\}]
\PYG{g+gp}{\PYGZgt{}\PYGZgt{}\PYGZgt{} }\PYG{n}{hsd\PYGZus{}importdata} \PYG{p}{(}\PYG{n}{vis}\PYG{o}{=}\PYG{p}{[}\PYG{l+s+s1}{\PYGZsq{}}\PYG{l+s+s1}{uid\PYGZus{}\PYGZus{}\PYGZus{}A002\PYGZus{}X30a93d\PYGZus{}X43e.ms}\PYG{l+s+s1}{\PYGZsq{}}\PYG{p}{]}\PYG{p}{)}
\end{sphinxVerbatim}
\begin{enumerate}
\sphinxsetlistlabels{\arabic}{enumi}{enumii}{}{.}%
\setcounter{enumi}{2}
\item {} 
\sphinxAtStartPar
Load a tarred ASDM in ../rawdata into the context.

\end{enumerate}

\begin{sphinxVerbatim}[commandchars=\\\{\}]
\PYG{g+gp}{\PYGZgt{}\PYGZgt{}\PYGZgt{} }\PYG{n}{hsd\PYGZus{}importdata} \PYG{p}{(}\PYG{n}{vis}\PYG{o}{=}\PYG{p}{[}\PYG{l+s+s1}{\PYGZsq{}}\PYG{l+s+s1}{../rawdata/uid\PYGZus{}\PYGZus{}\PYGZus{}A002\PYGZus{}X30a93d\PYGZus{}X43e.tar.gz}\PYG{l+s+s1}{\PYGZsq{}}\PYG{p}{]}\PYG{p}{)}
\end{sphinxVerbatim}
\begin{enumerate}
\sphinxsetlistlabels{\arabic}{enumi}{enumii}{}{.}%
\setcounter{enumi}{3}
\item {} 
\sphinxAtStartPar
Import a list of MeasurementSets.

\end{enumerate}

\begin{sphinxVerbatim}[commandchars=\\\{\}]
\PYG{g+gp}{\PYGZgt{}\PYGZgt{}\PYGZgt{} }\PYG{n}{myvislist} \PYG{o}{=} \PYG{p}{[}\PYG{l+s+s1}{\PYGZsq{}}\PYG{l+s+s1}{uid\PYGZus{}\PYGZus{}\PYGZus{}A002\PYGZus{}X30a93d\PYGZus{}X43e.ms}\PYG{l+s+s1}{\PYGZsq{}}\PYG{p}{,} \PYG{l+s+s1}{\PYGZsq{}}\PYG{l+s+s1}{uid\PYGZus{}A002\PYGZus{}x30a93d\PYGZus{}X44e.ms}\PYG{l+s+s1}{\PYGZsq{}}\PYG{p}{]}
\PYG{g+gp}{\PYGZgt{}\PYGZgt{}\PYGZgt{} }\PYG{n}{hsd\PYGZus{}importdata}\PYG{p}{(}\PYG{n}{vis}\PYG{o}{=}\PYG{n}{myvislist}\PYG{p}{)}
\end{sphinxVerbatim}

\end{fulllineitems}


\sphinxstepscope


\section{pipeline.hsd.cli.hsd\_k2jycal}
\label{\detokenize{_autosummary/pipeline.hsd.cli.hsd_k2jycal:pipeline-hsd-cli-hsd-k2jycal}}\label{\detokenize{_autosummary/pipeline.hsd.cli.hsd_k2jycal::doc}}\index{hsd\_k2jycal() (in module pipeline.hsd.cli)@\spxentry{hsd\_k2jycal()}\spxextra{in module pipeline.hsd.cli}}

\begin{fulllineitems}
\phantomsection\label{\detokenize{_autosummary/pipeline.hsd.cli.hsd_k2jycal:pipeline.hsd.cli.hsd_k2jycal}}
\pysigstartsignatures
\pysiglinewithargsret
{\sphinxbfcode{\sphinxupquote{hsd\_k2jycal}}}
{\sphinxparam{\DUrole{n}{dbservice}\DUrole{o}{=}\DUrole{default_value}{None}}\sphinxparamcomma \sphinxparam{\DUrole{n}{endpoint}\DUrole{o}{=}\DUrole{default_value}{None}}\sphinxparamcomma \sphinxparam{\DUrole{n}{reffile}\DUrole{o}{=}\DUrole{default_value}{None}}\sphinxparamcomma \sphinxparam{\DUrole{n}{infiles}\DUrole{o}{=}\DUrole{default_value}{None}}\sphinxparamcomma \sphinxparam{\DUrole{n}{caltable}\DUrole{o}{=}\DUrole{default_value}{None}}}
{}
\pysigstopsignatures
\sphinxAtStartPar
Derive Kelvin to Jy calibration tables.

\sphinxAtStartPar
Derive the Kelvin to Jy calibration for list of MeasurementSets.
\begin{quote}\begin{description}
\sphinxlineitem{Parameters}\begin{itemize}
\item {} 
\sphinxAtStartPar
\sphinxstylestrong{dbservice} \textendash{} Whether or not accessing Jy/K DB to retrieve conversion factors.

\item {} 
\sphinxAtStartPar
\sphinxstylestrong{endpoint} \textendash{} Which endpoints to use for query. Options: ‘asdm’, ‘model\sphinxhyphen{}fit’, ‘interpolation’.

\item {} 
\sphinxAtStartPar
\sphinxstylestrong{reffile} \textendash{} Path to a file containing Jy/K factors for science data, which must be provided by associating calibrator reduction or the observatory
measurements. Jy/K factor must take into account all efficiencies, i.e.,
it must be a direct conversion factor from Ta* to Jy. The file must be
in either MS\sphinxhyphen{}based or session\sphinxhyphen{}based format. The MS\sphinxhyphen{}based format must
be in an CSV format with five fields: MS name, antenna name, spectral
window id, polarization string, and Jy/K conversion factor. Example for
the file is as follows:
MS,Antenna,Spwid,Polarization,Factor
uid\_\_\_A002\_X316307\_X6f.ms,CM03,5,XX,10.0
uid\_\_\_A002\_X316307\_X6f.ms,CM03,5,YY,12.0
uid\_\_\_A002\_X316307\_X6f.ms,PM04,5,XX,2.0
uid\_\_\_A002\_X316307\_X6f.ms,PM04,5,YY,5.0
The first line in the above example is a header which may or may not
exist. Example for the session\sphinxhyphen{}based format is as follows:
\#OUSID=XXXXXX
\#OBJECT=Uranus
\#FLUXJY=yy,zz,aa
\#FLUXFREQ=YY,ZZ,AA
\#sessionID,ObservationStartDate(UTC),ObservationEndDate(UTC),
Antenna,BandCenter(MHz),BandWidth(MHz),POL,Factor
1,2011\sphinxhyphen{}11\sphinxhyphen{}11 01:00:00,2011\sphinxhyphen{}11\sphinxhyphen{}11 01:30:00,CM02,86243.0,500.0,I,10.0
1,2011\sphinxhyphen{}11\sphinxhyphen{}11 01:00:00,2011\sphinxhyphen{}11\sphinxhyphen{}11 01:30:00,CM02,86243.0,1000.0,I,30.0
1,2011\sphinxhyphen{}11\sphinxhyphen{}11 01:00:00,2011\sphinxhyphen{}11\sphinxhyphen{}11 01:30:00,CM03,86243.0,500.0,I,50.0
1,2011\sphinxhyphen{}11\sphinxhyphen{}11 01:00:00,2011\sphinxhyphen{}11\sphinxhyphen{}11 01:30:00,CM03,86243.0,1000.0,I,70.0
1,2011\sphinxhyphen{}11\sphinxhyphen{}11 01:00:00,2011\sphinxhyphen{}11\sphinxhyphen{}11 01:30:00,ANONYMOUS,86243.0,500.0,I,30.0
1,2011\sphinxhyphen{}11\sphinxhyphen{}11 01:00:00,2011\sphinxhyphen{}11\sphinxhyphen{}11 01:30:00,ANONYMOUS,86243.0,1000.0,I,50.0
2,2011\sphinxhyphen{}11\sphinxhyphen{}13 01:45:00,2011\sphinxhyphen{}11\sphinxhyphen{}13 02:15:00,PM04,86243.0,500.0,I,90.0
2,2011\sphinxhyphen{}11\sphinxhyphen{}13 01:45:00,2011\sphinxhyphen{}11\sphinxhyphen{}13 02:15:00,PM04,86243.0,1000.0,I,110.0
2,2011\sphinxhyphen{}11\sphinxhyphen{}13 01:45:00,2011\sphinxhyphen{}11\sphinxhyphen{}13 02:15:00,ANONYMOUS,86243.0,500.0,I,90.0
2,2011\sphinxhyphen{}11\sphinxhyphen{}13 01:45:00,2011\sphinxhyphen{}11\sphinxhyphen{}13 02:15:00,ANONYMOUS,86243.0,1000.0,I,110.0
Lines starting with ‘\#’ are meta data and header.
The header must exist. The factor to apply is identified by matching the
session ID, antenna name, frequency and polarization of data in each line of
the file. Note the observation date is supplementary information and not used
for the matching so far. The lines whose antenna name is ‘ANONYMOUS’ are used
when there is no measurement for specific antenna in the session. In the above
example, if science observation of session 1 contains the antenna PM04, Jy/K
factor for ANONYMOUS antenna will be applied since there is no measurement for
PM04 in session 1.
If no file name is specified or specified file doesn’t exist, all Jy/K factors
are set to 1.0.
example: reffile=’’, reffile=’working/jyperk.csv’

\item {} 
\sphinxAtStartPar
\sphinxstylestrong{infiles} \textendash{} List of input MeasurementSets.
example: vis=’ngc5921.ms’

\item {} 
\sphinxAtStartPar
\sphinxstylestrong{caltable} \textendash{} Name of output gain calibration tables.
example: caltable=’ngc5921.gcal’

\end{itemize}

\sphinxlineitem{Returns}
\sphinxAtStartPar
The results object for the pipeline task is returned.

\end{description}\end{quote}
\subsubsection*{Examples}
\begin{enumerate}
\sphinxsetlistlabels{\arabic}{enumi}{enumii}{}{.}%
\item {} 
\sphinxAtStartPar
Compute the Kevin to Jy calibration tables for a list of MeasurementSets:

\end{enumerate}

\begin{sphinxVerbatim}[commandchars=\\\{\}]
\PYG{g+gp}{\PYGZgt{}\PYGZgt{}\PYGZgt{} }\PYG{n}{hsd\PYGZus{}k2jycal}\PYG{p}{(}\PYG{p}{)}
\end{sphinxVerbatim}

\end{fulllineitems}


\sphinxstepscope


\section{pipeline.hsd.cli.hsd\_restoredata}
\label{\detokenize{_autosummary/pipeline.hsd.cli.hsd_restoredata:pipeline-hsd-cli-hsd-restoredata}}\label{\detokenize{_autosummary/pipeline.hsd.cli.hsd_restoredata::doc}}\index{hsd\_restoredata() (in module pipeline.hsd.cli)@\spxentry{hsd\_restoredata()}\spxextra{in module pipeline.hsd.cli}}

\begin{fulllineitems}
\phantomsection\label{\detokenize{_autosummary/pipeline.hsd.cli.hsd_restoredata:pipeline.hsd.cli.hsd_restoredata}}
\pysigstartsignatures
\pysiglinewithargsret
{\sphinxbfcode{\sphinxupquote{hsd\_restoredata}}}
{\sphinxparam{\DUrole{n}{vis}\DUrole{o}{=}\DUrole{default_value}{None}}\sphinxparamcomma \sphinxparam{\DUrole{n}{session}\DUrole{o}{=}\DUrole{default_value}{None}}\sphinxparamcomma \sphinxparam{\DUrole{n}{products\_dir}\DUrole{o}{=}\DUrole{default_value}{None}}\sphinxparamcomma \sphinxparam{\DUrole{n}{copytoraw}\DUrole{o}{=}\DUrole{default_value}{None}}\sphinxparamcomma \sphinxparam{\DUrole{n}{rawdata\_dir}\DUrole{o}{=}\DUrole{default_value}{None}}\sphinxparamcomma \sphinxparam{\DUrole{n}{lazy}\DUrole{o}{=}\DUrole{default_value}{None}}\sphinxparamcomma \sphinxparam{\DUrole{n}{bdfflags}\DUrole{o}{=}\DUrole{default_value}{None}}\sphinxparamcomma \sphinxparam{\DUrole{n}{ocorr\_mode}\DUrole{o}{=}\DUrole{default_value}{None}}\sphinxparamcomma \sphinxparam{\DUrole{n}{asis}\DUrole{o}{=}\DUrole{default_value}{None}}\sphinxparamcomma \sphinxparam{\DUrole{n}{hm\_rasterscan}\DUrole{o}{=}\DUrole{default_value}{None}}}
{}
\pysigstopsignatures
\sphinxAtStartPar
Restore flagged and calibration single dish data from a pipeline run.

\sphinxAtStartPar
The hsd\_restoredata task restores flagged and calibrated MeasurementSets
from archived ASDMs and pipeline flagging and calibration data products.

\sphinxAtStartPar
hsd\_restoredata assumes that the ASDMs to be restored are present in the
directory specified by the \sphinxcode{\sphinxupquote{rawdata\_dir}} (default: ‘../rawdata’).

\sphinxAtStartPar
By default (\sphinxcode{\sphinxupquote{copytoraw}} = True), hsd\_restoredata assumes that for each
ASDM in the input list, the corresponding pipeline flagging and calibration
data products (in the format produced by the hsd\_exportdata task) are
present in the directory specified by \sphinxcode{\sphinxupquote{products\_dir}} (default: ‘../products’).
At the start of the task, these products are copied from the \sphinxcode{\sphinxupquote{products\_dir}}
to the \sphinxcode{\sphinxupquote{rawdata\_dir}}.

\sphinxAtStartPar
If \sphinxcode{\sphinxupquote{copytoraw}} = False, hsd\_restoredata assumes that these products are
to be found in \sphinxcode{\sphinxupquote{rawdata\_dir}} along with the ASDMs.

\sphinxAtStartPar
The expected flagging and calibration products (for each ASDM) include:
\begin{itemize}
\item {} 
\sphinxAtStartPar
a compressed tar file of the final flagversions file, e.g.
uid\_\_\_A002\_X30a93d\_X43e.ms.flagversions.tar.gz

\item {} 
\sphinxAtStartPar
a text file containing the applycal instructions, e.g.
uid\_\_\_A002\_X30a93d\_X43e.ms.calapply.txt

\item {} 
\sphinxAtStartPar
a compressed tar file containing the caltables for the parent session,
e.g. uid\_\_\_A001\_X74\_X29.session\_3.caltables.tar.gz

\end{itemize}

\sphinxAtStartPar
hsd\_restoredata performs the following operations:
\begin{itemize}
\item {} 
\sphinxAtStartPar
imports the ASDM(s)

\item {} 
\sphinxAtStartPar
removes the default MS.flagversions directory created by the filler

\item {} 
\sphinxAtStartPar
restores the final MS.flagversions directory stored by the pipeline

\item {} 
\sphinxAtStartPar
restores the final set of pipeline flags to the MS

\item {} 
\sphinxAtStartPar
restores the final calibration state of the MS

\item {} 
\sphinxAtStartPar
restores the final calibration tables for each MS

\item {} 
\sphinxAtStartPar
applies the calibration tables to each MS

\end{itemize}

\sphinxAtStartPar
When importing the ASDM and converting it to a Measurement Set (MS), if the
output MS already exists in the output directory, then the importasdm
conversion step is skipped, and the existing MS will be imported instead.
\begin{quote}\begin{description}
\sphinxlineitem{Parameters}\begin{itemize}
\item {} 
\sphinxAtStartPar
\sphinxstylestrong{vis} \textendash{} List of raw visibility data files to be restored. Assumed to be in the directory specified by rawdata\_dir.

\sphinxAtStartPar
example: vis={[}‘uid\_\_\_A002\_X30a93d\_X43e’{]}

\item {} 
\sphinxAtStartPar
\sphinxstylestrong{session} \textendash{} List of sessions one per visibility file.
example: session={[}‘session\_3’{]}

\item {} 
\sphinxAtStartPar
\sphinxstylestrong{products\_dir} \textendash{} Name of the data products directory to copy calibration products from.
Default: ‘../products’

\sphinxAtStartPar
The parameter is effective only when \sphinxcode{\sphinxupquote{copytoraw}} = True.
When \sphinxcode{\sphinxupquote{copytoraw}} = False, calibration products in
\sphinxcode{\sphinxupquote{rawdata\_dir}} will be used.

\sphinxAtStartPar
example: products\_dir=’myproductspath’

\item {} 
\sphinxAtStartPar
\sphinxstylestrong{copytoraw} \textendash{} Copy calibration and flagging tables from \sphinxcode{\sphinxupquote{products\_dir}} to \sphinxcode{\sphinxupquote{rawdata\_dir}} directory.
Default: True.

\sphinxAtStartPar
example: copytoraw=False

\item {} 
\sphinxAtStartPar
\sphinxstylestrong{rawdata\_dir} \textendash{} Name of the raw data directory. Default: ‘../rawdata’

\sphinxAtStartPar
example: rawdata\_dir=’myrawdatapath’

\item {} 
\sphinxAtStartPar
\sphinxstylestrong{lazy} \textendash{} Use the lazy filler option Default: False.

\sphinxAtStartPar
example: lazy=True

\item {} 
\sphinxAtStartPar
\sphinxstylestrong{bdfflags} \textendash{} Set the BDF flags Default: True.

\sphinxAtStartPar
example: bdfflags=False

\item {} 
\sphinxAtStartPar
\sphinxstylestrong{ocorr\_mode} \textendash{} Set ocorr\_mode. Default: ‘ao’.

\sphinxAtStartPar
example: ocorr\_mode=’ca’

\item {} 
\sphinxAtStartPar
\sphinxstylestrong{asis} \textendash{} Creates verbatim copies of the ASDM tables in the output MS. The value given to this option must be a list of table names separated by space characters.
Default: ‘SBSummary ExecBlock Annotation Antenna Station Receiver Source CalAtmosphere CalWVR’.

\sphinxAtStartPar
example: asis=’Source Receiver’

\item {} 
\sphinxAtStartPar
\sphinxstylestrong{hm\_rasterscan} \textendash{} Heuristics method for raster scan analysis. Two analysis modes, time\sphinxhyphen{}domain analysis (‘time’) and direction analysis (‘direction’), are available.
Default: ‘time’

\end{itemize}

\sphinxlineitem{Returns}
\sphinxAtStartPar
The results object for the pipeline task is returned.

\end{description}\end{quote}
\subsubsection*{Examples}
\begin{enumerate}
\sphinxsetlistlabels{\arabic}{enumi}{enumii}{}{.}%
\item {} 
\sphinxAtStartPar
Restore the pipeline results for a single ASDM in a single session

\end{enumerate}

\begin{sphinxVerbatim}[commandchars=\\\{\}]
\PYG{g+gp}{\PYGZgt{}\PYGZgt{}\PYGZgt{} }\PYG{n}{hsd\PYGZus{}restoredata} \PYG{p}{(}\PYG{n}{vis}\PYG{o}{=}\PYG{p}{[}\PYG{l+s+s1}{\PYGZsq{}}\PYG{l+s+s1}{uid\PYGZus{}\PYGZus{}\PYGZus{}A002\PYGZus{}X30a93d\PYGZus{}X43e}\PYG{l+s+s1}{\PYGZsq{}}\PYG{p}{]}\PYG{p}{,} \PYG{n}{session}\PYG{o}{=}\PYG{p}{[}\PYG{l+s+s1}{\PYGZsq{}}\PYG{l+s+s1}{session\PYGZus{}1}\PYG{l+s+s1}{\PYGZsq{}}\PYG{p}{]}\PYG{p}{,} \PYG{n}{ocorr\PYGZus{}mode}\PYG{o}{=}\PYG{l+s+s1}{\PYGZsq{}}\PYG{l+s+s1}{ao}\PYG{l+s+s1}{\PYGZsq{}}\PYG{p}{)}
\end{sphinxVerbatim}

\end{fulllineitems}


\sphinxstepscope


\section{pipeline.hsd.cli.hsd\_skycal}
\label{\detokenize{_autosummary/pipeline.hsd.cli.hsd_skycal:pipeline-hsd-cli-hsd-skycal}}\label{\detokenize{_autosummary/pipeline.hsd.cli.hsd_skycal::doc}}\index{hsd\_skycal() (in module pipeline.hsd.cli)@\spxentry{hsd\_skycal()}\spxextra{in module pipeline.hsd.cli}}

\begin{fulllineitems}
\phantomsection\label{\detokenize{_autosummary/pipeline.hsd.cli.hsd_skycal:pipeline.hsd.cli.hsd_skycal}}
\pysigstartsignatures
\pysiglinewithargsret
{\sphinxbfcode{\sphinxupquote{hsd\_skycal}}}
{\sphinxparam{\DUrole{n}{calmode}\DUrole{o}{=}\DUrole{default_value}{None}}\sphinxparamcomma \sphinxparam{\DUrole{n}{fraction}\DUrole{o}{=}\DUrole{default_value}{None}}\sphinxparamcomma \sphinxparam{\DUrole{n}{noff}\DUrole{o}{=}\DUrole{default_value}{None}}\sphinxparamcomma \sphinxparam{\DUrole{n}{width}\DUrole{o}{=}\DUrole{default_value}{None}}\sphinxparamcomma \sphinxparam{\DUrole{n}{elongated}\DUrole{o}{=}\DUrole{default_value}{None}}\sphinxparamcomma \sphinxparam{\DUrole{n}{parallel}\DUrole{o}{=}\DUrole{default_value}{None}}\sphinxparamcomma \sphinxparam{\DUrole{n}{infiles}\DUrole{o}{=}\DUrole{default_value}{None}}\sphinxparamcomma \sphinxparam{\DUrole{n}{field}\DUrole{o}{=}\DUrole{default_value}{None}}\sphinxparamcomma \sphinxparam{\DUrole{n}{spw}\DUrole{o}{=}\DUrole{default_value}{None}}\sphinxparamcomma \sphinxparam{\DUrole{n}{scan}\DUrole{o}{=}\DUrole{default_value}{None}}}
{}
\pysigstopsignatures
\sphinxAtStartPar
Calibrate data.

\sphinxAtStartPar
The hsd\_skycal generates a caltable for sky calibration that stores
reference spectra, which is to be subtracted from on\sphinxhyphen{}source spectra to filter
out non\sphinxhyphen{}source contribution.
\begin{quote}\begin{description}
\sphinxlineitem{Parameters}\begin{itemize}
\item {} 
\sphinxAtStartPar
\sphinxstylestrong{calmode} \textendash{} Calibration mode. Available options are ‘auto’ (default), ‘ps’, ‘otf’, and ‘otfraster’. When ‘auto’ is set, the task will
use preset calibration mode that is determined by inspecting data.
‘ps’ mode is simple position switching using explicit reference
scans. Other two modes, ‘otf’ and ‘otfraster’, will generate
reference data from scans at the edge of the map. Those modes
are intended for OTF observation and the former is defined for
generic scanning pattern such as Lissajous, while the latter is
specific use for raster scan.
options: ‘auto’, ‘ps’, ‘otf’, ‘otfraster’

\item {} 
\sphinxAtStartPar
\sphinxstylestrong{fraction} \textendash{} Sub\sphinxhyphen{}parameter for calmode. Edge marking parameter for ‘otf’ and ‘otfraster’ mode. It specifies a number of OFF scans
as a fraction of total number of data points.
options: String style like ‘20\%’, or float value less than 1.0.
For ‘otfraster’ mode, you can also specify ‘auto’.

\item {} 
\sphinxAtStartPar
\sphinxstylestrong{noff} \textendash{} Sub\sphinxhyphen{}parameter for calmode. Edge marking parameter for ‘otfraster’ mode. It is used to specify a number of OFF scans near edge directly
instead to specify it by fractional number by ‘fraction’. If it is
set, the value will come before setting by ‘fraction’.
options: any positive integer value

\item {} 
\sphinxAtStartPar
\sphinxstylestrong{width} \textendash{} Sub\sphinxhyphen{}parameter for calmode. Edge marking parameter for ‘otf’ mode. It specifies pixel width with respect to a median spatial
separation between neighboring two data in time. Default will
be fine in most cases.
options: any float value

\item {} 
\sphinxAtStartPar
\sphinxstylestrong{elongated} \textendash{} Sub\sphinxhyphen{}parameter for calmode. Edge marking parameter for ‘otf’ mode. Please set True only if observed area is elongated
in one direction.

\item {} 
\sphinxAtStartPar
\sphinxstylestrong{parallel} \textendash{} Execute using CASA HPC functionality, if available. Options: ‘automatic’, ‘true’, ‘false’, True, False.
default: None (equivalent to ‘automatic’)

\item {} 
\sphinxAtStartPar
\sphinxstylestrong{infiles} \textendash{} List of data files. These must be a name of MeasurementSets that are registered to context via hsd\_importdata task.
example: vis={[}‘X227.ms’, ‘X228.ms’{]}

\item {} 
\sphinxAtStartPar
\sphinxstylestrong{field} \textendash{} Data selection by field name.

\item {} 
\sphinxAtStartPar
\sphinxstylestrong{spw} \textendash{} Data selection by spw. (defalut all spws) example: ‘3,4’ (generate caltable for spw 3 and 4),
{[}‘0’,’2’{]} (spw 0 for first data, 2 for second)

\item {} 
\sphinxAtStartPar
\sphinxstylestrong{scan} \textendash{} Data selection by scan number. (default all scans) example: ‘22,23’ (use scan 22 and 23 for calibration),
{[}‘22’,’24’{]} (scan 22 for first data, 24 for second)

\end{itemize}

\sphinxlineitem{Returns}
\sphinxAtStartPar
The results object for the pipeline task is returned.

\end{description}\end{quote}
\subsubsection*{Examples}
\begin{enumerate}
\sphinxsetlistlabels{\arabic}{enumi}{enumii}{}{.}%
\item {} 
\sphinxAtStartPar
Generate caltables for all data managed by context.

\end{enumerate}

\begin{sphinxVerbatim}[commandchars=\\\{\}]
\PYG{g+gp}{\PYGZgt{}\PYGZgt{}\PYGZgt{} }\PYG{n}{default}\PYG{p}{(}\PYG{n}{hsd\PYGZus{}skycal}\PYG{p}{)}
\PYG{g+gp}{\PYGZgt{}\PYGZgt{}\PYGZgt{} }\PYG{n}{hsd\PYGZus{}skycal}\PYG{p}{(}\PYG{p}{)}
\end{sphinxVerbatim}

\end{fulllineitems}


\sphinxstepscope


\section{pipeline.hsd.cli.hsd\_tsysflag}
\label{\detokenize{_autosummary/pipeline.hsd.cli.hsd_tsysflag:pipeline-hsd-cli-hsd-tsysflag}}\label{\detokenize{_autosummary/pipeline.hsd.cli.hsd_tsysflag::doc}}\index{hsd\_tsysflag() (in module pipeline.hsd.cli)@\spxentry{hsd\_tsysflag()}\spxextra{in module pipeline.hsd.cli}}

\begin{fulllineitems}
\phantomsection\label{\detokenize{_autosummary/pipeline.hsd.cli.hsd_tsysflag:pipeline.hsd.cli.hsd_tsysflag}}
\pysigstartsignatures
\pysiglinewithargsret
{\sphinxbfcode{\sphinxupquote{hsd\_tsysflag}}}
{\sphinxparam{\DUrole{n}{vis}\DUrole{o}{=}\DUrole{default_value}{None}}\sphinxparamcomma \sphinxparam{\DUrole{n}{caltable}\DUrole{o}{=}\DUrole{default_value}{None}}\sphinxparamcomma \sphinxparam{\DUrole{n}{flag\_nmedian}\DUrole{o}{=}\DUrole{default_value}{None}}\sphinxparamcomma \sphinxparam{\DUrole{n}{fnm\_limit}\DUrole{o}{=}\DUrole{default_value}{None}}\sphinxparamcomma \sphinxparam{\DUrole{n}{fnm\_byfield}\DUrole{o}{=}\DUrole{default_value}{None}}\sphinxparamcomma \sphinxparam{\DUrole{n}{flag\_derivative}\DUrole{o}{=}\DUrole{default_value}{None}}\sphinxparamcomma \sphinxparam{\DUrole{n}{fd\_max\_limit}\DUrole{o}{=}\DUrole{default_value}{None}}\sphinxparamcomma \sphinxparam{\DUrole{n}{flag\_edgechans}\DUrole{o}{=}\DUrole{default_value}{None}}\sphinxparamcomma \sphinxparam{\DUrole{n}{fe\_edge\_limit}\DUrole{o}{=}\DUrole{default_value}{None}}\sphinxparamcomma \sphinxparam{\DUrole{n}{flag\_fieldshape}\DUrole{o}{=}\DUrole{default_value}{None}}\sphinxparamcomma \sphinxparam{\DUrole{n}{ff\_refintent}\DUrole{o}{=}\DUrole{default_value}{None}}\sphinxparamcomma \sphinxparam{\DUrole{n}{ff\_max\_limit}\DUrole{o}{=}\DUrole{default_value}{None}}\sphinxparamcomma \sphinxparam{\DUrole{n}{flag\_birdies}\DUrole{o}{=}\DUrole{default_value}{None}}\sphinxparamcomma \sphinxparam{\DUrole{n}{fb\_sharps\_limit}\DUrole{o}{=}\DUrole{default_value}{None}}\sphinxparamcomma \sphinxparam{\DUrole{n}{flag\_toomany}\DUrole{o}{=}\DUrole{default_value}{None}}\sphinxparamcomma \sphinxparam{\DUrole{n}{tmf1\_limit}\DUrole{o}{=}\DUrole{default_value}{None}}\sphinxparamcomma \sphinxparam{\DUrole{n}{tmef1\_limit}\DUrole{o}{=}\DUrole{default_value}{None}}\sphinxparamcomma \sphinxparam{\DUrole{n}{metric\_order}\DUrole{o}{=}\DUrole{default_value}{None}}\sphinxparamcomma \sphinxparam{\DUrole{n}{normalize\_tsys}\DUrole{o}{=}\DUrole{default_value}{None}}\sphinxparamcomma \sphinxparam{\DUrole{n}{filetemplate}\DUrole{o}{=}\DUrole{default_value}{None}}}
{}
\pysigstopsignatures
\sphinxAtStartPar
Flag deviant system temperature measurements.

\sphinxAtStartPar
Flag deviant system temperature measurements for single dish measurements. This is done by running a
sequence of flagging sub\sphinxhyphen{}tasks (tests), each looking for a different type of possible error.

\sphinxAtStartPar
If a file with manual Tsys flags is provided with the ‘filetemplate’
parameter, then these flags are applied prior to the evaluation of the
flagging heuristics listed below.

\sphinxAtStartPar
The tests are:
\begin{quote}
\begin{enumerate}
\sphinxsetlistlabels{\arabic}{enumi}{enumii}{}{.}%
\item {} 
\sphinxAtStartPar
Flag Tsys spectra with high median values

\item {} 
\sphinxAtStartPar
Flag Tsys spectra with high median derivatives. This is meant to spot

\end{enumerate}
\begin{quote}

\sphinxAtStartPar
spectra that are ‘ringing’.
\end{quote}
\begin{enumerate}
\sphinxsetlistlabels{\arabic}{enumi}{enumii}{}{.}%
\setcounter{enumi}{2}
\item {} 
\sphinxAtStartPar
Flag the edge channels of the Tsys spectra in each SpW.

\item {} 
\sphinxAtStartPar
Flag Tsys spectra whose shape is different from that associated with

\end{enumerate}
\begin{quote}

\sphinxAtStartPar
the BANDPASS intent.
\end{quote}
\begin{enumerate}
\sphinxsetlistlabels{\arabic}{enumi}{enumii}{}{.}%
\setcounter{enumi}{4}
\item {} 
\sphinxAtStartPar
Flag ‘birdies’.

\item {} 
\sphinxAtStartPar
Flag the Tsys spectra of all antennas in a timestamp and spw if

\end{enumerate}
\begin{quote}

\sphinxAtStartPar
proportion of antennas already flagged in this timestamp and spw exceeds
a threshold, and flag Tsys spectra for all antennas and all timestamps
in a spw, if proportion of antennas that are already entirely flagged
in all timestamps exceeds a threshold.
\end{quote}
\end{quote}
\begin{quote}\begin{description}
\sphinxlineitem{Parameters}\begin{itemize}
\item {} 
\sphinxAtStartPar
\sphinxstylestrong{vis} \textendash{} List of input MeasurementSets (Not used)

\item {} 
\sphinxAtStartPar
\sphinxstylestrong{caltable} \textendash{} List of input Tsys calibration tables.
default: {[}{]} \sphinxhyphen{} Use the table currently stored in the pipeline context.
example: caltable={[}‘X132.ms.tsys.s2.tbl’{]}

\item {} 
\sphinxAtStartPar
\sphinxstylestrong{flag\_nmedian} \textendash{} True to flag Tsys spectra with high median value.

\item {} 
\sphinxAtStartPar
\sphinxstylestrong{fnm\_limit} \textendash{} Flag spectra with median value higher than fnm\_limit * median of this measure over all spectra.

\item {} 
\sphinxAtStartPar
\sphinxstylestrong{fnm\_byfield} \textendash{} Evaluate the nmedian metric separately for each field.

\item {} 
\sphinxAtStartPar
\sphinxstylestrong{flag\_derivative} \textendash{} True to flag Tsys spectra with high median derivative.

\item {} 
\sphinxAtStartPar
\sphinxstylestrong{fd\_max\_limit} \textendash{} Flag spectra with median derivative higher than fd\_max\_limit * median of this measure over all spectra.

\item {} 
\sphinxAtStartPar
\sphinxstylestrong{flag\_edgechans} \textendash{} True to flag edges of Tsys spectra.

\item {} 
\sphinxAtStartPar
\sphinxstylestrong{fe\_edge\_limit} \textendash{} Flag channels whose channel to channel difference \textgreater{} fe\_edge\_limit * median across spectrum.

\item {} 
\sphinxAtStartPar
\sphinxstylestrong{flag\_fieldshape} \textendash{} True to flag Tsys spectra with a radically different shape to those of the ff\_refintent.

\item {} 
\sphinxAtStartPar
\sphinxstylestrong{ff\_refintent} \textendash{} Data intent that provides the reference shape for ‘flag\_fieldshape’.

\item {} 
\sphinxAtStartPar
\sphinxstylestrong{ff\_max\_limit} \textendash{} Flag Tsys spectra with ‘fieldshape’ metric values \textgreater{} ff\_max\_limit.

\item {} 
\sphinxAtStartPar
\sphinxstylestrong{flag\_birdies} \textendash{} True to flag channels covering sharp spectral features.

\item {} 
\sphinxAtStartPar
\sphinxstylestrong{fb\_sharps\_limit} \textendash{} Flag channels bracketing a channel to channel difference \textgreater{} fb\_sharps\_limit.

\item {} 
\sphinxAtStartPar
\sphinxstylestrong{flag\_toomany} \textendash{} True to flag Tsys spectra for which a proportion of antennas for given timestamp and/or proportion of antennas that are
entirely flagged in all timestamps exceeds their respective thresholds.

\item {} 
\sphinxAtStartPar
\sphinxstylestrong{tmf1\_limit} \textendash{} Flag Tsys spectra for all antennas in a timestamp and spw if proportion of antennas already flagged in this timestamp and spw exceeds
tmf1\_limit.

\item {} 
\sphinxAtStartPar
\sphinxstylestrong{tmef1\_limit} \textendash{} Flag Tsys spectra for all antennas and all timestamps in a spw, if proportion of antennas that are already entirely flagged
in all timestamps exceeds tmef1\_limit.

\item {} 
\sphinxAtStartPar
\sphinxstylestrong{metric\_order} \textendash{} Order in which to evaluate the flagging metrics that are enabled. Disabled metrics are skipped.

\item {} 
\sphinxAtStartPar
\sphinxstylestrong{normalize\_tsys} \textendash{} True to create a normalized Tsys table that is used to evaluate the Tsys flagging metrics. All newly found flags are also applied
to the original Tsys caltable that continues to be used for subsequent
calibration.

\item {} 
\sphinxAtStartPar
\sphinxstylestrong{filetemplate} \textendash{} The name of a text file that contains the manual Tsys flagging template. If the template flags file is undefined, a name of the form
‘msname.flagtsystemplate.txt’ is assumed.

\end{itemize}

\sphinxlineitem{Returns}
\sphinxAtStartPar
The results object for the pipeline task is returned.

\end{description}\end{quote}
\subsubsection*{Examples}
\begin{enumerate}
\sphinxsetlistlabels{\arabic}{enumi}{enumii}{}{.}%
\item {} 
\sphinxAtStartPar
Flag Tsys measurements using currently recommended tests:

\end{enumerate}

\begin{sphinxVerbatim}[commandchars=\\\{\}]
\PYG{g+gp}{\PYGZgt{}\PYGZgt{}\PYGZgt{} }\PYG{n}{hsd\PYGZus{}tsysflag}\PYG{p}{(}\PYG{p}{)}
\end{sphinxVerbatim}

\sphinxAtStartPar
2. Flag Tsys measurements using all recommended tests apart from that
using the ‘fieldshape’ metric:

\begin{sphinxVerbatim}[commandchars=\\\{\}]
\PYG{g+gp}{\PYGZgt{}\PYGZgt{}\PYGZgt{} }\PYG{n}{hsd\PYGZus{}tsysflag}\PYG{p}{(}\PYG{n}{flag\PYGZus{}fieldshape}\PYG{o}{=}\PYG{k+kc}{False}\PYG{p}{)}
\end{sphinxVerbatim}

\end{fulllineitems}


\sphinxstepscope


\chapter{pipeline.hsdn.cli}
\label{\detokenize{_autosummary/pipeline.hsdn.cli:module-pipeline.hsdn.cli}}\label{\detokenize{_autosummary/pipeline.hsdn.cli:pipeline-hsdn-cli}}\label{\detokenize{_autosummary/pipeline.hsdn.cli::doc}}\index{module@\spxentry{module}!pipeline.hsdn.cli@\spxentry{pipeline.hsdn.cli}}\index{pipeline.hsdn.cli@\spxentry{pipeline.hsdn.cli}!module@\spxentry{module}}
\sphinxAtStartPar
Nobeyama Tasks
\subsubsection*{Functions}


\begin{savenotes}\sphinxattablestart
\sphinxthistablewithglobalstyle
\sphinxthistablewithnovlinesstyle
\centering
\begin{tabulary}{\linewidth}[t]{\X{1}{2}\X{1}{2}}
\sphinxtoprule
\sphinxtableatstartofbodyhook
\sphinxAtStartPar
{\hyperref[\detokenize{_autosummary/pipeline.hsdn.cli.hsdn_exportdata:pipeline.hsdn.cli.hsdn_exportdata}]{\sphinxcrossref{\sphinxcode{\sphinxupquote{hsdn\_exportdata}}}}}
&
\sphinxAtStartPar
Prepare single dish data for export.
\\
\sphinxhline
\sphinxAtStartPar
{\hyperref[\detokenize{_autosummary/pipeline.hsdn.cli.hsdn_importdata:pipeline.hsdn.cli.hsdn_importdata}]{\sphinxcrossref{\sphinxcode{\sphinxupquote{hsdn\_importdata}}}}}
&
\sphinxAtStartPar
Imports Nobeyama data into the single dish pipeline.
\\
\sphinxhline
\sphinxAtStartPar
{\hyperref[\detokenize{_autosummary/pipeline.hsdn.cli.hsdn_restoredata:pipeline.hsdn.cli.hsdn_restoredata}]{\sphinxcrossref{\sphinxcode{\sphinxupquote{hsdn\_restoredata}}}}}
&
\sphinxAtStartPar
Restore flagged and calibration single dish data from a pipeline run.
\\
\sphinxbottomrule
\end{tabulary}
\sphinxtableafterendhook\par
\sphinxattableend\end{savenotes}

\sphinxstepscope


\section{pipeline.hsdn.cli.hsdn\_exportdata}
\label{\detokenize{_autosummary/pipeline.hsdn.cli.hsdn_exportdata:pipeline-hsdn-cli-hsdn-exportdata}}\label{\detokenize{_autosummary/pipeline.hsdn.cli.hsdn_exportdata::doc}}\index{hsdn\_exportdata() (in module pipeline.hsdn.cli)@\spxentry{hsdn\_exportdata()}\spxextra{in module pipeline.hsdn.cli}}

\begin{fulllineitems}
\phantomsection\label{\detokenize{_autosummary/pipeline.hsdn.cli.hsdn_exportdata:pipeline.hsdn.cli.hsdn_exportdata}}
\pysigstartsignatures
\pysiglinewithargsret
{\sphinxbfcode{\sphinxupquote{hsdn\_exportdata}}}
{\sphinxparam{\DUrole{n}{pprfile}\DUrole{o}{=}\DUrole{default_value}{None}}\sphinxparamcomma \sphinxparam{\DUrole{n}{targetimages}\DUrole{o}{=}\DUrole{default_value}{None}}\sphinxparamcomma \sphinxparam{\DUrole{n}{products\_dir}\DUrole{o}{=}\DUrole{default_value}{None}}}
{}
\pysigstopsignatures
\sphinxAtStartPar
Prepare single dish data for export.

\sphinxAtStartPar
The hsdn\_exportdata task exports the data defined in the pipeline context
and exports it to the data products directory, converting and or packing
it as necessary.

\sphinxAtStartPar
The current version of the task exports the following products
\begin{itemize}
\item {} 
\sphinxAtStartPar
a FITS image for each selected science target source image

\item {} 
\sphinxAtStartPar
a tar file per ASDM containing the final flags version and blparam

\item {} 
\sphinxAtStartPar
a tar file containing the file web log

\end{itemize}
\begin{quote}\begin{description}
\sphinxlineitem{Parameters}\begin{itemize}
\item {} 
\sphinxAtStartPar
\sphinxstylestrong{pprfile} \textendash{} Name of the pipeline processing request to be exported. Defaults to a file matching the template ‘PPR\_*.xml’.
example: pprfile={[}‘PPR\_GRB021004.xml’{]}

\item {} 
\sphinxAtStartPar
\sphinxstylestrong{targetimages} \textendash{} List of science target images to be exported. Defaults to all science target images recorded in the pipeline context.
example: targetimages={[}‘r\_aqr.CM02.spw5.line0.XXYY.sd.im’, ‘r\_aqr.CM02.spw5.XXYY.sd.cont.im’{]}

\item {} 
\sphinxAtStartPar
\sphinxstylestrong{products\_dir} \textendash{} Name of the data products subdirectory. Defaults to ‘./’. Example: products\_dir=’../products’.

\end{itemize}

\sphinxlineitem{Returns}
\sphinxAtStartPar
The results object for the pipeline task is returned.

\end{description}\end{quote}
\subsubsection*{Examples}

\sphinxAtStartPar
1. Export the pipeline results for a single session to the data products
directory

\begin{sphinxVerbatim}[commandchars=\\\{\}]
\PYGZgt{}\PYGZgt{}\PYGZgt{} !mkdir ../products
\PYGZgt{}\PYGZgt{}\PYGZgt{} hsdn\PYGZus{}exportdata (products\PYGZus{}dir=\PYGZsq{}../products\PYGZsq{})
\end{sphinxVerbatim}

\end{fulllineitems}


\sphinxstepscope


\section{pipeline.hsdn.cli.hsdn\_importdata}
\label{\detokenize{_autosummary/pipeline.hsdn.cli.hsdn_importdata:pipeline-hsdn-cli-hsdn-importdata}}\label{\detokenize{_autosummary/pipeline.hsdn.cli.hsdn_importdata::doc}}\index{hsdn\_importdata() (in module pipeline.hsdn.cli)@\spxentry{hsdn\_importdata()}\spxextra{in module pipeline.hsdn.cli}}

\begin{fulllineitems}
\phantomsection\label{\detokenize{_autosummary/pipeline.hsdn.cli.hsdn_importdata:pipeline.hsdn.cli.hsdn_importdata}}
\pysigstartsignatures
\pysiglinewithargsret
{\sphinxbfcode{\sphinxupquote{hsdn\_importdata}}}
{\sphinxparam{\DUrole{n}{vis}\DUrole{o}{=}\DUrole{default_value}{None}}\sphinxparamcomma \sphinxparam{\DUrole{n}{session}\DUrole{o}{=}\DUrole{default_value}{None}}\sphinxparamcomma \sphinxparam{\DUrole{n}{hm\_rasterscan}\DUrole{o}{=}\DUrole{default_value}{None}}\sphinxparamcomma \sphinxparam{\DUrole{n}{datacolumns}\DUrole{o}{=}\DUrole{default_value}{None}}\sphinxparamcomma \sphinxparam{\DUrole{n}{overwrite}\DUrole{o}{=}\DUrole{default_value}{None}}\sphinxparamcomma \sphinxparam{\DUrole{n}{nocopy}\DUrole{o}{=}\DUrole{default_value}{None}}\sphinxparamcomma \sphinxparam{\DUrole{n}{createmms}\DUrole{o}{=}\DUrole{default_value}{None}}}
{}
\pysigstopsignatures
\sphinxAtStartPar
Imports Nobeyama data into the single dish pipeline.

\sphinxAtStartPar
Imports Nobeyama data into the single dish pipeline.
The hsdn\_importdata task loads the specified visibility data into the pipeline
context unpacking and / or converting it as necessary.

\sphinxAtStartPar
If the \sphinxcode{\sphinxupquote{overwrite}} input parameter is set to False and the task is asked to
convert an input ASDM input to an MS, then when the output MS already exists in
the output directory, the importasdm conversion step is skipped, and the
existing MS will be imported instead.
\begin{quote}\begin{description}
\sphinxlineitem{Parameters}\begin{itemize}
\item {} 
\sphinxAtStartPar
\sphinxstylestrong{vis} \textendash{} List of visibility data files. These may be ASDMs, tar files of ASDMs, MSes, or tar files of MSes, If ASDM files are specified, they will be
converted to MS format.
example: vis={[}‘X227.ms’, ‘asdms.tar.gz’{]}

\item {} 
\sphinxAtStartPar
\sphinxstylestrong{session} \textendash{} List of sessions to which the visibility files belong. Defaults to a single session containing all the visibility files, otherwise
a session must be assigned to each vis file.
example: session={[}‘Session\_1’, ‘Sessions\_2’{]}

\item {} 
\sphinxAtStartPar
\sphinxstylestrong{hm\_rasterscan} \textendash{} Heuristics method for raster scan analysis. Two analysis modes, time\sphinxhyphen{}domain analysis (‘time’) and direction analysis (‘direction’), are available.
Default is ‘time’.

\item {} 
\sphinxAtStartPar
\sphinxstylestrong{datacolumns} \textendash{} Dictionary defining the data types of existing columns. The format is:
\{‘data’: ‘data type 1’\}
or
\{‘data’: ‘data type 1’, ‘corrected’: ‘data type 2’\}.
For ASDMs the data type can only be RAW and one
can only specify it for the data column.
For MSes one can define two different data types
for the DATA and CORRECTED\_DATA columns and they
can be any of the known data types (RAW,
REGCAL\_CONTLINE\_ALL, REGCAL\_CONTLINE\_SCIENCE,
SELFCAL\_CONTLINE\_SCIENCE, REGCAL\_LINE\_SCIENCE,
SELFCAL\_LINE\_SCIENCE, BASELINED, ATMCORR). The
intent selection strings \_ALL or \_SCIENCE can be
skipped. In that case the task determines this
automatically by inspecting the existing intents
in the dataset.
Usually, a single datacolumns dictionary is used
for all datasets. If necessary, one can define a
list of dictionaries, one for each EB, with
different setups per EB.
If no type is specified, \{‘data’:’raw’\} will
be assumed.

\item {} 
\sphinxAtStartPar
\sphinxstylestrong{overwrite} \textendash{} Overwrite existing files on import. When converting ASDM to MS, if overwrite=False and the MS already
exists in output directory, then this existing MS dataset will be used
instead.

\item {} 
\sphinxAtStartPar
\sphinxstylestrong{nocopy} \textendash{} Disable copying of MS to working directory.

\item {} 
\sphinxAtStartPar
\sphinxstylestrong{createmms} \textendash{} Create an MMS

\end{itemize}

\sphinxlineitem{Returns}
\sphinxAtStartPar
The results object for the pipeline task is returned.

\end{description}\end{quote}
\subsubsection*{Examples}
\begin{enumerate}
\sphinxsetlistlabels{\arabic}{enumi}{enumii}{}{.}%
\item {} 
\sphinxAtStartPar
Load an ASDM list in the ../rawdata subdirectory into the context:

\end{enumerate}

\begin{sphinxVerbatim}[commandchars=\\\{\}]
\PYG{g+gp}{\PYGZgt{}\PYGZgt{}\PYGZgt{} }\PYG{n}{hsdn\PYGZus{}importdata} \PYG{p}{(}\PYG{n}{vis}\PYG{o}{=}\PYG{p}{[}\PYG{l+s+s1}{\PYGZsq{}}\PYG{l+s+s1}{../rawdata/uid\PYGZus{}\PYGZus{}\PYGZus{}A002\PYGZus{}X30a93d\PYGZus{}X43e}\PYG{l+s+s1}{\PYGZsq{}}\PYG{p}{,} \PYG{l+s+s1}{\PYGZsq{}}\PYG{l+s+s1}{../rawdata/uid\PYGZus{}A002\PYGZus{}x30a93d\PYGZus{}X44e}\PYG{l+s+s1}{\PYGZsq{}}\PYG{p}{]}\PYG{p}{)}
\end{sphinxVerbatim}
\begin{enumerate}
\sphinxsetlistlabels{\arabic}{enumi}{enumii}{}{.}%
\setcounter{enumi}{1}
\item {} 
\sphinxAtStartPar
Load an MS in the current directory into the context:

\end{enumerate}

\begin{sphinxVerbatim}[commandchars=\\\{\}]
\PYG{g+gp}{\PYGZgt{}\PYGZgt{}\PYGZgt{} }\PYG{n}{hsdn\PYGZus{}importdata} \PYG{p}{(}\PYG{n}{vis}\PYG{o}{=}\PYG{p}{[}\PYG{l+s+s1}{\PYGZsq{}}\PYG{l+s+s1}{uid\PYGZus{}\PYGZus{}\PYGZus{}A002\PYGZus{}X30a93d\PYGZus{}X43e.ms}\PYG{l+s+s1}{\PYGZsq{}}\PYG{p}{]}\PYG{p}{)}
\end{sphinxVerbatim}
\begin{enumerate}
\sphinxsetlistlabels{\arabic}{enumi}{enumii}{}{.}%
\setcounter{enumi}{2}
\item {} 
\sphinxAtStartPar
Load a tarred ASDM in ../rawdata into the context:

\end{enumerate}

\begin{sphinxVerbatim}[commandchars=\\\{\}]
\PYG{g+gp}{\PYGZgt{}\PYGZgt{}\PYGZgt{} }\PYG{n}{hsdn\PYGZus{}importdata} \PYG{p}{(}\PYG{n}{vis}\PYG{o}{=}\PYG{p}{[}\PYG{l+s+s1}{\PYGZsq{}}\PYG{l+s+s1}{../rawdata/uid\PYGZus{}\PYGZus{}\PYGZus{}A002\PYGZus{}X30a93d\PYGZus{}X43e.tar.gz}\PYG{l+s+s1}{\PYGZsq{}}\PYG{p}{]}\PYG{p}{)}
\end{sphinxVerbatim}
\begin{enumerate}
\sphinxsetlistlabels{\arabic}{enumi}{enumii}{}{.}%
\setcounter{enumi}{3}
\item {} 
\sphinxAtStartPar
Import a list of MeasurementSets:

\end{enumerate}

\begin{sphinxVerbatim}[commandchars=\\\{\}]
\PYG{g+gp}{\PYGZgt{}\PYGZgt{}\PYGZgt{} }\PYG{n}{myvislist} \PYG{o}{=} \PYG{p}{[}\PYG{l+s+s1}{\PYGZsq{}}\PYG{l+s+s1}{uid\PYGZus{}\PYGZus{}\PYGZus{}A002\PYGZus{}X30a93d\PYGZus{}X43e.ms}\PYG{l+s+s1}{\PYGZsq{}}\PYG{p}{,} \PYG{l+s+s1}{\PYGZsq{}}\PYG{l+s+s1}{uid\PYGZus{}A002\PYGZus{}x30a93d\PYGZus{}X44e.ms}\PYG{l+s+s1}{\PYGZsq{}}\PYG{p}{]}
\PYG{g+gp}{\PYGZgt{}\PYGZgt{}\PYGZgt{} }\PYG{n}{hsdn\PYGZus{}importdata}\PYG{p}{(}\PYG{n}{vis}\PYG{o}{=}\PYG{n}{myvislist}\PYG{p}{)}
\end{sphinxVerbatim}

\end{fulllineitems}


\sphinxstepscope


\section{pipeline.hsdn.cli.hsdn\_restoredata}
\label{\detokenize{_autosummary/pipeline.hsdn.cli.hsdn_restoredata:pipeline-hsdn-cli-hsdn-restoredata}}\label{\detokenize{_autosummary/pipeline.hsdn.cli.hsdn_restoredata::doc}}\index{hsdn\_restoredata() (in module pipeline.hsdn.cli)@\spxentry{hsdn\_restoredata()}\spxextra{in module pipeline.hsdn.cli}}

\begin{fulllineitems}
\phantomsection\label{\detokenize{_autosummary/pipeline.hsdn.cli.hsdn_restoredata:pipeline.hsdn.cli.hsdn_restoredata}}
\pysigstartsignatures
\pysiglinewithargsret
{\sphinxbfcode{\sphinxupquote{hsdn\_restoredata}}}
{\sphinxparam{\DUrole{n}{vis}\DUrole{o}{=}\DUrole{default_value}{None}}\sphinxparamcomma \sphinxparam{\DUrole{n}{caltable}\DUrole{o}{=}\DUrole{default_value}{None}}\sphinxparamcomma \sphinxparam{\DUrole{n}{reffile}\DUrole{o}{=}\DUrole{default_value}{None}}\sphinxparamcomma \sphinxparam{\DUrole{n}{products\_dir}\DUrole{o}{=}\DUrole{default_value}{None}}\sphinxparamcomma \sphinxparam{\DUrole{n}{copytoraw}\DUrole{o}{=}\DUrole{default_value}{None}}\sphinxparamcomma \sphinxparam{\DUrole{n}{rawdata\_dir}\DUrole{o}{=}\DUrole{default_value}{None}}\sphinxparamcomma \sphinxparam{\DUrole{n}{hm\_rasterscan}\DUrole{o}{=}\DUrole{default_value}{None}}}
{}
\pysigstopsignatures
\sphinxAtStartPar
Restore flagged and calibration single dish data from a pipeline run.

\sphinxAtStartPar
The hsdn\_restoredata task restores flagged and calibrated data from archived
ASDMs and pipeline flagging and calibration data products.

\sphinxAtStartPar
hsdn\_restoredata assumes that the ASDMs to be restored are present in the
directory specified by the \sphinxcode{\sphinxupquote{rawdata\_dir}} (default: ‘../rawdata’).

\sphinxAtStartPar
By default (\sphinxcode{\sphinxupquote{copytoraw}} = True), hsdn\_restoredata assumes that for each
ASDM in the input list, the corresponding pipeline flagging and calibration
data products (in the format produced by the hsdn\_exportdata task) are
present in the directory specified by \sphinxcode{\sphinxupquote{products\_dir}} (default: ‘../products’).
At the start of the task, these products are copied from the \sphinxcode{\sphinxupquote{products\_dir}}
to the \sphinxcode{\sphinxupquote{rawdata\_dir}}.

\sphinxAtStartPar
If \sphinxcode{\sphinxupquote{copytoraw}} = False, hsdn\_restoredata assumes that these products are
to be found in \sphinxcode{\sphinxupquote{rawdata\_dir}} along with the ASDMs.

\sphinxAtStartPar
The expected flagging and calibration products (for each ASDM) include:
\begin{itemize}
\item {} 
\sphinxAtStartPar
a compressed tar file of the final flagversions file, e.g.
uid\_\_\_A002\_X30a93d\_X43e.ms.flagversions.tar.gz

\item {} 
\sphinxAtStartPar
a text file containing the applycal instructions, e.g.
uid\_\_\_A002\_X30a93d\_X43e.ms.calapply.txt

\item {} 
\sphinxAtStartPar
a compressed tar file containing the caltables for the parent session,
e.g. uid\_\_\_A001\_X74\_X29.session\_3.caltables.tar.gz

\end{itemize}

\sphinxAtStartPar
hsdn\_restoredata performs the following operations:
\begin{itemize}
\item {} 
\sphinxAtStartPar
imports the ASDM(s)

\item {} 
\sphinxAtStartPar
removes the default MS.flagversions directory created by the filler

\item {} 
\sphinxAtStartPar
restores the final MS.flagversions directory stored by the pipeline

\item {} 
\sphinxAtStartPar
restores the final set of pipeline flags to the MS

\item {} 
\sphinxAtStartPar
restores the final calibration state of the MS

\item {} 
\sphinxAtStartPar
restores the final calibration tables for each MS

\item {} 
\sphinxAtStartPar
applies the calibration tables to each MS

\end{itemize}

\sphinxAtStartPar
When importing the ASDM and converting it to a Measurement Set (MS), if the
output MS already exists in the output directory, then the importasdm
conversion step is skipped, and the existing MS will be imported instead.
\begin{quote}\begin{description}
\sphinxlineitem{Parameters}\begin{itemize}
\item {} 
\sphinxAtStartPar
\sphinxstylestrong{vis} \textendash{} List of raw visibility data files to be restored. Assumed to be in the directory specified by rawdata\_dir.
example: vis={[}‘uid\_\_\_A002\_X30a93d\_X43e’{]}

\item {} 
\sphinxAtStartPar
\sphinxstylestrong{caltable} \textendash{} Name of output gain calibration tables. example: caltable=’ngc5921.gcal’

\item {} 
\sphinxAtStartPar
\sphinxstylestrong{reffile} \textendash{} Path to a file containing scaling factors between beams. The format is equals to jyperk.csv with five fields:
MS name, beam name (instead of antenna name), spectral window id,
polarization string, and the scaling factor.
Example for the file is as follows:
\#MS,Beam,Spwid,Polarization,Factor
mg2\sphinxhyphen{}20181016165248\sphinxhyphen{}181017.ms,NRO\sphinxhyphen{}BEAM0,0,I,1.000000000
mg2\sphinxhyphen{}20181016165248\sphinxhyphen{}181017.ms,NRO\sphinxhyphen{}BEAM0,1,I,1.000000000
mg2\sphinxhyphen{}20181016165248\sphinxhyphen{}181017.ms,NRO\sphinxhyphen{}BEAM0,2,I,1.000000000
mg2\sphinxhyphen{}20181016165248\sphinxhyphen{}181017.ms,NRO\sphinxhyphen{}BEAM0,3,I,1.000000000
mg2\sphinxhyphen{}20181016165248\sphinxhyphen{}181017.ms,NRO\sphinxhyphen{}BEAM1,0,I,3.000000000
mg2\sphinxhyphen{}20181016165248\sphinxhyphen{}181017.ms,NRO\sphinxhyphen{}BEAM1,1,I,3.000000000
mg2\sphinxhyphen{}20181016165248\sphinxhyphen{}181017.ms,NRO\sphinxhyphen{}BEAM1,2,I,3.000000000
mg2\sphinxhyphen{}20181016165248\sphinxhyphen{}181017.ms,NRO\sphinxhyphen{}BEAM1,3,I,3.000000000
mg2\sphinxhyphen{}20181016165248\sphinxhyphen{}181017.ms,NRO\sphinxhyphen{}BEAM2,0,I,0.500000000
mg2\sphinxhyphen{}20181016165248\sphinxhyphen{}181017.ms,NRO\sphinxhyphen{}BEAM2,1,I,0.500000000
mg2\sphinxhyphen{}20181016165248\sphinxhyphen{}181017.ms,NRO\sphinxhyphen{}BEAM2,2,I,0.500000000
mg2\sphinxhyphen{}20181016165248\sphinxhyphen{}181017.ms,NRO\sphinxhyphen{}BEAM2,3,I,0.500000000
mg2\sphinxhyphen{}20181016165248\sphinxhyphen{}181017.ms,NRO\sphinxhyphen{}BEAM3,0,I,2.000000000
mg2\sphinxhyphen{}20181016165248\sphinxhyphen{}181017.ms,NRO\sphinxhyphen{}BEAM3,1,I,2.000000000
mg2\sphinxhyphen{}20181016165248\sphinxhyphen{}181017.ms,NRO\sphinxhyphen{}BEAM3,2,I,2.000000000
mg2\sphinxhyphen{}20181016165248\sphinxhyphen{}181017.ms,NRO\sphinxhyphen{}BEAM3,3,I,2.000000000
If no file name is specified or specified file doesn’t exist,
all the factors are set to 1.0.
example: reffile=’’, reffile=’nroscalefactor.csv’

\item {} 
\sphinxAtStartPar
\sphinxstylestrong{products\_dir} \textendash{} Name of the data products directory. Default: ‘../products’.
example: products\_dir=’myproductspath’

\item {} 
\sphinxAtStartPar
\sphinxstylestrong{copytoraw} \textendash{} Copy calibration and flagging tables to raw data directory. Default: True.
example: copytoraw=False

\item {} 
\sphinxAtStartPar
\sphinxstylestrong{rawdata\_dir} \textendash{} Name of the raw data directory. Default: ‘../rawdata’.
example: rawdata\_dir=’myrawdatapath’

\item {} 
\sphinxAtStartPar
\sphinxstylestrong{hm\_rasterscan} \textendash{} Heuristics method for raster scan analysis. Two analysis modes, time\sphinxhyphen{}domain analysis (‘time’) and direction analysis (‘direction’), are available.
Default is ‘time’.

\end{itemize}

\sphinxlineitem{Returns}
\sphinxAtStartPar
The results object for the pipeline task is returned.

\end{description}\end{quote}
\subsubsection*{Examples}
\begin{enumerate}
\sphinxsetlistlabels{\arabic}{enumi}{enumii}{}{.}%
\item {} 
\sphinxAtStartPar
Restore the pipeline results for a single ASDM in a single session

\end{enumerate}

\begin{sphinxVerbatim}[commandchars=\\\{\}]
\PYG{g+gp}{\PYGZgt{}\PYGZgt{}\PYGZgt{} }\PYG{n}{hsdn\PYGZus{}restoredata} \PYG{p}{(}\PYG{n}{vis}\PYG{o}{=}\PYG{p}{[}\PYG{l+s+s1}{\PYGZsq{}}\PYG{l+s+s1}{mg2\PYGZhy{}20181016165248\PYGZhy{}190320.ms}\PYG{l+s+s1}{\PYGZsq{}}\PYG{p}{]}\PYG{p}{,} \PYG{n}{reffile}\PYG{o}{=}\PYG{l+s+s1}{\PYGZsq{}}\PYG{l+s+s1}{nroscalefactor.csv}\PYG{l+s+s1}{\PYGZsq{}}\PYG{p}{)}
\end{sphinxVerbatim}

\end{fulllineitems}



\chapter{Indices and tables}
\label{\detokenize{latex:indices-and-tables}}\begin{itemize}
\item {} 
\sphinxAtStartPar
\DUrole{xref}{\DUrole{std}{\DUrole{std-ref}{genindex}}}

\item {} 
\sphinxAtStartPar
\DUrole{xref}{\DUrole{std}{\DUrole{std-ref}{modindex}}}

\item {} 
\sphinxAtStartPar
\DUrole{xref}{\DUrole{std}{\DUrole{std-ref}{search}}}

\end{itemize}


\renewcommand{\indexname}{Python Module Index}
\begin{sphinxtheindex}
\let\bigletter\sphinxstyleindexlettergroup
\bigletter{p}
\item\relax\sphinxstyleindexentry{pipeline.h.cli}\sphinxstyleindexpageref{_autosummary/pipeline.h.cli:\detokenize{module-pipeline.h.cli}}
\item\relax\sphinxstyleindexentry{pipeline.h.cli.cli}\sphinxstyleindexpageref{_autosummary/pipeline.h.cli.cli:\detokenize{module-pipeline.h.cli.cli}}
\item\relax\sphinxstyleindexentry{pipeline.h.cli.utils}\sphinxstyleindexpageref{_autosummary/pipeline.h.cli.utils:\detokenize{module-pipeline.h.cli.utils}}
\item\relax\sphinxstyleindexentry{pipeline.hif.cli}\sphinxstyleindexpageref{_autosummary/pipeline.hif.cli:\detokenize{module-pipeline.hif.cli}}
\item\relax\sphinxstyleindexentry{pipeline.hifa.cli}\sphinxstyleindexpageref{_autosummary/pipeline.hifa.cli:\detokenize{module-pipeline.hifa.cli}}
\item\relax\sphinxstyleindexentry{pipeline.hifv.cli}\sphinxstyleindexpageref{_autosummary/pipeline.hifv.cli:\detokenize{module-pipeline.hifv.cli}}
\item\relax\sphinxstyleindexentry{pipeline.hsd.cli}\sphinxstyleindexpageref{_autosummary/pipeline.hsd.cli:\detokenize{module-pipeline.hsd.cli}}
\item\relax\sphinxstyleindexentry{pipeline.hsdn.cli}\sphinxstyleindexpageref{_autosummary/pipeline.hsdn.cli:\detokenize{module-pipeline.hsdn.cli}}
\end{sphinxtheindex}

\renewcommand{\indexname}{Index}
\printindex
\end{document}